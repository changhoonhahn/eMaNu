\section{HADES and Quijote Simulation Suites} \label{sec:hades} 
The HADES\footnote{https://franciscovillaescusa.github.io/hades.html} and 
Quijote\footnote{https://github.com/franciscovillaescusa/Quijote-simulations}
suites are sets of, 43000 total, $N$-body simulations run on multiple cosmologies,
including those with massive neutrinos ($\smnu > 0$ eV). In this work, 
we use a subset of the HADES and Quijote simulations. Below, we briefly describe these simulations; 
a summary of the simulations can be found in Table~\ref{tab:sims}. 
The HADES simulations start from Zel'dovich approximated initial conditions 
generated at $z=99$ using the~\cite{zennaro2017a} rescaling method and follow 
the gravitational evolution of $N_{\rm cdm}=512^3$ CDM, plus $N_{\nu}=512^3$ 
neutrino particles for $\smnu > 0$ eV cosmologies, to $z=0$. They are run using 
the {\sc GADGET-III} TreePM+SPH code~\citep{springel2005} in a periodic 
$(1~h^{-1}{\rm Gpc})^3$ box. All of the HADES simulations share the following 
cosmological parameter values, which are in good agreement with Planck 
constraints~\cite{ade2016a}: $\Om{=}0.3175, \Ob{=}0.049, \OL{=}0.6825, n_s{=}0.9624, h{=}0.6711$, 
and $k_{\rm pivot} = 0.05~h{\rm Mpc}^{-1}$. 

The HADES suite includes $N$-body simulations with degenerate massive neutrinos 
of $\smnu = $ 0.06, 0.10, and 0.15 eV. These simulations are run using the 
``particle method'', where neutrinos are described as a collisionless 
and pressureless fluid and therefore modeled as particles, same as 
CDM~\citep{brandbyge2008,viel2010}. HADES also includes simulations with massless 
neutrino and different values of $\sigma_8$ to examine the $\smnu-\sigma_8$ 
degeneracy. The $\sigma_8$ values in these simulations were chosen to match {\em either} 
$\sigma_8^m$ or $\sigma_8^{c}$ ($\sigma_8$ computed with respect to total matter, 
CDM + baryons + $\nu$, or CDM + baryons) of the massive neutrino simulations: 
$\sigma_8 = 0.822, 0.818, 0.807$, and $0.798$. Each model has $100$ independent 
realizations and we focus on the snapshots saved at $z = 0$. Halos closely 
trace the CDM+baryon field rather than the total matter field and neutrinos 
have negligible contribution to halo masses~\citep[\emph{e.g.}][]{ichiki2012, castorina2014, loverde2014, villaescusa-navarro2014}.
Hence, dark matter halos are identified in each realization using the Friends-of-Friends 
algorithm~\cite[FoF;][]{davis1985} with linking length $b=0.2$ on the CDM + baryon
distribution. We limit the catalogs to halos with masses above 
$M_{\rm lim} = 3.2\times 10^{13} h^{-1}M_\odot$. We refer readers
to~\cite{villaescusa-navarro2018} for more details on the HADES simulations. 

In addition to HADES, we use simulations from the Quijote suite, a
set of 42,000 $N$-body simulations that in total contain more than 8 trillion 
($8\times10^{12}$) particles over a volume of $42000 (h^{-1}{\rm Gpc})^3$. 
These simulations were designed to quantify the information content of 
different cosmological observables using Fisher matrix forecasting 
technique~(Section~\ref{sec:forecasts}). They are therefore constructed to accurately 
calculate the covariance matrices of observables and the derivatives of observables with 
respect to cosmological parameters: 
$\Om$, $\Ob$, $h$, $n_s$, $\sig$, and $\smnu$.

To calculate covariance matrices, Quijote includes $N_{\rm cov}{=}15,000$ $N$-body 
simulations at a fiducial cosmology ($\Om{=}0.3175$, $\Ob{=}0.049$, $h{=}0.6711$, 
$n_s{=}0.9624$, $\sig{=}0.834$, and $\smnu{=}0.0$ eV). It also includes sets 
of 500 $N$-body simulations run at different cosmologies where only one parameter 
is varied from the fiducial cosmology at a time for the derivatives. Along $\Om$, 
$\Ob$, $h$, $n_s$, and $\sig$, the fiducial cosmology is adjusted by either a 
small step above or below the fiducial value: 
$\{\Om^{+}, \Om^{-}, \Ob^{+}, \Ob^{-}, h^+, h^-, n_s^{+}, n_s^{-}, \sig^{+}, \sig^-\}$ . 
Along $\smnu$, because $\smnu \ge 0.0$ eV and the derivative of certain observable 
with respect to $\smnu$ is noisy, Quijote includes sets of 500 simulations for 
$\smnu = 0.1$, 0.2, and 0.4 eV. Table~\ref{tab:sims} lists the cosmologies included 
in the Quijote suite. 

The initial conditions for all Quijote simulations were generated at $z=127$ using 
2LPT for simulations with massless neutrinos and the Zel’dovich approximation for 
massive neutrinos. The suite also includes a set of simulations at the fiducial 
$\smnu = 0$ eV cosmology with Zel'dovich approximation initial conditions, which 
we use later in Section~\ref{sec:forecasts}. Like HADES, the initial conditions 
of simulations with massive neutrinos take their scale-dependent growth factors/rates 
into account using the \cite{zennaro2017a} method. From the initial conditions, all 
of the simulations follow the gravitational evolution of $512^3$ dark matter particles, 
and $512^3$ neutrino particles for massive neutrino models, to $z=0$ using {\sc Gadget-III}
TreePM+SPH code (same as HADES). Halos are then identified using the same FoF 
scheme and mass limit as HADES. For the fiducial cosmology, the halo catalogs 
have ${\sim}156,000$ halos ($\bar{n} \sim 1.56 \times 10^{-4}~h^3{\rm Mpc}^{-3}$) 
with $\bar{n} P_0(k=0.1)\sim 3.23$.
% The simulations run at the fiducial cosmology for covariance matrix estimation are standard $N$-body simulations. However, the rest are paired fixed simulations, which greatly reduce cosmic variance without introducing bias for a large set of statistics~\citep{angulo2016,pontzen2016,villaescusa-navarro2018}.  We confirm that the paired fixed simulations do not introduce any bias for the redshift-space halo bispectrum (the observable we consider in this paper). 
We refer readers to~\cite{villaescusa-navarro2019} for further details on the Quijote simulations.

\begin{figure}
\begin{center}
\includegraphics[width=0.9\textwidth]{figs/haloPlk_rsd_ratio.pdf}
    \caption{Impact of $\smnu$ and $\sig$ on the redshift-space halo power 
    spectrum monopole, $P_0$, and quadrupole, $P_2$, of the HADES simulation 
    suite,   
    \edt{
        which includes simulations with $\smnu = 0.0$, 0.06, 0.10, 
        and 0.15 eV (orange, green, and red) as well as massless neutrino 
        simulations with $\sig = 0.822$, 0.818, 0.807, and 0.798 
        (black solid, dashed, dot-dashed, and dotted). $P_\ell$ at 
        each of these cosmology is averaged over 100 $N$-body realizations. 
        While neutrinos suppress small-scale matter power spectrum,  
        the halo $P_\ell$ increases in amplitude with higher $\smnu$ 
        because a lower value of $\sig$ translates into larger halo 
        bias with a fixed halo mass limit. 
    }
    $\smnu$ and $\sig$ produce almost identical effects on halo clustering 
    on small scales ($k > 0.1\hmpc$). This degeneracy can be partially 
    broken through the quadrupole; however, {\em $\smnu$ and $\sig$ 
    produce almost the same effect on two-point clustering --- within a 
    few percent}.
    }
\label{fig:plk}
\end{center}
\end{figure}


%%%%%%%%%%%%%%%%%%%%%%%%%%%%%%%%%%%%%%%%%%
% simulation table
%%%%%%%%%%%%%%%%%%%%%%%%%%%%%%%%%%%%%%%%%%
\begin{table}
\caption{Specifications of the HADES and Quijote simulation suites.} 
\begin{center}
\begin{tabular}{cccccccccc} \toprule
    Name  &$\smnu$ & $\Omega_m$ & $\Omega_b$ & $h$ & $n_s$ & $\sigma^m_8$ & $\sigma^c_8$ & ICs & realizations \\
      &({\footnotesize eV}) & & & & & ({\footnotesize $10^{10}h^{-1}M_\odot$}) & ({\footnotesize $10^{10}h^{-1}M_\odot$}) & \\[3pt] \hline\hline
    \multicolumn{9}{c}{HADES suite} \\ \hline
    Fiducial    & 0.0   & 0.3175 & 0.049 & 0.6711 & 0.9624 & 0.833 & 0.833 & Zel'dovich & 100 \\ 
                & 0.06  & 0.3175 & 0.049 & 0.6711 & 0.9624 & 0.819 & 0.822 & Zel'dovich & 100 \\ 
                & 0.10  & 0.3175 & 0.049 & 0.6711 & 0.9624 & 0.809 & 0.815 & Zel'dovich & 100 \\ 
                & 0.15  & 0.3175 & 0.049 & 0.6711 & 0.9624 & 0.798 & 0.806 & Zel'dovich & 100 \\ 
                & 0.0   & 0.3175 & 0.049 & 0.6711 & 0.9624 & 0.822 & 0.822 & Zel'dovich & 100 \\ 
                & 0.0   & 0.3175 & 0.049 & 0.6711 & 0.9624 & 0.818 & 0.818 & Zel'dovich & 100 \\ 
                & 0.0   & 0.3175 & 0.049 & 0.6711 & 0.9624 & 0.807 & 0.807 & Zel'dovich & 100 \\ 
                & 0.0   & 0.3175 & 0.049 & 0.6711 & 0.9624 & 0.798 & 0.798 & Zel'dovich & 100 \\[3pt]
    \hline \hline
    \multicolumn{9}{c}{Qujiote suite} \\ \hline
    Fiducial 	    & 0.0         & 0.3175 & 0.049 & 0.6711 & 0.9624 & 0.834 & 0.834 & 2LPT & 15,000 \\ 
    Fiducial ZA     & 0.0         & 0.3175 & 0.049 & 0.6711 & 0.9624 & 0.834 & 0.834 & Zel'dovich& 500 \\ 
    $\smnu^+$       & \underline{0.1}   & 0.3175 & 0.049 & 0.6711 & 0.9624 & 0.834 & 0.834 & Zel'dovich & 500 \\ 
    $\smnu^{++}$    & \underline{0.2}   & 0.3175 & 0.049 & 0.6711 & 0.9624 & 0.834 & 0.834 & Zel'dovich & 500 \\ 
    $\smnu^{+++}$   & \underline{0.4}   & 0.3175 & 0.049 & 0.6711 & 0.9624 & 0.834 & 0.834 & Zel'dovich & 500 \\ 
    $\Omega_m^+$    & 0.0   & \underline{ 0.3275} & 0.049 & 0.6711 & 0.9624 & 0.834 & 0.834 & 2LPT & 500 \\ 
    $\Omega_m^-$    & 0.0   & \underline{ 0.3075} & 0.049 & 0.6711 & 0.9624 & 0.834 & 0.834 & 2LPT & 500 \\ 
    $\Omega_b^+$    & 0.0   & 0.3175 & \underline{0.051} & 0.6711 & 0.9624 & 0.834 & 0.834 & 2LPT & 500 \\ 
    $\Omega_b^-$    & 0.0   & 0.3175 & \underline{0.047} & 0.6711 & 0.9624 & 0.834 & 0.834 & 2LPT & 500 \\ 
    $h^+$           & 0.0   & 0.3175 & 0.049 & \underline{0.6911} & 0.9624 & 0.834 & 0.834 & 2LPT & 500 \\ 
    $h^-$           & 0.0   & 0.3175 & 0.049 & \underline{0.6511} & 0.9624 & 0.834 & 0.834 & 2LPT & 500 \\ 
    $n_s^+$         & 0.0   & 0.3175 & 0.049 & 0.6711 & \underline{0.9824} & 0.834 & 0.834 & 2LPT & 500 \\ 
    $n_s^-$         & 0.0   & 0.3175 & 0.049 & 0.6711 & \underline{0.9424} & 0.834 & 0.834 & 2LPT & 500 \\ 
    $\sigma_8^+$    & 0.0   & 0.3175 & 0.049 & 0.6711 & 0.9624 & \underline{0.849} & \underline{0.849} & 2LPT & 500 \\ 
    $\sigma_8^-$    & 0.0   & 0.3175 & 0.049 & 0.6711 & 0.9624 & \underline{0.819} & \underline{0.819} & 2LPT & 500 \\[3pt]
    \hline
\end{tabular} \label{tab:sims}
\end{center}
    {\bf \em Top}: The HADES suite includes sets of 100 $N$-body simulations with degenerate massive neutrinos 
    of $\smnu = $ 0.06, 0.10, and 0.15 eV as well as sets of simulations with $\smnu = 0.0$ eV and 
    $\sigma_8 = $ 0.822, 0.818, 0.807, and 0.798 to examine the $\smnu-\sigma_8$ degeneracy. 
    {\bf \em Bottom}: The Quijote suite includes 15,000 $N$-body simulations at the fiducial 
    cosmology to accurately estimate the covariance matrices. It also includes sets of 500 
    simulations at 13 different cosmologies, where only one parameter is varied from the fiducial 
    value (underlined), to estimate derivatives of observables along the cosmological parameters.
\end{table}
%%%%%%%%%%%%%%%%%%%%%%%%%%%%%%%%%%%%%%%%%%
