\begin{figure}
\begin{center}
    \includegraphics[width=0.6\textwidth]{figs/quijote_bkCov_kmax05.png} 
    \caption{Covariance matrix of the redshift-space halo bispectrum estimated 
    using $N_{\rm cov} = 15,000$ realizations of the Qujiote simulation suite at 
    the fiducial cosmology: $\Om{=}0.3175$, $\Ob{=}0.049$, $h{=}0.6711$, $n_s{=}0.9624$, $\sig{=}0.834$, 
    and $\smnu{=}0.0$ eV. We include all possible triangle configurations with 
    $k_1, k_2, k_3 \leq k_{\rm max} = 0.5~h/{\rm Mpc}$ and order the configurations 
    (bins) in the same way as Figures~\ref{fig:bk_amp} and~\ref{fig:dbk_amp}. We 
    use the covariance matrix above for the Fisher matrix forecasts presented in 
    Section~\ref{sec:forecasts}. 
    }
\label{fig:bk_cov}
\end{center}
\end{figure}

\begin{figure}
\begin{center}
    \includegraphics[width=0.6\textwidth]{figs/quijote_dlogBdthetas_reg_fin.pdf} 
    \caption{Derivatives of the redshift-space halo bispectrum, ${\rm d}\log B_0 / {\rm d} \theta$ 
    with respect to $\Om$, $\Ob$, $h$, $n_s$, $\sig$, and $\smnu$ as a function 
    of all $1898$ triangle configurations with $k_1, k_2, k_3 \leq 0.5~\mpc$ 
    (top to bottom panels). We estimate the derivatives at the fiducial parameters 
    using $N_{\rm deriv} = 1,500$ $N$-body realizations from the Quijote suite. 
    The configurations above are ordered by looping through $k_3$ in the outer 
    most loop and $k_1$ in the inner most loop satisfying $k_1 \le k_2 \le k_3$. 
    By using $N$-body simulations for the derivatives, we exploit their accuracy 
    in the nonlinear regime and rely on fewer assumptions and approximations 
    than analytic methods like perturbation theory. 
    }
\label{fig:bk_deriv}
\end{center}
\end{figure}


%\begin{figure}
%\begin{center}
%    \includegraphics[width=\textwidth]{figs/quijote_pbkFisher_freeMmin_dmnu_fin_kmax02.pdf}
%    \caption{Fisher matrix constraints for $\smnu$ and other cosmological parameters 
%    for the redshift-space halo bispectrum monopole (orange). For comparison, we 
%    include Fisher parameter constraints for the redshift-space halo powerspectrum monopole 
%    in blue. The contours mark the $68\%$ and $95\%$ confidence interals. We set 
%    $k_{\rm max} = 0.2~h/{\rm Mpc}$ for both power spectrum and bispectrum. We 
%    include in our forecasts $b'$ and $M_{\rm min}$, a free amplitude scaling factor
%    and halo mass limit, respectively. They serve as a simplistic bias model and we 
%    marginalize over them to that our constraints do not include extra constraining 
%    power from the difference in bias/number density in the different Quijote cosmologies.
%    The bispectrum {\em substantially} improves constraints on all of the cosmological parameters 
%    over the power spectrum. {\em For $\smnu$, the bispectrum improves the constraint
%    from $\sigma_{\smnu}{=}~1.163$ to $0.112$ eV --- an order of magnitude improvement 
%    over the power spectrum}.}
%\label{fig:bk_fish_02}
%\end{center}
%\end{figure}

\subsection{$\smnu$ and other Cosmological Parameter Forecasts} \label{sec:forecasts}
We demonstrate in the previous section with the HADES simulations, that 
the bispectrum helps break the $\smnu$--$\sig$ degeneracy, a major 
challenge in precisely constraining $\smnu$ with the power spectrum. 
While this establishes the bispectrum as a promising probe for $\smnu$, 
we are ultimately interested in determining the constraining power of the 
bispectrum for an analysis that include cosmological parameters beyond 
$\smnu$ and $\sig$--- \emph{i.e.} $\Om$, $\Ob$, $h$, and $n_s$. The Quijote 
simulation suite is \emph{specifically} designed to answer this question
through Fisher matrix forecast.

First, the Quijote suite includes $N_{\rm cov} = 15,000$ $N$-body realizations run at a 
fiducial cosmology: $\smnu{=}0.0$eV, $\Om{=}0.3175, \Ob{=}0.049, 
n_s{=}0.9624, h{=}0.6711$, and $\sig{=}0.834$ (see Table~\ref{tab:sims}). 
This allows us to robustly estimate the $1898\times1898$ covariance matrix 
of the {\em full} bispectrum, $\bfi{C}$ (Figure~\ref{fig:bk_cov}). Second, the Quijote 
suite includes $500$ $N$-body realizations evaluated at $13$ different 
cosmologies, each a small step away from the fiducial cosmology parameter 
values along one parameter (Section~\ref{sec:hades} and~Table~\ref{tab:sims}). 
We apply redshift-space distortions along 3 different directions for 
these $500$ realizations, which then effectively gives us $N_{\rm deriv.} = 1,500$ 
realizations. These simulations allow us to precisely estimate the derivatives 
of the bispectrum with respect to each of the cosmological parameters. 

\begin{figure}
\begin{center}
    \includegraphics[width=\textwidth]{figs/quijote_pbkFisher_bMmin_dmnu_fin_kmax0_50_reg.pdf} 
    \caption{Fisher matrix constraints for $\smnu$ and other cosmological parameters 
        for the redshift-space halo bispectrum monopole (orange). We include Fisher
        parameter constraints for the redshift-space halo powerspectrum monopole in blue 
        for comparison. The contours mark the $68\%$ and $95\%$ confidence interals. 
        We set $k_{\rm max} = 0.5~h/{\rm Mpc}$ for both power spectrum and bispectrum. 
        We include in our forecasts $b'$ and $M_{\rm min}$, a free amplitude scaling 
        factor and halo mass limit, respectively. They serve as a simplistic bias model 
        and we marginalize over them so that our constraints do not include extra 
        constraining power from the difference in bias/number density in the different 
        Quijote cosmologies. The bispectrum {\em substantially} improves constraints on all of the cosmological 
        parameters over the power spectrum. Constraints on $\Om$, $\Ob$, $h$, $n_s$, and $\sig$
        improve by factors of 3.1, 4.1, 5.1, 5.9, and 3.3, respectively. {\em For $\smnu$, the
        bispectrum improves $\sigma_{\smnu}$ from $0.1962$ to $0.0342$ eV --- a factor 
        of ${\sim}6$ improvement over the power spectrum}.}
\label{fig:bk_fish_05}
\end{center}
\end{figure}

Since their introduction to cosmology over two decades ago, Fisher information 
matrices have been ubiquitously used to forecast the constraining power of future 
experiments~\citep[\emph{e.g.}][]{jungman1996,tegmark1997,dodelson2003,heavens2009,verde2010}. 
Defined as 
\beq 
F_{ij} = - \bigg \langle \frac{\partial^2 \mathrm{ln} \mathcal{L}}{\partial \theta_i \partial \theta_j} \bigg \rangle,
\eeq
where $\mathcal{L}$ is the likelihood, the Fisher matrix for the bispectrum can 
be written as 
\beq \label{eq:fullfish} 
F_{ij} = \frac{1}{2}~\mathrm{Tr} \Bigg[\bfi{C}^{-1}\parti{\bfi{C}}\bfi{C}^{-1}\partj{\bfi{C}} + \bfi{C}^{-1} \left(\parti{\overline{B}_0}\partj{\overline{B}_0}^T + \parti{\overline{B}_0}^T \partj{\overline{B}_0} \right)\Bigg].
\eeq
Since we assume that the $B_0$ likelihood is Gaussian, including the first 
term in Eq.~\ref{eq:fullfish} runs the risk of incorrectly including information 
from the covariance already included in the mean~\citep{carron2013}. We, therefore,
conservatively neglect the first term and calculate the Fisher matrix as, 
\beq \label{eq:fisher}
F_{ij} = \frac{1}{2}~\mathrm{Tr} \Bigg[\bfi{C}^{-1} \left(\parti{\overline{B}_0}\partj{\overline{B}_0}^T + \parti{\overline{B}_0}^T \partj{\overline{B}_0} \right)\Bigg],
\eeq
directly with $\bfi{C}$ and $\partial B_0/\partial \theta_i$ along each cosmological 
parameter from the Quijote simulations. 

For $\Om$, $\Ob$, $h$, $n_s$, and $\sig$, we estimate 
\beq \label{eq:dbkdt} 
\frac{\partial \overline{B}_0}{\partial \theta_i} \approx \frac{\overline{B}_0(\theta_i^{+})-\overline{B}_0(\theta_i^{-})}{\theta_i^+ - \theta_i^-}, 
\eeq
where $\overline{B}_0(\theta_i^{+})$ and $\overline{B}_0(\theta_i^{-})$ are 
the average bispectrum of the $1,500$ realizations at $\theta^{+}$ and $\theta^{-}$, 
respectively. Meanwhile, for $\smnu$, where the fiducial value is 0.0 eV and we 
cannot have negative $\smnu$, we use the Quijote simulations at $\smnu^+$, 
$\smnu^{++}$, 
$\smnu^{+++}=0.1, 0.2, 0.4$ eV (Table~\ref{tab:sims}) to estimate 
\beq \label{eq:dbkdmnu} 
\frac{\partial \overline{B}_0}{\partial \smnu} \approx \frac{-21\overline{B}_0(\smnu^{\rm fid}) + 
32 \overline{B}_0(\smnu^{+}) - 12 \overline{B}_0(\smnu^{++}) + \overline{B}_0(\smnu^{+++})}{1.2}, 
\eeq
which provides a $\mathcal{O}(\delta \smnu^2)$ order approximation. By using these 
$N$-body simulations, instead of analytic methods (\emph{e.g.} perturbation theory), 
we exploit the accuracy of the simulations in the nonlinear regime and rely on fewer 
assumptions and approximations. In fact, these $N$-body simulation estimated 
derivatives are key in enabling us to quantify, for the first time, the full information 
content of the redshift-space bispectrum in the non-linear regime. We present the 
bispectrum derivatives for all triangle configurations in Figure~\ref{fig:bk_deriv} and  
discuss subtleties of the bispectrum derivative and tests of convegence and stability 
in Appendix~\ref{sec:numerical}. 

We present the constraints on $\smnu$ and other cosmological parameters 
$\{\Om, \Ob, h, n_s, \sig\}$ derived from the redshift-space halo bispectrum 
Fisher matrix for $k_{\rm max} = 0.5~h/{\rm Mpc}$ in Figure~\ref{fig:bk_fish_05}. 
We include Fisher constraints for the redshift-space halo power spectrum 
monopole with the same $k_{\rm max}$ for comparison (blue). The shaded contours 
mark the $68\%$ and $95\%$ confidence interals. We include in our 
Fisher constraints the following nuisance parameters: $b'$ and $M_{\rm min}$. 
$b'$ is a scaling factor on the bispectrum amplitude, analogous to linear bias, 
and $M_{\rm min}$ is the halo mass limit, which we choose as a nuisance
parameter to address the difference in the number densities among the Quijote
cosmologies, which impacts the derivatives $\partial \overline{B}_0/\partial \theta_i$. 
For instance, the $\sig^{+}$ and $\sig^{-}$ cosmologies have halo 
$\bar{n} = 1.586\times10^{-4}$ and $1.528 \times 10^{-4}(h^{-1}{\rm Mpc})^{-3}$. 
These parameters serve as a simplistic bias model and by marginalizing 
over them we aim to ensure that our Fisher constraints do not include extra 
constraining power from the difference in bias or number density. $b'$ is a 
multiplicative factor so $\partial \overline{B}_0/\partial b' = \overline{B}_0$. 
$\partial \overline{B}_0/\partial M_{\rm min}$, we estimate numerically using 
$\overline{B}_0$ evaluated at $M^{+}_{\rm min}{=}3.3\times10^{13}h^{-1}M_\odot$ 
and $M^{-}_{\rm min}{=}3.1\times10^{13}h^{-1}M_\odot$ with all other parameters 
set to the fiducial value. 

{\em The bispectrum substantially improves constraints on all parameters 
over the power spectrum.} For $k_{\rm max} = 0.5~h/{\rm Mpc}$, the 
bispectrum tightens the marginalized $1\sigma$ constraints, $\sigma_\theta$, of $\Om$, 
$\Ob$, $h$, $n_s$, and $\sig$ by factors of $\sim$3, 4, 5, 6, and 3 over 
the power spectrum. {\em For $\smnu$, the bispectrum improves the constraint 
from $\sigma_{\smnu}{=}~0.1962$ to $0.0342$~eV --- a factor of 6 improvement 
over the power spectrum}. This $\sigma_{\smnu}{=}0.0342$~eV constraint is for 
the {\em bispectrum alone} and only for a $1h^{-1}{\rm Gpc}$ box. For a larger 
volume, V, $\sigma_\theta$ scales as $\propto1/\sqrt{V}$. We list the precise
marginalized Fisher parameter constraints of both cosmological and nuisance 
parameters for $P_0$ and $B_0$ in Table~\ref{tab:forecast}. 

Even below $k_{\rm max} < 0.5~h/{\rm Mpc}$, the bispectrum significantly 
improves cosmological parameter constraints. We compare $\sigma_\theta$ 
of $\Om$, $\Ob$, $h$, $n_s$, $\sig$, and $\smnu$ as a function of $k_{\rm max}$ 
for $B_0$ (orange) and $P_0$ (blue) in Figure~\ref{fig:fish_kmax}. We only 
include the $k_{\rm max}$ range where the Fisher forecast is well defined --- 
\emph{i.e.} more data bins than the number of parameters: $k_{\rm max} > 8~k_{\rm f} \approx 0.05~\mpc$ 
for $P_0$ and $k_{\rm max} > 12~k_{\rm f} \approx 0.075~h/{\rm Mpc}$ for 
$B_0$. Figure~\ref{fig:fish_kmax} reveals that the improvement of the
bispectrum $\sigma_\theta$ over the power spectrum $\sigma_\theta$ is 
larger at higher $k_{\rm max}$. Although limited by the $k_{\rm max}$ range, 
the figure suggests that on large scales ($k_{\rm max}\lesssim 0.1~h/{\rm Mpc}$) 
$\sigma_\theta$ of $P_0$ crosses over $\sigma_\theta$ of $B_0$ so $P_0$ has more 
constraining power than $B_0$, as expected on linear scales. At slightly 
larger $k_{\rm max}$, $k_{\rm max} = 0.2~h/{\rm Mpc}$, we find that 
the bispectrum substantially improves $\sigma_\theta$ by factors of
$\sim 1.8$, 2.1, 2.6, 2.5, 2.6, and $3.1$ for $\Om$, $\Ob$, $h$, $n_s$, $\sig$, 
and $\smnu$ respectively. 

%%%%%%%%%%%%%%%%%%%%%%%%%%%%%%%%%%%%%%%%%%
% forecast table
%%%%%%%%%%%%%%%%%%%%%%%%%%%%%%%%%%%%%%%%%%
\begin{table}
    \caption{Marginalized Fisher parameter constraints from the redshift-space halo power 
    spectrum (top) and bispectrum (bottom) for different values of $k_{\rm max}$. We list 
    constraints for cosmological parameters $\smnu$, $\Omega_m$, $\Omega_b$, $h$, $n_s$, 
    and $\sig$ as well as nuisance parameters $b'$ and $M_{\rm min}$.} 
\begin{center} 
    \begin{tabular}{cccccccccc} \toprule
         & $k_{\rm max}$ & $\smnu$ & $\Omega_m$ & $\Omega_b$ & $h$ & $n_s$ & $\sig$ & $b'$ & $M_{\rm min}$ \\
         & ({\footnotesize $h/{\rm Mpc}$}) &({\footnotesize eV}) & & & & & & & ({\footnotesize $10^{13} h^{-1}M_\odot$}) \\[3pt] \hline\hline

            &     & 0.0 & 0.3175 & 0.049 & 0.6711 & 0.9624 & 0.834 & 1. & 3.2  \\ 
    $P_0$  & 0.2 &$\pm0.333$ & $\pm0.052$ & $\pm0.030$ & $\pm0.372$ & $\pm0.347$ & $\pm0.128$ & $\pm0.649$ & $\pm5.045$\\
           & 0.3 &$\pm0.277$ & $\pm0.044$ & $\pm0.023$ & $\pm0.273$ & $\pm0.276$ & $\pm0.069$ & $\pm0.383$ & $\pm2.457$\\
           & 0.4 &$\pm0.228$ & $\pm0.040$ & $\pm0.020$ & $\pm0.235$ & $\pm0.240$ & $\pm0.059$ & $\pm0.226$ & $\pm1.270$\\
           & 0.5 &$\pm${\bf 0.196}& $\pm0.036$ & $\pm0.018$ & $\pm0.207$ & $\pm0.213$ & $\pm0.048$ & $\pm0.157$ & $\pm0.807$\\\hline
%    {\footnotesize with Planck priors} & 0.5 &$\pm0.080$& $\pm0.013$ & $\pm0.0012$ & $\pm0.0086$ & $\pm0.0048$ & $\pm0.019$ & $\pm0.067$ & $\pm0.362$\\\hline
                   
    $B_0$  & 0.2 &$\pm0.107$ & $\pm0.029$ & $\pm0.014$ & $\pm0.144$ & $\pm0.140$ & $\pm0.050$ & $\pm0.265$ & $\pm1.317$\\
           & 0.3 &$\pm0.065$ & $\pm0.020$ & $\pm0.008$ & $\pm0.077$ & $\pm0.074$ & $\pm0.023$ & $\pm0.143$ & $\pm0.657$\\
           & 0.4 &$\pm0.043$ & $\pm0.015$ & $\pm0.006$ & $\pm0.052$ & $\pm0.047$ & $\pm0.016$ & $\pm0.088$ & $\pm0.369$\\
           & 0.5 &$\pm${\bf 0.034}& $\pm0.012$ & $\pm0.004$ & $\pm0.040$ & $\pm0.036$ & $\pm0.014$ & $\pm0.070$ & $\pm0.269$\\[3pt]

%B sigmas 0.0086, 0.0008, 0.0060, 0.0043, 0.0085, 0.0297, 0.0679, 0.2549
    \hline
\end{tabular} \label{tab:forecast}
\end{center}
\end{table}
%%%%%%%%%%%%%%%%%%%%%%%%%%%%%%%%%%%%%%%%%%

%%%%%%%%%%%%%%%%%%%%%%%%%%%%%%%%%%%%%%%%%%
% sigma(kmax)  
%%%%%%%%%%%%%%%%%%%%%%%%%%%%%%%%%%%%%%%%%%
\begin{figure}
\begin{center}
    \includegraphics[width=0.9\textwidth]{figs/quijote_Fisher_kmax_bMmin_dmnu_fin_reg_planck.pdf} 
    \caption{Marginalized $1\sigma$ constraints, $\sigma_\theta$, of the cosmological 
    parameters $\Om$, $\Ob$, $h$, $n_s$, $\sig$, and $\smnu$ as a function 
    of $k_{\rm max}$ for the redshift-space halo bispectrum (orange) and power 
    spectrum (blue). The constraints are marginalized over the nuisance parameters 
    $b'$ and $M_{\rm min}$ in our forecast (Section~\ref{sec:forecasts}). We impose 
    $k_{\rm max} > 8~k_{\rm f}$ for $P_0$ and $k_{\rm max} > 12~k_{\rm f}$ 
    for $B_0$, the $k$ ranges where we have more data bins than number of parameters. 
    We also include $\sigma_\theta$ constraints with {\em Planck} priors (dotted). 
    Even below $k_{\rm max} < 0.5~h/{\rm Mpc}$, the bispectrum significantly 
    improves cosmological parameter constraints. The improvement, however, is larger 
    for higher $k_{\rm max}$. At $k_{\rm max} = 0.2~h/{\rm Mpc}$, the bispectrum 
    improves constraints on $\Om$, $\Ob$, $h$, $n_s$, $\sig$, and $\smnu$ by factors 
    of $\sim 1.8$, 2.1, 2.6, 2.5, 2.6, and 3.1 over the power spectrum. Even with {\em Planck}
    priors, $B_0$ significantly improves $\sigma_\theta$ for $k_{\rm max} \gtrsim 0.2~\mpc$. 
    While the constraining power of $P_0$ saturates at $k_{\rm max} = 0.2~\mpc$,
    the constraining power of $B_0$ continues to increase out to $k_{\rm max} = 0.5~\mpc$.}
\label{fig:fish_kmax}
\end{center}
\end{figure}
%%%%%%%%%%%%%%%%%%%%%%%%%%%%%%%%%%%%%%%%%%
Our forecasts demonstrate that the bispectrum has significant constraining power 
beyond the powerspectrum in the weakly nonlinear regime ($k > 0.1~\mpc$). 
This constraining power 
comes from the bispectrum breaking degeneracies among the cosmological and 
nuisance parameters. This is evident when we compare the unmarginalized 
constraints from $P_0$ and $B_0$: $1/\sqrt{F_{ii}}$ where $F_{ii}$ is a 
diagonal element of the Fisher matrix. For $k < 0.4~\mpc$, the unmarginalized 
cosntraints from $P_0$ are tighter than those from $B_0$. Yet, once we 
marginalize the constraints over the other parameters, the $B_0$ constraints 
are tigher than $P_0$ for $k > 0.1~\mpc$. The derivatives, 
$\partial B_0/\partial \theta_i$, also shed light on how $B_0$ breaks parameter 
degeneracies. The parameter degeneracies in the $P_0$ forecasts of Figure~\ref{fig:bk_fish_05} 
are consistent with similarities in the shape and scale dependence of 
$P_0$ derivatives $\partial P_0/\partial \theta$. On the other hand, the 
$B_0$ derivatives with respect to the parameters have significant different 
scale and triangle shape dependences, as seem in Figure~\ref{fig:bk_deriv}.

By exploiting the massive number of $N$-body simulations of the Quijote 
suite, we present for the first time the full information content of the 
redshift-space bispectrum beyond the linear regime. The information content 
of the bispectrum has previously been examined using perturbation 
theory. Previous works, for instance, measure the signal-to-noise ratio (SNR) 
of the bispectrum derived from covariance matrices estimated using perturbation 
theory~\citep{sefusatti2005, chan2017}. More recently, \cite{chan2017},
used covariance matrices that include non-Gaussian contributions calibrated with 
$N$-body simulations to find that the cumulative SNR of the halo bispectrum is 
$\sim 30\%$ of the SNR of the halo power spectrum at $k_{\rm max} \sim 0.1~\mpc$ 
and increases to $\sim 40\%$ at $k_{\rm max}\sim 0.35~\mpc$. While these simple 
SNR measurements cannot be easily compared to Fisher analysis~\citep{repp2015, blot2016}, 
we note that they are roughly consistent with the unmarginalized constraints, 
which loosely represent the SNRs of the derivatives. 
Also, when we measure the the halo power spectrum and bispectrum SNRs using our 
covariance matrices (Figure~\ref{fig:bk_cov}), we find a relation between the 
SNRs consistent with \cite{chan2017}. Outside the $k$ range explored by \cite{chan2017},  
$k_{\rm max}>0.35~\mpc$, we find that the SNR of $B_0$ continues to increase 
at higher $k_{\rm max}$ in constract to the $P_0$ SNR, which saturates at 
$k_{\rm max}\sim 0.1~\mpc$. At $k_{\rm max} = 0.75~\mpc$, the largest $k$ we 
measure the $B_0$, the SNR of $B_0$ is $\sim75\%$ of the SNR of $P_0$.

Beyond these signal-to-noise calculations, a number of previous works 
have quantified the information content of the bispectrum with Fisher 
forecasts~\citep{scoccimarro2004, sefusatti2006, sefusatti2007, song2015, tellarini2016, yamauchi2017a, karagiannis2018, yankelevich2019}. 
While most of these works fix most cosmological parameters and focus soley 
on forecasting constraints of primordial non-Gaussianity and bias parameters, 
\cite{sefusatti2006} and \cite{yankelevich2019} provide 
bispectrum forecasts for full sets of cosmological parameters. In \cite{sefusatti2006}, 
they present likelihood analysis forecasts for $\omega_d$, $\omega_b$, $\OL$, 
$n_s$, $A_s$, $w$, $\tau$. %$\omega_d = \Omega_{\rm cdm}h^2$ and $\omega_b = \Ob h^2$ are the physical dark matter and baryon densities,  $\tau$ is the reionization optical depth, $A_s$ is the primordial amplitude of scalar fluctuations, and $w$ is dark energy equation of state parameter.  
For $\Lambda$CDM, with fixed bias parameters, and $k_{\rm max} = 0.3~\mpc$, 
they find constraints on $\Om$, $\Ob$, $h$, $n_s$, and $\sig$ from WMAP, 
$P_0$, and $B_0$ is $\sim 1.5$ times tighter than constraints from WMAP 
and $P_0$. In comparison, for 
$k_{\rm max} = 0.3~\mpc$ our $B_0$ constraints are tighter than $P_0$ constraints 
by factors of 2.1, 2.9, 3.5, 3.7, and 3.0. Both \cite{sefusatti2006} and our 
analysis find significantly tighter constraints with the bispectrum. They 
however include the WMAP likelihood in their forecast and use perturbation 
theory models, which break down on small scales. In a comparison to $N$-body 
simulations, \cite{lazanu2016} find that 
perturbation theory models of the matter bispectrum deviate by $>5\%$ 
at $k_{\rm max} \gtrsim 0.15~\mpc$. 

\cite{yankelevich2019} present Fisher forecasts for $\Omega_{\rm cdm}$, $\Ob$, 
$h$, $n_s$, $A_s$, $w_0$, and $w_0$ for a 
Euclid-like survey~\citep{laureijs2011} in 14 non-overlapping redshift 
bins over $0.65 < z < 2.05$. They use the full redshift-space bispectrum,
rather than just the monopole, and a more sophisticated bias expansion 
than \cite{sefusatti2006} but use a perturbation theory bispectrum model, 
which consequently limit their forecast to $k_{\rm max} = 0.15~\mpc$. 
They find similar constraining power on cosmological parameters from $B$ 
alone as $P$.  They also find that combining the bispectrum with the power 
spectrum only moderately improves parameter constraints because posterior 
correlations are similar for $P$ and $B$. While this seemingly conflicts 
with the results we present, there are significant differences between our 
forecasts. For instance, their forecasts are at higher redshifts, $z > 0.7$,
where we expect the constraining power of $B$ to be weaker than at $z=0$. 
They also forecast the {\em galaxy} $P$ and $B$ and marginalizes over 
56 nuisance parameters (14 $z$ bins each with 3 bias parameters and 1 
RSD parameter). They also neglect non-Gaussian contributions to the $B$ 
covariance matrix, which substantially impact the constraints especially 
on small scales~\citep{chan2017}. %; for $k_{\rm max} \sim 0.14~\mpc$, neglecting these contributions overestimates the SNR of the bispectrum by a factor of $\sim 3$~\citep{chan2017}. 
Despite differences, \cite{yankelevich2019} find that the constraining 
power of $B$ relative to $P$ increases for higher $k_{\rm max}$, consistent 
with our forecasts as a function of $k_{\rm max}$ (Figure~\ref{fig:fish_kmax}). 
Also, consistent with their results, for $k_{\rm max} = 0.15~\mpc$, 
we find similar posterior correlations between the $P_0$ and $B_0$ 
constraints. At $k_{\rm max} = 0.5~\mpc$, however, we find the posterior 
correlations are no longer similar, which contribute to the constraining
power of $B_0$~(Figure~\ref{fig:bk_fish_05}).

Various differences between our forecast and previous work prevent more 
thorough comparisons. Three crucial aspects, however, distinguish our
forecasts from other works. We present the first bispectrum forecasts 
for a full set of cosmological parameters using bispectrum measured entirely 
from $N$-body simulations. This allows us to go beyond perturbation theory 
forecasts and quantify the full information content of the redshift-space 
bispectrum out to nonlinear regimes. Second, we present the first forecast 
of the full bispectrum of all 1898 triangles out to $k_{\rm max} = 0.5~\mpc$, 
which is only made possible with the immense number (42,000) of $N$-body
simulations in the Quijote suite. Lastly, we present the first bispectrum 
forecast of cosmological parameters that includes neutrinos and demonstrates 
the constraining power of the bispectrum for $\smnu$. Below, we underline a 
few caveats of our forecasts.

Our forecasts are derived from Fisher matrices. Such forecasts make 
the assumption that the posterior is approximately Gaussian and, as a result, 
they underestimate the constraints for posteriors that are highly 
non-elliptical or asymmetric~\cite{wolz2012}. Fisher matrices also rely 
on the stability, and in our case also convergence, of numerical derivatives. 
We examine the stability of the $P_0$ and $B_0$ derivatives with respect 
to $\smnu$ by comparing the derivatives computed using $N$-body simulations 
at three different sets of cosmologies: (1) \{fiducial, $\smnu^{+}$, $\smnu^{++}$, 
and $\smnu^{+++}$\} (Eq.~\ref{eq:dbkdmnu}), (2) \{fiducial, $\smnu^{+}$, 
and $\smnu^{++}$\}, and (3) \{fiducial and $\smnu^{+}$ \}~(see Appendix~\ref{sec:numerical}; Figure~\ref{fig:dPBdmnu}). 
The derivatives computed using the different set of cosmologies, do not impact 
the $\Om$, $\Ob$, $h$, $n_s$, and $\sig$ constraints. They do however affect the 
$\smnu$ constraints; but because $P_0$ and $B_0$ derivatives are affected by 
the same factor, the relative improvement of the $B_0$ $\smnu$ constraint over 
the $P_0$ constraints is not impacted. 
%$\sigma_\smnu = 0.1962$, 0.2385, and 0.3774 eV with $P_0$ and $\sigma_\smnu = 0.0342$, 0.0416, and 0.0657 eV with $B_0$. 
In addition to the stability, because we use $N$-body simulations we test 
whether the convergence of our covariance matrix and derivatives impact 
our forecasts by varying the number of simulations 
used to estimate them: $N_{\rm cov}$ and $N_{\rm deriv}$, repsectively. 
For $N_{\rm cov}$, we find $< 5\%$ variation in the Fisher matrix elements, 
$F_{ij}$, for $N_{\rm cov} > 5000$ and $< 1\%$ variation in $\sigma_\theta$
for $N_{\rm cov} > 12000$.  For $N_{\rm deriv}$, we find $< 5\%$ variation 
in the $F_{ij}$ elements and $< 5\%$ variation in $\sigma_\theta$ for 
$N_{\rm deriv} > 1200$. Since our constraints vary by $< 10\%$ for sufficient 
$N_{\rm cov}$ and $N_{\rm deriv}$, the convergence of the covariance matrix 
and derivatives do not impact our forecasts to the accuracy level of Fisher 
forecasting. We refer readers to Apprendix~\ref{sec:numerical} for a more 
details on the robustness of our results to the stability of the derivatives 
and convergence. 

We argue that the constraining power of the bispectrum and its improvement 
over the power spectrum come from breaking degeneracies among the cosmological 
parameters. However, numerical noise can impact our forecasts when we invert 
the Fisher matrix. Since {\em Planck} constrain $\{\Om$, $\Ob$, $h$, $n_s$, $\sig\}$ 
tighter than the $P_0$ and $B_0$ alone, the elements of the {\em Planck} prior 
matrix are larger than the elements of $P_0$ 
and $B_0$ Fisher matrices. Including {\em Planck} priors (\emph{i.e.} adding the 
prior matrix to the Fisher matrix) increases the numerical stability of the 
matrix inversion. It also reveals whether the bispectrum still improves parameter 
constraints once we include CMB constraints. With {\em Planck} priors and $P_0$ to 
$k_{\rm max} = 0.5~\mpc$, we derive the following constraints: 
$\sigma_{\Om} = 0.0125$, $\sigma_{\Ob} = 0.0012$, $\sigma_h=0.0086$, 
$\sigma_{n_s}=0.0048$, and $\sigma_{\sig}=0.0194$. 
Including {\em Planck} priors expectedly tighten the constraints from $P_0$ alone. 
Meanwhile, with {\em Planck} priors and $B_0$ to $k_{\rm max} = 0.5~\mpc$, we get 
$\sigma_{\Om} = 0.0086$, $\sigma_{\Ob} = 0.0008$, $\sigma_h=0.0006$, 
$\sigma_{n_s}=0.0043$, and $\sigma_{\sig}=0.0085$, 1.5, 1.6, 1.4, 1.1, and 2.3 
times tighter constraints. For $\smnu$, $\sigma_{\smnu} = 0.0802$ eV for $P_0$ 
and $\sigma_{\smnu} = 0.0297$ eV, a factor of 2.7 improvement. Since we find
substantial improvements in parameter constraints with the {\em Planck} prior, 
the improvement from $B_0$ are numerically robust. Furthermore, $\sigma_\theta$ 
as a function of $k_{\rm max}$ with {\em Planck} priors reveal that while the 
constraining power of $P_0$ saturates at $k_{\rm max} = 0.2~\mpc$, the constraining 
power of $B_0$ continues to increase out to $k_{\rm max} = 0.5~\mpc$ 
(dotted; Figure~\ref{fig:fish_kmax}). 

Our forecasts are derived using the power spectrum and bispectrum in
\emph{periodic boxes}. We do not consider a realistic geometry or radial selection 
function of actual observations from galaxy 
surveys. A realistic selection function will smooth out the triangle 
configuration dependence and consequently degrade the constraining power 
of the bispectrum. In \cite{sefusatti2005}, for instance, they find that the 
signal-to-noise of the bispectrum is significantly reduced once survey geometry 
is included in their forecast. We also do not include super-sample covariance, 
which may also degrade constraints. Survey geometry and super-sample covariance, 
however, also degrades the signal-to-noise of their power spectrum forecasts. 
Hence, with the substantial improvement in the $\smnu$ constraints of the bispectrum, 
even with survey geometry we expect the bispectrum will significantly improve 
$\smnu$ constraints over the power spectrum.  

We include the nuisance parameter $\mmin$ in our forecasts to address the 
difference in halo bias and number densities among the Quijote cosmologies. Although 
we marginalize over $\mmin$, this may not fully account for the extra 
information from $\bar{n}$ and nonlinear bias leaking into the derivatives. 
To test this, we include extra nuisance parameters, $\{A_{\rm SN}$, 
$B_{\rm SN}$, $b_2$, $\gamma_2\}$, and examine their impact on our forecasts. 
$A_{\rm SN}$ and $B_{\rm SN}$ are multiplicative factors of the first and
second terms of Eq.~\ref{eq:bk_sn}, which we include to account for any 
$\bar{n}$ dependence that may be introduced from the shotnoise correction 
(Eq.~\ref{eq:bk_sn}). $b_2$ and $\gamma_2$ are the quadratic bias and nonlocal bias 
parameters~\citep{chan2012, sheth2013} to account for information from 
nonlinear bias. Marginalizing over $b'$, $\mmin$, $A_{\rm SN}$, $B_{\rm SN}$, 
$b_2$ and $\gamma_2$, we obtain the following constraints for $B_0$ 
with $k_{\rm max} = 0.5~\mpc$:
$\sigma_{\Om} = 0.0129$, $\sigma_{\Ob} = 0.0044$, $\sigma_h=0.0404$, 
$\sigma_{n_s}=0.0455$, $\sigma_{\sig}=0.0228$, and $\sigma_{\smnu}=0.0343$. 
While constraints on $n_s$ and $\sig$ are broadened from our fiducial 
forecasts, by $27\%$ and $60\%$, the other parameters, especially $\smnu$, 
are not significantly impacted by marginalizing over the extra nuisance 
parameters. As another test, we calculate derivatives using halo catalogs 
from Quijote $\theta^{-}$ and $\theta^{+}$ cosmologies with fixed $\bar{n}$. 
We similarly find no significant impact on the $B_0$ parameter constraints.
Forecasts using additional nuisance parameters and with fixed $\bar{n}$ 
derivatives, both support the robustness of our forecast. Yet these 
tests do not ensure that our forecast entirely marginalizes over halo bias. 

In this paper, we focus on the halo bispectrum and power spectrum. However,   
constraints on $\smnu$ will ultimately be derived from the distribution of 
galaxies. Besides the cosmological parameters, bias and nuisance parameters 
that allow us to marginalize over galaxy bias need to be incorporated to 
forecast $\smnu$ and other cosmological parameter constraints for the 
galaxy bispectrum. Although we include a \emph{naive} bias model through $b'$ 
and $M_{\rm min}$, and even $b_2$ and $\gamma_2$ in our tests, this is 
insufficient to describe how galaxies trace matter. A more realistic bias model 
such as a halo occupation distribution (HOD) model involve extra parameters 
that describe the distribution of central and satellite galaxies in 
halos~\citep[\emph{e.g.}][]{zheng2005,leauthaud2012a,tinker2013,zentner2016,vakili2019}. 
We, therefore, refrain from a more exhaustive investigation of the impact of 
halo bias on our results and focus quantifying the constraing power of 
the galaxy bispectrum in the next paper of this series. 

Marginalizing over galaxy bias parameters, will likely reduce the constraining 
power at high $k$. \cite{hand2017}, for instance, with their 13 parameter 
model only find a 15-30\% improvement in $f\sig$ when they extend their power 
spectrum multipole analysis from $0.2~\mpc$ to $0.4~\mpc$. However, jointly 
analyzing power spectrum and bispectrum will help break parameter degeneracies 
and improve constraints on cosmological parameters~\citep{sefusatti2006, yankelevich2019}. 
We again emphasize that the constraints we present is for a $1h^{-1}{\rm Gpc}$ box, 
a substantially smaller volume than upcoming surveys. Thus, even if the constraining 
power at high $k$ is reduced, our forecasts suggest that the bispectrum offers 
significant improvements over the power spectrum, especially for constraining $\smnu$. 
