\begin{figure}
\begin{center}
    \includegraphics[width=0.6\textwidth]{figs/quijote_bkCov_kmax05.png} 
    \caption{Covariance matrix of the redshift-space halo bispectrum estimated 
    using $N_{\rm cov} = 15,000$ realizations of the Qujiote simulation suite at 
    the fiducial cosmology: $\Om{=}0.3175$, $\Ob{=}0.049$, $h{=}0.6711$, $n_s{=}0.9624$, $\sig{=}0.834$, 
    and $\smnu{=}0.0$ eV. We include all possible triangle configurations with 
    $k_1, k_2, k_3 \leq k_{\rm max} = 0.5~h/{\rm Mpc}$ and order the configurations 
    (bins) in the same way as Figures~\ref{fig:bk_amp} and~\ref{fig:dbk_amp}. We 
    use the covariance matrix above for the Fisher matrix forecasts presented in 
    Section~\ref{sec:forecasts}. 
    }
\label{fig:bk_cov}
\end{center}
\end{figure}

%\begin{figure}
%\begin{center}
%    \includegraphics[width=\textwidth]{figs/quijote_pbkFisher_freeMmin_dmnu_fin_kmax02.pdf}
%    \caption{Fisher matrix constraints for $\smnu$ and other cosmological parameters 
%    for the redshift-space halo bispectrum monopole (orange). For comparison, we 
%    include Fisher parameter constraints for the redshift-space halo powerspectrum monopole 
%    in blue. The contours mark the $68\%$ and $95\%$ confidence interals. We set 
%    $k_{\rm max} = 0.2~h/{\rm Mpc}$ for both power spectrum and bispectrum. We 
%    include in our forecasts $b'$ and $M_{\rm min}$, a free amplitude scaling factor
%    and halo mass limit, respectively. They serve as a simplistic bias model and we 
%    marginalize over them to that our constraints do not include extra constraining 
%    power from the difference in bias/number density in the different Quijote cosmologies.
%    The bispectrum {\em substantially} improves constraints on all of the cosmological parameters 
%    over the power spectrum. {\em For $\smnu$, the bispectrum improves the constraint
%    from $\sigma_{\smnu}{=}~1.163$ to $0.112$ eV --- an order of magnitude improvement 
%    over the power spectrum}.}
%\label{fig:bk_fish_02}
%\end{center}
%\end{figure}

\subsection{$\smnu$ and other Cosmological Parameter Forecasts} \label{sec:forecasts}
We demonstrate in the previous section with the HADES simulations, that 
the bispectrum helps break the $\smnu$--$\sig$ degeneracy, a major 
challenge in precisely constraining $\smnu$ with the power spectrum. 
This establishes the bispectrum as a promising probe for $\smnu$. However, 
we are ultimately interested in determining the constraining power of the 
bispectrum for an analysis that include cosmological parameters beyond 
$\smnu$ and $\sig$--- \emph{i.e.} $\Om$, $\Ob$, $h$, and $n_s$. The Quijote 
suite of simulations is \emph{specifically} designed to answer this question
through Fisher matrix forecast.

First, the Quijote suite includes $N_{\rm cov} = 15,000$ $N$-body realizations run at a 
fidicial cosmology: $\smnu{=}0.0$eV, $\Om{=}0.3175, \Ob{=}0.049, 
n_s{=}0.9624, h{=}0.6711$, and $\sig=0.834$ (see Table~\ref{tab:sims}. 
This allows us to robustly estimate the covariance matrix of the bispectrum, 
$\bfi{C}$, which has ${\sim}1,800$ triangle configurations 
(Figure~\ref{fig:bk_cov}). Second, the Quijote suite includes $500$ $N$-body 
realizations evaluated at $13$ different cosmologies, each a small step away 
from the fiducial cosmology parameter values along one parameter 
(Section~\ref{sec:hades} and~Table~\ref{tab:sims}). On top of this, we apply 
redshift-space distortions along 3 different directions for these $500$ 
realizations, which then gives us $N_{\rm deriv.} = 1,500$ realizations. 
These realizations allow us to precisely estimate the derivatives of the 
bispectrum with respect to each of the cosmological parameters. 

\begin{figure}
\begin{center}
    \includegraphics[width=\textwidth]{figs/quijote_pbkFisher_bMmin_dmnu_fin_kmax0_50_reg.pdf} 
    \caption{Fisher matrix constraints for $\smnu$ and other cosmological parameters 
        for the redshift-space halo bispectrum monopole (orange). We include Fisher
        parameter constraints for the redshift-space halo powerspectrum monopole in blue 
        for comparison. The contours mark the $68\%$ and $95\%$ confidence interals. 
        We set $k_{\rm max} = 0.5~h/{\rm Mpc}$ for both power spectrum and bispectrum. 
        We include in our forecasts $b'$ and $M_{\rm min}$, a free amplitude scaling 
        factor and halo mass limit, respectively. They serve as a simplistic bias model 
        and we marginalize over them so that our constraints do not include extra 
        constraining power from the difference in bias/number density in the different 
        Quijote cosmologies. The bispectrum {\em substantially} improves constraints on all of the cosmological 
        parameters over the power spectrum. Constraints on $\Om$, $\Ob$, $h$, $n_s$, and $\sig$
        improve by factors of 3.1, 4.1, 5.1, 5.9, and 3.3, respectively. {\em For $\smnu$, the
        bispectrum improves $\sigma_{\smnu}$ from $0.1962$ to $0.0342$ eV --- a factor 
        of ${\sim}6$ improvement over the power spectrum}.}
\label{fig:bk_fish_05}
\end{center}
\end{figure}

Since their introduction to cosmology over two decades ago, Fisher Information 
matrices have been ubiquitously used to forecast the constraining power of future 
experiments~\citep[\emph{e.g.}][]{jungman1996,tegmark1997,dodelson2003,heavens2009,verde2010}. 
Defined as 
\beq 
F_{ij} = - \bigg \langle \frac{\partial^2 \mathrm{ln} \mathcal{L}}{\partial \theta_i \partial \theta_j} \bigg \rangle,
\eeq
where $\mathcal{L}$ is the likelihood, the Fisher matrix for the bispectrum can 
be written as 
\beq \label{eq:fullfish} 
F_{ij} = \frac{1}{2}~\mathrm{Tr} \Bigg[\bfi{C}^{-1}\parti{\bfi{C}}\bfi{C}^{-1}\partj{\bfi{C}} + \bfi{C}^{-1} \left(\parti{\overline{B}_0}\partj{\overline{B}_0}^T + \parti{\overline{B}_0}^T \partj{\overline{B}_0} \right)\Bigg].
\eeq
Since we assume that the $B_0$ likelihood is Gaussian, including the first 
term in Eq.~\ref{eq:fullfish} runs the risk of incorrectly including information 
from the covariance already included in the mean~\citep{carron2013}. We, therefore,
conservatively neglect the first term and calculate the Fisher matrix, 
\beq \label{eq:fisher}
F_{ij} = \frac{1}{2}~\mathrm{Tr} \Bigg[\bfi{C}^{-1} \left(\parti{\overline{B}_0}\partj{\overline{B}_0}^T + \parti{\overline{B}_0}^T \partj{\overline{B}_0} \right)\Bigg],
\eeq
directly with $\bfi{C}$ and $\partial B_0/\partial \theta_i$ along each cosmological 
parameter from the Quijote simulations. 

For $\Om$, $\Ob$, $h$, $n_s$, and $\sig$, we estimate 
\beq \label{eq:dbkdt} 
\frac{\partial \overline{B}_0}{\partial \theta_i} \approx \frac{\overline{B}_0(\theta_i^{+})-\overline{B}_0(\theta_i^{-})}{\theta_i^+ - \theta_i^-}, 
\eeq
where $\overline{B}_0(\theta_i^{+})$ and $\overline{B}_0(\theta_i^{-})$ are 
the average bispectrum of the $1,500$ realizations at $\theta^{+}$ and $\theta^{-}$. 
Meanwhile, for $\smnu$, where the fiducial value is 0.0 eV and we cannot have 
negative $\smnu$, we use the Quijote simulations at $\smnu^+$, $\smnu^{++}$, 
$\smnu^{+++}=0.1, 0.2, 0.4$ eV (Table~\ref{tab:sims}) to estimate 
\beq \label{eq:dbkdmnu} 
\frac{\partial \overline{B}_0}{\partial \smnu} \approx \frac{-21\overline{B}_0(\smnu^{\rm fid}) + 
32 \overline{B}_0(\smnu^{+}) - 12 \overline{B}_0(\smnu^{++}) + \overline{B}_0(\smnu^{+++})}{1.2}, 
\eeq
which provides a $\mathcal{O}(\delta \smnu^2)$ order approximation. By using these 
$N$-body simulations, instead of analytic methods (\emph{e.g.} perturbation theory), 
we exploit the accuracy of numerical simulations in the nonlinear regime and rely on 
fewer assumptions and approximations. In fact, \emph{these $N$-body simulation estimated 
derivatives are the key ingredients that enables us to quantify, for the first time, 
the full information content of the redshift-space bispectrum in the non-linear regime}. 
We discuss subtleties of Eq.~\ref{eq:dbkdmnu} bispectrum derivative and tests of 
convegence in Appendix~\ref{sec:numerical}. 

We present the constraints on $\smnu$ and other cosmological parameters 
$\{\Om, \Ob, h, n_s, \sig\}$ derived from the redshift-space halo bispectrum 
Fisher matrix (Eq.~\ref{eq:fisher}) for $k_{\rm max} = 0.5~h/{\rm Mpc}$ 
in Figure~\ref{fig:bk_fish_05}, respectively. 
We include Fisher constraints for the redshift-space halo power spectrum 
monopole with the same $k_{\rm max}$ for comparison (blue). We mark the 
$68\%$ and $95\%$ confidence interals with the contours. We include in our 
Fisher constraints the following nuisance parameters: $b'$, a scaling factor 
on the bispectrum amplitude, and $M_{\rm min}$, the halo mass limit. $b'$ is 
analogous to linear bias. Meanwhile, we choose $M_{\rm min}$ as a nuisance
parameter to address the difference in the number densities among the Quijote
cosmologies, which impacts the derivatives $\partial \overline{B}_0/\partial \theta_i$. 
For instance, the $\sig^{+}$ and $\sig^{-}$ cosmologies have halo 
$\bar{n} = 1.586\times10^{-4}$ and $1.528 \times 10^{-4}(h^{-1}{\rm Mpc})^3$. 
These parameters serve as a simplistic bias model and by marginalizing 
over them we aim to ensure that our Fisher constraints do not include extra 
constraining power from the difference in bias or number density. $b'$ is a 
multiplicative factor so $\partial \overline{B}_0/\partial b' = \overline{B}_0$. 
Meanwhile, we numerically estimate $\partial \overline{B}_0/\partial M_{\rm min}$ 
using $\overline{B}_0$ evaluated at $M^{+}_{\rm min}{=}3.3\times10^{13}h^{-1}M_\odot$ 
and $M^{-}_{\rm min}{=}3.1\times10^{13}h^{-1}M_\odot$, with all other parameters 
set to the fiducial value. 

The bispectrum {\em substantially} improves constraints on all parameters 
over the power spectrum. For $k_{\rm max} = 0.5~h/{\rm Mpc}$, the 
bispectrum improves the marginalized constraints, $\sigma_\theta$ of $\Om$, 
$\Ob$, $h$, $n_s$, and $\sig$ by factors of $\sim$3, 4, 5, 6, and 3 over 
the power spectrum. {\em For $\smnu$, the bispectrum improves the constraint 
from $\sigma_{\smnu}{=}~0.1962$ to $0.0342$eV --- a factor of 6 improvement 
over the power spectrum}. We emphasize that this $\sigma_{\smnu}{=}0.0342$eV 
constraint is for the {\em bispectrum alone} and only for a $1h^{-1}{\rm Gpc}$ 
box. We list the precise marginalized Fisher parameter constraints of both 
cosmological and nuisance parameters for $P_0$ and $B_0$ in Table~\ref{tab:forecast}. 

Even below $k_{\rm max} < 0.5~h/{\rm Mpc}$, the bispectrum significantly 
improves cosmological parameter constraints. We compare $\sigma_\theta$, 
the marginalized $1\sigma$ constraints of $\Om$, $\Ob$, $h$, $n_s$, $\sig$, 
and $\smnu$, as a function of $k_{\rm max}$ for $B_0$ (orange) and $P_0$ 
(blue) in Figure~\ref{fig:fish_kmax}. We focus only on the $k_{\rm max}$ range 
where the Fisher forecast is well defined --- \emph{i.e.} more data bins 
than the number of parameters: $k_{\rm max} > 8 k_{\rm f} \approx 0.05~h/{\rm Mpc}$ 
for $P_0$ and $k_{\rm max} > 12 k_{\rm f} \approx 0.075~h/{\rm Mpc}$ for 
$B_0$. Figure~\ref{fig:fish_kmax} reveals that the improvement of the
bispectrum $\sigma_\theta$ over the power spectrum $\sigma_\theta$ is 
larger at higher $k_{\rm max}$. Although limited by the $k_{\rm max}$ range, 
the figure suggests that on large scales ($k_{\rm max}\lesssim 0.1~h/{\rm Mpc}$) 
$P_0$ $\sigma_\theta$ crosses over $B_0$ $\sigma_\theta$ so $P_0$ has more 
constraining power than $B_0$--- as expected on linear scales. On slightly 
larger $k_{\rm max}$, even at $k_{\rm max} = 0.2~h/{\rm Mpc}$, we find that 
the bispectrum substantially improves $\sigma_\theta$ by factors of
$\sim 1.8$, 2.1, 2.6, 2.5, 2.6, and $3.1$ for $\Om$, $\Ob$, $h$, $n_s$, $\sig$, 
and $\smnu$ respectively. 

%%%%%%%%%%%%%%%%%%%%%%%%%%%%%%%%%%%%%%%%%%
% forecast table
%%%%%%%%%%%%%%%%%%%%%%%%%%%%%%%%%%%%%%%%%%
\begin{table}
    \caption{Marginalized Fisher parameter constraints from the redshift-space halo power 
    spectrum (top) and bispectrum (bottom) for different values of $k_{\rm max}$. We list 
    constraints for cosmological parameters $\smnu$, $\Omega_m$, $\Omega_b$, $h$, $n_s$, 
    and $\sig$ as well as nuisance parameters $b'$ and $M_{\rm min}$} 
\begin{center} 
    \begin{tabular}{cccccccccc} \toprule
         & $k_{\rm max}$ & $\smnu$ & $\Omega_m$ & $\Omega_b$ & $h$ & $n_s$ & $\sig$ & $b'$ & $M_{\rm min}$ \\
         & ({\footnotesize $h/{\rm Mpc}$}) &({\footnotesize eV}) & & & & & & & ({\footnotesize $10^{13} h^{-1}M_\odot$}) \\[3pt] \hline\hline

            &     & 0.0 & 0.3175 & 0.049 & 0.6711 & 0.9624 & 0.834 & 1. & 3.2  \\ 
    $P_0$  & 0.2 & 0.333 & $\pm0.052$ & $\pm0.030$ & $\pm0.372$ & $\pm0.347$ & $\pm0.128$ & $\pm0.649$ & $\pm5.045$\\
           & 0.3 & 0.277 & $\pm0.044$ & $\pm0.023$ & $\pm0.273$ & $\pm0.276$ & $\pm0.069$ & $\pm0.383$ & $\pm2.457$\\
           & 0.4 & 0.228 & $\pm0.040$ & $\pm0.020$ & $\pm0.235$ & $\pm0.240$ & $\pm0.059$ & $\pm0.226$ & $\pm1.270$\\
           & 0.5 &$\pm${\bf 0.196}& $\pm0.036$ & $\pm0.018$ & $\pm0.207$ & $\pm0.213$ & $\pm0.048$ & $\pm0.157$ & $\pm0.807$\\
                      
    $B_0$  & 0.2 & 0.107 & $\pm0.029$ & $\pm0.014$ & $\pm0.144$ & $\pm0.140$ & $\pm0.050$ & $\pm0.265$ & $\pm1.317$\\
           & 0.3 & 0.065 & $\pm0.020$ & $\pm0.008$ & $\pm0.077$ & $\pm0.074$ & $\pm0.023$ & $\pm0.143$ & $\pm0.657$\\
           & 0.4 & 0.043 & $\pm0.015$ & $\pm0.006$ & $\pm0.052$ & $\pm0.047$ & $\pm0.016$ & $\pm0.088$ & $\pm0.369$\\
           & 0.5 &$\pm${\bf 0.034}& $\pm0.012$ & $\pm0.004$ & $\pm0.040$ & $\pm0.036$ & $\pm0.014$ & $\pm0.070$ & $\pm0.269$\\[3pt]
    \hline
\end{tabular} \label{tab:forecast}
\end{center}
\end{table}
%%%%%%%%%%%%%%%%%%%%%%%%%%%%%%%%%%%%%%%%%%

%%%%%%%%%%%%%%%%%%%%%%%%%%%%%%%%%%%%%%%%%%
% sigma(kmax)  
%%%%%%%%%%%%%%%%%%%%%%%%%%%%%%%%%%%%%%%%%%
\begin{figure}
\begin{center}
    \includegraphics[width=0.9\textwidth]{figs/quijote_Fisher_kmax_bMmin_dmnu_fin_reg.pdf} 
    \caption{Marginalized $1\sigma$ constraints of the cosmological parameters 
    $\Om$, $\Ob$, $h$, $n_s$, $\sig$, and $\smnu$ ($\sigma_\theta$) as a function 
    of $k_{\rm max}$ for the redshift-space halo bispectrum (orange) and power spectrum (blue). 
    Though not included in the figure, we marginalize over the nuisance parameters
    $b'$ and $M_{\rm min}$ in our forecast (Section~\ref{sec:forecasts}). We only 
    include $k_{\rm max} > 8 k_{\rm f}$ for $P_0$ and $k_{\rm max} > 12 k_{\rm f}$ 
    for $B_0$ --- $k_{\rm max}$ ranges where we have more data bins than number of 
    parameters. Even at $k_{\rm max} < 0.5~h/{\rm Mpc}$, the bispectrum significantly 
    improves cosmological parameter constraints. The improvement, however, is larger 
    for higher $k_{\rm max}$. At $k_{\rm max} = 0.2~h/{\rm Mpc}$, the bispectrum 
    improves constraints on $\Om$, $\Ob$, $h$, $n_s$, $\sig$, and $\smnu$ by factors 
    of $\sim 1.8$, 2.1, 2.6, 2.5, 2.6, and 3.1 over the power spectrum.}
\label{fig:fish_kmax}
\end{center}
\end{figure}
%%%%%%%%%%%%%%%%%%%%%%%%%%%%%%%%%%%%%%%%%%
Our forecasts demonstrate that the bispectrum has significant constraining power 
in the weakly non-linear regime ($k > 0.1~\mpc$) beyond the power spectrum. This constraining power 
comes from the bispectrum breaking degeneracies among the cosmological and 
nuisance parameters. This is evident when we compare the unmarginalized 
constraints from $P_0$ and $B_0$: $1/\sqrt{F_{ii}}$, where $F_{ii}$ is a 
diagonal element of the Fisher matrix. For $k < 0.4~\mpc$, the unmarginalized 
cosntraints from $P_0$ are tighter than those from $B_0$. Yet, once we 
marginalize the constraints over the other parameters, the $B_0$ constraints 
are tigher than $P_0$ for $k > 0.1~\mpc$. The derivatives, 
$\partial B_0/\partial \theta_i$, also shed light on how $B_0$ breaks parameter 
degeneracies. The parameter degeneracies in the $P_0$ forecasts of Figure~\ref{fig:bk_fish_05} 
are consistent with similarities among the shape and scale dependence of 
$P_0$ derivatives $\partial P_0/\partial \theta$. On the other hand, the 
$B_0$ derivatives with respect to the parameters have significant different 
scale and triangle shape dependences. \ch{derivative figure?} 

By exploiting the unprecedented number of $N$-body simulations of the Quijote 
suite, we present for the first time the full information content of the 
redshift-space bispectrum beyond the linear regime. The information content 
of the bispectrum, however, has previously been examined using perturbation 
theory. Previous works, for instance, measure the signal-to-noise ratio (SNR) 
of the bispectrum derived from covariance matrices estimated using perturbation 
theory~\citep[\emph{e.g.}][]{sefusatti2005, chan2017}. Most recently, \cite{chan2017},
using covariance matrices that include non-Gaussian contributions calibrated with 
$N$-body simulations, find that the cumulative SNR of the halo bispectrum is 
$\sim 30\%$ of the SNR of the halo power spectrum at $k_{\rm max} \sim 0.1~\mpc$ 
and increases to $\sim 40\%$ at $k_{\rm max}\sim 0.35~\mpc$. While these simple 
SNR measurements cannot be easily compared to our Fisher analysis, we note that 
they are loosely consistent with the unmarginalized constraints, where we find 
tighter unmarginalized constraints from $P_0$ than $B_0$ at $k_{\rm max} < 0.4~\mpc$. 
Also, when we measure the the halo power spectrum and bispectrum SNRs using our 
covariance matrices (Figure~\ref{fig:bk_cov}), we find a relation between the 
SNRs consistent with \cite{chan2017}. Beyond the $k$ range of \cite{chan2017} 
($k_{\rm max}>0.35~\mpc$), we find that the SNR of $B_0$ continues to increase 
at higher $k_{\rm max}$ in constract to the $P_0$ SNR, which saturates at 
$k_{\rm max}\sim 0.1~\mpc$. At $k_{\rm max} = 0.75~\mpc$, the largest $k$ we 
probe, the $B_0$ SNR is $\sim75\%$ of the $P_0$ SNR.

Beyond these signal-to-noise calculations, a number of previous works quantify 
the information content of the bispectrum with Fisher 
forecasts~\citep{scoccimarro2004, sefusatti2006, sefusatti2007, song2015, tellarini2016, yamauchi2017a, karagiannis2018, yankelevich2019}. While many of these works fix most cosmological parameters
and focus soley on forecasting constraints of primordial non-Gaussianity 
($f_{\rm NL}$) and bias parameters, \cite{sefusatti2006} and \cite{yankelevich2019} 
forecast full sets of cosmological parameters. \cite{sefusatti2006} present 
likelihood analysis forecasts for $\omega_d$, $\omega_b$, $\OL$, $n_s$, 
$A_s$, $w$, $\tau$, and bias parameters. %$\omega_d = \Omega_{\rm cdm}h^2$ and $\omega_b = \Ob h^2$ are the physical dark matter and baryon densities,  $\tau$ is the reionization optical depth, $A_s$ is the primordial amplitude of scalar fluctuations, and $w$ is dark energy equation of state parameter.  
For $\Lambda$CDM, with fixed bias parameters, and $k_{\rm max} = 0.3~\mpc$, 
they find constraints on $\Om$, $\Ob$, $h$, $n_s$, and $\sig$ from WMAP, 
$P_0$, and $B_0$ is a factor of 1.71, 1.3, 1.4, 1.2, and 1.5 tighter 
than constraints from WMAP and $P_0$ alone. In comparison, for 
$k_{\rm max} = 0.3~\mpc$ our $B_0$ constraints are tighter than $P_0$ constraints 
by factors of 2.1, 2.9, 3.5, 3.7, and 3.0. Both \cite{sefusatti2006} and our 
analysis find significant improvements on the constraints with the 
bispectrum. They however include the WMAP likelihood in their forecast 
and only present constraints with both $P_0$ and $B_0$. We also emphasize 
that they use perturbation theory bispectrum models, which break down on 
small scales. In a comparison with $N$-body simulations, \cite{lazanu2016} 
find that perturbation theory models of the matter bispectrum deviate by 
$>5\%$ at $k_{\rm max} \gtrsim 0.15~\mpc$. 

Meanwhile, \cite{yankelevich2019} present Fisher forecasts for 
$\Omega_{\rm cdm}$, $\Ob$, $h$, $n_s$, $A_s$, $w_0$, and $w_0$ for a Euclid-like
survey~\citep{laureijs2011} in 14 non-overlapping redshift bins. They use the full redshift-space 
bispectrum rather than just $B_0$. They include forecasts of dynamical 
dark-energy models and use a more sophisticated bias expansion than 
\cite{sefusatti2006} that depend on the tidal field. However, same as 
\cite{sefusatti2006} and all other forecasts, they use a pertrubative 
model for the bispectrum and consequently limit their forecast to 
$k_{\rm max} = 0.15~\mpc$. They also marginalize over 56 nuisance 
parameters (14 $z$ bins each with 3 bias parameters and 1 RSD parameter).

These forecasts also do not account for the 
non-Gaussian contributions to the covariance matrix, which substantially impact the
constraints especially on small scales where non-linear effects strongly enhance 
the non-Gaussian contributions~\citep{chan2017}. For $k_{\rm max} \sim 0.14~\mpc$, 
\cite{chan2017} find that neglecting non-Gaussian contributions overestimates the 
SNR of the bispectrum by a factor of $\sim 3$. 
%- fisher forecast for the following sets of cosmological parameters: {Ocdm, Ob, h, ns, A}, {Ocdm, Ob, h, ns, A, w}, {Ocdm, Ob, h, ns, A, w0, wa}. Along with $b_1$, $b_2$, $b_{s^2}$, and $\sigma_p$ nuisance parameters.
%- Improvements over Sefusatti et al. (2016): 
%  - consider the full galaxy bispectrum in redshift space instead of its monopole moment
%  - make forecasts for dynamical dark-energy models
%  - account for a more sophisticated bias expansion that also depends on the tidal field and which represents the current state of the art
%- Gaussian covariance only
%- kmax = 0.15 (because of PT where accounting for galaxy biasing, RSD and discreteness effects provide additional challenges for the per- turbative models and reduces their range of validity.)
%https://arxiv.org/abs/1506.00083, https://arxiv.org/abs/1512.05383
We're the first to use simulations. There's really nothing comparable. especially because there are no forecasts looking at neutrnios
They also don't look at neutrinos\cite{ruggeri2018} . 

Our results also demonstrated the potential of 
the bispectrum in constraining $\smnu$ (an order of magnitude improvement over 
$P_0$). Below, we underline a few caveats of the results we present above. 
First, the parameter constraints were derived using the Fisher matrix. This 
assumes 

{\bf Q}: How robust are Fisher forecasts \\ 
{\bf A}: There are many caveats to the Fisher forecast. One of the main limitation is that 
Fisher forecasts assume the posteriors are Gaussian, this can result in incorrect posteriors \cite{wolz2012}. 
Convergence of the covariance matrix and derivatives are also key. We discuss this in the appendix
and they do not seem to be an issue. 

{\bf Q}: A lot of the advantages from breaking parameter degeneracies. How robust is this? \\
{\bf A}: We address this by imposing Planck priors on our forecasts. This expectedly has a 
large impact on P constraints, but doesn't change B constraints much. B still does better 
by a lot. This suggests that the breaking of parameter degeneracies are relatively stable 
to poorly constrained derivatives spuriously breaking degeneracies so this is robust. Also 
demonstrates that the gains are relatively stable to Fisher forecasts caveats 
(i.e. projecting a banana). 

Another caveat is that our parameter constraints were derived using the power 
spectrum and bispectrum of halo in a periodic box. We do not consider a 
realistic survey geometry or radial selection function. A realistic selection 
function will smooth out the triangle configuration dependence and consequently 
degrade the constraining power of the bispectrum. In \cite{sefusatti2005}, for 
instance, they find that the signal-to-noise of the bispectrum is significantly 
reduced once survey geometry is included in their forecast. Survey geometry, 
however, also degrade the signal-to-noise of their power spectrum forecasts. 
Hence, with the order of magnitude improve in the $\smnu$ constraining power 
of the bispectrum, even with survey geometry, including the bispectrum will 
improve $\smnu$ constraints. 

{\bf Q}: extra information leaking in due to non-linear bias and number density differences? \\
{\bf A}: We ran tests that include SN correction parameters in case number density 
information is leaking in from there. Constraints barely change. We also include and marginalize 
over $b_2$ and $g_2$ but nothing changes. Explain fixed nbar tests that we ran, which also came 
out short. Talk about how these tests don't definitely answer whether non-linear bias information
is leaking into the forecasts since $b'$, $M_{\rm min}$, $b_2$, and $g_2$ may not a sufficient
model of halo bias. We suspect this is the reason why B does better than P for all $k_{\rm max}$ 
even though on linear scales we expect P to have more constraining power. 
However since the ultimate goal is to constrain $\smnu$ with the galaxy 
bispectrum.  For galaxy bispectrum, HOD is a more robust bias model. The halo bispectrum is an
intermediate step and a proof of concept, which our results show is promising. We therefore 
refrain from an exhaustive investigation of halo bias and go directly to HOD in the next paper 
of this series.

Although we focus on the halo bispectrum and power spectrum in this paper, 
constraints on $\smnu$ will ultimately be derived from the distribution of 
galaxies. Besides the cosmological parameters, bias and nuisance parameters 
that allow us to marginalize over the galaxy---halo connection need to be
incorporated to forecast $\smnu$ and other cosmological parameter constraints 
for the galaxy bispectrum. Although we include a \emph{naive} bias model 
through $b'$ and $M_{\rm min}$, this is insufficient to describe how galaxies
occupy halos. A more realistic bias model such as a halo occupation distribution 
(HOD) model involve extra parameters that describe the distribution of central 
and satellite galaxies in halos~\citep[\emph{e.g.}][]{zheng2005,leauthaud2012a,tinker2013,zentner2016,vakili2019}. 
\ch{maybe something about Uros and Nick's model involving a lot of parameters.}
Marginalizing over these extra parameters, will likely reduce the constraining 
power at high $k$. Even if the constraining power at high $k$ is reduced, the
bispectrum still offers significant improvements over the power spectrum at 
$k_{\rm max} \sim 0.2$. Jointly analyzing power spectrum and bispectrum will 
help constrain these extra bias parameters. 
\ch{talk about \cite{yankelevich2019} how they don't find as much of an improvement 
for the galaxy bispectrum alone. But they do find that jointly analyzing P and B 
improve results significantly. Don't read too into this because of all the differences 
that we mentioned.} Furthermore, we again emphasize that
the constraints we present in this paper is for a $1h^{-1}{\rm Gpc}$ box. In 
Hahn et al. (in preparation), we will include a realistic HOD model and quantify 
the information content and constraining power of a joint galaxy power spectrum 
and bispectrum analysis. 
