\documentclass[12pt, letterpaper, preprint]{aastex62}
%%% This file is generated by the Makefile.
\newcommand{\giturl}{\url{https://github.com/changhoonhahn/eMaNu}}
\newcommand{\githash}{112c8e2}\newcommand{\gitdate}{2019-09-20}\newcommand{\gitauthor}{changhoonhahn}

%\usepackage[breaklinks,colorlinks, urlcolor=blue,citecolor=blue,linkcolor=blue]{hyperref}
\usepackage{color}
\usepackage{amsmath}
\usepackage{natbib}
\usepackage{ctable}
\usepackage{bm}
\usepackage[normalem]{ulem} % Added by MS for \sout -> not required for final version
\usepackage{xspace}
\usepackage{csvsimple} 

% typesetting shih
\linespread{1.08} % close to 10/13 spacing
\setlength{\parindent}{1.08\baselineskip} % Bringhurst
\setlength{\parskip}{0ex}
\let\oldbibliography\thebibliography % killin' me.
\renewcommand{\thebibliography}[1]{%
  \oldbibliography{#1}%
  \setlength{\itemsep}{0pt}%
  \setlength{\parsep}{0pt}%
  \setlength{\parskip}{0pt}%
  \setlength{\bibsep}{0ex}
  \raggedright
}
\setlength{\footnotesep}{0ex} % seriously?

% citation alias

% math shih
\newcommand{\setof}[1]{\left\{{#1}\right\}}
\newcommand{\given}{\,|\,}
\newcommand{\lss}{{\small{LSS}}\xspace}

\newcommand{\Om}{\Omega_{\rm m}} 
\newcommand{\Ob}{\Omega_{\rm b}} 
\newcommand{\OL}{\Omega_\Lambda}
\newcommand{\smnu}{M_\nu}
\newcommand{\sig}{\sigma_8} 
\newcommand{\mmin}{M_{\rm min}}
\newcommand{\BOk}{\widehat{B}_0} 
\newcommand{\hmpc}{\,h/\mathrm{Mpc}}
\newcommand{\bfi}[1]{\textbf{\textit{#1}}}
\newcommand{\parti}[1]{\frac{\partial #1}{\partial \theta_i}}
\newcommand{\partj}[1]{\frac{\partial #1}{\partial \theta_j}}
\newcommand{\mpc}{h/{\rm Mpc}}

\newcommand{\specialcell}[2][c]{%
  \begin{tabular}[#1]{@{}c@{}}#2\end{tabular}}
% text shih
\newcommand{\foreign}[1]{\textsl{#1}}
\newcommand{\etal}{\foreign{et~al.}}
\newcommand{\opcit}{\foreign{Op.~cit.}}
\newcommand{\documentname}{\textsl{Article}}
\newcommand{\equationname}{equation}
\newcommand{\bitem}{\begin{itemize}}
\newcommand{\eitem}{\end{itemize}}
\newcommand{\beq}{\begin{equation}}
\newcommand{\eeq}{\end{equation}}

%% collaborating
\definecolor{orange}{rgb}{1,0.5,0}
\newcommand{\ch}[1]{{\color{orange}{\bf CH:} #1}}

\begin{document}\sloppy\sloppypar\frenchspacing 

\title{Constraining $\smnu$ with the Bispectrum I: Breaking Parameter Degeneracies} 
\date{\texttt{DRAFT~---~\githash~---~\gitdate~---~NOT READY FOR DISTRIBUTION}}

\newcounter{affilcounter}
\author{ChangHoon Hahn}
\altaffiliation{hahn.changhoon@gmail.com}
\affil{Lawrence Berkeley National Laboratory, 1 Cyclotron Rd, Berkeley CA 94720, USA}
\affil{Berkeley Center for Cosmological Physics, University of California, Berkeley, CA 94720, USA}

\author{Francisco Villaescusa-Navarro} 
\affil{Center for Computational Astrophysics, Flatiron Institute, 162 5th Avenue, New York, NY 10010, USA} 

\author{Emanuele Castorina} 
\affil{Berkeley Center for Cosmological Physics, University of California, Berkeley, CA 94720, USA}
\affil{Lawrence Berkeley National Laboratory, 1 Cyclotron Rd, Berkeley CA 94720, USA}

\author{Roman Scoccimarro} 
\affil{Center for Cosmology and Particle Physics, Department of Physics, New York University, NY 10003, New York, USA}

\begin{abstract}
    Massive neutrinos suppress the growth of structure below their free-streaming scale and leave an 
    imprint on large-scale structure. Measuring this imprint allows us to constrain the sum of neutrino 
    masses, $\smnu$, a key parameter in particle physics beyond the Standard Model. However, degeneracies 
    among cosmological parameters, especially between $\smnu$ and $\sig$, limit the constraining power 
    of standard two-point clustering statistics. In this work, we investigate whether we can break these 
    degerancies and constrain $\smnu$ with the next higher-order correlation function --- the bispectrum. 
    We first examine the redshift-space halo bispectrum of $800$ $N$-body simulations from the HADES suite 
    and demonstrate that the bispectrum helps break the $\smnu$--$\sig$ degeneracy. Then using over 42,000
    $N$-body simulations of the Quijote suite, we quantify for the first time the full information content 
    of the redshift-space halo bispectrum using a Fisher matrix forecast of $\{\Om$, $\Ob$, $h$, $n_s$, $\sig$, $\smnu\}$ 
    down to nonlinear scales. For $k_{\rm max}{=}0.5~\mpc$, the bispectrum produces
    $\Om$, $\Ob$, $h$, $n_s$, and $\sig$ constraints 3.1, 4.1, 5.1, 5.9, and 3.3 times tighter than the 
    power spectrum. For $\smnu$, the bispectrum improves the 1$\sigma$ constraint from 
    0.1962 to 0.0342 eV --- 6 times tigher than the power spectrum. Even with priors from {\em Planck}, 
    the bispectrum improves $\smnu$ constraints by a factor of 2.7. Although we reserve marginalizing 
    over bias parameters to the next paper of the series, these constraints are derived for a 
    $(1~h^{-1}{\rm Gpc})^3$ box, a substantially smaller volume than upcoming surveys. Thus, our results 
    demonstrate that the bispectrum offers significant improvements over the power spectrum, especially 
    for constraining $\smnu$.  
\end{abstract}

\keywords{
cosmology: 
---
}

% --- intro ---
\section{Introduction} \label{sec:intro} 

\cite{pontzen2016}
\cite{angulo2016}
\cite{villaescusa-navarro2018a}

\beq
\delta(\bfi{x}) = \frac{\rho(\bfi{x}) - \bar{\rho}}{\bar{\rho}}
\eeq

\beq
\delta(\bfi{k}) = \frac{1}{(2\pi)^3} \int {\rm d}^3\bfi{x} e^{-i\bfi{k} \cdot \bfi{x}} \delta(\bfi{x}) = A e^{i\theta}
\eeq

For Gaussian random field, $\theta$ is uniformly sampled from $0$ to $2\pi$ 
and $A$ is sampled from Rayleigh distribution:
\beq
p(A) {\rm d}A = \frac{A}{\sigma^2} e^{-A^2/2\sigma^2} {\rm d}A
\eeq
where $\sigma^2 = V~P(k)/(16\pi^3)$. The mean of this distribution is 
\beq
\langle A\rangle = \int_0^\infty \frac{A^2}{\sigma^2} e^{-A^2/2\sigma^2} {\rm d}A = \sqrt{\frac{V\,P(k)}{32\pi^2}}.
\eeq
Also, 
\beq
\langle \delta(\bfi{k})\delta^*(\bfi{k}) \rangle = \langle A^2\rangle = \int_0^\infty \frac{A^3}{\sigma^2} e^{-A^2/2\sigma^2} {\rm d}A = \frac{V\,P(k)}{(2\pi)^3}.
\eeq

A paired Gaussian field is where you have two fields $\delta_1$ and $\delta_2$ 
where $\delta_2(k) = A e^{i(\theta + \pi)} = - \delta_1(k)$. 

A fixed field is when the amplitude is fixed, 
\beq
A = \sqrt{\frac{V~P(k)}{(2\pi)^3}},
\eeq
such that the power spectrum is the same. 

Paired fixed is when you do both. 


% --- hades+quijote ---
% brief summary of the Quijote suite. 
\section{The Quijote Simulation Suite} \label{sec:sims}
We use a subset of simulations from the \quij suite, a set of 43,000 $N$-body 
simulations that spans over 7000 cosmological models and contains, at a single 
redshift, over 8.5 trillion particles. The \quij suite was designed to quantify
the information content of cosmological observables and also to train machine 
learning algorithms. Hence, the suite includes enough realizations to accurately 
estimate the covariance matrices of high-dimensional observables such as the 
bispectrum as well as the derivatives of these observables with respect to 
cosmological parameters. For the derivatives, the suite includes sets of 
simulations run at different cosmologies where only one parameter is varied 
from the fiducial cosmology  
($\Om{=}0.3175$, $\Ob{=}0.049$, $h{=}0.6711$, $n_s{=}0.9624$, $\sig{=}0.834$, 
and $\smnu{=}0.0$ eV). Along $\Om$, $\Ob$, $h$, $n_s$, and $\sig$, the fiducial 
cosmology is adjusted by either a small step above or below the fiducial value:
$\{\Om^{+}, \Om^{-}, \Ob^{+}, \Ob^{-}, h^+, h^-, n_s^{+}, n_s^{-}, \sig^{+}, \sig^-\}$ . 
Along $\smnu$, because $\smnu \ge 0.0$ eV and the derivative of certain observable 
with respect to $\smnu$ is noisy, \quij includes sets of simulations for 
$\{\smnu^{+}, \smnu^{++}, \smnu^{+++} \} = \{0.1, 0.2, 0.4~{\rm eV}\}$.
At each of these 14 cosmologies, \quij includes sets of standard $N$-body and 
suppressed variance simulations. 

The initial conditions for all the simulations were generated at $z=127$ using 
second-order perturbation theory for simulations with massless neutrinos 
($\smnu = 0.0$ eV) and the Zel’dovich approximation for massive neutrinos 
($\smnu > 0.0$ eV). The initial conditions with massive neutrinos take 
their scale-dependent growth factors/rates into account using the 
\cite{zennaro2017a} method, while for the massless neutrino case we use 
the traditional scale-independent rescaling. 
%The \textsc{Quijote} simulations contain standard, fixed and paired fixed simulations. The differences between those is the way the initial conditions are generated. Consider a Fourier-space mode, $\delta(\vec{k})$. Since it is in general a complex number, we can write it as $\delta(\vec{k})=Ae^{i\theta}$, where both the amplitude $A$, and the phase $\theta$, depends on the considered wavenumber $\vec{k}$. In standard simulations, $A$ follows a Rayleigh distribution and $\theta$ is drawn from an uniform distribution between 0 and $\pi$. We note that this is the traditional way to generate ICs for cosmological simulations. In fixed simulations, while $\theta$ is still drawn from an uniform distribution between 0 and $2\pi$, the value of $A$ is fixed to the square root of the variance of the previous Rayleigh distribution. Finally, paired fixed simulations are two fixed simulations where the phases of the two pairs differ by $\pi$. We refer the reader to \citep{Pontzen_2016,RP_16,Paco_2018b} for further details.


From the initial conditions, 
the simulations follow the gravitational evolution of $512^3$ dark matter
particles, and $512^3$ neutrino particles for massive neutrino models, to 
$z=0$ using {\sc Gadget-III} TreePM+SPH code~\citep{springel2005}. 
Simulations with massive neutrinos are run using the ``particle method'', 
where neutrinos are described as a collisionless and pressureless fluid 
and therefore modeled as particles, same as CDM~\citep{brandbyge2008,viel2010}.
Halos are then identified using the Friends-of-Friends algorithm~\citep[FoF;][]{davis1985} 
with linking length $b=0.2$ on the CDM + baryon distribution. We limit 
the halo catalogs to halos with masses above 
$M_{\rm lim} = 3.2\times 10^{13} h^{-1}M_\odot$.
For the fiducial cosmology, the halo catalogs have ${\sim}156,000$ halos 
($\bar{n} \sim 1.56 \times 10^{-4}~h^3{\rm Gpc}^{-3}$) with $\bar{n} P_0(k=0.1)\sim 3.23$. 
We refer readers to Villaescusa-Navarro et al. (in preparation) and Hahn et al. (2019) 
for further details on the \quij simulations.



% --- bispectrum ---
\begin{figure}
\begin{center}
    \includegraphics[width=\textwidth]{figs/haloBk_shape_kmax05_rsd.pdf} 
    \caption{The redshift-space halo bispectrum, $\BOk(k_1, k_2, k_3)$, as a 
    function of triangle configuration shape for $\smnu = 0.0, 0.06, 0.10$, 
    and $0.15\,\mathrm{eV}$ (upper panels) and $\sig = 0.822, 0.818$, and 
    $0.807$ (lower panels). The HADES simulations of the top and bottom 
    panels in the three right-most columns, have matching $\sig$ values 
    (Section~\ref{sec:hades}). We describe the triangle configuration shape 
    by the ratio of the triangle sides: $k_3/k_1$ and $k_2/k_1$. The upper 
    left bin contains squeezed triangles ($k_1 = k_2 \gg k_3$); the upper 
    right bin contains equilateral triangles ($k_1 = k_2 = k_3$); and the 
    bottom center bin contains folded triangles ($k_1 = 2 k_2 = 2 k_3$). 
    We include all triangle configurations with $k_1, k_2, k_3 \leq k_{\rm max} = 0.5~h/{\rm Mpc}$. 
    and use the $\BOk$ estimator in Section~\ref{sec:bk}.}
\label{fig:bk_shape}
\end{center}
\end{figure}

\begin{figure}
\begin{center}
\includegraphics[width=0.9\textwidth]{figs/haloBk_amp_kmax05_rsd.pdf}
    \caption{The redshift-space halo bispectrum, $\BOk(k_1, k_2, k_3)$, as a
    function of triangle configurations for $\smnu = 0.0, 0.06, 0.10$, 
    and $0.15\,\mathrm{eV}$ (top panel) and $\smnu = 0.0$ eV, $\sig = 0.822, 0.818, 0.807$, 
    and $0.798$ (lower panel). We include all possible triangle configurations 
    with $k_1, k_2, k_3 \leq k_{\rm max} = 0.5~h/{\rm Mpc}$ where we order
    the configurations by looping through $k_3$ in the inner most loop and 
    $k_1$ in the outer most loop satisfying $k_1 \leq k_2 \leq k_3$. In the 
    insets of the panels we zoom into triangle configurations with 
    $k_1 = 0.113$, $0.226 \leq k_2 \leq 0.283$, and $0.283 \leq k_3 \leq 0.377~h/{\rm Mpc}$.}
\label{fig:bk_amp}
\end{center}
\end{figure}

%%%%%%%%%%%%%%%%%%%%%%%%%%%%%%%%%%%%%%%%%%%%%%%%%%%%%%%%%%%%%%%%%%%%%%%%
% Section   
%%%%%%%%%%%%%%%%%%%%%%%%%%%%%%%%%%%%%%%%%%%%%%%%%%%%%%%%%%%%%%%%%%%%%%%%
\section{Bispectrum} \label{sec:bk} 
We're interested in breaking parameter degeneracies that limit the constraining 
power on $\smnu$ of two-point clustering analyses using three-point clustering 
statistics --- \emph{i.e.} the bispectrum. In this section, we describe the 
bispectrum estimator used throughout the paper. We focus on the bispectrum monopole 
($\ell = 0$) and use an estimator that exploits Fast Fourier Transpforms (FFTs). 
Our estimator is similar to the estimators described in \cite{scoccimarro2015,sefusatti2016}; 
we also follow their formalism in our description below. Although \cite{sefusatti2016} 
and \cite{scoccimarro2015} respectively describe estimators in redshift- and real-space,  
since we focus on the bispectrum monopole, we note that there is no difference. 

To measure the bispectrum of our halo catalogs, we begin by interpolating the halo
positions to a grid, $\delta({\bfi x})$ and Fourier transforming the grid to get $\delta({\bfi k})$.
We use a fourth-order interpolation to interlaced grids, which has advantageous 
anti-aliasing properties~\citep{hockney1981,sefusatti2016} that allow unbias measurements 
up to the Nyquist frequency. Then using $\delta({\bfi k})$, we measure the bispectrum monopole as 
\beq \label{eq:bk} 
\widehat{B}_{\ell = 0}(k_1,k_2, k_3) = \frac{1}{V_B} \int\limits_{k_1}{\rm d}^3q_1 \int\limits_{k_2}{\rm d}^3q_2 \int\limits_{k_3}{\rm d}^3q_3~\delta_{\rm D}({\bfi q_{123}})~\delta({\bfi q_1})~\delta({\bfi q_2})~\delta({\bfi q_3}) - B^{\rm SN}_{\ell = 0}
\eeq
$\delta_{\rm D}$ above is a Dirac delta function and hence $\delta_{\rm D}({\bfi q_{123}}) = \delta_{\rm D}({\bfi q_1} + {\bfi q_2} + {\bfi q_3})$ 
ensures that the ${\bfi q}_i$ triplet actually form a closed triangle. Each of the integrals 
above represent an integral over a spherical shell in $k$-space with radius $\delta k$ 
centered at ${\bfi k}_i$ --- \emph{i.e.} 
\beq
\int_{k_i}{\rm d}^3q \equiv \int\limits_{k_i-\delta k/2}^{k_i+\delta k/2}{\rm d}q~q^2\int {\rm d}\Omega.
\eeq
$V_B$ is a normalization factor proportional to the number of triplets ${\bfi q_1}$, 
${\bfi q_2}$, and ${\bfi q_3}$ that can be found in the triangle bin defined by 
$k_1$, $k_2$, and $k_3$ with width $\delta k$: 
\beq
V_B = \int\limits_{k_1}{\rm d}^3q_1 \int\limits_{k_2}{\rm d}^3q_2 \int\limits_{k_3}{\rm d}^3q_3~\delta_{\rm D}({\bfi q_{123}})
\eeq
Lastly, $B^{\rm SN}_{\ell=0}$ is the correction for the Poisson shot noise, which 
contributes due to the self-correlation of individual objects: 
\beq \label{eq:bk_sn} 
B^{\rm SN}_{\ell=0}(k_1, k_2, k_3) = \frac{1}{\bar{n}} \big(P_0(k_1) + P_0(k_2) + P_0(k_3) \big) + \frac{1}{\bar{n}^2}. 
\eeq
$\bar{n}$ is the number density of objects (halos) and $P_0$ is the powerspectrum monopole. 

In order to evaluate the integrals in Eq.~\ref{eq:bk}, we take advantage of the plane-wave 
representation of the Dirac delta function and rewrite the equation as
\begin{align} \label{eq:bk2} 
    \widehat{B}_{\ell = 0}(k_1,k_2, k_3) &= \frac{1}{V_B} \int\frac{{\rm d}^3x}{(2\pi)^3} \int\limits_{k_1}{\rm d}^3q_1 \int\limits_{k_2}{\rm d}^3q_2 \int\limits_{k_3}{\rm d}^3q_3~\delta({\bfi q_1})~\delta({\bfi q_2})~\delta({\bfi q_3})~e^{i{\bfi q_{123}}\cdot{\bfi x}} - B^{\rm SN}_{\ell = 0}\\ 
    &= \frac{1}{V_B} \int\frac{{\rm d}^3x}{(2\pi)^3}\prod\limits_{i=1}^{3} I_{k_i}({\bfi x}) - B^{\rm SN}_{\ell = 0} 
\end{align}
where 
\beq
I_{k_i}({\bfi x}) = \int\limits_k {\rm d}^3q~\delta({\bfi q})~e^{i {\bfi q}\cdot{\bfi x}}. 
\eeq
At this point, we measure $\widehat{B}_{\ell = 0}(k_1, k_2, k_3)$ by calcuating the 
$I_{k_i}$s with inverse FFTs and summing over in real space.\footnote{The code that we use to 
evaluate $\widehat{B}_{\ell = 0}$ is publicly available at \url{https://github.com/changhoonhahn/pySpectrum}}
For $\widehat{B}_{\ell = 0}$
measurements throughout the paper, we use $\delta({\bfi x})$ grids with $N_{\rm grid} = 360$ 
and triangle configurations defined by $k_1, k_2, k_3$ bins of width 
$\Delta k = 3 k_f = 0.01885~h/{\rm Mpc}$, three times the fundamental mode
$k_f = 2\pi/(1000~h/{\rm Mpc})$ given the box size. 

We present the redshift-space halo bispectrum of the HADE simulations measured using 
the estimator above in two ways: one that emphasizes the triangle shape dependence 
(Figure~\ref{fig:bk_shape}) and the other that emphasizes the amplitude 
(Figure~\ref{fig:bk_amp}). In Figure~\ref{fig:bk_shape}, we plot $\BOk(k_1, k_2, k_3)$ 
as a function of $k_2/k_1$ and $k_3/k_1$, which describe the triangle configuration shape. 
In each panel, the colormap in each ($k_2/k_1$, $k_3/k_1$) bin is the weighted average $\BOk$ amplitude 
of all triangle configurations in the bin. The upper left bins contain squeezed triangles 
($k_1 = k_2 \gg k_3$); the upper right bins contain equilateral triangles 
($k_1 = k_2 = k_3$); and the bottom center bins contain folded triangles ($k_1 = 2 k_2 = 2 k_3$). 
We include all possible triangle configurations with $k_1, k_2, k_3 < k_{\rm max} = 0.5~h/{\rm Mpc}$. 
The $\BOk$ in the upper panels are HADES models with $\smnu = 0.0$ (fiducial), $0.06, 0.10$, 
and $0.15\,\mathrm{eV}$; $\BOk$ in the lower panels are  HADES models with $\smnu 0.0$ eV and 
$\sig = 0.822, 0.818$, and $0.807$. The top and bottom panels of the three right-most 
columns have matching $\sig$ values (Section~\ref{sec:hades}).

Next, in Figure~\ref{fig:bk_amp}, we plot $\BOk(k_1, k_2, k_3)$ for all possible triangle 
configurations with $k_1, k_2, k_3 < k_{\rm max} = 0.5~h/{\rm Mpc}$ where we order the 
configurations by looping through $k_3$ in the inner most loop and $k_1$ in the outer most 
loop with $k_1 \leq k_2 \leq k_3$. In the top panel, we present $\BOk$ of HADES models 
with $\smnu = 0.0, 0.06, 0.10$, and $0.15\,\mathrm{eV}$; in the lower panel, we present 
$\BOk$ of HADES models with $\smnu 0.0$ eV and $\sig = 0.822, 0.818$, and $0.807$. We 
zoom into triangle configurations with $k_1 = 0.113$, $0.226 \leq k_2 \leq 0.283$, 
and $0.283 \leq k_3 \leq 0.377~h/{\rm Mpc}$ in the insets of the panels. 
 

% --- results ---
\section{Results} \label{sec:results} 
\begin{figure}
\begin{center}
\includegraphics[width=0.8\textwidth]{figs/haloBk_dshape_kmax05_rsd.pdf} 
    \caption{The shape dependence of the $\smnu$ and $\sig$ imprint on 
    the redshift-space halo bispectrum, $\Delta \BOk/\BOk^\mathrm{fid}$. 
    We align the $\smnu{=}~0.06, 0.10$, and $0.15$ eV HADES simulations 
    in the upper panels with $\smnu{=}~0.0$~eV $\sig{=}~0.822$, $0.818$, and $0.807$ 
    simulations on the bottom such that the top and bottom panels in each column 
    have matching $\sig^{c}$, which produce mostly degenerate imprints on the 
    redshift-space power spectrum. $\BOk^{\rm fid}$ is measured from the 
    fiducial HADES simulation: $\smnu{=}~0.0$~eV and $\sig{=}~0.833$. 
    The difference between the top and bottom 
    panels highlight that $\smnu$ leaves a imprint distinct from $\sig$ on 
    elongated and isosceles triangles, bins along the bottom left and bottom 
    right edges, respectively. {\em The imprint of $\smnu$ has a distinct shape 
    dependence on the bispectrum that cannot be replicated by varying $\sig$}. 
    }
\label{fig:dbk_shape}
\end{center}
\end{figure}

\begin{figure}
\begin{center}
\includegraphics[width=0.9\textwidth]{figs/haloBk_residual_kmax05_rsd.pdf}
    \caption{The impact of $\smnu$ and $\sig$ on the redshift-space halo bispectrum,
    $\Delta \BOk/\BOk^\mathrm{fid}$, for all $1898$ triangle configurations with 
    $k_1, k_2, k_3 \leq 0.5~\mpc$. We compare $\Delta \BOk/\BOk^\mathrm{fid}$ 
    of the $\smnu = 0.06$ (top), $0.10$ (middle), and $0.15$ eV (bottom) HADES simulations 
    to $\Delta \BOk/\BOk^\mathrm{fid}$ of $\smnu{=}~0.0$ eV, $\sig{=}~0.822$, $0.818$, and 
    $0.807$ simulations. The imprint of $\smnu$ on the bispectrum has a significantly different 
    amplitude than the imprint of $\sig$. For instance, $\smnu{=}~0.15\,\mathrm{eV}$ (red) 
    has a $\sim 5\%$ stronger impact on the bispectrum than $\smnu{=}~0.0$ eV  
    $\sig{=}~0.798$ (black) even though their power spectra only differ by $< 1\%$ 
    (Figure~\ref{fig:plk}). {\em The distinct imprint of $\smnu$ on the bispectrum 
    illustrate that the bispectrum can break the degeneracy between $\smnu$ and $\sig$ 
    that degrade constraints from two-point analyses}.
    }
\label{fig:dbk_amp}
\end{center}
\end{figure}
\subsection{Breaking the $\smnu$-- $\sig$ degeneracy} \label{sec:mnusig}
One major bottleneck of constraining $\smnu$ with the power spectrum alone is the 
strong $\smnu$ -- $\sig$ degeneracy. The imprints of $\smnu$ and $\sig$ on the power 
spectrum are degenerate and for models with the same $\sig^c$, the power spectrum
only differ by $< 1\%$ (see Figure~\ref{fig:plk} and~\citealt{villaescusa-navarro2018}). 
The HADES suite, which has simulations with $\smnu = 0.0, 0.06, 0.10$, 
and $0.15$ eV as well as $\smnu = 0.0$ eV simulations with matching $\sig^{c}$, 
%--- $\sig{=}~0.822$, $0.818$, and $0.807$ , 
provides an ideal set of simulations to disentangle the impact of $\smnu$ and 
examine the degeneracy between $\smnu$ and $\sig$ (Section~\ref{sec:hades} and Table~\ref{tab:sims}). 
We measure the bispectrum of the HADES simulations (Figure~\ref{fig:bk_shape} and~\ref{fig:bk_amp}) 
and present how the bispectrum can significantly improve $\smnu$ constraints 
by breaking the $\smnu$ -- $\sig$ degeneracy. 

We begin by examining the triangle shape dependent imprint of $\smnu$ on the 
redshift-space halo bispectrum versus $\sig$ alone. In Figure~\ref{fig:dbk_shape}, 
we present the fractional residual, $(\Delta \BOk = \BOk - \BOk^{\rm fid})/\BOk^{\rm fid}$,
as a function of $k_2/k_1$ and $k_3/k_1$ for $\smnu=0.06, 0.10$, and $0.15$~eV 
in the upper panels and 0.0~eV $\sig{=}~0.822$, $0.818$, and $0.807$ in the 
bottom panels. The simulations in the top and bottom panels of each column 
have matching $\sig^{c}$. Overall, as $\smnu$ increases, the amplitude of the 
bispectrum increases for all triangle shapes (top panels). This increase is due to halo 
bias~\citep[][see also Figure~\ref{fig:plk}]{villaescusa-navarro2018}. Since we 
impose a fixed $M_{\rm lim}$ on our halos, lower values of $\sig$ translate 
to a larger halo bias, which boosts the amplitude of the bispectrum. Within the 
overall increase in amplitude, however, there is a significant triangle dependence. 
Equilateral triangles (upper left) have the largest increase. For $\smnu=0.15$ eV, 
the bispectrum is $\sim 15\%$ higher than $\BOk^{\rm fid}$ for equilateral triangles 
while only $\sim 8\%$ higher for folded triangles (lower center). The noticeable 
difference in $\Delta \BOk/\BOk^{\rm fid}$ between equilateral and squeezed 
triangles (upper left) is roughly consistent with the comparison in Figure 7 
of~\cite{ruggeri2018}. They, however, fix $A_s$ in their simulations and 
measure the real-space halo bispectrum so we refrain from any detailed 
comparisons. 

As $\sig$ increases with $\smnu = 0.0$ eV, the bispectrum also increases 
overall for all triangle shapes (bottom panels). However, the comparison of the
top and bottom panels in each column reveals significant differences in 
$\Delta \BOk/\BOk^{\rm fid}$ for $\smnu$ versus $\sig$ alone. Between 
$\smnu=0.15$ eV and \{$0.0$ eV, $\sig = 0.807$\} cosmologies, there is an overall 
$\gtrsim 5\%$ difference in the bispectrum. In addition, the shape dependence 
of the $\Delta \BOk/\BOk^{\rm fid}$ increase is different for $\smnu$ than $\sig$. 
This is particularly clear in the differences between $0.1$ eV (top center panel) 
and $\{$0.0 eV, $\sig=0.807\}$ 
(bottom right panel): near equilateral triangles in the two panels have similar 
$\Delta \BOk/\BOk^{\rm fid}$ while triangle shapes near the lower left edge from 
the squeezed to folded triangles have significantly different $\Delta \BOk/\BOk^{\rm fid}$. 
Hence, $\smnu$ leaves an imprint on the bispectrum with a distinct triangle 
shape dependence than $\sig$ alone. In other words, the triangle shape dependent 
imprint of $\smnu$ on the bispectrum cannot be replicated by varying $\sig$ --- 
unlike the power spectrum. 

We next examine the amplitude of the $\smnu$ imprint on the redshift-space halo 
bispectrum versus $\sig$ alone as a function of all triangle configurations. We present 
$\Delta \BOk/\BOk^{\rm fid}$ for all $1898$ possible triangle configurations 
with $k_1, k_2, k_3 < k_{\rm max} = 0.5~h/{\rm Mpc}$ in Figure~\ref{fig:dbk_amp}. 
We compare $\Delta \BOk/\BOk^{\rm fid}$ of the $\smnu = 0.06, 0.10$, and 
$0.15$ eV HADES models to the $\Delta \BOk/\BOk^\mathrm{fid}$ of 
$\smnu{=}~0.0$ eV $\sig{=}~0.822$, $0.818$, and $0.807$ models in the
top, middle, and bottom panels, respectively. The comparison confirms the 
difference in overall amplitude of varying $\smnu$ and $\sig$ (Figure~\ref{fig:dbk_shape}). 
For instance, $\smnu{=}~0.15\,\mathrm{eV}$ (red) has a $\sim 5\%$ stronger 
impact on the bispectrum than $\smnu{=}~0.0$ eV $\sig{=}~0.798$ (black) 
even though their power spectra differ by $< 1\%$ (Figure~\ref{fig:plk}).

The comparison in the panels of Figure~\ref{fig:dbk_amp} confirms the difference 
in the configuration dependence in $\Delta \BOk/\BOk^\mathrm{fid}$ between $\smnu$ 
versus $\sig$. The triangle configurations are ordered by looping through $k_3$ 
in the inner most loop and $k_1$ in the outer most loop such that $k_1 \geq k_2 \geq k_3$. 
In this ordering, $k_1$ increases from left to right (see Figure~\ref{fig:bk_details} 
and Appendix~\ref{sec:bk_details}). $\Delta \BOk/\BOk^\mathrm{fid}$ 
of $\smnu$ expectedly increases with $k_1$: for small $k_1$ (on large scales), 
neutrinos behave like CDM and therefore the impact is reduced. However, 
$\Delta \BOk/\BOk^\mathrm{fid}$ of $\smnu$ has a smaller $k_1$ dependence than 
$\Delta \BOk/\BOk^\mathrm{fid}$ of $\sig$. The distinct imprint of $\smnu$ on the redshift-space 
halo bispectrum illustrates that the bispectrum can break the degeneracy between 
$\smnu$ and $\sig$. Therefore, by including the bispectrum, we can more precisely
constrain $\smnu$ than with the power spectrum. 
 
\begin{figure}
\begin{center}
    \includegraphics[width=0.6\textwidth]{figs/quijote_bkCov_kmax05.png} 
    \caption{Covariance matrix of the redshift-space halo bispectrum estimated 
    using $N_{\rm cov} = 15,000$ realizations of the Qujiote simulation suite at 
    the fiducial cosmology: $\Om{=}0.3175$, $\Ob{=}0.049$, $h{=}0.6711$, $n_s{=}0.9624$, $\sig{=}0.834$, 
    and $\smnu{=}0.0$ eV. We include all possible triangle configurations with 
    $k_1, k_2, k_3 \leq k_{\rm max} = 0.5~h/{\rm Mpc}$ and order the configurations 
    (bins) in the same way as Figures~\ref{fig:bk_amp} and~\ref{fig:dbk_amp}. We 
    use the covariance matrix above for the Fisher matrix forecasts presented in 
    Section~\ref{sec:forecasts}. 
    }
\label{fig:bk_cov}
\end{center}
\end{figure}

%\begin{figure}
%\begin{center}
%    \includegraphics[width=\textwidth]{figs/quijote_pbkFisher_freeMmin_dmnu_fin_kmax02.pdf}
%    \caption{Fisher matrix constraints for $\smnu$ and other cosmological parameters 
%    for the redshift-space halo bispectrum monopole (orange). For comparison, we 
%    include Fisher parameter constraints for the redshift-space halo powerspectrum monopole 
%    in blue. The contours mark the $68\%$ and $95\%$ confidence interals. We set 
%    $k_{\rm max} = 0.2~h/{\rm Mpc}$ for both power spectrum and bispectrum. We 
%    include in our forecasts $b'$ and $M_{\rm min}$, a free amplitude scaling factor
%    and halo mass limit, respectively. They serve as a simplistic bias model and we 
%    marginalize over them to that our constraints do not include extra constraining 
%    power from the difference in bias/number density in the different Quijote cosmologies.
%    The bispectrum {\em substantially} improves constraints on all of the cosmological parameters 
%    over the power spectrum. {\em For $\smnu$, the bispectrum improves the constraint
%    from $\sigma_{\smnu}{=}~1.163$ to $0.112$ eV --- an order of magnitude improvement 
%    over the power spectrum}.}
%\label{fig:bk_fish_02}
%\end{center}
%\end{figure}

\subsection{$\smnu$ and other Cosmological Parameter Forecasts} \label{sec:forecasts}
We demonstrate in the previous section with the HADES simulations, that 
the bispectrum helps break the $\smnu$--$\sig$ degeneracy, a major 
challenge in precisely constraining $\smnu$ with the power spectrum. 
This establishes the bispectrum as a promising probe for $\smnu$. However, 
we are ultimately interested in determining the constraining power of the 
bispectrum for an analysis that include cosmological parameters beyond 
$\smnu$ and $\sig$--- \emph{i.e.} $\Om$, $\Ob$, $h$, and $n_s$. The Quijote 
suite of simulations is \emph{specifically} designed to answer this question
through Fisher matrix forecast.

First, the Quijote suite includes $N_{\rm cov} = 15,000$ $N$-body realizations run at a 
fidicial cosmology: $\smnu{=}0.0$eV, $\Om{=}0.3175, \Ob{=}0.049, 
n_s{=}0.9624, h{=}0.6711$, and $\sig=0.834$ (see Table~\ref{tab:sims}. 
This allows us to robustly estimate the covariance matrix of the bispectrum, 
$\bfi{C}$, which has ${\sim}1,800$ triangle configurations 
(Figure~\ref{fig:bk_cov}). Second, the Quijote suite includes $500$ $N$-body 
realizations evaluated at $13$ different cosmologies, each a small step away 
from the fiducial cosmology parameter values along one parameter 
(Section~\ref{sec:hades} and~Table~\ref{tab:sims}). On top of this, we apply 
redshift-space distortions along 3 different directions for these $500$ 
realizations, which then gives us $N_{\rm deriv.} = 1,500$ realizations. 
These realizations allow us to precisely estimate the derivatives of the 
bispectrum with respect to each of the cosmological parameters. 

\begin{figure}
\begin{center}
    \includegraphics[width=\textwidth]{figs/quijote_pbkFisher_bMmin_dmnu_fin_kmax0_50_reg.pdf} 
    \caption{Fisher matrix constraints for $\smnu$ and other cosmological parameters 
        for the redshift-space halo bispectrum monopole (orange). We include Fisher
        parameter constraints for the redshift-space halo powerspectrum monopole in blue 
        for comparison. The contours mark the $68\%$ and $95\%$ confidence interals. 
        We set $k_{\rm max} = 0.5~h/{\rm Mpc}$ for both power spectrum and bispectrum. 
        We include in our forecasts $b'$ and $M_{\rm min}$, a free amplitude scaling 
        factor and halo mass limit, respectively. They serve as a simplistic bias model 
        and we marginalize over them so that our constraints do not include extra 
        constraining power from the difference in bias/number density in the different 
        Quijote cosmologies. The bispectrum {\em substantially} improves constraints on all of the cosmological 
        parameters over the power spectrum. Constraints on $\Om$, $\Ob$, $h$, $n_s$, and $\sig$
        improve by factors of 3.1, 4.1, 5.1, 5.9, and 3.3, respectively. {\em For $\smnu$, the
        bispectrum improves $\sigma_{\smnu}$ from $0.1962$ to $0.0342$ eV --- a factor 
        of ${\sim}6$ improvement over the power spectrum}.}
\label{fig:bk_fish_05}
\end{center}
\end{figure}

Since their introduction to cosmology over two decades ago, Fisher Information 
matrices have been ubiquitously used to forecast the constraining power of future 
experiments~\citep[\emph{e.g.}][]{jungman1996,tegmark1997,dodelson2003,heavens2009,verde2010}. 
Defined as 
\beq 
F_{ij} = - \bigg \langle \frac{\partial^2 \mathrm{ln} \mathcal{L}}{\partial \theta_i \partial \theta_j} \bigg \rangle,
\eeq
where $\mathcal{L}$ is the likelihood, the Fisher matrix for the bispectrum can 
be written as 
\beq \label{eq:fullfish} 
F_{ij} = \frac{1}{2}~\mathrm{Tr} \Bigg[\bfi{C}^{-1}\parti{\bfi{C}}\bfi{C}^{-1}\partj{\bfi{C}} + \bfi{C}^{-1} \left(\parti{\overline{B}_0}\partj{\overline{B}_0}^T + \parti{\overline{B}_0}^T \partj{\overline{B}_0} \right)\Bigg].
\eeq
Since we assume that the $B_0$ likelihood is Gaussian, including the first 
term in Eq.~\ref{eq:fullfish} runs the risk of incorrectly including information 
from the covariance already included in the mean~\citep{carron2013}. We, therefore,
conservatively neglect the first term and calculate the Fisher matrix, 
\beq \label{eq:fisher}
F_{ij} = \frac{1}{2}~\mathrm{Tr} \Bigg[\bfi{C}^{-1} \left(\parti{\overline{B}_0}\partj{\overline{B}_0}^T + \parti{\overline{B}_0}^T \partj{\overline{B}_0} \right)\Bigg],
\eeq
directly with $\bfi{C}$ and $\partial B_0/\partial \theta_i$ along each cosmological 
parameter from the Quijote simulations. 

For $\Om$, $\Ob$, $h$, $n_s$, and $\sig$, we estimate 
\beq \label{eq:dbkdt} 
\frac{\partial \overline{B}_0}{\partial \theta_i} \approx \frac{\overline{B}_0(\theta_i^{+})-\overline{B}_0(\theta_i^{-})}{\theta_i^+ - \theta_i^-}, 
\eeq
where $\overline{B}_0(\theta_i^{+})$ and $\overline{B}_0(\theta_i^{-})$ are 
the average bispectrum of the $1,500$ realizations at $\theta^{+}$ and $\theta^{-}$. 
Meanwhile, for $\smnu$, where the fiducial value is 0.0 eV and we cannot have 
negative $\smnu$, we use the Quijote simulations at $\smnu^+$, $\smnu^{++}$, 
$\smnu^{+++}=0.1, 0.2, 0.4$ eV (Table~\ref{tab:sims}) to estimate 
\beq \label{eq:dbkdmnu} 
\frac{\partial \overline{B}_0}{\partial \smnu} \approx \frac{-21\overline{B}_0(\smnu^{\rm fid}) + 
32 \overline{B}_0(\smnu^{+}) - 12 \overline{B}_0(\smnu^{++}) + \overline{B}_0(\smnu^{+++})}{1.2}, 
\eeq
which provides a $\mathcal{O}(\delta \smnu^2)$ order approximation. By using these 
$N$-body simulations, instead of analytic methods (\emph{e.g.} perturbation theory), 
we exploit the accuracy of numerical simulations in the nonlinear regime and rely on 
fewer assumptions and approximations. In fact, \emph{these $N$-body simulation estimated 
derivatives are the key ingredients that enables us to quantify, for the first time, 
the full information content of the redshift-space bispectrum in the non-linear regime}. 
We discuss subtleties of Eq.~\ref{eq:dbkdmnu} bispectrum derivative and tests of 
convegence in Appendix~\ref{sec:numerical}. 

We present the constraints on $\smnu$ and other cosmological parameters 
$\{\Om, \Ob, h, n_s, \sig\}$ derived from the redshift-space halo bispectrum 
Fisher matrix (Eq.~\ref{eq:fisher}) for $k_{\rm max} = 0.5~h/{\rm Mpc}$ 
in Figure~\ref{fig:bk_fish_05}, respectively. 
We include Fisher constraints for the redshift-space halo power spectrum 
monopole with the same $k_{\rm max}$ for comparison (blue). We mark the 
$68\%$ and $95\%$ confidence interals with the contours. We include in our 
Fisher constraints the following nuisance parameters: $b'$, a scaling factor 
on the bispectrum amplitude, and $M_{\rm min}$, the halo mass limit. $b'$ is 
analogous to linear bias. Meanwhile, we choose $M_{\rm min}$ as a nuisance
parameter to address the difference in the number densities among the Quijote
cosmologies, which impacts the derivatives $\partial \overline{B}_0/\partial \theta_i$. 
For instance, the $\sig^{+}$ and $\sig^{-}$ cosmologies have halo 
$\bar{n} = 1.586\times10^{-4}$ and $1.528 \times 10^{-4}(h^{-1}{\rm Mpc})^3$. 
These parameters serve as a simplistic bias model and by marginalizing 
over them we aim to ensure that our Fisher constraints do not include extra 
constraining power from the difference in bias or number density. $b'$ is a 
multiplicative factor so $\partial \overline{B}_0/\partial b' = \overline{B}_0$. 
Meanwhile, we numerically estimate $\partial \overline{B}_0/\partial M_{\rm min}$ 
using $\overline{B}_0$ evaluated at $M^{+}_{\rm min}{=}3.3\times10^{13}h^{-1}M_\odot$ 
and $M^{-}_{\rm min}{=}3.1\times10^{13}h^{-1}M_\odot$, with all other parameters 
set to the fiducial value. 

The bispectrum {\em substantially} improves constraints on all parameters 
over the power spectrum. For $k_{\rm max} = 0.5~h/{\rm Mpc}$, the 
bispectrum improves the marginalized constraints, $\sigma_\theta$ of $\Om$, 
$\Ob$, $h$, $n_s$, and $\sig$ by factors of $\sim$3, 4, 5, 6, and 3 over 
the power spectrum. {\em For $\smnu$, the bispectrum improves the constraint 
from $\sigma_{\smnu}{=}~0.1962$ to $0.0342$eV --- a factor of 6 improvement 
over the power spectrum}. We emphasize that this $\sigma_{\smnu}{=}0.0342$eV 
constraint is for the {\em bispectrum alone} and only for a $1h^{-1}{\rm Gpc}$ 
box. We list the precise marginalized Fisher parameter constraints of both 
cosmological and nuisance parameters for $P_0$ and $B_0$ in Table~\ref{tab:forecast}. 

Even below $k_{\rm max} < 0.5~h/{\rm Mpc}$, the bispectrum significantly 
improves cosmological parameter constraints. We compare $\sigma_\theta$, 
the marginalized $1\sigma$ constraints of $\Om$, $\Ob$, $h$, $n_s$, $\sig$, 
and $\smnu$, as a function of $k_{\rm max}$ for $B_0$ (orange) and $P_0$ 
(blue) in Figure~\ref{fig:fish_kmax}. We focus only on the $k_{\rm max}$ range 
where the Fisher forecast is well defined --- \emph{i.e.} more data bins 
than the number of parameters: $k_{\rm max} > 8 k_{\rm f} \approx 0.05~h/{\rm Mpc}$ 
for $P_0$ and $k_{\rm max} > 12 k_{\rm f} \approx 0.075~h/{\rm Mpc}$ for 
$B_0$. Figure~\ref{fig:fish_kmax} reveals that the improvement of the
bispectrum $\sigma_\theta$ over the power spectrum $\sigma_\theta$ is 
larger at higher $k_{\rm max}$. Although limited by the $k_{\rm max}$ range, 
the figure suggests that on large scales ($k_{\rm max}\lesssim 0.1~h/{\rm Mpc}$) 
$P_0$ $\sigma_\theta$ crosses over $B_0$ $\sigma_\theta$ so $P_0$ has more 
constraining power than $B_0$--- as expected on linear scales. On slightly 
larger $k_{\rm max}$, even at $k_{\rm max} = 0.2~h/{\rm Mpc}$, we find that 
the bispectrum substantially improves $\sigma_\theta$ by factors of
$\sim 1.8$, 2.1, 2.6, 2.5, 2.6, and $3.1$ for $\Om$, $\Ob$, $h$, $n_s$, $\sig$, 
and $\smnu$ respectively. 

%%%%%%%%%%%%%%%%%%%%%%%%%%%%%%%%%%%%%%%%%%
% forecast table
%%%%%%%%%%%%%%%%%%%%%%%%%%%%%%%%%%%%%%%%%%
\begin{table}
    \caption{Marginalized Fisher parameter constraints from the redshift-space halo power 
    spectrum (top) and bispectrum (bottom) for different values of $k_{\rm max}$. We list 
    constraints for cosmological parameters $\smnu$, $\Omega_m$, $\Omega_b$, $h$, $n_s$, 
    and $\sig$ as well as nuisance parameters $b'$ and $M_{\rm min}$} 
\begin{center} 
    \begin{tabular}{cccccccccc} \toprule
         & $k_{\rm max}$ & $\smnu$ & $\Omega_m$ & $\Omega_b$ & $h$ & $n_s$ & $\sig$ & $b'$ & $M_{\rm min}$ \\
         & ({\footnotesize $h/{\rm Mpc}$}) &({\footnotesize eV}) & & & & & & & ({\footnotesize $10^{13} h^{-1}M_\odot$}) \\[3pt] \hline\hline

            &     & 0.0 & 0.3175 & 0.049 & 0.6711 & 0.9624 & 0.834 & 1. & 3.2  \\ 
    $P_0$  & 0.2 &$\pm0.333$ & $\pm0.052$ & $\pm0.030$ & $\pm0.372$ & $\pm0.347$ & $\pm0.128$ & $\pm0.649$ & $\pm5.045$\\
           & 0.3 &$\pm0.277$ & $\pm0.044$ & $\pm0.023$ & $\pm0.273$ & $\pm0.276$ & $\pm0.069$ & $\pm0.383$ & $\pm2.457$\\
           & 0.4 &$\pm0.228$ & $\pm0.040$ & $\pm0.020$ & $\pm0.235$ & $\pm0.240$ & $\pm0.059$ & $\pm0.226$ & $\pm1.270$\\
           & 0.5 &$\pm${\bf 0.196}& $\pm0.036$ & $\pm0.018$ & $\pm0.207$ & $\pm0.213$ & $\pm0.048$ & $\pm0.157$ & $\pm0.807$\\\hline
%    {\footnotesize with Planck priors} & 0.5 &$\pm0.080$& $\pm0.013$ & $\pm0.0012$ & $\pm0.0086$ & $\pm0.0048$ & $\pm0.019$ & $\pm0.067$ & $\pm0.362$\\\hline
                   
    $B_0$  & 0.2 &$\pm0.107$ & $\pm0.029$ & $\pm0.014$ & $\pm0.144$ & $\pm0.140$ & $\pm0.050$ & $\pm0.265$ & $\pm1.317$\\
           & 0.3 &$\pm0.065$ & $\pm0.020$ & $\pm0.008$ & $\pm0.077$ & $\pm0.074$ & $\pm0.023$ & $\pm0.143$ & $\pm0.657$\\
           & 0.4 &$\pm0.043$ & $\pm0.015$ & $\pm0.006$ & $\pm0.052$ & $\pm0.047$ & $\pm0.016$ & $\pm0.088$ & $\pm0.369$\\
           & 0.5 &$\pm${\bf 0.034}& $\pm0.012$ & $\pm0.004$ & $\pm0.040$ & $\pm0.036$ & $\pm0.014$ & $\pm0.070$ & $\pm0.269$\\[3pt]

%B sigmas 0.0086, 0.0008, 0.0060, 0.0043, 0.0085, 0.0297, 0.0679, 0.2549
    \hline
\end{tabular} \label{tab:forecast}
\end{center}
\end{table}
%%%%%%%%%%%%%%%%%%%%%%%%%%%%%%%%%%%%%%%%%%

%%%%%%%%%%%%%%%%%%%%%%%%%%%%%%%%%%%%%%%%%%
% sigma(kmax)  
%%%%%%%%%%%%%%%%%%%%%%%%%%%%%%%%%%%%%%%%%%
\begin{figure}
\begin{center}
    \includegraphics[width=0.9\textwidth]{figs/quijote_Fisher_kmax_bMmin_dmnu_fin_reg.pdf} 
    \caption{Marginalized $1\sigma$ constraints of the cosmological parameters 
    $\Om$, $\Ob$, $h$, $n_s$, $\sig$, and $\smnu$ ($\sigma_\theta$) as a function 
    of $k_{\rm max}$ for the redshift-space halo bispectrum (orange) and power spectrum (blue). 
    Though not included in the figure, we marginalize over the nuisance parameters
    $b'$ and $M_{\rm min}$ in our forecast (Section~\ref{sec:forecasts}). We only 
    include $k_{\rm max} > 8 k_{\rm f}$ for $P_0$ and $k_{\rm max} > 12 k_{\rm f}$ 
    for $B_0$ --- $k_{\rm max}$ ranges where we have more data bins than number of 
    parameters. Even at $k_{\rm max} < 0.5~h/{\rm Mpc}$, the bispectrum significantly 
    improves cosmological parameter constraints. The improvement, however, is larger 
    for higher $k_{\rm max}$. At $k_{\rm max} = 0.2~h/{\rm Mpc}$, the bispectrum 
    improves constraints on $\Om$, $\Ob$, $h$, $n_s$, $\sig$, and $\smnu$ by factors 
    of $\sim 1.8$, 2.1, 2.6, 2.5, 2.6, and 3.1 over the power spectrum.}
\label{fig:fish_kmax}
\end{center}
\end{figure}
%%%%%%%%%%%%%%%%%%%%%%%%%%%%%%%%%%%%%%%%%%
Our forecasts demonstrate that the bispectrum has significant constraining power 
in the weakly non-linear regime ($k > 0.1~\mpc$) beyond the power spectrum. This constraining power 
comes from the bispectrum breaking degeneracies among the cosmological and 
nuisance parameters. This is evident when we compare the unmarginalized 
constraints from $P_0$ and $B_0$: $1/\sqrt{F_{ii}}$, where $F_{ii}$ is a 
diagonal element of the Fisher matrix. For $k < 0.4~\mpc$, the unmarginalized 
cosntraints from $P_0$ are tighter than those from $B_0$. Yet, once we 
marginalize the constraints over the other parameters, the $B_0$ constraints 
are tigher than $P_0$ for $k > 0.1~\mpc$. The derivatives, 
$\partial B_0/\partial \theta_i$, also shed light on how $B_0$ breaks parameter 
degeneracies. The parameter degeneracies in the $P_0$ forecasts of Figure~\ref{fig:bk_fish_05} 
are consistent with similarities among the shape and scale dependence of 
$P_0$ derivatives $\partial P_0/\partial \theta$. On the other hand, the 
$B_0$ derivatives with respect to the parameters have significant different 
scale and triangle shape dependences. \ch{derivative figure?} 

By exploiting the unprecedented number of $N$-body simulations of the Quijote 
suite, we present for the first time the full information content of the 
redshift-space bispectrum beyond the linear regime. The information content 
of the bispectrum, however, has previously been examined using perturbation 
theory. Previous works, for instance, measure the signal-to-noise ratio (SNR) 
of the bispectrum derived from covariance matrices estimated using perturbation 
theory~\citep[\emph{e.g.}][]{sefusatti2005, chan2017}. Most recently, \cite{chan2017},
using covariance matrices that include non-Gaussian contributions calibrated with 
$N$-body simulations, find that the cumulative SNR of the halo bispectrum is 
$\sim 30\%$ of the SNR of the halo power spectrum at $k_{\rm max} \sim 0.1~\mpc$ 
and increases to $\sim 40\%$ at $k_{\rm max}\sim 0.35~\mpc$. While these simple 
SNR measurements cannot be easily compared to our Fisher analysis, we note that 
they are loosely consistent with the unmarginalized constraints, where we find 
tighter unmarginalized constraints from $P_0$ than $B_0$ at $k_{\rm max} < 0.4~\mpc$. 
Also, when we measure the the halo power spectrum and bispectrum SNRs using our 
covariance matrices (Figure~\ref{fig:bk_cov}), we find a relation between the 
SNRs consistent with \cite{chan2017}. Beyond the $k$ range of \cite{chan2017} 
($k_{\rm max}>0.35~\mpc$), we find that the SNR of $B_0$ continues to increase 
at higher $k_{\rm max}$ in constract to the $P_0$ SNR, which saturates at 
$k_{\rm max}\sim 0.1~\mpc$. At $k_{\rm max} = 0.75~\mpc$, the largest $k$ we 
probe, the $B_0$ SNR is $\sim75\%$ of the $P_0$ SNR.
% https://arxiv.org/abs/1506.00083, https://arxiv.org/abs/1512.05383

Beyond these signal-to-noise calculations, a number of previous works 
quantify the information content of the bispectrum with Fisher 
forecasts~\citep{scoccimarro2004, sefusatti2006, sefusatti2007, song2015, tellarini2016, yamauchi2017a, karagiannis2018, yankelevich2019}. 
While most of these works fix most cosmological parameters and focus soley 
on forecasting constraints of primordial non-Gaussianity ($f_{\rm NL}$) and 
bias parameters, \cite{sefusatti2006} and \cite{yankelevich2019} provide 
bispectrum forecasts for full sets of cosmological parameters. In \cite{sefusatti2006}, 
they present likelihood analysis forecasts for $\omega_d$, $\omega_b$, $\OL$, 
$n_s$, $A_s$, $w$, $\tau$. %$\omega_d = \Omega_{\rm cdm}h^2$ and $\omega_b = \Ob h^2$ are the physical dark matter and baryon densities,  $\tau$ is the reionization optical depth, $A_s$ is the primordial amplitude of scalar fluctuations, and $w$ is dark energy equation of state parameter.  
For $\Lambda$CDM, with fixed bias parameters, and $k_{\rm max} = 0.3~\mpc$, 
they find constraints on $\Om$, $\Ob$, $h$, $n_s$, and $\sig$ from WMAP, 
$P_0$, and $B_0$ is a factor of 1.71, 1.3, 1.4, 1.2, and 1.5 tighter 
than constraints from WMAP and $P_0$ alone. In comparison, for 
$k_{\rm max} = 0.3~\mpc$ our $B_0$ constraints are tighter than $P_0$ constraints 
by factors of 2.1, 2.9, 3.5, 3.7, and 3.0. Both \cite{sefusatti2006} and our 
analysis find significantly tighter constraints with the bispectrum. They 
however include the WMAP likelihood in their forecast and only present 
constraints with both $P_0$ and $B_0$. We also emphasize that they use 
perturbation theory bispectrum models, which break down on small scales. 
In a comparison with $N$-body simulations, \cite{lazanu2016} find that 
perturbation theory models of the matter bispectrum deviate by $>5\%$ 
at $k_{\rm max} \gtrsim 0.15~\mpc$. 

Meanwhile, in \cite{yankelevich2019}, they present Fisher forecasts for 
$\Omega_{\rm cdm}$, $\Ob$, $h$, $n_s$, $A_s$, $w_0$, and $w_0$ for a 
Euclid-like survey~\citep{laureijs2011} in 14 non-overlapping redshift 
bins over $0.65 < z < 2.05$. They use the full redshift-space bispectrum,
rather than just the monopole, and a more sophisticated bias expansion 
than \cite{sefusatti2006} that depend on the tidal field. They use a 
pertrubative model for the bispectrum, which consequently limit their 
forecast to $k_{\rm max} = 0.15~\mpc$. From their forecasts, they find 
similar constraining power on cosmological parameters from $B$ alone as $P$. 
They also find that combining the bispectrum with the power spectrum 
only moderately improves parameter constraints because posterior 
correlations are similar for $P$ and $B$. While this seemingly conflicts 
with the results we present, there are significant differences between our 
forecasts. For instance, their forecasts are at higher redshifts, $z > 0.7$,
where we expect the constraining power of $B$ to be weaker than at $z=0$. 
They also forecast the {\em galaxy} $P$ and $B$ and marginalizes over 
56 nuisance parameters (14 $z$ bins each with 3 bias parameters and 1 
RSD parameter). They also neglect non-Gaussian contributions to the $B$ 
covariance matrix, which substantially impact the constraints especially 
on small scales~\citep{chan2017}. %; for $k_{\rm max} \sim 0.14~\mpc$, neglecting these contributions overestimates the SNR of the bispectrum by a factor of $\sim 3$~\citep{chan2017}. 
Despite differences, \cite{yankelevich2019} find that the constraining 
power of $B$ relative to $P$ increases for higher $k_{\rm max}$, consistent 
with our forecasts as a function of $k_{\rm max}$ (Figure~\ref{fig:fish_kmax}). 
Also, similar to their forecast, for for $k_{\rm max} = 0.15~\mpc$, 
we also find similar posterior correlations between the $P_0$ and $B_0$ 
constraints. At $k_{\rm max} = 0.5~\mpc$, however, we find significantly 
less correlations~(Figure~\ref{fig:bk_fish_05}). Based on our forecasts, 
at higher $k_{\rm max}$ adding $B_0$ to the analysis improve parameter 
constraints beyond the improvement found in \cite{yankelevich2019}. 

While the various difference between our forecast and previous work 
prevent more thorough comparisons, the main difference is that we 
present the first bispectrum forecasts for a full set of cosmological 
parameters using bispectrum measurements entirely from $N$-body 
simulations. This allows us to go beyond previous perturbation theory 
based forecasts and quantify the full information content of the 
redshift-space bispectrum all the way out to the non-linear regime. 
Moreover, we also present the first bispectrum forecast of cosmological
parameters including neutrinos and demonstrate the potential of the 
bispectrum in constraining $\smnu$. Below, we underline a few caveats 
of our forecasts.

Our forecasts are derived from Fisher matrices. Such forecasts make 
the assumption that the posterior is approximately Gaussian and, as a result, 
they underestimate the constraints for posteriors that are highly 
non-elliptical or asymmetric~\cite{wolz2012}. Fisher matrices also rely 
on the stability, and in our case convergence, of numerical derivatives. 
We examine the stability of the $P_0$ and $B_0$ derivatives with respect 
to $\smnu$ by comparing the derivatives computed using $N$-body simulations 
at three different sets of cosmologies: (1) \{fiducial, $\smnu^{+}$, $\smnu^{++}$, 
and $\smnu^{+++}$\} (Eq.~\ref{eq:dbkdmnu}), (2) \{fiducial, $\smnu^{+}$, 
and $\smnu^{++}$\}, and (3) \{fiducial and $\smnu^{+}$ \}~(Appendix~\ref{sec:numerical}; Figure~\ref{fig:dPBdmnu}). 
Compared to the derivatives derivatives computed with the (1) cosmologies, 
the derivatives with (2) and (3) differ by $\sim 20$ and $\sim 50\%$ for 
both $P_0$ and $B_0$. These differences do not impact the $\Om$, $\Ob$, $h$, 
$n_s$, and $\sig$ constraints. They do however affect the $\smnu$ constraints; 
but because $P_0$ and $B_0$ derivatives are scaled by the same factor, 
{\em the relative improvement of the $B_0$ constraint over the $P_0$ constraints 
is not impacted}. 
%$\sigma_\smnu = 0.1962$, 0.2385, and 0.3774 eV with $P_0$ and $\sigma_\smnu = 0.0342$, 0.0416, and 0.0657 eV with $B_0$. 

In addition to the stability of the derivatives, because we use $N$-body 
simulations, we test whether the convergence of our covariance matrix and 
derivatives impact our forecasts by varying the number of simulations 
used to estimate them: $N_{\rm cov}$ and $N_{\rm deriv}$, repsectively. 
For $N_{\rm cov}$, we find $< 5\%$ variation in the Fisher matrix elements, 
$F_{i,j}$ for $N_{\rm cov} > 5000$. Meanwhile, the marginalized parameter
constraints, $\sigma_\theta$, vary by $< 1\%$ for $N_{\rm cov} > 12000$. 
When we vary $N_{\rm deriv}$, we find $< 5\%$ variation in the $F_{i,j}$ 
elements and $\sigma_\theta$ for $N_{\rm deriv} > 1200$. 
Since our constraints vary by $< 10\%$ for sufficient $N_{\rm cov}$ and 
$N_{\rm deriv}$, {\em we conclude that the convergence of the covariance matrix 
and derivatives do not impact our forecasts}. We refer readers to 
Apprendix~\ref{sec:numerical} for a more details on the robustness of 
our results to the stability of the derivatives and convergence. 

We argue that the constraining power of the bispectrum and its improvement 
over the power spectrum come from breaking degeneracies among the cosmological 
parameters. However, numerical noise can impact our forecasts when we invert 
the Fisher matrix. Since Planck constrain $\Om$, $\Ob$, $h$, $n_s$, and $\sig$ 
tighter than the $P_0$ and $B_0$ alone (\ch{@paco: citation to planck prior}), 
the elements of the Planck prior matrix are larger than the elements of $P_0$ 
and $B_0$ Fisher matrices. By including Planck priors (\emph{i.e.} adding the 
prior matrix to the Fisher matrix) increases the numerical stability of our 
forecasts. It also reveals whether the bispectrum still improves parameter 
constraints once we include CMB constraints. With Planck priors and $P_0$ to 
$k_{\rm max} = 0.5~\mpc$, we derive the following constraints: 
$\sigma_{\Om} = 0.0125$, $\sigma_{\Ob} = 0.0012$, $\sigma_h=0.0086$, 
$\sigma_{n_s}=0.0048$, and $\sigma_{\sig}=0.0194$. 
Including Planck priors expectedly tighten the constraints from $P_0$ alone. 
Meanwhile, with Planck priors and $B_0$ to $k_{\rm max} = 0.5~\mpc$, we get 
$\sigma_{\Om} = 0.0086$, $\sigma_{\Ob} = 0.0008$, $\sigma_h=0.0006$, 
$\sigma_{n_s}=0.0043$, and $\sigma_{\sig}=0.0085$, 1.5, 1.6, 1.4, 1.1, and 2.3 
times tighter constraints. For $\smnu$, $\sigma_{\smnu} = 0.0802$ eV for $P_0$ 
and $\sigma_{\smnu} = 0.0297$ eV, a factor of 2.7 improvement.  Hence, even with 
Planck priors, $B_0$ improves substantially parameter constraints, especially for 
$\smnu$. This also suggests that the improvement from $B_0$ we find in our 
forecasts are numerically robust. 

So far in our analysis, our forecasts are derived using the power spectrum
and bispectrum in a \emph{periodic box}. We do not consider a realistic 
geometry or radial selection function of actual observations from galaxy 
surveys. A realistic selection function will smooth out the triangle 
configuration dependence and consequently degrade the constraining power 
of the bispectrum. In \cite{sefusatti2005}, for instance, they find that the 
signal-to-noise of the bispectrum is significantly reduced once survey geometry 
is included in their forecast. Survey geometry, however, also degrades the 
signal-to-noise of their power spectrum forecasts. Hence, with the substantial 
improvement in the $\smnu$ constraints of the bispectrum, even with survey 
geometry we expect the bispectrum will significantly improve $\smnu$ constraints 
over the power spectrum. 

We include the nuisance parameter $\mmin$ in our forecasts to address the 
difference in halo number densities among the Quijote cosmologies. Although 
we marginalize over $\mmin$, this may not fully account for the extra 
information from $\bar{n}$ and nonlinear bias leaking into the derivatives. 
To test this, we include extra nuisance parameters, $A_{\rm SN}$, 
$B_{\rm SN}$, $b_2$, and $\gamma_2$, and examine their impact on our forecasts. 
$A_{\rm SN}$ and $B_{\rm SN}$ are multiplicative factors of the first and
second terms of Eq.~\ref{eq:bk_sn}, which we include to account for any 
$\bar{n}$ dependence that may be introduced from the $B_0$ shotnoise correction. 
$b_2$ and $\gamma_2$ are the quadratic bias and nonlocal bias 
parameters~\citep{chan2012, sheth2013} to account for information from 
nonlinear bias. Marginalizing over $b'$, $\mmin$, $A_{\rm SN}$, $B_{\rm SN}$, 
$b_2$ and $\gamma_2$, we obtain the following constraints from $B_0$ 
with $k_{\rm max} = 0.5~\mpc$:
$\sigma_{\Om} = 0.0129$, $\sigma_{\Ob} = 0.0044$, $\sigma_h=0.0404$, 
$\sigma_{n_s}=0.0455$, $\sigma_{\sig}=0.0228$, and $\sigma_{\smnu}=0.0343$. 
Only constraints on $n_s$ and $\sig$ are broadened from our fiducial 
forecasts, by $27\%$ and $60\%$. Other parameters, especially $\smnu$, 
are not impacted by marginalizing over the extra nuisance parameters. 
As another test to further isolate the impact of the differing $\bar{n}$, 
we calculate derivatives using halo catalogs from Quijote $\theta^{-}$ 
and $\theta^{+}$ cosmologies--- with fixed $\bar{n}$. We similarly find 
no significant impact on the parameter constraints from $B_0$. Forecasts 
using additional nuisance parameters and with fixed $\bar{n}$ derivatives, 
both support the robustness of our forecast. Yet these tests do not 
ensure that our forecast entirely marginalizes over halo bias. 

In this paper, we focus on the halo bispectrum and power spectrum. However,   
constraints on $\smnu$ will ultimately be derived from the distribution of 
galaxies. Besides the cosmological parameters, bias and nuisance parameters 
that allow us to marginalize over galaxy bias need to be incorporated to 
forecast $\smnu$ and other cosmological parameter constraints for the 
galaxy bispectrum. Although we include a \emph{naive} bias model through $b'$ 
and $M_{\rm min}$, and even $b_2$ and $\gamma_2$ in our tests, this is 
insufficient to describe how galaxies trace matter. A more realistic bias model 
such as a halo occupation distribution (HOD) model involve extra parameters 
that describe the distribution of central and satellite galaxies in 
halos~\citep[\emph{e.g.}][]{zheng2005,leauthaud2012a,tinker2013,zentner2016,vakili2019}. 
We, therefore, refrain from a more exhaustive investigation of the impact of 
halo bias on our results and focus quantifying the constraing power of 
the galaxy bispectrum in the next paper of this series. 

Marginalizing over galaxy bias parameters, will likely reduce the constraining 
power at high $k$. \cite{hand2017}, for instance, find for their 13 parameter 
model only find a 15-30\% improvement in $f\sig$ when they extend their power 
spectrum multipole analysis from $0.2~\mpc$ to $0.4~\mpc$. Furthermore, jointly 
analyzing power spectrum and bispectrum will help break parameter degeneracies 
and improve constraints on cosmological parameters~\citep{sefusatti2006, yankelevich2019}. 
We again emphasize that the constraints we present is for a $1h^{-1}{\rm Gpc}$ box, 
a substantially volume than upcoming surveys. Thus, even if the constraining 
power at high $k$ is reduced, our forecasts suggest that the bispectrum still 
offers significant improvements over the power spectrum, especially for 
constraining $\smnu$. 
 

% --- summary ---
\section{Summary} \label{sec:summary} 

Afterwards, we will apply it to future surveys. 
\ch{rough numbers of DESI, PFS, and Euclid} 

% what are the implications of our results? 
%The constraints from our forecasts are derived for a $(1~h^{-1}{\rm Gpc})^3$ 
%volume. These constraints roughly scale as $\propto1/\sqrt{V}$ with volume. 
%Upcoming spectroscopic galaxy surveys will map out vastly larger cosmic
%volumes: PFS $\sim 9~h^{-3}{\rm Gpc}^3$ and DESI $\sim 50~h^{-3}{\rm Gpc}^3$~\citep{takada2014, desicollaboration2016}. 
%Euclid and WFIRST, space-based surveys, will expand these volumes to 
%higher redshifts. A key science goal in these surveys will be constraining $\smnu$. 
%Scaling our 0.0572 eV $\smnu$ constraint from the bispectrum to the survey 
%volumes gives 0.0191 eV and 0.0081 eV $1\sigma$ constraints for PFS and DESI. 
%Galaxy samples in these surveys will also have significantly higher 
%number densities than the halos used in our forecast: \emph{e.g.} 
%the DESI BGS at $z{\sim}0.3$, DESI LRG, and PFS at $z \sim 1.3$ 
%will have ${\sim}20, 3$, and $5\times$ higher number densities, 
%respectively. With such precision, we would detect $\smnu$ by $>3\sigma$ and 
%distinguish between the normal and inverted neutrino mass hierarchy scenarios 
%by $>2\sigma$ from the bispectrum alone. Such naive projections, however, 
%should be taken with more than a grain of salt. Nevertheless, the substantial 
%improvement we find in constraints with the bispectrum over the power spectrum, 
%strongly advocate for analyzing future surveys with more than the power spectrum. 
%Our results demonstrate the potential of the bispectrum to tightly constrain 
%$\smnu$ with unprecedented precision.
 

\section*{Acknowledgements}
It's a pleasure to thank 
    Enea Di Dio, 
    Daniel Eisenstein, 
    Simone Ferraro, 
    Shirley Ho, 
    Emmaneul Schaan, 
    Zachary Slepian, 
    David N.~Spergel, 
    and Benjamin D.~Wandelt
    for valuable discussions and comments. 

\appendix
\begin{figure}
\begin{center}
    \includegraphics[width=0.95\textwidth]{figs/quijote_B_details.pdf}
    \caption{
        {\em Top}: We highlight the triangle configurations of the bispectrum that fall within: 
        $k_{\rm max} <$ 0.1 (purple), 0.2 (red), 0.3 (green), 0.4 (orange), and $0.5~\mpc$ (blue). The number of 
        triangle configurations increase significantly at higher $k_{\rm max}$, which contributes to the 
        significant constraining power on nonlinear scales. 
        {\em Center}: We mark triangle configurations with three different shapes:
        equilateral (black triangle), folded (green star), and squeezed (red cross) triangles. Equilateral 
        triangles have $k_1 = k_2 = k_3$; folded triangles have $k_1 = k_2 + k_3$; squeezed triangles have 
        $k_1, k_2 \gg k_3$, above we mark configurations with $k_1, k_2 > 2.4k_3$ as squeezed. 
        {\em Bottom}: Comparison of the real (orange) and redshift-space (blue) halo bispectrum for 
        fiducial Quijote simulations. The redshift-space $\BOk$ has a significantly higher 
        overall amplitude than the real-space $\BOk$, with a shape and $k_{\rm max}$ dependence. 
        As a result, the redshift-space $\BOk$ has a higher signal-to-noise and more 
        constraining power than the real-space $\BOk$.  
    } 
\label{fig:bk_details}
\end{center}
\end{figure}
\section{Redshift-space Bispectrum} \label{sec:bk_details}
In this work, we present how the redshift-space bispectrum helps break degeneracies 
among cosmological parameters, especially between $\smnu$ and $\sig$, and improve 
constraints on $\smnu$ as well as the other cosmological parameters. We measure 
the redshift-space bispectrum monopole, $\BOk$, of the HADES and Quijote simulations 
(Section~\ref{sec:hades}) using an FFT based estimator similar to the ones in 
\cite{sefusatti2005a}, \cite{scoccimarro2015}, and \cite{sefusatti2016} 
(see Section~\ref{sec:bk} for details). We use triangle configurations defined by 
$k_1$, $k_2$, and $k_3$ bins of width $\Delta k = 3k_f = 0.01885~h/{\rm Mpc}$. 
For $k_{\rm max} = 0.5~h/{\rm Mpc}$ we have $\BOk$ for 1898 
triangle configurations. For $k_{\rm max} = 0.4$, 0.3, 0.2, and $0.1~h/{\rm Mpc}$, 
we have 1056, 428, 150, and 28 triangle configurations respectively. The number
of triangle configurations sharply increases for higher $k_{\rm max}$. This
contributes to the significant increase in constraining power as we go to increasingly 
nonlinear scales (Figure~\ref{fig:fish_kmax}). We highlight the triangle 
configurations at different $k_{\rm max}$ for the $\BOk$ of the Quijote simulation 
at the fiducial cosmology in the top panel of Figure~\ref{fig:bk_details}. We mark 
the configurations within $k_{\rm max} = 0.1, 0.2, 0.3, 0.4$, and $0.5~h/{\rm Mpc}$ 
in purple, red, green, orange, and blue respectively. 

In Figure~\ref{fig:bk_details}, as well as in Figures~\ref{fig:bk_amp}, \ref{fig:dbk_amp}, 
\ref{fig:bk_deriv}, we order the triangle configurations by looping through 
$k_3$ in the inner most loop and $k_1$ in the outer most loop such that 
$k_1 \geq k_2 \geq k_3$. This ordering is different from the ordering in~\cite{gil-marin2017}, 
which loops through $k_3$ in the inner most loop and $k_1$ in the outer most 
increasing loop but with $k_1 \leq k_2 \leq k_3$. As the top panel demonstrates, 
our ordering clearly reflects the $k_{\rm max}$ range of the configurations. 
Furthermore, the repeated configuration sequences in our ordering cycles through
all available triangle configurations for a given $k_1$. In the center panel 
of Figure~\ref{fig:bk_details}, we mark equilateral (black triangle), folded 
(green star), and squeezed (red cross) triangle configurations. Equilateral 
triangles have $k_1 = k_2 = k_3$, folded triangles have $k_1 = k_2 + k_3$, and 
squeezed triangles have $k_1, k_2 \gg k_3$ --- $k_1, k_2 > 2.4k_3$ in Figure~\ref{fig:bk_details}. 
These shapes correspond to the three vertices of the bispectrum shape plots of 
Figures~\ref{fig:bk_shape} and \ref{fig:dbk_shape}. 

Lastly, in the bottom panel of Figure~\ref{fig:bk_details} we compare the 
redshift-space bispectrum (blue) to the real-space bispectrum of the same 
Quijote simulations at the fiducial cosmology (orange). Overall, the 
redshift-space $\BOk$ has a higher amplitude than the real-space $\BOk$ with 
a significant triangle shape dependence: equilateral triangles have the 
largest difference in amplitude while folded triangles have the smallest. 
The relative amplitude difference also depends significantly on $k_{\rm max}$ 
where the difference is larger for configurations with higher $k_1$. With 
its higher amplitude, the redshift-space $\BOk$ has a higher signal-to-noise 
and more constraining power than the real-space $\BOk$.  
 
\section{Fisher Forecasts using $N$-body simulations} \label{sec:numerical}
The two key elements in calculating the Fisher matrices we use in our forecasts
are the bispectrum covariance matrix ($\bfi{C}$; Figure~\ref{fig:bk_cov}) and 
the derivatives of the bispectrum along the cosmological and nuisance parameters, 
$\partial B_0/\partial \theta_i$ (Section~\ref{sec:forecasts}). We compute 
both these elements directly using the $N$-body simulations of the Quijote suite 
(Section~\ref{sec:hades}). This takes advantage of the $N$-body simulations and 
allows us to accurately quantify the constraining power of the bispectrum that 
come from the nonlinear regime. However, to trust our forecast, we must ensure that 
both $\bfi{C}$ has converged and that the numerically calculated 
$\partial B_0/\partial \theta_i$ do not introduce any biases. Below, we tests 
the convergence of $\bfi{C}$ and $\partial B_0/\partial \theta_i$ and discuss
some of the subtleties and caveats of numerically calculating $\partial B_0/\partial \theta_i$ 
from the Quijote simulations. 
\ch{mention somewhere the P convergence looks good} 

To estimate $\bfi{C}$, we use 15,000 Quijote $N$-body simulations at the 
fiducial cosmology. This is a \emph{significantly} larger number of simulations 
than previous bispectrum analyses; however, we also consider a larger number
of triangle configurations --- $1898$ triangles out to $k_{\rm max} = 0.5~h/{\rm Mpc}$. 
For reference, the recent \cite{gil-marin2017} analysis used 2048 simulations 
to estimate the covariance matrix of the bispectrum with $825$ configurations. 
We, therefore, check the convergence of $\bfi{C}$ by varying $N_{\rm fid}$, 
the number of simulations used to estimate $\bfi{C}$, and examining whether 
this significantly impacts the Fisher parameter constraints. We present 
$\sigma_\theta(N_{\rm fid})/\sigma_\theta(N_{\rm fid} = 15,000)$, the ratio of 
the 1$\sigma$ Fisher constraint for $\theta = \Om$, $\Ob$, $h$, $n_s$, $\sig$, 
and $\smnu$ calculated with $N_{\rm fid}$ over the constraint calculated with 
$N_{\rm fid} = 15,000$, as a function of $N_{\rm fid}$ (Figure~\ref{fig:converge} 
left panel). The 1$\sigma$ Fisher constraints on the parameters 
vary by $<10\%$ for $N_{\rm fid} > 7000$; in fact, the constraints vary by 
$<1\%$ for $N_{\rm fid} > 14,000$. Hence, we conclude that we have a sufficient 
number of simulations to estimate the bispectrum $\bfi{C}$ and our forecasts 
are robust to the convergence of $\bfi{C}$. 

We estimate $\partial B_0/\partial \theta_i$ numerically using 13 sets of 
$N_{\rm fp} = 500$ fixed paired simulations~(Table~\ref{tab:sims}). To check 
the convergence of $\partial B_0/\partial \theta_i$ and its impact on our forecast 
we present the ratio of the 1$\sigma$ Fisher constraint for $\theta$ calculated 
using $N_{\rm fp}$ simulations over the constraint calculated with $N_{\rm fp} = 500$, 
$\sigma_\theta(N_{\rm fp})/\sigma_\theta(N_{\rm fp} = 500)$, as a function of 
$N_{\rm fp}$ (Figure~\ref{fig:converge} right panel). Unlike $\sigma_\theta(N_{\rm fid})$, 
$\sigma_\theta(N_{\rm fp})$ depend significantly on $\theta$. For instance, 
$\sigma_\theta$ for $\sig$ and $\Om$ vary by $< 10\%$ for $N_{\rm fp} > 300$ 
and $< 2\%$ for $N_{\rm fp} > 450$. $\sigma_\theta$ for the other parameter 
vary significantly more. Nonetheless, for $N_{\rm fp} > 400$ and $450$ they vary by 
$< 10$ and $5\%$, respectively. 

For $\Om$, $\Ob$, $h$, $n_s$, $\sig$, and $M_{\rm lim}$ we estimate 
$\partial B_0/\partial \theta_i$ using a centered difference approximation 
(Eq.~\ref{eq:dbkdt}). However, for $\smnu$, where we cannot have values below 
0.0 eV, we cannot estimate the derivative with the same method. If we use the 
forward difference approximation, 
\beq 
\frac{\partial \overline{B}_0}{\partial \smnu} \approx \frac{\overline{B}_0(\smnu^{\rm fid}+\delta \smnu) - \overline{B}_0(\smnu^{\rm fid})}{\delta \smnu}, 
\eeq
the error goes as $\mathcal{O}(\delta \smnu)$. Instead, we use Eq.~\ref{eq:dbkdmnu},
which provides a $\mathcal{O}(\delta \smnu^2)$ order approximation. In our 
$\partial B_0/\partial \smnu$ approximation, we use the Quijote simulations at 
$\smnu^{+}$, $\smnu^{++}$, and $\smnu^{+++}$. We compare 
$\partial \log B(k_1, k_2, k_3)/\partial M_\nu$ (right) and 
$\partial \log P(k)/\partial M_\nu$ (left), computed using Eq.~\ref{eq:dbkdmnu} (blue) 
and the forward difference approximation (green) in Figure~\ref{fig:dPBdmnu}.
We also include $P$ and $B$ derivatives approximated using only the $\smnu^{+}$ and 
$\smnu^{++}$ simulations in orange. \ch{update numbers:} The three approximations for the 
derivatives differ from one another by roughly $10\%$ with Eq.~\ref{eq:dbkdmnu} producing 
the largest estimate for both $P_0$ and $B_0$. If we use the $\smnu^{+}$ and $\smnu^{++}$ 
and forward difference derivatives instead of the Eq.~\ref{eq:dbkdmnu} for our Fisher 
forecasts, we find the following marginalized $\smnu$ constraints for $k_{\rm max} = 0.5~h/{\rm Mpc}$: 
$0.196$ and $0.294$eV for $P_0$ and  $0.0308$ and $0.0483$eV for $B_0$. These correspond 
to a $\sim20$ and $80\%$ relative increase from our forecasts in Section~\ref{sec:forecasts}. 
The forward difference derivatives have a significant impact on our forecasts; however, 
we emphasize that this is a 
$\mathcal{O}(\delta \smnu)$ approximation, unlike the other $\mathcal{O}(\delta \smnu^2)$ 
approximations. Moreover, because the discrepancies in the derivative propogate similarly
to the $P_0$ and $B_0$ constraints, the relative improvement of $B_0$ over $P_0$ remains
roughly the same. Hence we conclude that the derivatives have sufficiently converged and 
robust for our Fisher forecasts.

\begin{figure}
\begin{center}
    \includegraphics[width=0.75\textwidth]{figs/quijote_bkFij_convergence_reg_dmnu_fin_kmax0_5.pdf}
\label{fig:fij_converge}
\end{center}
\end{figure}

\begin{figure}
\begin{center}
    \includegraphics[width=0.8\textwidth]{figs/quijote_bkFisher_convergence_dmnu_fin_kmax05.pdf}
    \caption{{\bf Left}: The ratio of the 1$\sigma$ Fisher constraint for $\theta=\Om$, 
    $\Ob$, $h$, $n_s$, $\sig$, and $\smnu$ calculated using $N_{\rm fid}$ Quijote 
    simulations over the constraint calculated with all 15,000 simulations, 
    $\sigma_\theta(N_{\rm fid})/\sigma_\theta(N_{\rm fid} = 15,000)$, as a 
    function of $N_{\rm fid}$. The $N_{\rm fid}$ simulations are used to estimate 
    $\bfi{C}$ used to calculate the Fisher matrix~(Eq.~\ref{eq:fisher}). The Fisher
    parameter constraints vary by $<10$ and $1\%$ for $N_{\rm fid} > 7000$ and 
    $14,000$, respectively. 
    {\bf Right}: The ratio of the 1$\sigma$ Fisher constraint for $\theta$ 
    calculated using  $N_{\rm fp}$ simulations over the constraint calculated with 
    all 500 fixed paired simulations, $\sigma_\theta(N_{\rm fp})/\sigma_\theta(N_{\rm fp} = 500)$, 
    as a function of $N_{\rm fp}$. The $N_{\rm fp}$ fixed paired simulations are 
    used to numerically estimate $\partial B_0/\partial \theta_i$ in Eq.~\ref{eq:fisher}. 
    Although $\sigma_\theta(N_{\rm fp})/\sigma_\theta(N_{\rm fp} = 500)$ vary among 
    the parameters, for $N_{\rm fp} > 400$ and $450$ they vary by $< 10$ and $5\%$, 
    respectively. Hence, {\em we have a sufficient number of simulations to estimate 
    $\bfi{C}$ and the derivatives of the bipsectrum and our forecasts are robust to 
    their convergence.} 
    }
\label{fig:converge}
\end{center}
\end{figure}

\begin{figure}
\begin{center}
    \includegraphics[width=\textwidth]{figs/quijote_dPBdMnu_dmnu_reg.pdf} 
    \caption{Comparison of $\partial \log B(k_1, k_2, k_3)/\partial M_\nu$ (right) 
    and $\partial \log P(k)/\partial M_\nu$ (left), computed using Eq.\ref{eq:dbkdmnu} (blue), 
    excluding $\smnu^{+++}$ (orange), and the forward difference approximation (green). 
    }
\label{fig:dPBdmnu}
\end{center}
\end{figure}


\bibliographystyle{yahapj}
\bibliography{emanu} 
\end{document}
