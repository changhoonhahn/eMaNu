\documentclass[12pt, letterpaper, preprint]{aastex62}
%%% This file is generated by the Makefile.
\newcommand{\giturl}{\url{https://github.com/changhoonhahn/eMaNu}}
\newcommand{\githash}{f673b16}\newcommand{\gitdate}{2019-07-23}\newcommand{\gitauthor}{changhoonhahn}

%\usepackage[breaklinks,colorlinks, urlcolor=blue,citecolor=blue,linkcolor=blue]{hyperref}
\usepackage{color}
\usepackage{amsmath}
\usepackage{natbib}
\usepackage{ctable}
\usepackage{bm}
\usepackage[normalem]{ulem} % Added by MS for \sout -> not required for final version
\usepackage{xspace}
\usepackage{csvsimple} 

% typesetting shih
\linespread{1.08} % close to 10/13 spacing
\setlength{\parindent}{1.08\baselineskip} % Bringhurst
\setlength{\parskip}{0ex}
\let\oldbibliography\thebibliography % killin' me.
\renewcommand{\thebibliography}[1]{%
  \oldbibliography{#1}%
  \setlength{\itemsep}{0pt}%
  \setlength{\parsep}{0pt}%
  \setlength{\parskip}{0pt}%
  \setlength{\bibsep}{0ex}
  \raggedright
}
\setlength{\footnotesep}{0ex} % seriously?

% citation alias

% math shih
\newcommand{\setof}[1]{\left\{{#1}\right\}}
\newcommand{\given}{\,|\,}
\newcommand{\lss}{{\small{LSS}}\xspace}

\newcommand{\Om}{\Omega_{\rm m}} 
\newcommand{\Ob}{\Omega_{\rm b}} 
\newcommand{\OL}{\Omega_\Lambda}
\newcommand{\smnu}{M_\nu}
\newcommand{\sig}{\sigma_8} 
\newcommand{\mmin}{M_{\rm min}}
\newcommand{\BOk}{\widehat{B}_0} 
\newcommand{\hmpc}{\,h/\mathrm{Mpc}}
\newcommand{\bfi}[1]{\textbf{\textit{#1}}}
\newcommand{\parti}[1]{\frac{\partial #1}{\partial \theta_i}}
\newcommand{\partj}[1]{\frac{\partial #1}{\partial \theta_j}}
\newcommand{\mpc}{h/{\rm Mpc}}

\newcommand{\specialcell}[2][c]{%
  \begin{tabular}[#1]{@{}c@{}}#2\end{tabular}}
% text shih
\newcommand{\foreign}[1]{\textsl{#1}}
\newcommand{\etal}{\foreign{et~al.}}
\newcommand{\opcit}{\foreign{Op.~cit.}}
\newcommand{\documentname}{\textsl{Article}}
\newcommand{\equationname}{equation}
\newcommand{\bitem}{\begin{itemize}}
\newcommand{\eitem}{\end{itemize}}
\newcommand{\beq}{\begin{equation}}
\newcommand{\eeq}{\end{equation}}

%% collaborating
\definecolor{orange}{rgb}{1,0.5,0}
\newcommand{\ch}[1]{{\color{orange}{\bf CH:} #1}}

\begin{document}\sloppy\sloppypar\frenchspacing 

\title{Constraining $\smnu$ with the Bispectrum I: Breaking Parameter Degeneracies} 
\date{\texttt{DRAFT~---~\githash~---~\gitdate~---~NOT READY FOR DISTRIBUTION}}

\newcounter{affilcounter}
\author{ChangHoon Hahn}
\altaffiliation{hahn.changhoon@gmail.com}
\affil{Lawrence Berkeley National Laboratory, 1 Cyclotron Rd, Berkeley CA 94720, USA}
\affil{Berkeley Center for Cosmological Physics, University of California, Berkeley, CA 94720, USA}

\author{Francisco Villaescusa-Navarro} 
\affil{Center for Computational Astrophysics, Flatiron Institute, 162 5th Avenue, New York, NY 10010, USA} 

\author{Emanuele Castorina} 
\affil{Berkeley Center for Cosmological Physics, University of California, Berkeley, CA 94720, USA}
\affil{Lawrence Berkeley National Laboratory, 1 Cyclotron Rd, Berkeley CA 94720, USA}

\author{Roman Scoccimarro} 
\affil{Center for Cosmology and Particle Physics, Department of Physics, New York University, NY 10003, New York, USA}

\begin{abstract}
    Massive neutrinos suppress the growth of structure below their free-streaming scale and leave an 
    imprint on large-scale structure. Measuring this imprint allows us to constrain the sum of neutrino 
    masses, $\smnu$, a key parameter in particle physics beyond the Standard Model. However, degeneracies 
    among cosmological parameters, especially between $\smnu$ and $\sig$, limit the constraining power 
    of standard two-point clustering statistics. In this work, we investigate whether we can break these 
    degerancies and constrain $\smnu$ with the next higher-order correlation function --- the bispectrum. 
    We first examine the redshift-space halo bispectrum of $800$ $N$-body simulations from the HADES suite 
    and demonstrate that the bispectrum helps break the $\smnu$--$\sig$ degeneracy. Then using over 23000
    $N$-body simulations of the Quijote suite, we quantify for the first time the full information content 
    of the redshift-space halo bispectrum using a Fisher matrix forecast of $\{\Om$, $\Ob$, $h$, $n_s$, $\sig$, $\smnu\}$ 
    down to nonlinear scales. For $k_{\rm max}{=}0.5~\mpc$, the bispectrum 
    constrains $\Om$, $\Ob$, $h$, $n_s$, and $\sig$ 3.1, 4.1, 5.1, 5.9, and 3.3 times tighter than the 
    power spectrum. For $\smnu$, the bispectrum improves the 1$\sigma$ constraint from 
    0.1962 to 0.0342 eV --- 6 times tigher than the power spectrum. Even with priors from {\em Planck}, 
    the bispectrum improves $\smnu$ constraints by a factor of 2.7. Although we reserve marginalizing 
    over bias parameters to the next paper of the series, these constraints are derived for a 
    $(1~h^{-1}{\rm Gpc})^3$ box, a substantially smaller volume than upcoming surveys. Thus, our results 
    demonstrate that the bispectrum offers significant improvements over the power spectrum, especially 
    for constraining $\smnu$.  
\end{abstract}

\keywords{
cosmology: 
---
}

% --- intro ---
\section{Introduction} \label{sec:intro}
More than two decades ago, neutrino oscillation experiments discovered the lower
bound on the sum of neutrino masses ($\smnu \gtrsim 0.06$ eV) and confirmed
physics beyond the Standard Model~\citep{fukuda1998, forero2014, gonzalez-garcia2016}. 
Since then, experiments have sought to more precisely measure $\smnu$ in order
to distinguish between the `normal' and `inverted' neutrino mass hierarchy
scenarios and further reveal the physics of neutrinos. Upcoming laboratory 
experiments (\eg~double beta decay and tritium beta decay), however, will not
be sufficient to distinguish between the mass hierarchies~\citep{bonn2011, drexlin2013}.
Fortunately, complementary and more precise constraints on $\smnu$ can be
placed by measuring the effect of neutrinos on the expansion history and growth
of cosmic structure. 

In the early Universe, neutrinos are relativistic and contribute to the 
energy density of radiation. Later, as they become non-relativistic, 
they contribute to the energy density of matter. This transition affects 
the expansion history of the Universe and leaves imprints on the cosmic
microwave background~\citep[CMB;][]{lesgourgues2012, lesgourgues2014}. 
Massive neutrinos also impact the growth of structure. 
While neutrino perturbations are indistinguishable from cold dark matter (CDM)
perturbations on large scales, below their free-streaming scale, neutrinos 
do not contribute to the clustering and reduce the 
amplitude of the total matter power spectrum. They also reduce the growth 
rate of CDM perturbations at late times. This combined suppression of 
the small-scale matter power spectrum leaves measurable imprints 
on the CMB as well as large-scale structure~\citep[for further details see][]{lesgourgues2012, lesgourgues2014, gerbino2018}. 

%\todo{transition to why LSS neutrino mass measurement is important} (condensed version of paper 1) 
The tightest cosmological constraints on $\smnu$ currently come from 
combining CMB temperature and large angle polarization data from the 
\planck satellite with Baryon Acoustic Oscillation and CMB lensing: 
$\smnu < 0.13$ eV~\citep{planckcollaboration2018}. Future improvements
will likely continue to come from combining CMB data on large scales 
with clustering/lensing data on small scales and low redshifts, where 
the suppression of power by neutrinos is strongest~\citep{brinckmann2019}. 
But they will heavily rely on a better determination of $\tau$, the optical
depth of reionization since CMB experiments measure the combined quantity $A_s
e^{-2\tau}$~\citep{allison2015, liu2016, archidiacono2017}.
Major upcoming CMB experiments, however, are ground-based (\eg~CMB-S4) and 
will not directly constrain $\tau$~\citep{abazajian2016}. Meanwhile, proposed
future space-based experiments such as
LiteBIRD\footnote{http://litebird.jp/eng/} and 
LiteCOrE\footnote{http://www.core-mission.org/}, which have the greatest 
potential to precisely measure $\tau$, have yet to be confirmed. 

Despite the $\tau$ bottleneck in the near future, measuring the $\smnu$ imprint 
on the 3D clustering of galaxies provides a promising avenue for improving $\smnu$ constraints. 
Upcoming galaxy surveys such as DESI\footnote{https://www.desi.lbl.gov/}, 
PFS\footnote{https://pfs.ipmu.jp/}, EUCLID\footnote{http://sci.esa.int/euclid/}, 
and the Roman Space Telescope\footnote{https://roman.gsfc.nasa.gov/}, 
with the unprecedented cosmic volumes they will probe, 
have the potential to tightly constrain 
$\smnu$~\citep{audren2013, font-ribera2014, petracca2016, sartoris2016, boyle2018}.
Constraining $\smnu$ from 3D galaxy clustering, however, faces two major 
challenges: (1) accurate theoretical modeling beyond linear scales, for bias
tracers in redshift-space and (2) parameter degeneracies that limit the
constraining power of standard two-point clustering analyses. 

For the former, simulations have made huge strides in accurately modeling 
nonlinear structure formation with massive neutrinos~\citep[\eg][]{brandbyge2008, 
villaescusa-navarro2013, castorina2015, adamek2017, emberson2017, banerjee2018, 
villaescusa-navarro2018, villaescusa-navarro2019}. Moreover, new simulation-based
approaches to modeling such as `emulation' enable us to tractably exploit the accuracy of 
$N$-body simulations and analyze galaxy clustering on nonlinear scales beyond
traditional perturbation theory methods. Recent works have applied
these simulation-based approaches to analyze small-scale galaxy clustering with
remarkable success~\citep[\eg][]{heitmann2009, kwan2015, euclidcollaboration2018, lange2019, zhai2019, wibking2019}. 
These developments present the opportunity to significantly improve $\smnu$
constraints by unlocking the information content in nonlinear clustering, where
the impact of massive neutrinos is strongest~\citep[\eg][]{brandbyge2008,
saito2008, wong2008, saito2009, viel2010, agarwal2011, marulli2011, bird2012,
castorina2015, banerjee2016, upadhye2016}.

For the latter, parameter degeneracies such as the $\smnu$--$\sig8$
degeneracy pose serious limitations on constraining 
%\cite{villaescusa-navarro2018} recently used more than 1000 $N$-body
%simulations from the {\sc Hades} suite to examine the redshift-space matter and
%halo power spectrum.  They found that the imprint of $\smnu$ and $\sig$ on the
%redshift-space halo power spectrum are degenerate and differ by $< 1\%$. This
%$\smnu$ -- $\sig$ degeneracy poses a serious limitation on constraining 
$\smnu$ with the power spectrum~\citep{villaescusa-navarro2018}. However, 
information in the nonlinear regime cascades
from the power spectrum to higher-order statistics such as the bispectrum 
and help break these degeneracies~\citep{hahn2020}. Previous studies have already demonstrated the
potential of the bispectrum for improving cosmological parameter
constraints~\citep{sefusatti2005, sefusatti2006, chan2017, yankelevich2019,
agarwal2020}.
\cite{chudaykin2019}, in particular, included $\smnu$ in their forecast and
found that the bispectrum significantly improves constraints on $\smnu$.
However, none of these perturbation theory based forecast include the
constraining power on nonlinear scales. 

In \cite{hahn2020}, the previous paper of this series, we used 22,000 $N$-body
simulations from the \quij suite to quantify the total information content and
constraining power of the redshift-space halo bispectrum down to nonlinear scales. 
For $k_{\rm max}{=}0.5~\mpc$, we found that the bispectrum achieves $\Om$,
$\Ob$, $h$, $n_s$, and $\sig$ constraints 1.9, 2.6, 3.1, 3.6, and 2.6 times
tighter than the power spectrum. For $\smnu$, the bispectrum improved 
constraints by 5 times over the power spectrum. In this forecast, we marginalized 
over linear bias, $b_1$, and halo mass limit, $M_{\rm lim}$, parameters. We also found that the
improvements from the bispectrum are not impacted when we include quadratic 
and nonlocal bias parameters in the forecast. Nevertheless, \cite{hahn2020}
focused on the halo bispectrum. Actual constraints on $\smnu$, however, will be 
derived from the distribution of galaxies and therefore require a more 
realistic and complete galaxy bias model, which we provide in this paper.

In this work, we present the total information content and constraining power
of the {\em redshift-space galaxy bispectrum} down to $k_{\rm max}= 0.5~\hmpc$. For our galaxy
bias model, we use the halo occupation distribution (HOD) framework, which provides a
statistical prescription for populating dark matter halos with central and satellite
galaxies. The HOD model has been successful in reproducing the observed galaxy
clustering~\citep[\emph{e.g.}][]{zheng2005, leauthaud2012, tinker2013, zentner2016, vakili2019}. 
It is also the primary framework used in simulation-based clustering
analyses~\citep[\eg][]{mcclintock2018, zhai2019, lange2019, wibking2019}. 
We first construct 195,000 galaxy mock catalogs from the \quij $N$-body
simulations then use them to calculate Fisher matrix forecasts. Afterwards, we
present the constraining power of the galaxy bispectrum on $\smnu$ and other 
cosmological parameters after marginalizing over the HOD parameters. This work
is the second paper in a series that aims to demonstrate the potential for
simulation-based galaxy bispectrum analyses in constraining $\smnu$. Later in
the series, we will also present methods to tackle challenges that come with
analyzing the full galaxy bispectrum, such as data compression to reduce its
dimensionality. The series will culminate in fully simulation-based $\Pgl$ and
$\Bg$ reanalysis of SDSS-III BOSS. 

In Sections~\ref{sec:sims} and~\ref{sec:hod}, we describe the \quij $N$-body simulation 
suites and the HOD framework we use to construct galaxy mock catalogs from them. We then 
describe in Section~\ref{sec:methods}, how we measure the bispectrum and
calculate the Fisher forecasts of the cosmological parameters from the galaxy
mocks. Finally, in Section~\ref{sec:results}, we present the full information
content of the galaxy bispectrum and demonstrate how it significantly improves
the constraints on the cosmological parameters: $\Om$, $\Ob$, $h$, $n_s$,
$\sig$, and {\em especially} $\smnu$. 


% --- hades+quijote ---
\section{HADES and Quijote Simulation Suites} \label{sec:hades} 
The HADES\footnote{https://franciscovillaescusa.github.io/hades.html} and Quijote 
suites are sets of, 42000 total, $N$-body simulations run on multiple cosmologies,
including those with massive neutrinos ($\smnu > 0$ eV). In this work, 
we use a subset of the HADES and Quijote simulations. Below, we briefly describe these simulations; 
a summary of the simulations can be found in Table~\ref{tab:sims}. 
The HADES simulations start from Zel'dovich approximated initial conditions 
generated at $z=99$ using the~\cite{zennaro2017a} rescaling method and follow 
the gravitational evolution of $N_{\rm cdm}=512^3$ CDM, plus $N_{\nu}=512^3$ 
neutrino particles for $\smnu > 0$ eV cosmologies, to $z=0$. They are run using 
the {\sc GADGET-III} TreePM+SPH code~\citep{springel2005} in a periodic 
$(1~h^{-1}{\rm Gpc})^3$ box. All of the HADES simulations share the following 
cosmological parameter values, which are in good agreement with Planck 
constraints~\cite{ade2016a}: $\Om{=}0.3175, \Ob{=}0.049, \OL{=}0.6825, n_s{=}0.9624, h{=}0.6711$, 
and $k_{\rm pivot} = 0.05~h{\rm Mpc}^{-1}$. 

The HADES suite includes $N$-body simulations with degenerate massive neutrinos 
of $\smnu = $ 0.06, 0.10, and 0.15 eV. These simulations are run using the 
``particle method'', where neutrinos are described as a collisionless 
and pressureless fluid and therefore modeled as particles, same as 
CDM~\citep{brandbyge2008,viel2010}. HADES also includes simulations with massless 
neutrino and different values of $\sigma_8$ to examine the $\smnu-\sigma_8$ 
degeneracy. The $\sigma_8$ values were chosen to match either $\sigma_8^m$ or 
$\sigma_8^{c}$ --- $\sigma_8$ computed with respect to total matter 
(CDM + baryons + $\nu$) or CDM + baryons --- of the massive neutrino simulations: 
$\sigma_8 = 0.822, 0.818, 0.807$, and $0.798$. Each model has $100$ independent 
realizations and we focus on the snapshots saved at $z = 0$. Halos closely 
trace the CDM+baryon field rather than the total matter field and neutrinos 
have negligible contribution to halo masses~\citep[\emph{e.g.}][]{ichiki2012, castorina2014, loverde2014, villaescusa-navarro2014}.
Hence, dark matter halos are identified in each realization using the Friends-of-Friends 
algorithm~\cite[FoF;][]{davis1985} with linking length $b=0.2$ on the CDM + baryon
distribution. We limit the halo catalogs to halos with masses above 
$M_{\rm lim} = 3.2\times 10^{13} h^{-1}M_\odot$. We refer readers
to~\cite{villaescusa-navarro2018} for more details on the HADES simulations. 

In addition to HADES, we use simulations from the Quijote suite, a
set of 42,000 $N$-body simulations that in total contain more than 8 trillion 
($8\times10^{12}$) particles over a volume of $42000 (h^{-1}{\rm Gpc})^3$. 
These simulations were designed to quantify the information content of 
different cosmological observables using Fisher matrix forecasting 
technique~(Section~\ref{sec:forecasts}). They are therefore designed to accurately 
calculate the covariance matrices of observables and the derivatives of observables with 
respect to cosmological parameters: 
$\Om$, $\Ob$, $h$, $n_s$, $\sig$, and $\smnu$.

To calculate covariance matrices, Quijote includes $N_{\rm cov}{=}15,000$ $N$-body 
simulations at a fiducial cosmology ($\Om{=}0.3175$, $\Ob{=}0.049$, $h{=}0.6711$, 
$n_s{=}0.9624$, $\sig{=}0.834$, and $\smnu{=}0.0$ eV). It also includes sets 
of 500 $N$-body simulations run at different cosmologies where only one parameter 
is varied from the fiducial cosmology at a time for the derivatives. Along $\Om$, 
$\Ob$, $h$, $n_s$, and $\sig$, the fiducial cosmology is adjusted by either a 
small step above or below the fiducial value: 
$\{\Om^{+}, \Om^{-}, \Ob^{+}, \Ob^{-}, h^+, h^-, n_s^{+}, n_s^{-}, \sig^{+}, \sig^-\}$ . 
Along $\smnu$, because $\smnu \ge 0.0$ eV and the derivative of certain observable 
with respect to $\smnu$ is noisy, Quijote includes sets of 500 simulations for 
$\smnu = 0.1$, 0.2, and 0.4 eV. Table~\ref{tab:sims} lists the cosmologies included 
in the Quijote suite. 

The intial conditions for all Quijote simulations were generated at $z=127$ using 
2LPT for simulations with massless neutrinos and the Zel’dovich approximation for 
massive neutrinos. Like HADES, the initial conditions of simulations with massive
neutrinos take their scale-dependent growth factors/rates into account using the
\cite{zennaro2017a} method. From the initial conditions, all of the simulations 
follow the gravitational evolution of $512^3$ dark matter particles, and $512^3$ 
neutrino particles for massive neutrino models, to $z=0$ using {\sc Gadget-III}
TreePM+SPH code (same as HADES). 
\ch{Halos are then identified using the same FoF scheme and mass limit as HADES. 
For the fiducial cosmology, the halo catalogs have ${\sim}156,000$ halos 
($\bar{n} \sim 1.56 \times 10^{-4}~h^3{\rm Gpc}^{-3}$) with $\bar{n} P_0(k=0.1)\sim 3.23$.}
% The simulations run at the fiducial cosmology for covariance matrix estimation are standard $N$-body simulations. However, the rest are paired fixed simulations, which greatly reduce cosmic variance without introducing bias for a large set of statistics~\citep{angulo2016,pontzen2016,villaescusa-navarro2018}.  We confirm that the paired fixed simulations do not introduce any bias for the redshift-space halo bispectrum (the observable we consider in this paper). 
We refer readers to Villaescusa-Navarro et al. (in preparation) for further details 
on the Quijote simulations.

\begin{figure}
\begin{center}
\includegraphics[width=0.9\textwidth]{figs/haloPlk_rsd_ratio.pdf}
    \caption{Impact of $\smnu$ and $\sig$ on the redshift-space halo power 
    spectrum monopole, $P_0$, and quadrupole, $P_2$, of the HADES simulation 
    suite. The $P_\ell$ measurements at each cosmology is
    averaged over 100 $N$-body realizations. $\smnu$ and $\sig$ produce 
    almost identical effects on halo clustering on small scales 
    ($k > 0.1\hmpc$). This degeneracy can be partially broken through 
    the quadrupole; however, {\em $\smnu$ and $\sig$ produce almost 
    the same effect on two-point clustering --- within a few percent}.
    }
\label{fig:plk}
\end{center}
\end{figure}


%%%%%%%%%%%%%%%%%%%%%%%%%%%%%%%%%%%%%%%%%%
% simulation table
%%%%%%%%%%%%%%%%%%%%%%%%%%%%%%%%%%%%%%%%%%
\begin{table}
\caption{Specifications of the HADES and Quijote simulation suites.} 
\begin{center}
\begin{tabular}{ccccccccc} \toprule
    Name  &$\smnu$ & $\Omega_m$ & $\Omega_b$ & $h$ & $n_s$ & $\sigma^m_8$ & $\sigma^c_8$ & realizations \\
      &({\footnotesize eV}) & & & & & ({\footnotesize $10^{10}h^{-1}M_\odot$}) & ({\footnotesize $10^{10}h^{-1}M_\odot$}) & \\[3pt] \hline\hline
    \multicolumn{9}{c}{HADES suite} \\ \hline
    Fiducial    & 0.0   & 0.3175 & 0.049 & 0.6711 & 0.9624 & 0.833 & 0.833 & 100 \\ 
                & 0.06  & 0.3175 & 0.049 & 0.6711 & 0.9624 & 0.819 & 0.822 & 100 \\ 
                & 0.10  & 0.3175 & 0.049 & 0.6711 & 0.9624 & 0.809 & 0.815 & 100 \\ 
                & 0.15  & 0.3175 & 0.049 & 0.6711 & 0.9624 & 0.798 & 0.806 & 100 \\ 
                & 0.0   & 0.3175 & 0.049 & 0.6711 & 0.9624 & 0.822 & 0.822 & 100 \\ 
                & 0.0   & 0.3175 & 0.049 & 0.6711 & 0.9624 & 0.818 & 0.818 & 100 \\ 
                & 0.0   & 0.3175 & 0.049 & 0.6711 & 0.9624 & 0.807 & 0.807 & 100 \\ 
                & 0.0   & 0.3175 & 0.049 & 0.6711 & 0.9624 & 0.798 & 0.798 & 100 \\[3pt]
    \hline \hline
    \multicolumn{9}{c}{Qujiote suite} \\ \hline
    Fiducial 	    & 0.0         & 0.3175 & 0.049 & 0.6711 & 0.9624 & 0.834 & 0.834 & 15,000 \\ 
    $\smnu^+$       & \underline{0.1}   & 0.3175 & 0.049 & 0.6711 & 0.9624 & 0.834 & 0.834 & 500 \\ 
    $\smnu^{++}$    & \underline{0.2}   & 0.3175 & 0.049 & 0.6711 & 0.9624 & 0.834 & 0.834 & 500 \\ 
    $\smnu^{+++}$   & \underline{0.4}   & 0.3175 & 0.049 & 0.6711 & 0.9624 & 0.834 & 0.834 & 500 \\ 
    $\Omega_m^+$    & 0.0   & \underline{ 0.3275} & 0.049 & 0.6711 & 0.9624 & 0.834 & 0.834 & 500 \\ 
    $\Omega_m^-$    & 0.0   & \underline{ 0.3075} & 0.049 & 0.6711 & 0.9624 & 0.834 & 0.834 & 500 \\ 
    $\Omega_b^+$    & 0.0   & 0.3175 & \underline{0.051} & 0.6711 & 0.9624 & 0.834 & 0.834 & 500 \\ 
    $\Omega_b^-$    & 0.0   & 0.3175 & \underline{0.047} & 0.6711 & 0.9624 & 0.834 & 0.834 & 500 \\ 
    $h^+$           & 0.0   & 0.3175 & 0.049 & \underline{0.6911} & 0.9624 & 0.834 & 0.834 & 500 \\ 
    $h^-$           & 0.0   & 0.3175 & 0.049 & \underline{0.6511} & 0.9624 & 0.834 & 0.834 & 500 \\ 
    $n_s^+$         & 0.0   & 0.3175 & 0.049 & 0.6711 & \underline{0.9824} & 0.834 & 0.834 & 500 \\ 
    $n_s^-$         & 0.0   & 0.3175 & 0.049 & 0.6711 & \underline{0.9424} & 0.834 & 0.834 & 500 \\ 
    $\sigma_8^+$    & 0.0   & 0.3175 & 0.049 & 0.6711 & 0.9624 & \underline{0.849} & \underline{0.849} & 500 \\ 
    $\sigma_8^-$    & 0.0   & 0.3175 & 0.049 & 0.6711 & 0.9624 & \underline{0.819} & \underline{0.819} & 500 \\[3pt]
    \hline
\end{tabular} \label{tab:sims}
\end{center}
    {\bf \em Top}: The HADES suite includes sets of 100 $N$-body simulations with degenerate massive neutrinos 
    of $\smnu = $ 0.06, 0.10, and 0.15 eV as well as sets of simulations with $\smnu = 0.0$ eV and 
    $\sigma_8 = $ 0.822, 0.818, 0.807, and 0.798 to examine the $\smnu-\sigma_8$ degeneracy. 
    {\bf \em Bottom}: The Quijote suite includes 15,000 $N$-body simulations at the fiducial 
    cosmology to accurately estimate the covariance matrices. It also includes sets of 500 
    simulations at different cosmologies, where only one parameter is varied from the fiducial 
    value (underlined), to estimate derivatives of observables along the cosmological parameters.
\end{table}
%%%%%%%%%%%%%%%%%%%%%%%%%%%%%%%%%%%%%%%%%%


% --- bispectrum ---
\begin{figure}
\begin{center}
    \includegraphics[width=\textwidth]{figs/haloBk_shape_kmax05_rsd.pdf} 
    \caption{The redshift-space halo bispectrum, $\BOk(k_1, k_2, k_3)$, as a 
    function of triangle configuration shape for $\smnu = 0.0, 0.06, 0.10$, 
    and $0.15\,\mathrm{eV}$ (upper panels) and $\smnu=0.0$ eV, $\sig = 0.822, 0.818$, and 
    $0.807$ (lower panels). The $\BOk$ for each cosmology (each panel) is 
    averaged over 100 $N$-body realizations. The HADES simulations of the top 
    and bottom panels in the three right-most columns, have matching $\sig$ values 
    (Section~\ref{sec:hades}). We describe the triangle configuration shape 
    by the ratio of the triangle sides: $k_3/k_1$ and $k_2/k_1$. As we describe 
    schematically in the lower leftmost panel, in each panel, the upper left bin 
    contains squeezed triangles ($k_1 = k_2 \gg k_3$), the upper right bin 
    contains equilateral triangles ($k_1 = k_2 = k_3$), and the bottom center bin 
    contains folded triangles ($k_1 = 2 k_2 = 2 k_3$). We include all 1898 triangle 
    configurations with $k_1, k_2, k_3 \leq k_{\rm max} = 0.5~\mpc$ and use the
    $\BOk$ estimator in Section~\ref{sec:bk}.}
\label{fig:bk_shape}
\end{center}
\end{figure}

\begin{figure}
\begin{center}
\includegraphics[width=0.9\textwidth]{figs/haloBk_amp_kmax05_rsd.pdf}
    \caption{The redshift-space halo bispectrum, $\BOk(k_1, k_2, k_3)$, as a
    function of triangle configurations for $\smnu = 0.0, 0.06, 0.10$, 
    and $0.15\,\mathrm{eV}$ (top panel) and $\smnu = 0.0$ eV, $\sig = 0.822, 0.818, 0.807$, 
    and $0.798$ (lower panel). Each $\BOk$ is averaged over 100 $N$-body 
    realizations. We include all possible triangle configurations 
    with $k_1, k_2, k_3 \leq k_{\rm max} = 0.5~\mpc$ where we order
    the configurations by looping through $k_3$ in the inner most loop and 
    $k_1$ in the outer most loop satisfying $k_1 \geq k_2 \geq k_3$ 
    (see Appendix~\ref{sec:bk_details} for more details). In the 
    insets of the panels we zoom into triangle configurations with 
    $k_1 = 0.320$, $0.170 \leq k_2 \leq 0.302$, and 
    $0.094 \leq k_3 \leq 0.302~h/{\rm Mpc}$.}
\label{fig:bk_amp}
\end{center}
\end{figure}

%%%%%%%%%%%%%%%%%%%%%%%%%%%%%%%%%%%%%%%%%%%%%%%%%%%%%%%%%%%%%%%%%%%%%%%%
% Section   
%%%%%%%%%%%%%%%%%%%%%%%%%%%%%%%%%%%%%%%%%%%%%%%%%%%%%%%%%%%%%%%%%%%%%%%%
\section{Bispectrum} \label{sec:bk} 
We're interested in breaking parameter degeneracies that limit the constraining 
power on $\smnu$ of two-point clustering analyses using three-point clustering 
statistics --- \emph{i.e.} the bispectrum. In this section, we describe the 
bispectrum estimator used throughout the paper. We focus on the bispectrum monopole 
($\ell = 0$) and use an estimator that exploits the speed of Fast Fourier 
Transforms (FFTs). Our estimator is similar to the estimators described in 
\cite{sefusatti2005a}, \cite{scoccimarro2015}, and \cite{sefusatti2016}; we also 
follow their formalism in our description below. Although \cite{sefusatti2016} 
and \cite{scoccimarro2015} respectively describe estimators in real- and 
redshift-space, since we focus on the bispectrum monopole, we note that there 
is no difference. 

To measure the bispectrum of our halo catalogs, we begin by interpolating the halo
positions to a grid, $\delta({\bfi x})$, and Fourier transforming the grid to get 
$\delta({\bfi k})$. We use a fourth-order interpolation to get interlaced grids, 
which has advantageous anti-aliasing properties that allow unbias measurements 
up to the Nyquist frequency~\citep{hockney1981,sefusatti2016}. Then using 
$\delta({\bfi k})$ we measure the bispectrum monopole as 
\beq \label{eq:bk} 
\widehat{B}_{\ell = 0}(k_1,k_2, k_3) = \frac{1}{V_B} \int\limits_{k_1}{\rm d}^3q_1 \int\limits_{k_2}{\rm d}^3q_2 \int\limits_{k_3}{\rm d}^3q_3~\delta_{\rm D}({\bfi q_{123}})~\delta({\bfi q_1})~\delta({\bfi q_2})~\delta({\bfi q_3}) - B^{\rm SN}_{\ell = 0}.
\eeq
$\delta_{\rm D}$ is the Dirac delta function and hence $\delta_{\rm D}({\bfi q_{123}}) = \delta_{\rm D}({\bfi q_1} + {\bfi q_2} + {\bfi q_3})$ 
ensures that the ${\bfi q}_i$ triplet actually forms a closed triangle. Each of the integrals 
above represent an integral over a spherical shell in $k$-space with radius $\delta k$ 
centered at ${\bfi k}_i$: 
\beq
\int_{k_i}{\rm d}^3q \equiv \int\limits_{k_i-\delta k/2}^{k_i+\delta k/2}{\rm d}q~q^2\int {\rm d}\Omega.
\eeq
$V_B$ is a normalization factor proportional to the number of triplets ${\bfi q_1}$, 
${\bfi q_2}$, and ${\bfi q_3}$ that can be found in the triangle bin defined by 
$k_1$, $k_2$, and $k_3$ with width $\delta k$: 
\beq
V_B = \int\limits_{k_1}{\rm d}^3q_1 \int\limits_{k_2}{\rm d}^3q_2 \int\limits_{k_3}{\rm d}^3q_3~\delta_{\rm D}({\bfi q_{123}}). 
\eeq
Lastly, $B^{\rm SN}_{\ell=0}$ is the correction for the Poisson shot noise, which 
contributes due to the self-correlation of individual objects: 
\beq \label{eq:bk_sn} 
B^{\rm SN}_{\ell=0}(k_1, k_2, k_3) = \frac{1}{\bar{n}} \big(P_0(k_1) + P_0(k_2) + P_0(k_3) \big) + \frac{1}{\bar{n}^2}
\eeq
where $\bar{n}$ is the number density of objects (halos) and $P_0$ is the power 
spectrum monopole. 

In order to evaluate the integrals in Eq.~\ref{eq:bk}, we take advantage of the plane-wave 
representation of the Dirac delta function and rewrite the equation as
\begin{align} \label{eq:bk2} 
    \widehat{B}_{\ell = 0}(k_1,k_2, k_3) &= \frac{1}{V_B} \int\frac{{\rm d}^3x}{(2\pi)^3} \int\limits_{k_1}{\rm d}^3q_1 \int\limits_{k_2}{\rm d}^3q_2 \int\limits_{k_3}{\rm d}^3q_3~\delta({\bfi q_1})~\delta({\bfi q_2})~\delta({\bfi q_3})~e^{i{\bfi q_{123}}\cdot{\bfi x}} - B^{\rm SN}_{\ell = 0}\\ 
    &= \frac{1}{V_B} \int\frac{{\rm d}^3x}{(2\pi)^3}\prod\limits_{i=1}^{3} I_{k_i}({\bfi x}) - B^{\rm SN}_{\ell = 0} 
\end{align}
where 
\beq
I_{k_i}({\bfi x}) = \int\limits_k {\rm d}^3q~\delta({\bfi q})~e^{i {\bfi q}\cdot{\bfi x}}. 
\eeq
At this point, we measure $\widehat{B}_{\ell = 0}(k_1, k_2, k_3)$ by calculating the 
$I_{k_i}$s with inverse FFTs and summing over in real space. For $\widehat{B}_{\ell = 0}$
measurements throughout the paper, we use $\delta({\bfi x})$ grids with $N_{\rm grid} = 360$ 
and triangle configurations defined by $k_1, k_2, k_3$ bins of width 
$\Delta k = 3 k_f = 0.01885~h/{\rm Mpc}$, three times the fundamental mode
$k_f = 2\pi/(1000~h^{-1}{\rm Mpc})$ given the box size\footnote{The code that we use to 
evaluate $\widehat{B}_{\ell = 0}$ is publicly available at \url{https://github.com/changhoonhahn/pySpectrum}}. 

We present the redshift-space halo bispectrum of the HADES simulations measured using 
the estimator above in two ways: one that emphasizes the triangle shape dependence 
(Figure~\ref{fig:bk_shape}) and the other that emphasizes the amplitude 
(Figure~\ref{fig:bk_amp}). In Figure~\ref{fig:bk_shape}, we plot $\BOk(k_1, k_2, k_3)$ 
as a function of $k_2/k_1$ and $k_3/k_1$, which describe the triangle configuration shapes. 
In each panel, the colormap of the ($k_2/k_1$, $k_3/k_1$) bins represent the weighted 
average $\BOk$ amplitude of all triangle configurations in the bins. The upper left 
bins contain squeezed triangles ($k_1 = k_2 \gg k_3$), the upper right bins contain 
equilateral triangles ($k_1 = k_2 = k_3$), and the bottom center bins contain folded 
triangles ($k_1 = 2 k_2 = 2 k_3$) as schematically highlighted in the lower leftmost panel. 
We include all possible 1898 triangle configurations with $k_1, k_2, k_3 < k_{\rm max} = 0.5~\mpc$. 
$\BOk$ in the upper panels are HADES models with 
$(\smnu,~\sig) = (0.0~{\rm eV},~0.833)$ (fiducial), $(0.06~{\rm eV},~0.822), (0.10~{\rm eV},~0.815)$, 
and $(0.15~{\rm eV},~0.806)$. $\BOk$ in the lower panels are  HADES models with $\smnu = 0.0$ eV and 
$\sig = 0.822, 0.818$, and $0.807$. The top and bottom panels of the three right-most 
columns have matching $\sig$ values (Section~\ref{sec:hades}).

Next, in Figure~\ref{fig:bk_amp} we plot $\BOk(k_1, k_2, k_3)$ for all possible triangle 
configurations with $k_1, k_2, k_3 < k_{\rm max} = 0.5~\mpc$ where we order the 
configurations by looping through $k_3$ in the inner most loop and $k_1$ in the outer most 
loop with $k_1 \geq k_2 \geq k_3$. In the top panel, we present $\BOk$ of HADES models 
with $\smnu = 0.0, 0.06, 0.10$, and $0.15\,\mathrm{eV}$; in the lower panel, we present 
$\BOk$ of HADES models with $\smnu = 0.0$ eV and $\sig = 0.822, 0.818$, and $0.807$. We 
zoom into triangle configurations with $k_1 = 0.320$, $0.170 \leq k_2 \leq 0.302$, and 
$0.094 \leq k_3 \leq 0.302~h/{\rm Mpc}$ in the insets of the panels. For further details on the
redshift-space bispectrum, we refer to Appendix~\ref{sec:bk_details}. 
 

% --- results ---
\section{Results} \label{sec:results} 
\begin{figure}
\begin{center}
\includegraphics[width=0.8\textwidth]{figs/haloBk_dshape_kmax05_rsd.pdf} 
    \caption{The shape dependence of the $\smnu$ and $\sig$ imprint on 
    the redshift-space halo bispectrum, $\Delta \BOk/\BOk^\mathrm{fid}$. 
    We align the $\smnu{=}~0.06, 0.10$, and $0.15$ eV HADES simulations 
    in the upper panels with $\smnu{=}~0.0$~eV $\sig{=}~0.822$, $0.818$, and $0.807$ 
    simulations on the bottom such that the top and bottom panels in each column 
    have matching $\sig^{c}$, which produce mostly degenerate imprints on the 
    redshift-space power spectrum. $\BOk^{\rm fid}$ is measured from the 
    fiducial HADES simulation: $\smnu{=}~0.0$~eV and $\sig{=}~0.833$. 
    The difference between the top and bottom 
    panels highlight that $\smnu$ leaves a imprint distinct from $\sig$ on 
    elongated and isosceles triangles, bins along the bottom left and bottom 
    right edges, respectively. {\em The imprint of $\smnu$ has a distinct shape 
    dependence on the bispectrum that cannot be replicated by varying $\sig$}. 
    }
\label{fig:dbk_shape}
\end{center}
\end{figure}

\begin{figure}
\begin{center}
\includegraphics[width=0.9\textwidth]{figs/haloBk_residual_kmax05_rsd.pdf}
    \caption{The impact of $\smnu$ and $\sig$ on the redshift-space halo bispectrum,
    $\Delta \BOk/\BOk^\mathrm{fid}$, for all $1898$ triangle configurations with 
    $k_1, k_2, k_3 \leq 0.5~\mpc$. We compare $\Delta \BOk/\BOk^\mathrm{fid}$ 
    of the $\smnu = 0.06$ (top), $0.10$ (middle), and $0.15$ eV (bottom) HADES simulations 
    to $\Delta \BOk/\BOk^\mathrm{fid}$ of $\smnu{=}~0.0$ eV $\sig{=}~0.822$, $0.818$, and 
    $0.807$ simulations. The imprint of $\smnu$ on the bispectrum has a significantly different 
    amplitude than the imprint of $\sig$. For instance, $\smnu{=}~0.15\,\mathrm{eV}$ (red) 
    has a $\sim 5\%$ stronger impact on the bispectrum than $\smnu{=}~0.0$ eV  
    $\sig{=}~0.798$ (black) even though their power spectrums only differ by $< 1\%$ 
    (Figure~\ref{fig:plk}). {\em The distinct imprint of $\smnu$ on the bispectrum 
    illustrate that the bispectrum can break the degeneracy between $\smnu$ and $\sig$ 
    that degrade constraints from two-point analyses}.
    }
\label{fig:dbk_amp}
\end{center}
\end{figure}
\subsection{Breaking the $\smnu$-- $\sig$ degeneracy} \label{sec:mnusig}
One major bottleneck of constraining $\smnu$ with the power spectrum alone is the 
strong $\smnu$ -- $\sig$ degeneracy. The imprint of $\smnu$ and $\sig$ on the power 
spectrum are degenerate and for models with the same $\sig^c$, the power spectrum
only differ by $< 1\%$ (see Figure~\ref{fig:plk} and~\citealt{villaescusa-navarro2018}). 
The HADES suite, which has simulations with $\smnu = 0.0, 0.06, 0.10$, 
and $0.15$ eV as well as $\smnu = 0.0$ eV simulations with matching $\sig^{c}$, 
%--- $\sig{=}~0.822$, $0.818$, and $0.807$ , 
provides an ideal set of simulations to separate the impact of $\smnu$ and 
examine the degeneracy between $\smnu$ and $\sig$ (Section~\ref{sec:hades} and Table~\ref{tab:sims}). 
We measure the bispectrum of the HADES simulations (Figure~\ref{fig:bk_shape} and~\ref{fig:bk_amp}), 
and present how the bispectrum can significantly improve $\smnu$ constraints 
by breaking the $\smnu$ -- $\sig$ degeneracy. 

We begin by examining the triangle shape dependent imprint of $\smnu$ on the 
redshift-space halo bispectrum versus $\sig$ alone. In Figure~\ref{fig:dbk_shape}, 
we present the fractional residual, $(\Delta \BOk = \BOk - \BOk^{\rm fid})/\BOk^{\rm fid}$,
as a function of $k_2/k_1$ and $k_3/k_1$ for $\smnu=0.06, 0.10$, and $0.15$~eV 
in the upper panels and 0.0~eV $\sig{=}~0.822$, $0.818$, and $0.807$ in the 
bottom panels. The simulations in the top and bottom panels of each column 
have matching $\sig^{c}$. Overall, as $\smnu$ increases, the amplitude of the 
bispectrum increases for all triangle shapes (top panels). This increase is due to halo 
bias~\citep[][see also Figure~\ref{fig:plk}]{villaescusa-navarro2018}. We 
impose a fixed $M_{\rm lim}$ on our halos so lower values of $\sig$ translate 
to a larger halo bias, which boosts the amplitude of the bispectrum. Within the 
overall increase in amplitude, however, there is a significant triangle dependence. 
Equilateral triangles (upper left) have the largest increase. For $\smnu=0.15$ eV, 
the bispectrum is $\sim 15\%$ higher than $\BOk^{\rm fid}$ for equilateral triangles 
while only $\sim 8\%$ higher for folded triangles (lower center). The noticeable 
difference in $\Delta \BOk/\BOk^{\rm fid}$ between equilateral and squeezed 
triangles (upper left) is roughly consistent with the comparison in Figure 7 
of~\cite{ruggeri2018}. They, however, fix $A_s$ in their simulations and 
measure the real-space halo bispectrum so we refrain from any detailed 
comparisons. 

As $\sig$ increases with $\smnu = 0.0$ eV, the bispectrum also increases 
overall for all triangle shapes (bottom panels). However, the comparison of the
top and bottom panels in each column reveals significant differences in 
$\Delta \BOk/\BOk^{\rm fid}$ for $\smnu$ versus $\sig$ alone. Between 
$\smnu=0.15$ eV and \{$0.0$ eV, $\sig = 0.807$\} cosmologies, there is an overall 
$\gtrsim 5\%$ difference in the bispectrum. In addition, the shape dependence 
of the $\Delta \BOk/\BOk^{\rm fid}$ increase is different for $\smnu$ than $\sig$. 
This is particularly clear in the differences between $0.1$ eV (top center panel) 
and $\{$0.0 eV, $\sig=0.807\}$ 
(bottom right panel): near equilateral triangles in the two panels have similar 
$\Delta \BOk/\BOk^{\rm fid}$ while triangle shapes near the lower left edge from 
the squeezed to folded triangles have significantly different $\Delta \BOk/\BOk^{\rm fid}$. 
Hence, $\smnu$ leaves an imprint on the bispectrum with a distinct triangle 
shape dependence than $\sig$ alone. In other words, the triangle shape dependent 
imprint of $\smnu$ on the bispectrum cannot be replicated by varying $\sig$ --- 
unlike the power spectrum. 

We next examine the amplitude of the $\smnu$ imprint on the redshift-space halo 
bispectrum versus $\sig$ alone for all triangle configurations. We present 
$\Delta \BOk/\BOk^{\rm fid}$ for all $1898$ possible triangle configurations 
with $k_1, k_2, k_3 < k_{\rm max} = 0.5~h/{\rm Mpc}$ in Figure~\ref{fig:dbk_amp}. 
We compare $\Delta \BOk/\BOk^{\rm fid}$ of the $\smnu = 0.06, 0.10$, and 
$0.15$ eV HADES models to the $\Delta \BOk/\BOk^\mathrm{fid}$ of 
$\smnu{=}~0.0$ eV $\sig{=}~0.822$, $0.818$, and $0.807$ models in the
top, middle, and bottom panels, respectively. The comparison confirms the 
difference in overall amplitude of varying $\smnu$ and $\sig$ (Figure~\ref{fig:dbk_shape}). 
For instance, $\smnu{=}~0.15\,\mathrm{eV}$ (red) has a $\sim 5\%$ stronger 
impact on the bispectrum than $\smnu{=}~0.0$ eV $\sig{=}~0.798$ (black) 
even though their power spectrums differ by $< 1\%$ (Figure~\ref{fig:plk}).

The comparison in the panels of Figure~\ref{fig:dbk_amp} confirms the difference 
in the configuration dependence in $\Delta \BOk/\BOk^\mathrm{fid}$ between $\smnu$ 
versus $\sig$. The triangle configurations are ordered by looping through $k_3$ 
in the inner most loop and $k_1$ in the outer most loop such that $k_1 \leq k_2 \leq k_3$. 
In this ordering, $k_1$ increases from left to right. $\Delta \BOk/\BOk^\mathrm{fid}$ 
of $\smnu$ expectedly increases with $k_1$: for small $k_1$ (on large scales), 
neutrinos behave like CDM and therefore the impact is reduced. However, 
$\Delta \BOk/\BOk^\mathrm{fid}$ of $\smnu$ has a smaller $k_1$ dependence than 
$\Delta \BOk/\BOk^\mathrm{fid}$ of $\sig$. The distinct imprint of $\smnu$ on the redshift-space 
halo bispectrum illustrates that the bispectrum can break the degeneracy between 
$\smnu$ and $\sig$. Therefore, by including the bispectrum, we can more precisely
constrain $\smnu$ than with the power spectrum. 
 
\begin{figure}
\begin{center}
    \includegraphics[width=\textwidth]{figs/quijote_bkCov_kmax05.png} 
    \caption{Covariance and correlation matrices of the redshift-space halo bispectrum 
    estimated using $N_{\rm cov} = 15,000$ realizations of the Qujiote simulation suite at 
    the fiducial cosmology: $\Om{=}0.3175$, $\Ob{=}0.049$, $h{=}0.6711$, $n_s{=}0.9624$, $\sig{=}0.834$, 
    and $\smnu{=}0.0$ eV. We include all possible triangle configurations with 
    $k_1, k_2, k_3 \leq k_{\rm max} = 0.5~h/{\rm Mpc}$ and order the configurations 
    (bins) in the same way as Figures~\ref{fig:bk_amp} and~\ref{fig:dbk_amp}. We 
    use the covariance matrix above for the Fisher matrix forecasts presented in 
    Section~\ref{sec:forecasts}. 
    }
\label{fig:bk_cov}
\end{center}
\end{figure}

\begin{figure}
\begin{center}
    \includegraphics[width=0.8\textwidth]{figs/quijote_dlogBdthetas_reg_fin.pdf} 
    \caption{Derivatives of the redshift-space halo bispectrum, ${\rm d}\log B_0 / {\rm d} \theta$ 
    with respect to $\Om$, $\Ob$, $h$, $n_s$, $\sig$, and $\smnu$ as a function 
    of all $1898$ triangle configurations with $k_1, k_2, k_3 \leq 0.5~\mpc$ 
    (top to bottom panels). We estimate the derivatives at the fiducial parameters 
    using $N_{\rm deriv} = 1,500$ $N$-body realizations from the Quijote suite. 
    The configurations above are ordered in the same way as Figures~\ref{fig:bk_amp} 
    and~\ref{fig:dbk_amp} (Appendix~\ref{sec:bk_details}). 
    \edt{
        The shaded region marks configurations with $k_1 = 72k_f$; the darker shaded 
        region marks configurations with $k_1 = 72 k_f$ and $k_2 = 69 k_f$. 
    }
    By using $N$-body simulations for the derivatives, we rely on fewer assumptions 
    and approximations than analytic methods (\emph{i.e.} perturbation theory). 
    \emph{Furthermore, with their accuracy on small scales, these derivatives enable 
    us to quantify, for the first time, the information content of the bispectrum in 
    the nonlinear regime.}
    }
\label{fig:bk_deriv}
\end{center}
\end{figure}

\subsection{$\smnu$ and other Cosmological Parameter Forecasts} \label{sec:forecasts}
We demonstrate in the previous section with the HADES simulations, that 
the bispectrum helps break the $\smnu$--$\sig$ degeneracy, a major 
challenge in precisely constraining $\smnu$ with the power spectrum. 
While this establishes the bispectrum as a promising probe for $\smnu$, 
we are ultimately interested in determining the constraining power of the 
bispectrum for an analysis that include cosmological parameters beyond 
$\smnu$ and $\sig$--- \emph{i.e.} $\Om$, $\Ob$, $h$, and $n_s$. The Quijote 
simulation suite is \emph{specifically} designed to answer this question
through Fisher matrix forecast.

First, the Quijote suite includes $N_{\rm cov} = 15,000$ $N$-body realizations run at a 
fiducial cosmology: $\smnu{=}0.0$eV, $\Om{=}0.3175, \Ob{=}0.049, 
n_s{=}0.9624, h{=}0.6711$, and $\sig{=}0.834$ (see Table~\ref{tab:sims}). 
This allows us to robustly estimate the $1898\times1898$ covariance matrix 
$\bfi{C}$ of the {\em full} bispectrum (Figure~\ref{fig:bk_cov}). Second, the 
Quijote suite includes $500$ $N$-body realizations evaluated at $13$ different 
cosmologies, each a small step away from the fiducial cosmology parameter 
values along one parameter (Section~\ref{sec:hades} and~Table~\ref{tab:sims}). 
We apply redshift-space distortions along 3 different directions for 
these $500$ realizations, which then effectively gives us $N_{\rm deriv.}{=}1,500$ 
realizations. These simulations allow us to precisely estimate the derivatives 
of the bispectrum with respect to each of the cosmological parameters. 

\begin{figure}
\begin{center}
    \includegraphics[width=\textwidth]{figs/quijote_p02bkFisher_bMmin_dmnu_fin_kmax0_50_reg.pdf} 
    \caption{Fisher matrix constraints for $\smnu$ and other cosmological parameters 
        for the redshift-space halo bispectrum monopole (orange). We include Fisher
        parameter constraints for the redshift-space halo power spectrum monopole and 
        quadrupole in blue for comparison. The contours mark the $68\%$ and $95\%$ 
        confidence intervals.  We set $k_{\rm max} = 0.5~h/{\rm Mpc}$ for both power 
        spectrum and bispectrum. We include in our forecasts $b'$ and $M_{\rm min}$, a 
        free amplitude scaling factor and halo mass limit, respectively. They serve as 
        a simplistic bias model and we marginalize over them so that our constraints do 
        not include extra constraining power from the difference in bias/number density 
        in the different Quijote cosmologies. The bispectrum {\em substantially} 
        improves constraints on all of the cosmological parameters over the power 
        spectrum. Constraints on $\Om$, $\Ob$, $h$, $n_s$, and $\sig$ improve by
        factors of 1.9, 2.6, 3.1, 3.6, and 2.6, respectively. {\em For $\smnu$, the
        bispectrum improves $\sigma_{\smnu}$ from 0.2968 to 0.0572 eV --- over a
        factor of ${\sim}5$ improvement over the power spectrum}.}
\label{fig:bk_fish_05}
\end{center}
\end{figure}

Since their introduction to cosmology over two decades ago, Fisher information 
matrices have been ubiquitously used to forecast the constraining power of future 
experiments~\citep[\emph{e.g.}][]{jungman1996,tegmark1997,dodelson2003,heavens2009,verde2010}. 
Defined as 
\beq 
F_{ij} = - \bigg \langle \frac{\partial^2 \mathrm{ln} \mathcal{L}}{\partial \theta_i \partial \theta_j} \bigg \rangle,
\eeq
where $\mathcal{L}$ is the likelihood, the Fisher matrix for the bispectrum can 
be written as 
\beq \label{eq:fullfish} 
F_{ij} = \frac{1}{2}~\mathrm{Tr} \Bigg[\bfi{C}^{-1}\parti{\bfi{C}}\bfi{C}^{-1}\partj{\bfi{C}} + \bfi{C}^{-1} \left(\parti{B_0}\partj{B_0}^T + \parti{B_0}^T \partj{B_0} \right)\Bigg].
\eeq
Since we assume that the $B_0$ likelihood is Gaussian, including the first 
term in Eq.~\ref{eq:fullfish} runs the risk of incorrectly including information 
from the covariance already included in the mean~\citep{carron2013}. We, therefore,
conservatively neglect the first term and calculate the Fisher matrix as, 
\beq \label{eq:fisher}
F_{ij} = \frac{1}{2}~\mathrm{Tr} \Bigg[\bfi{C}^{-1} \left(\parti{B_0}\partj{B_0}^T + \parti{B_0}^T \partj{B_0} \right)\Bigg],
\eeq
directly with $\bfi{C}$ and $\partial B_0/\partial \theta_i$ along each cosmological 
parameter from the Quijote simulations. 

For $\Om$, $\Ob$, $h$, $n_s$, and $\sig$, we estimate 
\beq \label{eq:dbkdt} 
\frac{\partial B_0}{\partial \theta_i} \approx \frac{B_0(\theta_i^{+})-B_0(\theta_i^{-})}{\theta_i^+ - \theta_i^-}, 
\eeq
where $B_0(\theta_i^{+})$ and $B_0(\theta_i^{-})$ are the average bispectrum of the 
$1,500$ realizations at $\theta_i^{+}$ and $\theta_i^{-}$, 
respectively. Meanwhile, for $\smnu$, where the fiducial value is 0.0 eV and we 
cannot have negative $\smnu$, we use the Quijote simulations at $\smnu^+$, 
$\smnu^{++}$, 
$\smnu^{+++}=0.1, 0.2, 0.4$ eV (Table~\ref{tab:sims}) to estimate 
\beq \label{eq:dbkdmnu} 
\frac{\partial B_0}{\partial \smnu} \approx \frac{-21 B_0(\theta_{\rm fid}^{\rm ZA}) + 
32 B_0(\smnu^{+}) - 12 B_0(\smnu^{++}) + B_0(\smnu^{+++})}{1.2}, 
\eeq
which provides a $\mathcal{O}(\delta \smnu^2)$ order approximation. 
Since the simulations at $\smnu^+$, $\smnu^{++}$, and $\smnu^{+++}$ are generated 
from Zel'dovich initial conditions, we use simulations at the fiducial cosmology 
also generated from Zel'dovich initial conditions ($\theta_{\rm fid}^{\rm ZA}$). 
By using these $N$-body simulations, instead of analytic methods (\emph{e.g.} perturbation theory), 
we exploit the accuracy of the simulations in the nonlinear regime and rely on fewer 
assumptions and approximations. In fact, these $N$-body simulation estimated 
derivatives are key in enabling us to quantify, for the first time, the total information 
content of the redshift-space bispectrum in the non-linear regime. We present the 
bispectrum derivatives for all triangle configurations in Figure~\ref{fig:bk_deriv} and  
discuss subtleties of the derivatives and tests of convergence and stability 
in Appendix~\ref{sec:numerical}. 


We present the constraints on $\smnu$ and other cosmological parameters 
$\{\Om, \Ob, h, n_s, \sig\}$ derived from the redshift-space halo bispectrum 
Fisher matrix for $k_{\rm max} = 0.5~h/{\rm Mpc}$ in Figure~\ref{fig:bk_fish_05}. 
We include Fisher constraints for the redshift-space halo power spectrum 
monopole and quadrupole with the same $k_{\rm max}$ for comparison (blue). 
The shaded contours mark the $68\%$ and $95\%$ confidence intervals. We include 
in our Fisher constraints the following nuisance parameters: $b'$ and $M_{\rm min}$. 
$b'$ is a scaling factor on the bispectrum amplitude, analogous to linear bias. 
And $M_{\rm min}$ is the halo mass limit, which we choose as a nuisance
parameter to address the difference in the number densities among the Quijote
cosmologies, which impacts the derivatives $\partial B_0/\partial \theta_i$. 
For instance, the $\sig^{+}$ and $\sig^{-}$ cosmologies have halo 
$\bar{n} = 1.586\times10^{-4}$ and $1.528 \times 10^{-4}~h^{3}{\rm Mpc}^{-3}$. 
These parameters serve as a simplistic bias model and by marginalizing 
over them we aim to ensure that our Fisher constraints do not include extra 
constraining power from the difference in bias or number density. $b'$ is a 
multiplicative factor so $\partial B_0/\partial b' = B_0$. 
$\partial B_0/\partial M_{\rm min}$, we estimate numerically using 
$B_0$ evaluated at $M^{+}_{\rm min}{=}3.3\times10^{13}h^{-1}M_\odot$ 
and $M^{-}_{\rm min}{=}3.1\times10^{13}h^{-1}M_\odot$ with all other parameters 
set to the fiducial value. 
\edt{
    In the power spectrum constraints that we include for comparison, we also 
    include $b'$ and $M_{\rm min}$, where $b'$ in this case is 
    a scaling factor on the power spectrum amplitude. 
} 

{\em The bispectrum substantially improves constraints on all parameters 
over the power spectrum.} For $k_{\rm max} = 0.5~h/{\rm Mpc}$, the 
bispectrum tightens the marginalized $1\sigma$ constraints, $\sigma_\theta$, of $\Om$, 
$\Ob$, $h$, $n_s$, and $\sig$ by factors of $\sim$1.9, 2.6, 3.1, 3.6, 2.6 over 
the power spectrum. {\em For $\smnu$, the bispectrum improves the constraint 
from $\sigma_{\smnu}{=}~0.2968$ to 0.0572 eV --- over a factor of 5 improvement 
over the power spectrum}. This $\sigma_{\smnu}{=}~0.0572$~eV constraint is from 
the {\em bispectrum alone} and only for a $1h^{-1}{\rm Gpc}$ box. For a larger 
volume, $V$, $\sigma_\theta$ scales roughly as $\propto1/\sqrt{V}$. 
\edt{
    The $\bar{n}$ of our halo catalogs ($\sim 1.56 \times 10^{-4}~h^3{\rm Mpc}^{-3}$) 
    is also significantly lower than, for instance, $\bar{n}$ of the SDSS-III BOSS 
    LOWZ + CMASS sample~\citep[$\sim 3 \times 10^{-4}~h^3{\rm Mpc}^{-3}$;][]{alam2015}. 
    A higher number density reduces the shot noise term and thus would
    further improve the constraining power of the bispectrum (see Eq.~\ref{eq:bk_sn}). 
} 
We list the precise marginalized Fisher parameter constraints of both cosmological 
and nuisance parameters for $P_\ell$ and $B_0$ in Table~\ref{tab:forecast}. 

Even below $k_{\rm max} < 0.5~h/{\rm Mpc}$, the bispectrum significantly 
improves cosmological parameter constraints. We compare $\sigma_\theta$ 
of $\Om$, $\Ob$, $h$, $n_s$, $\sig$, and $\smnu$ as a function of $k_{\rm max}$ 
for $B_0$ (orange) and $P_{\ell = 0,2}$ (blue) in Figure~\ref{fig:fish_kmax}. We only 
include the $k_{\rm max}$ range where the Fisher forecast is well defined --- 
\emph{i.e.} more data bins than the number of parameters: $k_{\rm max} > 5~k_{\rm f} \approx 0.03~\mpc$ 
for $P_\ell$ and $k_{\rm max} > 12~k_{\rm f} \approx 0.075~h/{\rm Mpc}$ for 
$B_0$. Figure~\ref{fig:fish_kmax} reveals that the improvement of the
bispectrum $\sigma_\theta$ over the power spectrum $\sigma_\theta$ is 
larger at higher $k_{\rm max}$. Although limited by the $k_{\rm max}$ range, 
the figure suggests that on large scales ($k_{\rm max}\lesssim 0.1~h/{\rm Mpc}$) 
$\sigma_\theta$ of $P_\ell$ crosses over $\sigma_\theta$ of $B_0$ so $P_\ell$ has more 
constraining power than $B_0$, as expected on linear scales. On slightly smaller 
scales, $k_{\rm max} = 0.2~h/{\rm Mpc}$, we find that the bispectrum improves 
$\sigma_\theta$ by factors of 
$\sim 1.3$, 1.1, 1.3, 1.5, 1.7, and 2.8 for $\Om$, $\Ob$, $h$, $n_s$, $\sig$, 
and $\smnu$ respectively. 

%%%%%%%%%%%%%%%%%%%%%%%%%%%%%%%%%%%%%%%%%%
% forecast table
%%%%%%%%%%%%%%%%%%%%%%%%%%%%%%%%%%%%%%%%%%
\begin{table}
    \caption{Marginalized Fisher parameter constraints from the redshift-space halo power 
    spectrum (top) and bispectrum (bottom) for different $k_{\rm max}$. We list 
    constraints for cosmological parameters $\smnu$, $\Omega_m$, $\Omega_b$, $h$, $n_s$, 
    and $\sig$ as well as nuisance parameters $b'$ and $M_{\rm min}$.} 
\begin{center} 
    \begin{tabular}{cccccccccc} \toprule
        & $k_{\rm max}$ & $\smnu$ & $\Omega_m$ & $\Omega_b$ & $h$ & $n_s$ & $\sig$ & $b'$ & $M_{\rm min}$ \\
        & ({\footnotesize $h/{\rm Mpc}$}) &({\footnotesize eV}) & & & & & & & ({\footnotesize $10^{13} h^{-1}M_\odot$}) \\[3pt] \hline\hline

        &     & 0.0 & 0.3175 & 0.049 & 0.6711 & 0.9624 & 0.834 & 1. & 3.2  \\ 
        $P_{\ell=0,2}$  & 0.2 & $\pm0.711$ & $\pm0.037$ & $\pm0.015$ & $\pm0.179$ & $\pm0.213$ & $\pm0.085$ & $\pm0.223$ & $\pm1.827$\\
        & 0.3 & $\pm0.509$ & $\pm0.029$ & $\pm0.013$ & $\pm0.151$ & $\pm0.167$ & $\pm0.047$ & $\pm0.117$ & $\pm0.725$\\
        & 0.4 & $\pm0.383$ & $\pm0.026$ & $\pm0.012$ & $\pm0.137$ & $\pm0.152$ & $\pm0.039$ & $\pm0.102$ & $\pm0.470$\\
        & 0.5 & $\pm{\bf 0.297}$ & $\pm0.023$ & $\pm0.012$ & $\pm0.123$ & $\pm0.130$ & $\pm0.037$ & $\pm0.096$ & $\pm0.376$\\
    {\footnotesize +Planck priors} & 0.5 &$\pm0.077$& $\pm0.012$ & $\pm0.0012$ & $\pm0.0084$ & $\pm0.0044$ & $\pm0.017$ & $\pm0.033$ & $\pm0.212$\\\hline
    $B_0$  & 0.2 & $\pm0.256$ & $\pm0.029$ & $\pm0.014$ & $\pm0.142$ & $\pm0.138$ & $\pm0.049$ & $\pm0.264$ & $\pm1.299$\\
            & 0.3 & $\pm0.130$ & $\pm0.021$ & $\pm0.008$ & $\pm0.077$ & $\pm0.074$ & $\pm0.023$ & $\pm0.143$ & $\pm0.654$\\
            & 0.4 & $\pm0.078$ & $\pm0.015$ & $\pm0.006$ & $\pm0.052$ & $\pm0.047$ & $\pm0.016$ & $\pm0.088$ & $\pm0.367$\\
            & 0.5 & $\pm{\bf 0.057}$ & $\pm0.012$ & $\pm0.004$ & $\pm0.040$ & $\pm0.036$ & $\pm0.014$ & $\pm0.070$ & $\pm0.268$\\
{\footnotesize +Planck priors} & 0.5 &$\pm0.043$& $\pm0.009$ & $\pm0.0008$ & $\pm0.0064$ & $\pm0.0043$ & $\pm0.010$ & $\pm0.068$ & $\pm0.253$\\[3pt]

%B sigmas 0.0086, 0.0008, 0.0060, 0.0043, 0.0085, 0.0297, 0.0679, 0.2549
    \hline
\end{tabular} \label{tab:forecast}
\end{center}
\end{table}
%%%%%%%%%%%%%%%%%%%%%%%%%%%%%%%%%%%%%%%%%%

%%%%%%%%%%%%%%%%%%%%%%%%%%%%%%%%%%%%%%%%%%
% sigma(kmax)  
%%%%%%%%%%%%%%%%%%%%%%%%%%%%%%%%%%%%%%%%%%
\begin{figure}
\begin{center}
    \includegraphics[width=0.9\textwidth]{figs/quijote_p02bFisher_kmax_bMmin_dmnu_fin_reg_planck.pdf} 
    \caption{Marginalized $1\sigma$ constraints, $\sigma_\theta$, of the cosmological 
    parameters $\Om$, $\Ob$, $h$, $n_s$, $\sig$, and $\smnu$ as a function 
    of $k_{\rm max}$ for the redshift-space halo bispectrum (orange) and power 
    spectrum ($\ell = 0, 2$; blue). The constraints are marginalized over the nuisance parameters 
    $b'$ and $M_{\rm min}$ in our forecast (Section~\ref{sec:forecasts}). We impose 
    $k_{\rm max} > 5~k_{\rm f}$ for $P_\ell$ and $k_{\rm max} > 12~k_{\rm f}$ 
    for $B_0$, the $k$ ranges where we have more data bins than number of parameters. 
    We also include $\sigma_\theta$ constraints with {\em Planck} priors (dotted). 
    Even below $k_{\rm max} < 0.5~h/{\rm Mpc}$, the bispectrum significantly 
    improves cosmological parameter constraints. The improvement, however, is larger 
    for higher $k_{\rm max}$. At $k_{\rm max} = 0.2~h/{\rm Mpc}$, the bispectrum 
    improves constraints on $\Om$, $\Ob$, $h$, $n_s$, $\sig$, and $\smnu$ by factors 
    of $\sim 1.3$, 1.1, 1.3, 1.5, 1.7, and 2.8 over the power spectrum. Even with 
    {\em Planck} priors, $B_0$ significantly improves $\sigma_\theta$ for $k_{\rm max} \gtrsim 0.2~\mpc$. 
    While the constraining power of $P_\ell$ saturates at $k_{\rm max} = 0.2~\mpc$,
    the constraining power of $B_0$ continues to increase out to $k_{\rm max} = 0.5~\mpc$.}
\label{fig:fish_kmax}
\end{center}
\end{figure}
%%%%%%%%%%%%%%%%%%%%%%%%%%%%%%%%%%%%%%%%%%
Our forecasts demonstrate that the bispectrum has significant constraining power 
beyond the power spectrum in the weakly nonlinear regime ($k > 0.1~\mpc$). 
This constraining power 
comes from the bispectrum breaking degeneracies among the cosmological and 
nuisance parameters. This is evident when we compare the unmarginalized 
constraints from $P_\ell$ and $B_0$: $1/\sqrt{F_{ii}}$ where $F_{ii}$ is a 
diagonal element of the Fisher matrix. For $k < 0.4~\mpc$, the unmarginalized 
constraints from $P_\ell$ are tighter than those from $B_0$. Yet, once we 
marginalize the constraints over the other parameters, the $P_\ell$ constraints 
are degraded and the $B_0$ constraints are tighter for $k > 0.1~\mpc$. The derivatives, 
$\partial B_0/\partial \theta_i$, also shed light on how $B_0$ breaks parameter 
degeneracies. The parameter degeneracies in the $P_\ell$ forecasts of Figure~\ref{fig:bk_fish_05} 
are consistent with similarities in the shape and scale dependence of 
$P_\ell$ derivatives $\partial P_0/\partial \theta$ and $\partial P_2/\partial \theta$. 
On the other hand, the $B_0$ derivatives with respect to the parameters 
have significant different scale and triangle shape dependencies (Figure~\ref{fig:bk_deriv}).

\edt{
    In Figure~\ref{fig:bk_deriv}, we mark the triangles with $k_1 = 72 k_f$ in the 
    shaded region and triangles with $k_1 = 72 k_f$ and $k_2 = 69 k_f$ in the darker 
    shaded region. In the shaded region, $k_2$ increases for triangles to the right 
    while $k_3$ and $\theta_{12}$, the angle between $k_1$ and $k_2$, increase for 
    the triangles to the right in the darker shaded region. $\partial \log B_0/\partial \smnu$
    has little scale dependence for triangles with small $\theta_{12}$ 
    ($k_1, k_2 \gg k_3$; folded or squeezed traingles). This is in contrast to the 
    $\Om$, $\Ob$, $h$, and $n_s$ derivatives, which have significant scale dependence 
    at $k < 0.2~\mpc$. For a given $k_1$ and $k_2$, $\partial \log B_0/\partial \smnu$
    decreases as $\theta_{12}$ increases. The $\Om$, $h$, and $n_s$ derivatives have 
    the opposite $\theta_{12}$ dependence. $\partial \log B_0/\partial \sig$
    decreases as $\theta_{12}$ increases and also has little scale dependence for 
    folded or squeezed triangles. However, the $\theta_{12}$ dependence of 
    $\partial \log B_0/\partial \sig$, is scale independent unlike $\partial \log B_0/\partial \smnu$,
    which has little $\theta_{12}$ dependence on large scales and stronger $\theta_{12}$ 
    dependence on small scales. The detailed differences in the shape dependence of 
    $\partial \log B_0/\partial \smnu$, $\partial \log B_0/\partial \sig$, other the 
    other derivatives, highlighted in the shaded regions, ultimately allow $B_0$ 
    to break parameter degeneracies. 
}

By exploiting the massive number of $N$-body simulations of the Quijote 
suite, we present for the first time the total information content of the 
full redshift-space bispectrum beyond the linear regime. The information content 
of the bispectrum has previously been examined using perturbation 
theory. Previous works, for instance, measure the signal-to-noise ratio (SNR) 
of the bispectrum derived from covariance matrices estimated using perturbation 
theory~\citep{sefusatti2005, sefusatti2006, chan2017}. More recently, \cite{chan2017},
used covariance matrices from %that include non-Gaussian contributions calibrated with 
$\sim 4700$ $N$-body simulations to find that the cumulative SNR of the halo bispectrum is 
$\sim 30\%$ of the SNR of the halo power spectrum at $k_{\rm max} \sim 0.1~\mpc$ 
and increases to $\sim 40\%$ at $k_{\rm max}\sim 0.35~\mpc$. While these simple 
SNR measurements cannot be easily compared to Fisher analysis~\citep{repp2015, blot2016}, 
we note that they are roughly consistent with the unmarginalized constraints, 
which loosely represent the SNRs of the derivatives. 
Also, when we measure the the halo power spectrum and bispectrum SNRs using our 
covariance matrices (Figure~\ref{fig:bk_cov}), we find a relation between the 
SNRs consistent with \cite{chan2017}. Beyond the $k$ range explored by \cite{chan2017},  
$k_{\rm max}>0.35~\mpc$, we find that the SNR of $B_0$ continues to increase 
at higher $k_{\rm max}$ in constrast to the $P_0$ SNR, which saturates at 
$k_{\rm max}\sim 0.1~\mpc$. At $k_{\rm max} = 0.75~\mpc$, the largest $k$ we 
measure the $B_0$, the SNR of $B_0$ is $\sim75\%$ of the SNR of $P_0$.

Beyond these signal-to-noise calculations, a number of previous works 
have quantified the information content of the bispectrum~\citep{scoccimarro2004, sefusatti2006, sefusatti2007, song2015, tellarini2016, yamauchi2017a, karagiannis2018, yankelevich2019, chudaykin2019, coulton2019}. 
While most of these works fix most cosmological parameters and focus soley 
on forecasting constraints of primordial non-Gaussianity and bias parameters, 
\cite{sefusatti2006}, \cite{yankelevich2019}, and \cite{chudaykin2019} provide 
bispectrum forecasts for full sets of cosmological parameters. In \cite{sefusatti2006}, 
they present likelihood analysis forecasts for $\omega_d$, $\omega_b$, $\OL$, 
$n_s$, $A_s$, $w$, $\tau$. %$\omega_d = \Omega_{\rm cdm}h^2$ and $\omega_b = \Ob h^2$ are the physical dark matter and baryon densities,  $\tau$ is the reionization optical depth, $A_s$ is the primordial amplitude of scalar fluctuations, and $w$ is dark energy equation of state parameter.  
For $\Lambda$CDM, with fixed bias parameters, and $k_{\rm max} = 0.3~\mpc$, 
they find constraints on $\Om$, $\Ob$, $h$, $n_s$, and $\sig$ from WMAP, 
$P_0$, and $B_0$ is $\sim 1.5$ times tighter than constraints from WMAP 
and $P_0$. In comparison, for 
$k_{\rm max} = 0.3~\mpc$ our $B_0$, constraints are tighter than $P_0$ constraints 
by factors of 1.4, 1.7, 2.0, 2.3, 2.0, and 3.9. Both \cite{sefusatti2006} and our 
analysis find significantly tighter constraints with the bispectrum. 
They however include the WMAP likelihood in their forecast and use 
perturbation theory models, which is limited to larger scales than used 
here~\citep{scoccimarro1998a, scoccimarro1999a, sefusatti2010, pollack2012, gil-marin2014, lazanu2016, eggemeier2019}.
%break down on small scales. In a comparison to $N$-body simulations, \cite{lazanu2016} find that perturbation theory models of the matter bispectrum deviate by $>5\%$ at $k_{\rm max} \gtrsim 0.15~\mpc$. 

\cite{yankelevich2019} present Fisher forecasts for $\Omega_{\rm cdm}$, 
$\Ob$, $h$, $n_s$, $A_s$, $w_0$, and $w_0$ for a 
Euclid-like survey~\citep{laureijs2011} in 14 non-overlapping redshift 
bins over $0.65 < z < 2.05$. They use the full redshift-space bispectrum,
rather than just the monopole, and a more sophisticated bias expansion 
than \cite{sefusatti2006} but use a perturbation theory bispectrum model, 
which consequently limit their forecast to $k_{\rm max} = 0.15~\mpc$. 
They find similar constraining power on cosmological parameters from $B$ 
alone as $P$. They also find that combining the bispectrum with the power 
spectrum only moderately improves parameter constraints because posterior 
correlations are similar for $P$ and $B$. While this seemingly conflicts 
with the results we present, there are significant differences between our 
forecasts. For instance, they forecasts the Euclid survey 
(\emph{i.e.} $z > 0.7$), while our forecasts are for $z = 0$.
They also forecast the {\em galaxy} $P$ and $B$ and marginalizes over 
56 nuisance parameters (14 $z$ bins each with 3 bias parameters and 1 
RSD parameter). They also neglect non-Gaussian contributions to the $B$ 
covariance matrix, which play a significant role on small scales~\citep{chan2017} 
and may impact the constraints.  % for $k_{\rm max} \sim 0.14~\mpc$, neglecting these contributions overestimates the SNR of the bispectrum by a factor of $\sim 3$~\citep{chan2017}. 
Despite differences, \cite{yankelevich2019} find that the constraining 
power of $B$ relative to $P_\ell$ increases for higher $k_{\rm max}$, consistent 
with our forecasts as a function of $k_{\rm max}$ (Figure~\ref{fig:fish_kmax}). 
Also, consistent with their results, for $k_{\rm max} = 0.15~\mpc$, 
we find similar posterior correlations between the $P_\ell$ and $B_0$ 
constraints. At $k_{\rm max} = 0.5~\mpc$, however, we find the posterior 
correlations are no longer similar, which contribute to the constraining
power of $B_0$~(Figure~\ref{fig:bk_fish_05}).

Finally, \cite{chudaykin2019} present power spectrum and bispectrum 
forecasts for $\omega_{\rm cdm}$, $\omega_b$, $h$, $n_s$, $A_s$, $n_s$, 
and $\smnu$ of a Euclid-like survey in 8 non-overlapping redshift bins 
over $0.5 < z < 2.1$. They use
a one-loop perturbation theory model for the redshift-space  power spectrum
multipoles ($\ell = 0, 2, 4$) and a tree-level bispectrum monopole model. 
Also, rather than imposing a $k_{\rm max}$ cutoff to restrict their 
forecasts to scales where their perturbation theory model can be trusted, 
they use a theoretical error covariance model approach from~\cite{baldauf2016}.% and extend down to $k_{\rm max} = 0.5~\mpc$. 
They find ${\sim}1.4$, 1.5, 1.2, 1.5, and 1.3 times tighter constraints 
from $P_\ell$ and $B_0$ than from $P_\ell$ alone. For $\smnu$, they find 
a factor of 1.4 improvement, from 0.038 eV to 0.028 eV. Overall, 
\cite{chudaykin2019} find more modest improvements from including $B_0$ than 
the improvements we find in our $k_{\rm max} = 0.5~\mpc$ $B_0$ constraints 
over the $P_\ell$ constraints. 

Among the differences from our forecast, most notably, \cite{chudaykin2019}  
marginalize over 64 parameters of their bias model (4 bias and 4 counterterm 
normalization parameters at each redshift bin). They also include the 
Alcock-Paczynski (AP) effect for $P_\ell$, which significantly improve the 
$P_\ell$ parameter constraints (\emph{e.g.} tightens $\smnu$ constraints by $\sim30\%$). 
They however, do not include AP effects in $B_0$. Furthermore, 
\cite{chudaykin2019} use theoretical error covariance to quantify the 
uncertainty of their perturbation theory models. Because they use the tree-level 
perturbation theory model for $B_0$, theoretical errors for $B_0$ quickly 
dominate at $k \gtrsim 0.1~\mpc$, where the one- and two-loop contribute 
significantly~\citep[\emph{e.g.}][]{lazanu2018}. Hence, their forecasts do not include 
the constraining power on nonlinear scales. Also different from our analysis,
\cite{chudaykin2019} use an Markov-Chain Monte-Carlo (MCMC) 
approach, which derives more accurate parameter constraints than our Fisher
approach. They, however, neglect non-Gaussian contributions to both the 
$P_\ell$ and $B_0$ covariance matrices and do not include the covariance 
between $P_\ell$ and $B_0$ for their joint constraints. 
\edt{
    While their theoretical error covariance, which couples different $k$-modes, 
    may partly take this into account, 
}
we find that neglecting the covariance between $P_\ell$ and $B_0$ overestimates 
constraints, tighter by ${\sim}20\%$ for $k_{\rm max} = 0.2~\mpc$. 
Nonetheless, \cite{chudaykin2019} 
and our results both find significant improves in cosmological parameter 
constraints from including $B_0$. In fact, taking the theoretical errors 
into account, the improvements from $B_0$ they find are loosely consistent 
with our results at $k_{\rm max} \sim  0.2~\mpc$. 
%to extend down to $k_{\rm max} = 0.5~\mpc$, well beyond the accuracy of their tree-level $B_0$ model. Consequently, at $k \gtrsim 0.2~\mpc$ these theoretical errors for $B_0$ quickly dominate~\citep[see Figure 1 in][]{chudaykin2019}.  Hence, their forecasts do not include the significant constraining power we find in $B_0$ on nonlinear scales. In addition, the theoretical error model they use is derived from the tree-level $B_0$ and is not validated against $N$-body simulations. They only validate their theoretical error model against the one and two-loop dark matter perturbation theory bispectra, which are only marginally more accurate than the tree-level~\citep{lazanu2018}. 

Various differences between our forecast and previous work prevent more 
thorough comparisons. However, crucial aspects of our simulation based 
approach distinguish our forecasts from other works. %Three crucial aspects, however, distinguish our forecasts from other works.
We present the first bispectrum forecasts for a full set of cosmological 
parameters using bispectrum measured entirely from $N$-body simulations.
By using the simulations, we go beyond perturbation theory models and 
accurately model the redshift-space bispectrum to the nonlinear regime. 
Furthermore, by exploiting the immense number of simulations, we 
accurately estimate the full high-dimensional covariance matrix of 
the bispectrum. With these advantages, we present the first forecast
of cosmological parameters from the bispectrum down to nonlinear scales
and demonstrate the constraining power of the bispectrum for $\smnu$. 
Below, we underline a few caveats of our forecasts.
%This allows us to go beyond perturbation theory forecasts and accurately 
%quantify the full information content of the redshift-space bispectrum out to 
%nonlinear regimes. Furthermore, we present the first forecast of the full 
%bispectrum of all 1898 triangles out to $k_{\rm max} = 0.5~\mpc$, which is 
%only made possible with the immense number ($\sim42,000$) of $N$-body 
%simulations in the Quijote suite. %Lastly, we present the first bispectrum forecast of cosmological parameters that includes neutrinos and demonstrates the constraining power of the bispectrum for $\smnu$. 

Our forecasts are derived from Fisher matrices. Such forecasts make 
the assumption that the posterior is approximately Gaussian and, as a result, 
they underestimate the constraints for posteriors that are highly 
non-elliptical or asymmetric~\citep{wolz2012}. Fisher matrices also rely 
on the stability, and in our case also convergence, of numerical derivatives. 
We examine the stability of the $B_0$ derivatives with respect to $\smnu$ 
by comparing the derivatives computed using $N$-body simulations 
at three different sets of cosmologies: (1) \{$\theta_{\rm fid}^{\rm ZA}$, $\smnu^{+}$, $\smnu^{++}$, 
$\smnu^{+++}$\} (Eq.~\ref{eq:dbkdmnu}), (2) \{$\theta_{\rm fid}^{\rm ZA}$, $\smnu^{+}$, 
$\smnu^{++}$\}, and (3) \{$\theta_{\rm fid}^{\rm ZA}$, $\smnu^{+}$ \}~(see Appendix~\ref{sec:numerical}; Figure~\ref{fig:dPBdmnu}). 
The derivatives computed using the different set of cosmologies, do not impact 
the $\Om$, $\Ob$, $h$, $n_s$, and $\sig$ constraints. They do however affect the 
$\smnu$ constraints; but because $P_\ell$ and $B_0$ derivatives are affected by 
the same factor, the relative improvement of the $B_0$ $\smnu$ constraint over 
the $P_\ell$ constraints is \emph{not} impacted. 
%$\sigma_\smnu = 0.1962$, 0.2385, and 0.3774 eV with $P_0$ and $\sigma_\smnu = 0.0342$, 0.0416, and 0.0657 eV with $B_0$. 
In addition to the stability, because we use $N$-body simulations, we test 
whether the convergence of our covariance matrix and derivatives impact 
our forecasts by varying the number of simulations 
used to estimate them: $N_{\rm cov}$ and $N_{\rm deriv}$, respectively. 
For $N_{\rm cov}$, we find $< 5\%$ variation in the Fisher matrix elements, 
$F_{ij}$, for $N_{\rm cov} > 5000$ and $< 1\%$ variation in $\sigma_\theta$
for $N_{\rm cov} > 12000$.  For $N_{\rm deriv}$, we find $< 5\%$ variation 
in the $F_{ij}$ elements and $< 5\%$ variation in $\sigma_\theta$ for 
$N_{\rm deriv} > 1200$. Since our constraints vary by $< 10\%$ for sufficient 
$N_{\rm cov}$ and $N_{\rm deriv}$, the convergence of the covariance matrix 
and derivatives do not impact our forecasts to the accuracy level of Fisher 
forecasting. We refer readers to Appendix~\ref{sec:numerical} for a more 
details on the robustness of our results to the stability of the derivatives 
and convergence. 

We argue that the constraining power of the bispectrum and its improvement 
over the power spectrum come from breaking degeneracies among the cosmological 
parameters. However, numerical noise can impact our forecasts when we invert 
the Fisher matrix. Since {\em Planck} constrain $\{\Om$, $\Ob$, $h$, $n_s$, $\sig\}$ 
tighter than either $P_\ell$ or $B_0$ alone, the elements of the {\em Planck} prior 
matrix are larger than the elements of $P_\ell$ and $B_0$ Fisher matrices. 
Including {\em Planck} priors (\emph{i.e.} adding the prior matrix to the 
Fisher matrix) increases the numerical stability of the matrix inversion. 
It also reveals whether the bispectrum still improves parameter constraints 
once we include CMB constraints. With {\em Planck} priors and $P_\ell$ to 
$k_{\rm max} = 0.5~\mpc$, we derive the following constraints: 
$\sigma_{\Om} = 0.0120$, $\sigma_{\Ob} = 0.0012$, $\sigma_h=0.0084$, 
$\sigma_{n_s}=0.0044$, and $\sigma_{\sig}=0.0169$. 
Including {\em Planck} priors expectedly tighten the constraints from $P_\ell$. 
Meanwhile, with {\em Planck} priors and $B_0$ to $k_{\rm max} = 0.5~\mpc$, we get 
$\sigma_{\Om} = 0.0090$, $\sigma_{\Ob} = 0.0009$, $\sigma_h=0.0064$, 
$\sigma_{n_s}=0.0043$, and $\sigma_{\sig}=0.0104$, 1.3, 1.4, 1.3, 1.0, and 1.6
times tighter constraints. For $\smnu$, $\sigma_{\smnu} = 0.0773$ eV for $P_\ell$ 
and $\sigma_{\smnu} = 0.0430$ eV for $B_0$, a factor of 1.8 improvement. Since we find
substantial improvements in parameter constraints with the {\em Planck} prior, 
the improvement from $B_0$ are numerically robust. Furthermore, $\sigma_\theta$ 
as a function of $k_{\rm max}$ with {\em Planck} priors reveal that while the 
constraining power of $P_\ell$ saturates at $k_{\rm max} = 0.2~\mpc$, the constraining 
power of $B_0$ continues to increase out to $k_{\rm max} = 0.5~\mpc$ 
(dotted; Figure~\ref{fig:fish_kmax}). 

Our forecasts are derived using the power spectrum and bispectrum in
\emph{periodic boxes}. We do not consider a realistic geometry or radial selection 
function of actual observations from galaxy 
surveys. A realistic selection function will smooth out the triangle 
configuration dependence and consequently degrade the constraining power 
of the bispectrum. In \cite{sefusatti2005}, for instance, they find that the 
signal-to-noise of the bispectrum is significantly reduced once survey geometry 
is included in their forecast. We also do not account for super-sample 
covariance~\citep[\emph{e.g.}][]{hamilton2006, sefusatti2006, takada2013, li2018}, 
which may also degrade constraints. Survey geometry and super-sample covariance, 
however, also degrades the signal-to-noise of their power spectrum forecasts. 
Hence, with the substantial improvement in the $\smnu$ constraints of the bispectrum, 
even with survey geometry we expect the bispectrum will significantly improve 
$\smnu$ constraints over the power spectrum.  

We include the nuisance parameter $\mmin$ in our forecasts to address the 
difference in halo bias and number densities among the Quijote cosmologies. Although 
we marginalize over $\mmin$, this may not fully account for the extra 
information from $\bar{n}$ and nonlinear bias leaking into the derivatives. 
To test this, we include extra nuisance parameters, $\{A_{\rm SN}$, 
$B_{\rm SN}$, $b_2$, $\gamma_2\}$, and examine their impact on our forecasts. 
\edt{ 
    $A_{\rm SN}$ and $B_{\rm SN}$ account for any $\bar{n}$ dependence that may 
    be introduced from the shot noise correction. They are multiplicative factors 
    of the first and second terms of Eq.~\ref{eq:bk_sn}. 
}
$b_2$ and $\gamma_2$ 
are the quadratic bias and nonlocal bias parameters~\citep{chan2012, sheth2013}
to account for information from nonlinear bias. Marginalizing over $\{b'$, 
$\mmin$, $A_{\rm SN}$, $B_{\rm SN}$, $b_2$, $\gamma_2\}$, we obtain the 
following constraints for $B_0$ with $k_{\rm max} = 0.5~\mpc$:
$\sigma_{\Om} = 0.0129$, $\sigma_{\Ob} = 0.0044$, $\sigma_h=0.0404$, 
$\sigma_{n_s}=0.0456$, $\sigma_{\sig}=0.0228$, and $\sigma_{\smnu}=0.0575$. 
While constraints on $n_s$ and $\sig$ are broadened from our fiducial 
forecasts, by $27\%$ and $60\%$, the other parameters, especially $\smnu$, 
are not significantly impacted by marginalizing over the extra nuisance 
parameters. As another test, we calculate derivatives using halo catalogs 
from Quijote $\theta^{-}$ and $\theta^{+}$ cosmologies with fixed $\bar{n}$. 
We similarly find no significant impact on the $B_0$ parameter constraints.
Forecasts using additional nuisance parameters and with fixed $\bar{n}$ 
derivatives, both support the robustness of our forecast. Yet these 
tests do not ensure that our forecast entirely marginalizes over halo bias. 

In this paper, we focus on the halo bispectrum and power spectrum. However,   
constraints on $\smnu$ will ultimately be derived from the distribution of 
galaxies. Besides the cosmological parameters, bias and nuisance parameters 
that allow us to marginalize over galaxy bias need to be incorporated to 
forecast $\smnu$ and other cosmological parameter constraints for the 
galaxy bispectrum. Although we include a \emph{naive} bias model through $b'$ 
and $M_{\rm min}$, and even $b_2$ and $\gamma_2$ in our tests, this is 
insufficient to describe how galaxies trace matter. A more realistic bias model 
such as a halo occupation distribution (HOD) model involve extra parameters 
that describe the distribution of central and satellite galaxies in 
halos~\citep[\emph{e.g.}][]{zheng2005,leauthaud2012,tinker2013,zentner2016,vakili2019}. 
We, therefore, refrain from a more exhaustive investigation of the impact of 
halo bias on our results and focus quantifying the constraining power of 
the galaxy bispectrum in the next paper of this series: Hahn et al. (in preparation). 

Marginalizing over galaxy bias parameters, will likely reduce the constraining 
power at high $k$. 
\edt{
    Improvements that come from extending to smaller scales will also be diminished 
    by the extra parameters needed to accurately model those scales. \cite{hand2017}, 
    for instance, using a 13 parameter model only find a 15-30\% improvement in 
    $f\sig$ when they extend their power spectrum multipole analysis from 
    $0.2~\mpc$ to $0.4~\mpc$. Although we focus on parameter constraints from 
    the bispectrum alone in this work, jointly analyzing the power spectrum and 
    bispectrum will improve matters in this regard. A combined analysis will help 
    constrain bias parameters, further break parameter degeneracies, 
    and improve constraints on cosmological parameters~\citep{sefusatti2006, yankelevich2019, chudaykin2019, coulton2019}.
    Furthermore, we emphasize that the constraints we present is for a $1h^{-1}{\rm Gpc}$ 
    box and $\bar{n} \sim 1.56 \times 10^{-4}~h^3{\rm Mpc}^{-3}$, a substantially 
    smaller volume and lower number density than upcoming surveys. 
}
Thus, even if the constraining 
power at high $k$ is reduced, our forecasts suggest that the bispectrum offers 
significant improvements over the power spectrum, especially for constraining $\smnu$. 
 

\section{Summary} 
\ch{talk about DESI, PFS, WFIRST} 

\section*{Acknowledgements}
It's a pleasure to thank 
    Enea Di Dio, 
    Daniel Eisenstein, 
    Simone Ferraro, 
    Shirley Ho, 
    Emmaneul Schaan, 
    Zachary Slepian, 
    David N.~Spergel, 
    and Benjamin D.~Wandelt
    for valuable discussions and comments. 

\appendix
\begin{figure}
\begin{center}
    \includegraphics[width=0.95\textwidth]{figs/quijote_B_details.pdf}
    \caption{
        {\em Top}: We highlight the triangle configurations of the bispectrum that fall within the range: 
        $k_{\rm max} <$ 0.1 (purple), 0.2 (red), 0.3 (green), 0.4 (orange), and 0.5 (blue). The number of 
        triangle configurations increase significantly at higher $k_{\rm max}$, which contributes to the 
        significant constraining power on nonlinear scales. 
        {\em Center}: We mark triangle configurations of the bispectrum with three different shapes:
        equilateral (black triangle), folded (green star), and squeezed (red cross) triangles. Equilateral 
        triangles have $k_1 = k_2 = k_3$; folded triangles have $k_1 = k_2 + k_3$; squeezed tiangles have 
        $k_1, k_2 \gg k_3$. More specifically we mark configurations with $k_1, k_2 > 2.4k_3$ as squeezed 
        triangles. 
        {\em Bottom}: Comparison of the real (orange) and redshift-space (blue) halo bispectrum for 
        fiducial Quijote simuluations. The redshift-space $\BOk$ has a significantly higher 
        overall amplitude than the real-space $\BOk$, with a significant shape and $k_{\rm max}$ 
        dependence. As a result, the redshift-space $\BOk$ has a higher signal-to-noise and more 
        constraining power than the real-space $\BOk$.  
    } 
\label{fig:bk_details}
\end{center}
\end{figure}
\section{Redshift-space Bispectrum} 
In this work, we present how the redshift-space bispectrum helps break degeneracies 
among cosmological parameters, especially between $\smnu$ and $\sig$, and constrain 
$\smnu$ as well as the other cosmological parameters. To do this, we measure the 
redshift-space bispectrum monopole, $\BOk$, of the HADES and Quijote simulations 
(Section~\ref{sec:hades}) using an FFT based estimator similar to the ones in 
\cite{sefusatti2005a}, \cite{scoccimarro2015}, and \cite{sefusatti2016} 
(see Section~\ref{sec:bk} for details). We use triangle configurations defined by 
$k_1$, $k_2$, and $k_3$ bins of width $\Delta k = 3k_f = 0.01885~h/{\rm Mpc}$. 
For $k_{\rm max} = 0.5~h/{\rm Mpc}$, our fiducial value, we have $\BOk$ for 1898 
triangle configurations. For $k_{\rm max} = 0.4$, 0.3, 0.2, and $0.1~h/{\rm Mpc}$, 
we have 1056, 428, 150, and 28 triangle configurations respectively. The number
of triangle configurations sharply increases for higher $k_{\rm max}$. This
contributes to the significant increase in constraining power as we go to increasingly 
nonlinear scales as we see Figure~\ref{fig:fish_kmax}. We highlight the triangle 
configurations at different $k_{\rm max}$ for the $\BOk$ of the Quijote simulation 
at the fiducial cosmology in the top panel of Figure~\ref{fig:bk_details}. We mark 
the configurations within $k_{\rm max} = 0.1, 0.2, 0.3, 0.4$, and $0.5~h/{\rm Mpc}$ 
in purple, red, green, orange, and blue respecitvely. 

In Figure~\ref{fig:bk_details}, as well as in Figures~\ref{fig:bk_amp}, \ref{fig:dbk_amp}, 
\ref{fig:bk_deriv}, we order the triangle configurations by looping through 
$k_3$ in the inner most loop and $k_1$ in the outer most loop such that 
$k_1 \geq k_2 \geq k_3$. This ordering is different from the ordering in~\cite{gil-marin2017}, 
which loops through $k_3$ in the inner most loop and $k_1$ in the outer most 
increasing loop but with $k_1 \leq k_2 \leq k_3$. As the top panel demonstrates, 
our ordering clearly reflects the $k_{\rm max}$ range of the configurations. 
Furthermore, the repeated configuration sequences in our ordering cycles through
all available triangle configurations for a given $k_1$. In the center panel 
of Figure~\ref{fig:bk_details}, we mark equilateral (black triangle), folded 
(green star), and squeezed (red cross) triangle configurations. Equilateral 
triangles have $k_1 = k_2 = k_3$, folded triangles have $k_1 = k_2 + k_3$, and 
squeezed tiangles have $k_1, k_2 \gg k_3$ --- $k_1, k_2 > 2.4k_3$ in Figure~\ref{fig:bk_details}. 
These shapes correspond to the three verticies of the bispectrum shape plots of 
Figures~\ref{fig:bk_shape} and \ref{fig:dbk_shape}. 

Lastly, in the bottom panel of Figure~\ref{fig:bk_details} we compare the 
redshift-space bispectrum (blue) to the real-space bispectrum of the same 
Quijote simulations at the fiducial cosmology (orange). Overall, the 
redshift-space $\BOk$ has a higher amplitude than the real-space $\BOk$ with 
a significant triangle shape dependence: equilateral triangles have the 
largest difference in amplitude while folded triangles have the smallest. 
The relative amplitude difference also depends significantly on $k_{\rm max}$ 
where the difference is larger for configurations with higher $k_1$. With 
the higher amplitude the redshift-space $\BOk$ has a higher signal-to-noise 
and more constraining power than the real-space $\BOk$.  
 
\section{Fisher Forecasts using $N$-body simulations} \label{sec:numerical}
The two key elements in calculating the Fisher matrices in our bispectrum forecasts
are the bispectrum covariance matrix ($\bfi{C}$; Figure~\ref{fig:bk_cov}) and 
the derivatives of the bispectrum along the cosmological and nuisance parameters 
($\partial B_0/\partial \theta$; Figure~\ref{fig:bk_deriv}). We compute 
both these elements directly using the $N$-body simulations of the Quijote suite 
(Section~\ref{sec:hades}). This exploits the accuracy of $N$-body simulations in 
the nonlinear regime and allows us to accurately quantify the constraining power of 
the bispectrum beyond perturbation theory models. However, we must ensure that 
both $\bfi{C}$ and $\partial B_0/\partial \theta$ have converged and that numerical 
effects do not introduce any biases that impact our results. Below, we tests 
the convergence of $\bfi{C}$ and $\partial B_0/\partial \theta$ and discuss
some of the subtleties and caveats of our $\partial B_0/\partial \theta$ 
calculations. 

We use 15,000 Quijote $N$-body simulations at the fiducial cosmology to 
estimate $\bfi{C}$. This is a \emph{significantly} larger number of simulations 
than any previous bispectrum analyses; however, we also consider 1898 triangle 
configurations out to $k_{\rm max} = 0.5~\mpc$. For reference, \cite{gil-marin2017} 
recently used 2048 simulations to estimate the covariance matrix of the 
bispectrum with $825$ configurations. To check the convergence of covariance 
matrix, we vary the number of simulations used to estimate $\bfi{C}$, 
$N_{\rm cov}$, and see whether this significantly impacts the elements 
of the Fisher matrix, $F_{ij}$, or the final marginalized Fisher parameter 
constraints, $\sigma_\theta$. In the left panel of Figure~\ref{fig:fij_converge}, 
we present the ratio between $F_{ij}(N_{\rm cov})$, $F_{ij}$ derived from 
$\bfi{C}$ calculated with $N_{\rm cov}$ simulations, and 
$F_{ij}(N_{\rm cov} = 15,000)$ for all 36 elements of the Fisher matrix. We 
shade $\pm5\%$ deviations in the ratios for reference. The $F_{ij}$ elements 
vary by $\lesssim 5\%$ for $N_{\rm cov} > 5000$ and $\lesssim 1\%$ for 
$N_{\rm cov} > 10,000$. Next, we present ratio between $\sigma_\theta(N_{\rm cov})$, 
the marginalized $1\sigma$ constraints for $\{\Om$, $\Ob$, $h$, $n_s$, $\sig$, $\smnu\}$ 
derived from $\bfi{C}$ calculated with $N_{\rm cov}$ simulations, and 
$\sigma_\theta(N_{\rm cov} = 15,000)$ in the left panel of Figure~\ref{fig:converge}. 
The constraints vary by $\lesssim 5\%$ for $N_{\rm cov} > 5000$ and $\lesssim 1\%$ 
for $N_{\rm cov} > 12000$. Hence, $N_{\rm cov} = 15,000$ is sufficient to 
accurately estimate $\bfi{C}$ and its convergence does not impact our 
forecasts.

We estimate $\partial B_0/\partial \theta$ using $N_{\rm deriv} = 1,500$ 
$N$-body simulations at 13 different cosmologies listed in Table~\ref{tab:sims} 
To check the convergence of $\partial B_0/\partial \theta$ and its impact on our 
results, we examine the ratio between $F_{ij}(N_{\rm deriv})$, the Fisher 
matrix element derived from $\partial B_0/\partial \theta$ calculated with 
$N_{\rm deriv}$ simulations, and $F_{ij}(N_{\rm deriv} = 1,500)$ for all 
36 elements of the Fisher matrix in the right panel of Figure~\ref{fig:fij_converge}.
For $N_{\rm deriv} > 1000$, $F_{ij}$ elements vary by $\lesssim 5\%$. Next, we
present the ratio between $\sigma_\theta(N_{\rm deriv})$, the marginalized 
$1\sigma$ constraints for $\{\Om$, $\Ob$, $h$, $n_s$, $\sig$, $\smnu\}$ 
derived from $\partial B_0/\partial \theta$ calculated with $N_{\rm deriv}$ 
simulations, and $\sigma_\theta(N_{\rm deriv} = 1,500)$ in the right panel 
of Figure~\ref{fig:converge}. Unlike $\sigma_\theta(N_{\rm cov})$, 
$\sigma_\theta(N_{\rm deriv})$ depend significantly on $\theta$. For instance, 
$\sigma_{\sig}$ and $\sigma_{\Om}$ vary by $\lesssim 10\%$ for $N_{\rm deriv} > 600$ 
and $\lesssim 1\%$ for $N_{\rm deriv} > 1200$. $\sigma_\theta$ for the other 
parameter vary significantly more. Nonetheless, for $N_{\rm deriv} > 800$ and 
$1200$ they vary by $\lesssim 10$ and $5\%$, respectively.

For $\Om$, $\Ob$, $h$, $n_s$, $\sig$, and also the nuisance parameter $M_{\rm lim}$ 
we estimate $\partial B_0/\partial \theta$ using a centered difference approximation 
(Eq.~\ref{eq:dbkdt}). However, for $\smnu$ we cannot have values below 0.0 eV 
and, thus, cannot estimate the derivative with the same method. If we use the 
forward difference approximation, 
\beq 
\frac{\partial \overline{B}_0}{\partial \smnu} \approx \frac{\overline{B}_0(\smnu^{\rm fid}+\delta \smnu) - \overline{B}_0(\smnu^{\rm fid})}{\delta \smnu}, 
\eeq
the error goes as $\mathcal{O}(\delta \smnu)$. Instead, we use a finite difference
approximation with the Quijote simulations at $\smnu^{+}$, $\smnu^{++}$, $\smnu^{+++}$, 
and the fiducial cosmology, a $\mathcal{O}(\delta \smnu^2)$ order approximation 
(Eq.~\ref{eq:dbkdmnu}). We can also use a finite difference approximation with 
simulations at $\smnu^{+}$, $\smnu^{++}$, and the fiducial cosmology. We examine
the stability of $\partial \log B(k_1, k_2, k_3)/\partial M_\nu$ (right) and 
$\partial \log P(k)/\partial M_\nu$ (left) by comparing the derivatives computing
using simulations at \{fiducial, $\smnu^{+}$, $\smnu^{++}$, $\smnu^{+++}$\} (blue), 
\{fiducial, $\smnu^{+}$, $\smnu^{++}$\} (orange), and \{fiducial, $\smnu^{+}$\} 
(green) in Figure~\ref{fig:dPBdmnu}. The three $\partial \log B/\partial \smnu$
approximations differ from one another by $\sim10\%$ with Eq.~\ref{eq:dbkdmnu} 
producing the largest estimate for both $P_0$ and $B_0$. If we the derivatives
from \{fiducial, $\smnu^{+}$, $\smnu^{++}$\} and \{fiducial, $\smnu^{+}$\} instead
of Eq.~\ref{eq:dbkdmnu} for our Fisher forecasts, the marginalized constraint on
$\smnu$ for $k_{\rm max} = 0.5~\mpc$ increases to $0.238$ and $0.377$eV for $P_0$ 
and $0.0416$ and $0.0657$eV for $B_0$. Compared to our $\sigma_{\smnu} = 0.0342$ eV 
$B_0$ forecast, these correspond to a $\sim20$ and $90\%$ relative increase. 
While the derivative estimated from \{fiducial, $\smnu^{+}$\} significantly impact 
the forecasts, we emphasize that this is a $\mathcal{O}(\delta \smnu)$ approximation, 
unlike the other $\mathcal{O}(\delta \smnu^2)$ approximations. In fact, the 
\{fiducial, $\smnu^{+}$\} derivatives are better $\mathcal{O}(\delta \smnu^2)$ estimates
for $\partial P/\partial \smnu$ and $\partial B/\partial \smnu$ at 0.05 eV. When we 
compare the derivative of the linear theory power spectrum, $P^{(\rm LT)}$, we find 
that $\partial P^{\rm LT}/\partial \smnu$ at 0.0 is larger than at 0.05 eV. Hence, 
the differences between the \{fiducial, $\smnu^{+}$\} derivatives and the other 
derivatives are not solely due to numerical stability. Moreover, because the 
discrepancies in the derivative propogate similarly to the $P_0$ and $B_0$ forecasts, 
the relative improvement of $B_0$ over $P_0$ remains roughly the same. Hence, we conclude 
that the derivatives have sufficiently converged and robust for our Fisher forecasts.

\begin{figure}
\begin{center}
    \includegraphics[width=0.75\textwidth]{figs/quijote_bkFij_convergence_reg_dmnu_fin_kmax0_5.pdf}
    \caption{
        {\em Left}: We examine the convergence of the Fisher matrix elements, $F_{ij}$, by varying
        the number of $N$-body simulations used to estimate the covariance matrix, $N_{\rm cov}$. 
        {\em Right}:
    }
\label{fig:fij_converge}
\end{center}
\end{figure}

\begin{figure}
\begin{center}
    \includegraphics[width=0.8\textwidth]{figs/quijote_bkFisher_convergence_reg_dmnu_fin_kmax0_5.pdf} %quijote_bkFisher_convergence_dmnu_fin_kmax05.pdf}
    \caption{{\bf Left}: The ratio of the 1$\sigma$ Fisher constraint for $\theta=\Om$, 
    $\Ob$, $h$, $n_s$, $\sig$, and $\smnu$ calculated using $N_{\rm fid}$ Quijote 
    simulations over the constraint calculated with all 15,000 simulations, 
    $\sigma_\theta(N_{\rm fid})/\sigma_\theta(N_{\rm fid} = 15,000)$, as a 
    function of $N_{\rm fid}$. The $N_{\rm fid}$ simulations are used to estimate 
    $\bfi{C}$ used to calculate the Fisher matrix~(Eq.~\ref{eq:fisher}). The Fisher
    parameter constraints vary by $<10$ and $1\%$ for $N_{\rm fid} > 7000$ and 
    $14,000$, respectively. 
    {\bf Right}: The ratio of the 1$\sigma$ Fisher constraint for $\theta$ 
    calculated using  $N_{\rm fp}$ simulations over the constraint calculated with 
    all 500 fixed paired simulations, $\sigma_\theta(N_{\rm fp})/\sigma_\theta(N_{\rm fp} = 500)$, 
    as a function of $N_{\rm fp}$. The $N_{\rm fp}$ fixed paired simulations are 
    used to numerically estimate $\partial B_0/\partial \theta_i$ in Eq.~\ref{eq:fisher}. 
    Although $\sigma_\theta(N_{\rm fp})/\sigma_\theta(N_{\rm fp} = 500)$ vary among 
    the parameters, for $N_{\rm fp} > 400$ and $450$ they vary by $< 10$ and $5\%$, 
    respectively. Hence, {\em we have a sufficient number of simulations to estimate 
    $\bfi{C}$ and the derivatives of the bipsectrum and our forecasts are robust to 
    their convergence.} 
    }
\label{fig:converge}
\end{center}
\end{figure}

\begin{figure}
\begin{center}
    \includegraphics[width=\textwidth]{figs/quijote_dPBdMnu_dmnu_reg.pdf} 
    \caption{Comparison of $\partial \log B(k_1, k_2, k_3)/\partial M_\nu$ (right) 
    and $\partial \log P(k)/\partial M_\nu$ (left), computed using Eq.\ref{eq:dbkdmnu} (blue), 
    excluding $\smnu^{+++}$ (orange), and the forward difference approximation (green). 
    }
\label{fig:dPBdmnu}
\end{center}
\end{figure}


\bibliographystyle{yahapj}
\bibliography{emanu} 
\end{document}
