\begin{figure}
\begin{center}
\includegraphics[width=0.8\textwidth]{figs/haloBk_dshape_kmax05_rsd.pdf} 
    \caption{The shape dependence of the $\smnu$ and $\sig$ imprint on 
    the redshift-space halo bispectrum, $\Delta \BOk/\BOk^\mathrm{fid}$. 
    We align the $\smnu{=}~0.06, 0.10$, and $0.15$ eV HADES simulations 
    in the upper panels with $\smnu{=}~0.0$~eV $\sig{=}~0.822$, $0.818$, and $0.807$ 
    simulations on the bottom such that the top and bottom panels in each column 
    have matching $\sig^{c}$, which produce mostly degenerate imprints on the 
    redshift-space power spectrum. $\BOk^{\rm fid}$ is measured from the 
    fiducial HADES simulation: $\smnu{=}~0.0$~eV and $\sig{=}~0.833$. 
    The difference between the top and bottom 
    panels highlight that $\smnu$ leaves a imprint distinct from $\sig$ on 
    elongated and isosceles triangles, bins along the bottom left and bottom 
    right edges, respectively. {\em The imprint of $\smnu$ has a distinct shape 
    dependence on the bispectrum that cannot be replicated by varying $\sig$}. 
    }
\label{fig:dbk_shape}
\end{center}
\end{figure}

\begin{figure}
\begin{center}
\includegraphics[width=0.9\textwidth]{figs/haloBk_residual_kmax05_rsd.pdf}
    \caption{The impact of $\smnu$ and $\sig$ on the redshift-space halo bispectrum,
    $\Delta \BOk/\BOk^\mathrm{fid}$, for all $1898$ triangle configurations with 
    $k_1, k_2, k_3 \leq 0.5~\mpc$. We compare $\Delta \BOk/\BOk^\mathrm{fid}$ 
    of the $\smnu = 0.06$ (top), $0.10$ (middle), and $0.15$ eV (bottom) HADES simulations 
    to $\Delta \BOk/\BOk^\mathrm{fid}$ of $\smnu{=}~0.0$ eV, $\sig{=}~0.822$, $0.818$, and 
    $0.807$ simulations. The imprint of $\smnu$ on the bispectrum has a significantly different 
    amplitude than the imprint of $\sig$. For instance, $\smnu{=}~0.15\,\mathrm{eV}$ (red) 
    has a $\sim 5\%$ stronger impact on the bispectrum than $\smnu{=}~0.0$ eV  
    $\sig{=}~0.798$ (black) even though their power spectra only differ by $< 1\%$ 
    (Figure~\ref{fig:plk}). {\em The distinct imprint of $\smnu$ on the bispectrum 
    illustrate that the bispectrum can break the degeneracy between $\smnu$ and $\sig$ 
    that degrade constraints from two-point analyses}.
    }
\label{fig:dbk_amp}
\end{center}
\end{figure}
\subsection{Breaking the $\smnu$-- $\sig$ degeneracy} \label{sec:mnusig}
One major bottleneck of constraining $\smnu$ with the power spectrum alone is the 
strong $\smnu$ -- $\sig$ degeneracy. The imprints of $\smnu$ and $\sig$ on the power 
spectrum are degenerate and for models with the same $\sig^c$, the power spectrum
only differ by $< 1\%$ (see Figure~\ref{fig:plk} and~\citealt{villaescusa-navarro2018}). 
The HADES suite, which has simulations with $\smnu = 0.0, 0.06, 0.10$, 
and $0.15$ eV as well as $\smnu = 0.0$ eV simulations with matching $\sig^{c}$, 
%--- $\sig{=}~0.822$, $0.818$, and $0.807$ , 
provides an ideal set of simulations to disentangle the impact of $\smnu$ and 
examine the degeneracy between $\smnu$ and $\sig$ (Section~\ref{sec:hades} and Table~\ref{tab:sims}). 
We measure the bispectrum of the HADES simulations (Figure~\ref{fig:bk_shape} and~\ref{fig:bk_amp}) 
and present how the bispectrum can significantly improve $\smnu$ constraints 
by breaking the $\smnu$ -- $\sig$ degeneracy. 

We begin by examining the triangle shape dependent imprint of $\smnu$ on the 
redshift-space halo bispectrum versus $\sig$ alone. In Figure~\ref{fig:dbk_shape}, 
we present the fractional residual, $(\Delta \BOk = \BOk - \BOk^{\rm fid})/\BOk^{\rm fid}$,
as a function of $k_2/k_1$ and $k_3/k_1$ for $\smnu=0.06, 0.10$, and $0.15$~eV 
in the upper panels and 0.0~eV $\sig{=}~0.822$, $0.818$, and $0.807$ in the 
bottom panels. The simulations in the top and bottom panels of each column 
have matching $\sig^{c}$. Overall, as $\smnu$ increases, the amplitude of the 
bispectrum increases for all triangle shapes (top panels). This increase is due to halo 
bias~\citep[][see also Figure~\ref{fig:plk}]{villaescusa-navarro2018}. Since we 
impose a fixed $M_{\rm lim}$ on our halos, lower values of $\sig$ translate 
to a larger halo bias, which boosts the amplitude of the bispectrum. Within the 
overall increase in amplitude, however, there is a significant triangle dependence. 
Equilateral triangles (upper left) have the largest increase. For $\smnu=0.15$ eV, 
the bispectrum is $\sim 15\%$ higher than $\BOk^{\rm fid}$ for equilateral triangles 
while only $\sim 8\%$ higher for folded triangles (lower center). The noticeable 
difference in $\Delta \BOk/\BOk^{\rm fid}$ between equilateral and squeezed 
triangles (upper left) is roughly consistent with the comparison in Figure 7 
of~\cite{ruggeri2018}. They, however, fix $A_s$ in their simulations and 
measure the real-space halo bispectrum so we refrain from any detailed 
comparisons. 

As $\sig$ increases with $\smnu = 0.0$ eV, the bispectrum also increases 
overall for all triangle shapes (bottom panels). However, the comparison of the
top and bottom panels in each column reveals significant differences in 
$\Delta \BOk/\BOk^{\rm fid}$ for $\smnu$ versus $\sig$ alone. Between 
$\smnu=0.15$ eV and \{$0.0$ eV, $\sig = 0.807$\} cosmologies, there is an overall 
$\gtrsim 5\%$ difference in the bispectrum. In addition, the shape dependence 
of the $\Delta \BOk/\BOk^{\rm fid}$ increase is different for $\smnu$ than $\sig$. 
This is particularly clear in the differences between $0.1$ eV (top center panel) 
and $\{$0.0 eV, $\sig=0.807\}$ 
(bottom right panel): near equilateral triangles in the two panels have similar 
$\Delta \BOk/\BOk^{\rm fid}$ while triangle shapes near the lower left edge from 
the squeezed to folded triangles have significantly different $\Delta \BOk/\BOk^{\rm fid}$. 
Hence, $\smnu$ leaves an imprint on the bispectrum with a distinct triangle 
shape dependence than $\sig$ alone. In other words, the triangle shape dependent 
imprint of $\smnu$ on the bispectrum cannot be replicated by varying $\sig$ --- 
unlike the power spectrum. 

We next examine the amplitude of the $\smnu$ imprint on the redshift-space halo 
bispectrum versus $\sig$ alone as a function of all triangle configurations. We present 
$\Delta \BOk/\BOk^{\rm fid}$ for all $1898$ possible triangle configurations 
with $k_1, k_2, k_3 < k_{\rm max} = 0.5~h/{\rm Mpc}$ in Figure~\ref{fig:dbk_amp}. 
We compare $\Delta \BOk/\BOk^{\rm fid}$ of the $\smnu = 0.06, 0.10$, and 
$0.15$ eV HADES models to the $\Delta \BOk/\BOk^\mathrm{fid}$ of 
$\smnu{=}~0.0$ eV $\sig{=}~0.822$, $0.818$, and $0.807$ models in the
top, middle, and bottom panels, respectively. The comparison confirms the 
difference in overall amplitude of varying $\smnu$ and $\sig$ (Figure~\ref{fig:dbk_shape}). 
For instance, $\smnu{=}~0.15\,\mathrm{eV}$ (red) has a $\sim 5\%$ stronger 
impact on the bispectrum than $\smnu{=}~0.0$ eV $\sig{=}~0.798$ (black) 
even though their power spectra differ by $< 1\%$ (Figure~\ref{fig:plk}).

The comparison in the panels of Figure~\ref{fig:dbk_amp} confirms the difference 
in the configuration dependence in $\Delta \BOk/\BOk^\mathrm{fid}$ between $\smnu$ 
versus $\sig$. The triangle configurations are ordered by looping through $k_3$ 
in the inner most loop and $k_1$ in the outer most loop such that $k_1 \geq k_2 \geq k_3$. 
In this ordering, $k_1$ increases from left to right (see Figure~\ref{fig:bk_details} 
and Appendix~\ref{sec:bk_details}). $\Delta \BOk/\BOk^\mathrm{fid}$ 
of $\smnu$ expectedly increases with $k_1$: for small $k_1$ (on large scales), 
neutrinos behave like CDM and therefore the impact is reduced. However, 
$\Delta \BOk/\BOk^\mathrm{fid}$ of $\smnu$ has a smaller $k_1$ dependence than 
$\Delta \BOk/\BOk^\mathrm{fid}$ of $\sig$. The distinct imprint of $\smnu$ on the redshift-space 
halo bispectrum illustrates that the bispectrum can break the degeneracy between 
$\smnu$ and $\sig$. Therefore, by including the bispectrum, we can more precisely
constrain $\smnu$ than with the power spectrum. 
