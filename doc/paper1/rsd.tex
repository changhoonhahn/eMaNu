\begin{figure}
\begin{center}
    \includegraphics[width=0.95\textwidth]{figs/quijote_B_details.pdf}
    \caption{
        {\em Top}: We highlight the triangle configurations of the bispectrum that fall within the range: 
        $k_{\rm max} <$ 0.1 (purple), 0.2 (red), 0.3 (green), 0.4 (orange), and 0.5 (blue). The number of 
        triangle configurations increase significantly at higher $k_{\rm max}$, which contributes to the 
        significant constraining power on nonlinear scales. 
        {\em Center}: We mark triangle configurations of the bispectrum with three different shapes:
        equilateral (black triangle), folded (green star), and squeezed (red cross) triangles. Equilateral 
        triangles have $k_1 = k_2 = k_3$; folded triangles have $k_1 = k_2 + k_3$; squeezed tiangles have 
        $k_1, k_2 \gg k_3$. More specifically we mark configurations with $k_1, k_2 > 2.4k_3$ as squeezed 
        triangles. 
        {\em Bottom}: Comparison of the real (orange) and redshift-space (blue) halo bispectrum for 
        fiducial Quijote simuluations. The redshift-space $\BOk$ has a significantly higher 
        overall amplitude than the real-space $\BOk$, with a significant shape and $k_{\rm max}$ 
        dependence. As a result, the redshift-space $\BOk$ has a higher signal-to-noise and more 
        constraining power than the real-space $\BOk$.  
    } 
\label{fig:bk_details}
\end{center}
\end{figure}
\section{Redshift-space Bispectrum} \label{sec:bk_details}
In this work, we present how the redshift-space bispectrum helps break degeneracies 
among cosmological parameters, especially between $\smnu$ and $\sig$, and constrain 
$\smnu$ as well as the other cosmological parameters. To do this, we measure the 
redshift-space bispectrum monopole, $\BOk$, of the HADES and Quijote simulations 
(Section~\ref{sec:hades}) using an FFT based estimator similar to the ones in 
\cite{sefusatti2005a}, \cite{scoccimarro2015}, and \cite{sefusatti2016} 
(see Section~\ref{sec:bk} for details). We use triangle configurations defined by 
$k_1$, $k_2$, and $k_3$ bins of width $\Delta k = 3k_f = 0.01885~h/{\rm Mpc}$. 
For $k_{\rm max} = 0.5~h/{\rm Mpc}$, our fiducial value, we have $\BOk$ for 1898 
triangle configurations. For $k_{\rm max} = 0.4$, 0.3, 0.2, and $0.1~h/{\rm Mpc}$, 
we have 1056, 428, 150, and 28 triangle configurations respectively. The number
of triangle configurations sharply increases for higher $k_{\rm max}$. This
contributes to the significant increase in constraining power as we go to increasingly 
nonlinear scales as we see Figure~\ref{fig:fish_kmax}. We highlight the triangle 
configurations at different $k_{\rm max}$ for the $\BOk$ of the Quijote simulation 
at the fiducial cosmology in the top panel of Figure~\ref{fig:bk_details}. We mark 
the configurations within $k_{\rm max} = 0.1, 0.2, 0.3, 0.4$, and $0.5~h/{\rm Mpc}$ 
in purple, red, green, orange, and blue respecitvely. 

In Figure~\ref{fig:bk_details}, as well as in Figures~\ref{fig:bk_amp}, \ref{fig:dbk_amp}, 
\ref{fig:bk_deriv}, we order the triangle configurations by looping through 
$k_3$ in the inner most loop and $k_1$ in the outer most loop such that 
$k_1 \geq k_2 \geq k_3$. This ordering is different from the ordering in~\cite{gil-marin2017}, 
which loops through $k_3$ in the inner most loop and $k_1$ in the outer most 
increasing loop but with $k_1 \leq k_2 \leq k_3$. As the top panel demonstrates, 
our ordering clearly reflects the $k_{\rm max}$ range of the configurations. 
Furthermore, the repeated configuration sequences in our ordering cycles through
all available triangle configurations for a given $k_1$. In the center panel 
of Figure~\ref{fig:bk_details}, we mark equilateral (black triangle), folded 
(green star), and squeezed (red cross) triangle configurations. Equilateral 
triangles have $k_1 = k_2 = k_3$, folded triangles have $k_1 = k_2 + k_3$, and 
squeezed tiangles have $k_1, k_2 \gg k_3$ --- $k_1, k_2 > 2.4k_3$ in Figure~\ref{fig:bk_details}. 
These shapes correspond to the three verticies of the bispectrum shape plots of 
Figures~\ref{fig:bk_shape} and \ref{fig:dbk_shape}. 

Lastly, in the bottom panel of Figure~\ref{fig:bk_details} we compare the 
redshift-space bispectrum (blue) to the real-space bispectrum of the same 
Quijote simulations at the fiducial cosmology (orange). Overall, the 
redshift-space $\BOk$ has a higher amplitude than the real-space $\BOk$ with 
a significant triangle shape dependence: equilateral triangles have the 
largest difference in amplitude while folded triangles have the smallest. 
The relative amplitude difference also depends significantly on $k_{\rm max}$ 
where the difference is larger for configurations with higher $k_1$. With 
the higher amplitude the redshift-space $\BOk$ has a higher signal-to-noise 
and more constraining power than the real-space $\BOk$.  
