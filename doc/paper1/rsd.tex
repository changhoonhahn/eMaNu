\begin{figure}
\begin{center}
    \includegraphics[width=0.95\textwidth]{figs/quijote_B_details.pdf}
    \caption{
        {\em Top}: We highlight the triangle configurations of the bispectrum that fall within the range: 
        $k_{\rm max} <$ 0.1 (purple), 0.2 (red), 0.3 (green), 0.4 (orange), and 0.5 (blue). The number of 
        triangle configurations increase significantly at higher $k_{\rm max}$, which contributes to the 
        significant constraining power on nonlinear scales. 
        {\em Center}: We mark triangle configurations of the bispectrum with three different shapes:
        equilateral (black triangle), folded (green star), and squeezed (red cross) triangles. Equilateral 
        triangles have $k_1 = k_2 = k_3$; folded triangles have $k_1 = k_2 + k_3$; squeezed tiangles have 
        $k_1, k_2 \gg k_3$. More specifically we mark configurations with $k_1, k_2 > 2.4k_3$ as squeezed 
        triangles. 
        {\em Bottom}: Comparison of the real (orange) and redshift-space (blue) halo bispectrum for 
        fiducial Quijote simuluations. The redshift-space $\BOk$ has a significantly higher 
        amplitude, and thus signal-to-noise, than the real-space $\BOk$.
    } 
\label{fig:real_vs_rsd}
\end{center}
\end{figure}
\section{Redshift-space Halo Bispectrum} 
Supplementary detials on the bispectrum in hopes of helping the readers get a better intuition. 
