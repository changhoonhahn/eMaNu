\begin{figure}
\begin{center}
    \includegraphics[width=\textwidth]{figs/haloBk_shape_kmax05_rsd.pdf} 
    \caption{The redshift-space halo bispectrum, $\BOk(k_1, k_2, k_3)$, as a 
    function of triangle configuration shape for $\smnu = 0.0, 0.06, 0.10$, 
    and $0.15\,\mathrm{eV}$ (upper panels) and $\smnu=0.0$ eV, $\sig = 0.822, 0.818$, and 
    $0.807$ (lower panels). The $\BOk$ for each cosmology (each panel) is 
    averaged over 100 $N$-body realizations. The HADES simulations of the top 
    and bottom panels in the three right-most columns, have matching $\sig$ values 
    (Section~\ref{sec:hades}). We describe the triangle configuration shape 
    by the ratio of the triangle sides: $k_3/k_1$ and $k_2/k_1$. As we describe 
    schematically in the lower leftmost panel, in each panel, the upper left bin 
    contains squeezed triangles ($k_1 = k_2 \gg k_3$), the upper right bin 
    contains equilateral triangles ($k_1 = k_2 = k_3$), and the bottom center bin 
    contains folded triangles ($k_1 = 2 k_2 = 2 k_3$). We include all 1898 triangle 
    configurations with $k_1, k_2, k_3 \leq k_{\rm max} = 0.5~\mpc$ and use the
    $\BOk$ estimator in Section~\ref{sec:bk}.}
\label{fig:bk_shape}
\end{center}
\end{figure}

\begin{figure}
\begin{center}
\includegraphics[width=0.9\textwidth]{figs/haloBk_amp_kmax05_rsd.pdf}
    \caption{The redshift-space halo bispectrum, $\BOk(k_1, k_2, k_3)$, as a
    function of triangle configurations for $\smnu = 0.0, 0.06, 0.10$, 
    and $0.15\,\mathrm{eV}$ (top panel) and $\smnu = 0.0$ eV, $\sig = 0.822, 0.818, 0.807$, 
    and $0.798$ (lower panel). Each $\BOk$ is averaged over 100 $N$-body 
    realizations. We include all possible triangle configurations 
    with $k_1, k_2, k_3 \leq k_{\rm max} = 0.5~\mpc$ where we order
    the configurations by looping through $k_3$ in the inner most loop and 
    $k_1$ in the outer most loop satisfying $k_1 \geq k_2 \geq k_3$ 
    (see Appendix~\ref{sec:bk_details} for more details). In the 
    insets of the panels we zoom into triangle configurations with 
    $k_1 = 0.320$, $0.170 \leq k_2 \leq 0.302$, and 
    $0.094 \leq k_3 \leq 0.302~h/{\rm Mpc}$.}
\label{fig:bk_amp}
\end{center}
\end{figure}

%%%%%%%%%%%%%%%%%%%%%%%%%%%%%%%%%%%%%%%%%%%%%%%%%%%%%%%%%%%%%%%%%%%%%%%%
% Section   
%%%%%%%%%%%%%%%%%%%%%%%%%%%%%%%%%%%%%%%%%%%%%%%%%%%%%%%%%%%%%%%%%%%%%%%%
\section{Bispectrum} \label{sec:bk} 
We're interested in breaking parameter degeneracies that limit the constraining 
power on $\smnu$ of two-point clustering analyses using three-point clustering 
statistics --- \emph{i.e.} the bispectrum. In this section, we describe the 
bispectrum estimator used throughout the paper. We focus on the bispectrum monopole 
($\ell = 0$) and use an estimator that exploits the speed of Fast Fourier 
Transforms (FFTs). Our estimator is similar to the estimators described in 
\cite{sefusatti2005a}, \cite{scoccimarro2015}, and \cite{sefusatti2016}; we also 
follow their formalism in our description below. Although \cite{sefusatti2016} 
and \cite{scoccimarro2015} respectively describe estimators in real- and 
redshift-space, since we focus on the bispectrum monopole, we note that there 
is no difference. 

To measure the bispectrum of our halo catalogs, we begin by interpolating the halo
positions to a grid, $\delta({\bfi x})$, and Fourier transforming the grid to get 
$\delta({\bfi k})$. We use a fourth-order interpolation to get interlaced grids, 
which has advantageous anti-aliasing properties that allow unbias measurements 
up to the Nyquist frequency~\citep{hockney1981,sefusatti2016}. Then using 
$\delta({\bfi k})$ we measure the bispectrum monopole as 
\beq \label{eq:bk} 
\widehat{B}_{\ell = 0}(k_1,k_2, k_3) = \frac{1}{V_B} \int\limits_{k_1}{\rm d}^3q_1 \int\limits_{k_2}{\rm d}^3q_2 \int\limits_{k_3}{\rm d}^3q_3~\delta_{\rm D}({\bfi q_{123}})~\delta({\bfi q_1})~\delta({\bfi q_2})~\delta({\bfi q_3}) - B^{\rm SN}_{\ell = 0}.
\eeq
$\delta_{\rm D}$ is the Dirac delta function and hence $\delta_{\rm D}({\bfi q_{123}}) = \delta_{\rm D}({\bfi q_1} + {\bfi q_2} + {\bfi q_3})$ 
ensures that the ${\bfi q}_i$ triplet actually forms a closed triangle. Each of the integrals 
above represent an integral over a spherical shell in $k$-space with radius $\delta k$ 
centered at ${\bfi k}_i$: 
\beq
\int_{k_i}{\rm d}^3q \equiv \int\limits_{k_i-\delta k/2}^{k_i+\delta k/2}{\rm d}q~q^2\int {\rm d}\Omega.
\eeq
$V_B$ is a normalization factor proportional to the number of triplets ${\bfi q_1}$, 
${\bfi q_2}$, and ${\bfi q_3}$ that can be found in the triangle bin defined by 
$k_1$, $k_2$, and $k_3$ with width $\delta k$: 
\beq
V_B = \int\limits_{k_1}{\rm d}^3q_1 \int\limits_{k_2}{\rm d}^3q_2 \int\limits_{k_3}{\rm d}^3q_3~\delta_{\rm D}({\bfi q_{123}}). 
\eeq
Lastly, $B^{\rm SN}_{\ell=0}$ is the correction for the Poisson shot noise, which 
contributes due to the self-correlation of individual objects: 
\beq \label{eq:bk_sn} 
B^{\rm SN}_{\ell=0}(k_1, k_2, k_3) = \frac{1}{\bar{n}} \big(P_0(k_1) + P_0(k_2) + P_0(k_3) \big) + \frac{1}{\bar{n}^2}
\eeq
where $\bar{n}$ is the number density of objects (halos) and $P_0$ is the power 
spectrum monopole. 

In order to evaluate the integrals in Eq.~\ref{eq:bk}, we take advantage of the plane-wave 
representation of the Dirac delta function and rewrite the equation as
\begin{align} \label{eq:bk2} 
    \widehat{B}_{\ell = 0}(k_1,k_2, k_3) &= \frac{1}{V_B} \int\frac{{\rm d}^3x}{(2\pi)^3} \int\limits_{k_1}{\rm d}^3q_1 \int\limits_{k_2}{\rm d}^3q_2 \int\limits_{k_3}{\rm d}^3q_3~\delta({\bfi q_1})~\delta({\bfi q_2})~\delta({\bfi q_3})~e^{i{\bfi q_{123}}\cdot{\bfi x}} - B^{\rm SN}_{\ell = 0}\\ 
    &= \frac{1}{V_B} \int\frac{{\rm d}^3x}{(2\pi)^3}\prod\limits_{i=1}^{3} I_{k_i}({\bfi x}) - B^{\rm SN}_{\ell = 0} 
\end{align}
where 
\beq
I_{k_i}({\bfi x}) = \int\limits_k {\rm d}^3q~\delta({\bfi q})~e^{i {\bfi q}\cdot{\bfi x}}. 
\eeq
At this point, we measure $\widehat{B}_{\ell = 0}(k_1, k_2, k_3)$ by calculating the 
$I_{k_i}$s with inverse FFTs and summing over in real space. For $\widehat{B}_{\ell = 0}$
measurements throughout the paper, we use $\delta({\bfi x})$ grids with $N_{\rm grid} = 360$ 
and triangle configurations defined by $k_1, k_2, k_3$ bins of width 
$\Delta k = 3 k_f = 0.01885~h/{\rm Mpc}$, three times the fundamental mode
$k_f = 2\pi/(1000~h^{-1}{\rm Mpc})$ given the box size\footnote{The code that we use to 
evaluate $\widehat{B}_{\ell = 0}$ is publicly available at \url{https://github.com/changhoonhahn/pySpectrum}}. 

We present the redshift-space halo bispectrum of the HADES simulations measured using 
the estimator above in two ways: one that emphasizes the triangle shape dependence 
(Figure~\ref{fig:bk_shape}) and the other that emphasizes the amplitude 
(Figure~\ref{fig:bk_amp}). In Figure~\ref{fig:bk_shape}, we plot $\BOk(k_1, k_2, k_3)$ 
as a function of $k_2/k_1$ and $k_3/k_1$, which describe the triangle configuration shapes. 
In each panel, the colormap of the ($k_2/k_1$, $k_3/k_1$) bins represent the weighted 
average $\BOk$ amplitude of all triangle configurations in the bins. The upper left 
bins contain squeezed triangles ($k_1 = k_2 \gg k_3$), the upper right bins contain 
equilateral triangles ($k_1 = k_2 = k_3$), and the bottom center bins contain folded 
triangles ($k_1 = 2 k_2 = 2 k_3$) as schematically highlighted in the lower leftmost panel. 
We include all possible 1898 triangle configurations with $k_1, k_2, k_3 < k_{\rm max} = 0.5~\mpc$. 
$\BOk$ in the upper panels are HADES models with 
$(\smnu,~\sig) = (0.0~{\rm eV},~0.833)$ (fiducial), $(0.06~{\rm eV},~0.822), (0.10~{\rm eV},~0.815)$, 
and $(0.15~{\rm eV},~0.806)$. $\BOk$ in the lower panels are  HADES models with $\smnu = 0.0$ eV and 
$\sig = 0.822, 0.818$, and $0.807$. The top and bottom panels of the three right-most 
columns have matching $\sig$ values (Section~\ref{sec:hades}).

Next, in Figure~\ref{fig:bk_amp} we plot $\BOk(k_1, k_2, k_3)$ for all possible triangle 
configurations with $k_1, k_2, k_3 < k_{\rm max} = 0.5~\mpc$ where we order the 
configurations by looping through $k_3$ in the inner most loop and $k_1$ in the outer most 
loop with $k_1 \geq k_2 \geq k_3$. In the top panel, we present $\BOk$ of HADES models 
with $\smnu = 0.0, 0.06, 0.10$, and $0.15\,\mathrm{eV}$; in the lower panel, we present 
$\BOk$ of HADES models with $\smnu = 0.0$ eV and $\sig = 0.822, 0.818$, and $0.807$. We 
zoom into triangle configurations with $k_1 = 0.320$, $0.170 \leq k_2 \leq 0.302$, and 
$0.094 \leq k_3 \leq 0.302~h/{\rm Mpc}$ in the insets of the panels. For further details on the
redshift-space bispectrum, we refer to Appendix~\ref{sec:bk_details}. 
