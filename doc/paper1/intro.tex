\section{Introduction}
The discovery from neutrino oscillation experiments that $\smnu > 0.06eV$
opens many doors of physics beyond the standard model. Tight constraints on $\smnu$ 
has the potential to \ch{insert implications}. 
Current and upcoming laboratory experiments have the potential to measure 
These measurements alone are not good enough. 
Neutrinos through the cosmic neutrino background affect the expansion history 
and the growth of cosmic structure. Measuring these effects with cosmological 
observables provides complementary and potentially more constraining 
measurements of $\smnu$. 

Neutrinos, in the early universe, are relativistic and contribute to the 
energy density of radiation. Later as they become non-relativistic, 
they contribute to the energy density of matter. This transition affects 
the expansion history of the universe and leaves imprints observable in 
the cosmic microwave background (CMB) anisotropy spectrum (citecitecite). 
Massive neutrinos also impact the growth of structure. On large scales, 
neutrino perturbations are indistinguishable from perturbations of cold 
dark matter (CDM). However, on scales smaller than their free-streaming 
scale, neutrinos do not contribute the clustering and thereby reduce the
total matter power spectrum. In addition, they also reduce the growth 
rate of CDM perturbations at late times. This combined suppression of 
the small-scale matter power spectrum also leaves measurable imprints 
on the CMB as well as large-scale structure. For more details the effect
of neutrinos in cosmological observables, we refer readers to 
\cite{lesgourgues2012,lesgourgues2014,gerbino2018}. 

The tightest cosmological constraints on $\smnu$ currently come from 
combining CMB data with other cosmological probes. Temperture and large 
angle polarization data from the {\em Planck} satellite places an upper 
bound of $\smnu < 0.54$ eV with 95\% confidence level~\citep{planckcollaboration2018}. 
Adding the Baryon Acoustic Oscillation (BAO) to the {\em Planck} 
likelihood breaks geometrical degeneracies ($\smnu$, $h$, $\Om$) 
and significantly tightens the upper bound to $\smnu < 0.16$ eV. CMB 
lensing further tightens the bound fruther to $\smnu < 0.13$ eV, though 
not as significantly. Combining these probes, however, introduce a strong 
correlation between $\smnu$ and $A_s$, the scale amplitude. Since CMB 
experiments measure the combined quantity $A_s e^{-2\tau}$, where 
$\tau$ is the optical depth of reionization, this leads to a strong 
degeneracy between $\smnu$ and $\tau$~\citep{allison2015, liu2016, archidiacono2017}. 

The $\tau-\smnu$ degeneracy will continue to be a major bottleneck 
for $\smnu$ constraints from CMB data. The best constraints on $\tau$ 
currently come from {\em Planck} ($\tau\sim0.06$). However, upcoming 
ground-based CMB experiments ({\em e.g.} CMB-S4) will not observe scales 
larger than $\ell < 30$, and therefore will not directly constrain 
$\tau$~\citep{abazajian2016}. Meanwhile, proposed future space-based 
experiments such as LiteBIRD\footnote{http://litebird.jp/eng/} and 
COrE\footnote{http://www.core-mission.org/}, which have the potential 
to precisely measure $\tau$, have yet to be confirmed. CMB data, however, 
is not the only way to improve $\smnu$ constraints. The imprint of 
neutrinos on 3D clustering can be measured to constrain $\smnu$ and 
with the sheer cosmic volumes mapped by upcoming surveys, \emph{e.g.} 
DESI\footnote{https://www.desi.lbl.gov/}, PFS\footnote{https://pfs.ipmu.jp/}, 
EUCLID\footnote{http://sci.esa.int/euclid/}, and WFIRST\footnote{https://wfirst.gsfc.nasa.gov/}, 
will be able tightly constrain $\smnu$~\citep{audren2013, font-ribera2014, petracca2016, sartoris2016, boyle2018}.

A major limitation of using 3D clustering is modeling beyond linear 
scales, for bias tracers, and in redshift space. Simulations have made
huge strides in accurately and efficiently modeling nonlinear structure
formation with massive neutrinos~\citep[\emph{e.g.}][]{brandbyge2008, 
villaescusa-navarro2013, castorina2015, adamek2017, emberson2017, villaescusa-navarro2018}. 
In conjuction, simulation based `emulation` methods that exploit the 
accuracy of $N$-body simulations while minimizing the computing budget 
have been applied to analyze small-scale galaxy clustering with remarkable 
success~\citep[\emph{e.g.}][]{heitmann2009, kwan2015, euclidcollaboration2018, mcclintock2018, zhai2018, wibking2019}. 
With these developments, analyzing 
small scale clustering, shows great promise. 

\cite{villaescusa-navarro2018} examined the power spectrum multipole 
using a suite of $N$-body simulations going beyond previous real-space 
matter power spectrum analyses (citecite). One major bottleneck he 
found when it comes to analyzing the power spectrum is the $\smnu$-$\sig$ degeneracy. 
sig8 and Mnu have very similiar signatures and only differ by 1\% or w.e. 
However, we don't have to settle for just two point statistics, there's three-point 
statistics i.e. the bispectrum. mention ruggeri paper and some broad statements about
the advantages of the bispectrum. 

In this paper we use n-body simulations with massive neutrinos to examine the 
redshift-space halo bispectrum and demonstrate how it breaks the mnu-sig8 degeneracy. 
Furthermore, we present the full information content of the bispectrum with a Fisher 
forecast where we calculate the covariance matrix and derivatives with an unprecedented 
number simulations of the Quijote suite. This paper is the first of a series of 
papers that aim at demonstrating the potential of the galaxy bispectrum analysis in 
constraining $\smnu$. In the subsequent paper, we will include HOD parameters in our 
forecasts to quantify the full information content of the galaxy bispectrum. We will
also present various ways to tackle the challenges of a full galaxy bispectrum analysis, 
such as data compression for reducing the dimensionality of the bispectrum. 

In Section~\ref{sec:hades} we describe the two simulation suites, HADES and Quijote, 
that we use throughout the paper. Then in Section~\ref{sec:bk}, we describe how 
we measure the bispectrum of these simulations. We then present in Section~\ref{sec:mnusig}, 
using the redshift-space halo bispectrum of HADES simulations, how $\smnu$ leaves 
a distinct imprint on the bispectrum which allows the bispectrum to break the 
degeneracy between $\smnu$ and $\sig$. Finally, in Section~\ref{sec:forecasts} 
we present the full information content of the halo bispectrum with a Fisher 
forecast of cosmological parameters using the Quijote simulations and demonstrate 
the how the bispectrum drastically improves the constraints on the cosmological 
parameters: $\Om$, $\Ob$, $h$, $n_s$, $\sig$, and {\em especially} $\smnu$. 


