\section{Introduction}
The lower bound on the sum of neutrino masses ($\smnu \gtrsim 0.06$ eV), 
discovered by neutrino oscillation experiments, provides conclusive evidence of 
physics beyond the Standard Model of particle physics~\citep{forero2014, gonzalez-garcia2016}. 
A more precise measurement of $\smnu$ has the potential to distinguish 
between the `normal' and `inverted' neutrino mass hierarchy scenarios 
and further reveal the physics of neutrinos. Neutrino oscillation 
experiments, however, are insensitive to the absolute neutrino mass scales. 
Other laboratory experiments sensitive to $\smnu$ (\emph{e.g.} double beta 
decay and tritium beta decay experiments) have the potential to place upper 
bounds of $\smnu < 0.2$ eV in upcoming experiments~\citep{bonn2011, drexlin2013}. 
These upper bound alone are not insufficient to distinguish between the mass
hiearchies. Neutrinos, through the cosmic neutrino background, affect the 
expansion history and the growth of cosmic structure. Measuring these effects 
with cosmological observables provides complementary and potentially more 
precise measurements of $\smnu$. 

Neutrinos, in the early Universe, are relativistic and contribute to the 
energy density of radiation. Later as they become non-relativistic, 
they contribute to the energy density of matter. This transition affects 
the expansion history of the Universe and leaves imprints observable in 
the cosmic microwave background (CMB) anisotropy spectrum~\citep{lesgourgues2012, lesgourgues2014}. 
Massive neutrinos also impact the growth of structure. On large scales, 
neutrino perturbations are indistinguishable from perturbations of cold 
dark matter (CDM). However, on scales smaller than their free-streaming 
scale, neutrinos do not contribute to the clustering and thereby reduce 
the amplitude of the total matter power spectrum. In addition, they also reduce the growth 
rate of CDM perturbations at late times. This combined suppression of 
the small-scale matter power spectrum leaves measurable imprints 
on the CMB as well as large-scale structure. For more on details the effect
of neutrinos in cosmological observables, we refer readers to 
\cite{lesgourgues2012,lesgourgues2014,gerbino2018}. 

The tightest cosmological constraints on $\smnu$ currently come from 
combining CMB data with other cosmological probes. Temperture and large 
angle polarization data from the {\em Planck} satellite places an upper 
bound of $\smnu < 0.54$ eV with 95\% confidence level~\citep{planckcollaboration2018}. 
Adding the Baryon Acoustic Oscillation (BAO) to the {\em Planck} 
likelihood breaks geometrical degeneracies ($\smnu$, $h$, $\Om$) 
and significantly tightens the upper bound to $\smnu < 0.16$ eV. CMB 
lensing further tightens the bound further to $\smnu < 0.13$ eV, though 
not as significantly. Combining these probes, however, introduce a strong 
correlation between $\smnu$ and $A_s$, the scale amplitude. Since CMB 
experiments measure the combined quantity $A_s e^{-2\tau}$, where 
$\tau$ is the optical depth of reionization, this leads to a strong 
degeneracy between $\smnu$ and $\tau$~\citep{allison2015, liu2016, archidiacono2017}. 

The $\tau-\smnu$ degeneracy will continue to be a major bottleneck 
for $\smnu$ constraints from CMB data. The best constraints on $\tau$ 
currently come from {\em Planck}: $\tau\sim0.06$. However, upcoming 
ground-based CMB experiments ({\em e.g.} CMB-S4) will not observe scales 
larger than $\ell < 30$, and therefore will not directly constrain 
$\tau$~\citep{abazajian2016}. Meanwhile, proposed future space-based 
experiments such as LiteBIRD\footnote{http://litebird.jp/eng/} and 
COrE\footnote{http://www.core-mission.org/}, which have the potential 
to precisely measure $\tau$, have yet to be confirmed. CMB data, however, 
is not the only way to improve $\smnu$ constraints. The imprint of 
neutrinos on 3D clustering of galaxies can be measured to constrain 
$\smnu$ and with the sheer cosmic volumes mapped by upcoming surveys, 
\emph{e.g.} DESI\footnote{https://www.desi.lbl.gov/}, PFS\footnote{https://pfs.ipmu.jp/}, 
EUCLID\footnote{http://sci.esa.int/euclid/}, and WFIRST\footnote{https://wfirst.gsfc.nasa.gov/}, 
will be able tightly constrain 
$\smnu$~\citep{audren2013, font-ribera2014, petracca2016, sartoris2016, boyle2018}.

A major limitation of using 3D clustering is obtaining accurate theoretical 
predictions beyond linear scales, for bias tracers, and in redshift space. 
Simulations have made huge strides in accurately and efficiently modeling 
nonlinear structure formation with massive neutrinos~\citep[\emph{e.g.}][]{brandbyge2008, 
villaescusa-navarro2013, castorina2015, adamek2017, emberson2017, villaescusa-navarro2018}. 
In conjuction, new simulation based `emulation` models that exploit the 
accuracy of $N$-body simulations while minimizing the computing budget 
have been applied to analyze small-scale galaxy clustering with remarkable 
success~\citep[\emph{e.g.}][]{heitmann2009, kwan2015, euclidcollaboration2018, mcclintock2018, zhai2018, wibking2019}. 
Developments on these fronts have the potential to unlock the information 
content in nonlinear clustering to constrain $\smnu$. 

% more real space papers Bird et al. 2012; Wagner et al. 2012; Ali-Haïmoud & Bird 2013; Lesgourgues et al. 2013; Villaescusa-Navarro et al. 2013b, 2015; Blas et al. 2014; LoVerde 2014b; Rossi et al. 2014, 2014; Upadhye et al. 2014; Castorina et al. 2015; Führer & Wong 2015; Inman et al. 2015; Peloso et al. 2015; Banerjee & Dalal 2016; Emberson et al. 2016}
Various works have examined the impact of neutrino masses on nonlinear clustering 
of matter in 
real-space~\citep[\emph{e.g.}][]{brandbyge2008, saito2008, wong2008, saito2009, viel2010, agarwal2011, bird2012, castorina2015, banerjee2016} 
and in redshift-space~\citep{marulli2011, castorina2015, upadhye2016}. Most recently, 
using a suite of more than 1000 simulations, \cite{villaescusa-navarro2018} 
examined the impact of $\smnu$ on the redshift-space matter and halo power 
spectrum to find that the imprint of $\smnu$ and $\sig$ on the power spectrum are 
degenerate and differ only by $< 1\%$. The strong $\smnu$ -- $\sig$ degeneracy
poses a serious limitation of constraining $\smnu$ with the power spectrum. 
However, information in the nonlinear regime cascade from the power spectrum 
to higher-order statistics--- \emph{e.g.} the bispectrum. In fact, the 
bispectrum has a comparable signal-to-noise ratio to the power spectrum
on nonlinear scales~\citep{sefusatti2005, chan2017}. Furthermore, although $\smnu$ 
is not included in their analysis, \cite{sefusatti2006} and \cite{yankelevich2019} 
have shown that including the bispectrum in analyses improves constraints on 
cosmological parameters. \cite{ruggeri2018} presented the first measurement of 
the bispectrum for $N$-body simulations with massive neutrinos. Yet, no work to 
date has quantified the full information content and constraining power of 
the entire bispectrum down to nonlinear scales --- especially for $\smnu$. 

In this work, we examine the effect of massive neutrinos on the redshift-space 
halo bispectrum using more than 42000 $N$-body simulations with massive neutrinos. 
We first demonstrate that the bispectrum helps break the $\smnu$--$\sig$ degeneracy 
found in the power spectrum. Then we present the full information content of the 
bispectrum for all triangle configurations down to $k_{\rm max} = 0.5~\mpc$ using 
a Fisher forecast where we estimate the covariance matrix and derivatives with 
the large set of simulations from the Quijote suite (Villaescusa-Navarro et al. in preparation). 
This paper is the first of a series of papers that aim to demonstrate the potential 
of the galaxy bispectrum analysis in constraining $\smnu$. In the subsequent paper, 
we will include HOD parameters in our forecasts to quantify the full information 
content of the galaxy bispectrum. In the series, we will also present methods to 
tackle challenges that come with analyzing the full galaxy bispectrum, such as data 
compression for reducing the dimensionality of the bispectrum. 

In Section~\ref{sec:hades} we describe the two simulation suites, HADES and Quijote, 
and the halo catalogs cosntructed from them, used throughout the paper. We then 
describe in Section~\ref{sec:bk}, how we measure the bispectrum using these simulations. 
We then use the redshift-space halo bispectra to demonstrate the distinct imprint of 
$\smnu$ on the bispectrum, which allows the bispectrum to break the degeneracy between 
$\smnu$ and $\sig$ in Section~\ref{sec:mnusig}. Finally, in Section~\ref{sec:forecasts} 
we present the full information content of the halo bispectrum with a Fisher forecast 
of cosmological parameters and demonstrate how the bispectrum dramatically improves 
the constraints on the cosmological parameters: $\Om$, $\Ob$, $h$, $n_s$, $\sig$, and {\em especially} $\smnu$. 
