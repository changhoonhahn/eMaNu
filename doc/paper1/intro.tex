\section{Introduction}
\ch{very brief into on neutrinos}
Brief intro on the impact of massive active neutrinos on the matter powerspectrum 
and how that's detectable with CMB and LSS

Quick summary of current cosmology constraints and where they come from. 


%The neutrino mass can be probed precisely in cos- mology because properties of the cosmic neutrino back- ground affect the growth of cosmic structure and the expansion history of the universe [2, 3]. A primary ef- fect targeted by future experiments is the suppression of growth of small scale structure caused by a nonzero neu- trino mass. The rest mass of the neutrinos, as they be- come non-relativistic, increases the neutrino contribution to the total mean energy density beyond what it would be in the massless case, thereby increasing the expansion rate and thus suppressing growth. A secondary effect is the scale dependence of this suppression: above the free-streaming scale the neutrinos act just like cold dark matter and therefore contribute to gravitational instabil- ity, with the net effect of canceling out the suppressive effect of the increased expansion rate [4–8]. This results in a fairly broad “step”-like feature in the matter power spectrum, where the size of the step is time dependent and grows approximately linearly with every e-fold of ex- pansion1 .
%To measure neutrino mass using this time-dependent suppression, the amplitude of structure at low redshift (probed by gravitational lensing, clusters, or redshift- space distortions) is typically compared with the initial, high redshift amplitude probed by the CMB. In particu- lar, the small-scale suppression is about 4% between the redshift of recombination and today for the minimal mass of 60 meV.
% This high-z amplitude As, in turn, is limited by how well we know the optical depth τ to the CMB, because the combination Ase−2τ (describ- ing the amplitude of the CMB power spectrum) is what is measured by CMB surveys [11, 12]. Since it is un- clear whether substantially improved τ constraints will be forthcoming, it is well motivated to seek methods by which the neutrino mass can be probed without relying on a knowledge of the CMB optical depth.


Talk about 
the CMB-LSS lever arm. The degeneracy between As and tau and how that's a bottleneck
short thing about how $\tau$ is hard to constrain.
Fortunately the imprint of neutrinos on the matter distribution leaves imprints
on clustering. So with clustering measurements alone we can derive constraints 
on $\smnu$ and at the very least tighten constraints.  

Brief summary of previous works that look at the powerspectrum.
%https://arxiv.org/abs/1310.0037
Then Discuss the 
shortcomings of the powerspectrum only analysis-- Not good enough. 

However, we don't have to settle for just two point statistics, three-point 
statistics such as the bispectrum and 3PCF... 

In Section blah 

\ch{List plans for paper 2}

