\section{Introduction}
Boiler plate intro paragraph on how oscillation experiments show neutrinos have mass. 
$\smnu > 0.06eV$
implications of neutrino mass --- physics beyond the standard model -- fame and glory. 
Followed up by, very briefly, the current state of particle experiments in constraining
neutrino mass, while suggesting they're not good enough. 
follow this up with a sentence on how cosmology provides complementary constraints on 
neutrinso

Brief intro on the impact of massive neutrinos on cosmology.
Then explain how this is being used to detect with CMB and LSS. 
summary the state-of-the-art constraints from Planck + BAO.
Talk about how BAO breaks parameter degeneracies between $\smnu$, 
$h$, and $\Om$. However, combining these probes introduces a strong
correlation between $\smnu$ and $A_s$. This leads to a degeneracy
between $\smnu$ and $\tau$. 
\cite{archidiacono2017}
CMB experiments, however, do not 
measure $A_s$ explicitly. They instead measure the combination 
$A_se^{-2\tau}$, where $\tau$ is the optical depth of reionization. 

%The neutrino mass can be probed precisely in cos- mology because properties of the cosmic neutrino back- ground affect the growth of cosmic structure and the expansion history of the universe [2, 3]. A primary ef- fect targeted by future experiments is the suppression of growth of small scale structure caused by a nonzero neu- trino mass. The rest mass of the neutrinos, as they be- come non-relativistic, increases the neutrino contribution to the total mean energy density beyond what it would be in the massless case, thereby increasing the expansion rate and thus suppressing growth. A secondary effect is the scale dependence of this suppression: above the free-streaming scale the neutrinos act just like cold dark matter and therefore contribute to gravitational instabil- ity, with the net effect of canceling out the suppressive effect of the increased expansion rate [4–8]. This results in a fairly broad “step”-like feature in the matter power spectrum, where the size of the step is time dependent and grows approximately linearly with every e-fold of ex- pansion1 .
%To measure neutrino mass using this time-dependent suppression, the amplitude of structure at low redshift (probed by gravitational lensing, clusters, or redshift- space distortions) is typically compared with the initial, high redshift amplitude probed by the CMB. In particu- lar, the small-scale suppression is about 4% between the redshift of recombination and today for the minimal mass of 60 meV.
% This high-z amplitude As, in turn, is limited by how well we know the optical depth τ to the CMB, because the combination Ase−2τ (describ- ing the amplitude of the CMB power spectrum) is what is measured by CMB surveys [11, 12]. Since it is un- clear whether substantially improved τ constraints will be forthcoming, it is well motivated to seek methods by which the neutrino mass can be probed without relying on a knowledge of the CMB optical depth.

$\tau$ is a major bottleneck for future experiments. 
short thing about how $\tau$ is hard to constrain and how major improvements in it 
will take a while. Future experiments such as LiteBIRD, which at the moment is still
under review, will not arrive within 5 years. 
Fortunately the imprint of neutrinos on the matter distribution leaves imprints
on clustering. So with clustering measurements alone we can derive constraints 
on $\smnu$ and at the very least tighten constraints.  

Until now, one limitation of using galaxy clustering beyond linear scales has been 
the difficulty in modeling nonlinear structure formation. $N$-body simulations with
massive neutrinos have made huge progress in this regard (citecitecite). In 
conjuction, methods like emulation which leverage the accuracy of $N$-body simulation
while minimizing the computing budget have been applied to analyze small-scale galaxy
clustering with remarkable success (citecitecite). With these developments, analyzing 
small scale clustering, shows great promise. 

Paco et al. examined the power spectrum multipole using n-body simulations going beyond
previous real-space matter powerspectrum analyses (citecite). One major bottleneck he 
found when it comes to analyzing the power spectrum is the Mnu-sig8 degeneracy. 
sig8 and Mnu have very similiar signatures and only differ by 1\% or w.e. 
However, we don't have to settle for just two point statistics, there's three-point 
statistics i.e. the bispectrum. mention ruggeri paper and some broad statements about
the advantages of the bispectrum. 

In this paper we use n-body simulations with massive neutrinos to examine the 
redshift-space halo bispectrum and demonstrate how it breaks the mnu-sig8 degeneracy. 
Furthermore, we present the full information content of the bispectrum with a Fisher 
forecast where we calculate the covariance matrix and derivatives with an unprecedented 
number simulations of the Quijote suite. This paper is the first of a series of 
papers that aim at demonstrating the potential of the galaxy bispectrum analysis in 
constraining $\smnu$. In the subsequent paper, we will include HOD parameters in our 
forecasts to quantify the full information content of the galaxy bispectrum. We will
also present various ways to tackle the challenges of a full galaxy bispectrum analysis, 
such as data compression for reducing the dimensionality of the bispectrum. 

In Section~\ref{sec:hades} we describe the two simulation suites, HADES and Quijote, 
that we use throughout the paper. Then in Section~\ref{sec:bk}, we describe how 
we measure the bispectrum of these simulations. We then present in Section~\ref{sec:mnusig}, 
using the redshift-space halo bispectrum of HADES simulations, how $\smnu$ leaves 
a distinct imprint on the bispectrum which allows the bispectrum to break the 
degeneracy between $\smnu$ and $\sig$. Finally, in Section~\ref{sec:forecasts} 
we present the full information content of the halo bispectrum with a Fisher 
forecast of cosmological parameters using the Quijote simulations and demonstrate 
the how the bispectrum drastically improves the constraints on the cosmological 
parameters: $\Om$, $\Ob$, $h$, $n_s$, $\sig$, and {\em especially} $\smnu$. 


