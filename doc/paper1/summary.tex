\section{Summary} 
% why $\smnu$ constraints with galaxy clustering? 
A precise measurement of $\smnu$ can distinguish between the `normal' 
and `inverted' neutrino mass hierarchy scenarios and reveal physics 
beyond the Standard Model of particle physics. The total neutrino mass, 
through impact the  expansion history and the growth of cosmic structure 
in the Universe, can be measured using cosmological observables (\emph{e.g.} 
CMB and large-scale structure). In fact, cosmological probes have the 
potential to more precisely measure $\smnu$ than current and upcoming 
laboratory based experiments. The current tighest cosmological $\smnu$ 
constraints come from combining CMB data with other cosmological probes. 
The degeneracy between $\smnu$ and $\tau$, the optical depth of reionization, 
however, is a major bottleneck for $\smnu$ constraints from CMB data 
that will {\em not} be addressed by upcoming ground-based CMB experiments. 
Measuring the imprint of neutrinos on the 3D clustering of galaxies 
provides a promising alternative to tightly constrain $\smnu$, especially 
with the unprecedented cosmic volumes that will be mapped by upcoming 
surveys. 

% why bispectrum?
Recent developments in simulation based emulation methods have improved 
the accuracy of theoretical predictions beyond linear scales and address 
the challenges of unlocking the information content in nonlinear clustering 
to constrain $\smnu$. Yet, for power spectrum analyses, the strong 
degeneracy between the imprints of $\smnu$ and $\sig$ poses a serious 
limitation for constraining $\smnu$. Information in the nonlinear 
regime, however, cascade from the power spectrum to higher-order statistics 
such as the bispectrum. Previous works have demonstrated that the bispectrum
has comparable signal-to-noise as the power spectrum on nonlinear scales~\citep{sefusatti2005, chan2017}
and that it improves constraints on $\Lambda$CDM cosmological parameters~\citep{sefusatti2006, yankelevich2019}. 
No work to date has quantified the full information content and constraining 
power of the bispectrum down to nonlinear scales --- especially for $\smnu$. 

% what's our methodology?
In this work, we examine the effect of massive neutrinos on the redshift-space 
halo bispectrum using more than 42,000 $N$-body simulations with massive neutrinos
from the HADES and Quijote suite. Using $N$-body simulations and with such a massive 
number of them, we directly address key challenges of accurately modeling the 
bispectrum in the nonlinear regime and estimating its high dimensional covariance 
matrix. More specifically, 
\begin{itemize}
    \item We examine the imprint of $\smnu$ and $\sig$ on the redshift-space halo bispetrum, 
        $\BOk$, using two sets of HADES simulations. One with massive neutrinos, $\smnu = 0.0$, 0.06, 
        0.1, and 0.15 eV; the other with $\smnu = 0.0$ eV and matching $\sig^{c}$. 
        $\smnu$ and $\sig$ leave distinct imprints on $\BOk$ with significantly different 
        scale and triangle shape dependences. Thus, we demonstrate that $\BOk$ helps 
        break the $\smnu$-$\sig$ degeneracy found in the power spectrum. 
    \item We quantify the full information content of $\BOk$ using a Fisher matrix forecast of 
        $\{\Om$, $\Ob$, $h$, $n_s$, $\sig$, $\smnu\}$ using all 1898 triangle configurations
        down to $k_{\rm max} = 0.5~\mpc$. The covariance matrix and derivatives for our forecasts 
        are directly estimated from the Quijte suite of 42,000 $N$-body simulations. 
    \item For $k_{\rm max}{=}0.5~\mpc$, the bispectrum produces $\Om$, $\Ob$, $h$, $n_s$, and 
        $\sig$ constraints 3.1, 4.1, 5.1, 5.9, and 3.3 times tighter than the power spectrum. 
        The bispectrum produces 1$\sigma$ $\smnu$ constraint of 0.0342 eV --- 6 times tigher 
        than the power spectrum.
\end{itemize}

% what are the caveats our results? 
paragraph on the caveats of our results. 
we don't marginalize over a full bias model yet and will do that in the next paper. 
sentence saying regardless our results are awesome.

% what are the implications of our results? 
implications for upcoming spectroscopic redshift surveys. We only use 1gpc. DESI PFS, Euclid and WFIRST are going to map out way more. In fact some of the main scientific goals of these surveys will be neutrinos. Based on our results, the bispectrum has the potentially to dramatically improve $\smnu$ constraints. Something about X sigma detection?  

%In this work, we examine the effect of massive neutrinos on the redshift-space 
%halo bispectrum using more than 23000 $N$-body simulations with massive neutrinos. 
%We first demonstrate that the bispectrum helps break the $\smnu$--$\sig$ degeneracy 
%found in the power spectrum. Then we present the full information content of the 
%bispectrum for all triangle configurations down to $k_{\rm max} = 0.5~\mpc$ using 
%a Fisher forecast where we estimate the covariance matrix and derivatives with 
%the large set of simulations from the Quijote suite (Villaescusa-Navarro et al. in preparation). 
%This paper is the first of a series of papers that aim to demonstrate the potential 
%of the galaxy bispectrum analysis in constraining $\smnu$. In the subsequent paper, 
%we will include HOD parameters in our forecasts to quantify the full information 
%content of the galaxy bispectrum. In the series, we will also present methods to 
%tackle challenges that come with analyzing the full galaxy bispectrum, such as data 
%compression for reducing the dimensionality of the bispectrum. 

%This paper is the first of a series of papers that aim to demonstrate the potential 
%of the galaxy bispectrum analysis in constraining $\smnu$. In the subsequent paper, 
%we will include HOD parameters in our forecasts to quantify the full information 
%content of the galaxy bispectrum. In the series, we will also present methods to 
%tackle challenges that come with analyzing the full galaxy bispectrum, such as data 
%compression for reducing the dimensionality of the bispectrum. 

%Massive neutrinos suppress the growth of structure below their free-streaming scale and leave an 
%imprint on large-scale structure. Measuring this imprint allows us to constrain the sum of neutrino 
%masses, $\smnu$, a key parameter in particle physics beyond the Standard Model. However, degeneracies 
%among cosmological parameters, especially between $\smnu$ and $\sig$, limit the constraining power 
%of standard two-point clustering statistics. In this work, we investigate whether we can break these 
%degerancies and constrain $\smnu$ with the next higher-order correlation function --- the bispectrum. 
%We first examine the redshift-space halo bispectrum of $800$ $N$-body simulations from the HADES suite 
%and demonstrate that the bispectrum helps break the $\smnu$--$\sig$ degeneracy. Then using over 23000
%$N$-body simulations of the Quijote suite, we quantify for the first time the full information content 
%of the redshift-space halo bispectrum using a Fisher matrix forecast of $\{\Om$, $\Ob$, $h$, $n_s$, $\sig$, $\smnu\}$ 
%down to nonlinear scales. For $k_{\rm max}{=}0.5~\mpc$, the bispectrum 
%constrains $\Om$, $\Ob$, $h$, $n_s$, and $\sig$ 3.1, 4.1, 5.1, 5.9, and 3.3 times tighter than the 
%power spectrum. For $\smnu$, the bispectrum improves the 1$\sigma$ constraint from 
%0.1962 to 0.0342 eV --- 6 times tigher than the power spectrum. Even with priors from {\em Planck}, 
%the bispectrum improves $\smnu$ constraints by a factor of 2.7. Although we reserve marginalizing 
%over bias parameters to the next paper of the series, these constraints are derived for a 
%$(1~h^{-1}{\rm Gpc})^3$ box, a substantially smaller volume than upcoming surveys. Thus, our results 
%demonstrate that the bispectrum offers significant improvements over the power spectrum, especially 
%for constraining $\smnu$.  
