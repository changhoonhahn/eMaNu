\section{Summary} 
% why $\smnu$ constraints with galaxy clustering? 
A precise measurement of $\smnu$ can distinguish between the `normal' 
and `inverted' neutrino mass hierarchy scenarios and reveal physics 
beyond the Standard Model of particle physics. The total neutrino mass, 
through impact the  expansion history and the growth of cosmic structure 
in the Universe, can be measured using cosmological observables (\emph{e.g.} 
CMB and large-scale structure). In fact, cosmological probes have the 
potential to more precisely measure $\smnu$ than current and upcoming 
laboratory experiments. The current tightest cosmological $\smnu$ 
constraints come from combining CMB data with other cosmological probes. 
The degeneracy between $\smnu$ and $\tau$, the optical depth of reionization, 
however, is a major bottleneck for $\smnu$ constraints from CMB data 
that will {\em not} be addressed by upcoming ground-based CMB experiments. 
Measuring the imprint of neutrinos on the 3D clustering of galaxies 
provides a promising alternative to tightly constrain $\smnu$, especially 
with the unprecedented cosmic volumes that will be mapped by upcoming 
surveys. 

% why bispectrum?
Recent developments in simulation based emulation methods have improved 
the accuracy of theoretical predictions beyond linear scales and address 
the challenges of unlocking the information content in nonlinear clustering 
to constrain $\smnu$. Yet, for power spectrum analyses, the strong 
degeneracy between the imprints of $\smnu$ and $\sig$ poses a serious 
limitation for constraining $\smnu$. Information in the nonlinear 
regime, however, cascade from the power spectrum to higher-order statistics 
such as the bispectrum. Previous works have demonstrated that the bispectrum
has comparable signal-to-noise as the power spectrum on nonlinear scales~\citep{sefusatti2005, chan2017}
and that it improves constraints on cosmological parameters~\citep{sefusatti2006, yankelevich2019, chudaykin2019}. 
No work to date has quantified the total information content and constraining 
power of the full bispectrum down to nonlinear scales. 

% what's our methodology?
In this work, we examined the effect of massive neutrinos on the redshift-space 
halo bispectrum using more than 22,000 $N$-body simulations with massive neutrinos
from the HADES and Quijote suite. Using $N$-body simulations and with such a massive 
number of them, we directly address key challenges of accurately modeling the 
bispectrum in the nonlinear regime and estimating its high dimensional covariance 
matrix. More specifically, 
\begin{itemize}
    \item We examine the imprint of $\smnu$ and $\sig$ on the redshift-space halo bispectrum, 
        $B_0$, using two sets of HADES simulations. One with massive neutrinos, $\smnu = 0.0$, 0.06, 
        0.1, and 0.15 eV; the other with $\smnu = 0.0$ eV and matching $\sig^{c}$. 
        $\smnu$ and $\sig$ leave distinct imprints on $B_0$ with significantly different 
        scale and triangle shape dependencies. Thus, we demonstrate that $B_0$ helps 
        break the $\smnu$-$\sig$ degeneracy found in the power spectrum. 
    \item We quantify the total information content of $B_0$ using a Fisher matrix forecast of 
        $\{\Om$, $\Ob$, $h$, $n_s$, $\sig$, $\smnu\}$. With bispectrum measured from 22,000 
        $N$-body simulations from the Quijote suite, we are able to derive stable and converged 
        covariance matrix and derivatives for our forecast. Furthermore, our simulation 
        based approach allows us to extend beyond perturbation theory and use all 1898 
        triangle configurations of the bispectrum down to $k_{\rm max} = 0.5~\mpc$.
    \item For $k_{\rm max}{=}0.5~\mpc$, the bispectrum produces $\Om$, $\Ob$, $h$, $n_s$, and 
        $\sig$ constraints 1.9, 2.6, 3.1, 3.6, and 2.6 times tighter than the power spectrum. 
        For $\smnu$, we derive 1$\sigma$ constraint of 0.0572 eV --- over 5 times tighter than 
        the power spectrum. Even with priors from {\em Planck}, the bispectrum improves 
        $\smnu$ constraints by a factor of 1.8. These constraints are derived for a 
        $(1~h^{-1}{\rm Gpc})^3$ box, a substantially smaller volume than upcoming surveys.
\end{itemize}

% what are the caveats our results? 
While our results clearly showcase the advantages of the bispectrum for
more precisely constraining $\smnu$, as well as the cosmological parameters, 
a number of assumptions go into our forecast. Fisher matrix forecasts assume 
that the posterior is approximately Gaussian and, as a result, they 
overestimate the constraints for posteriors that are highly non-elliptical 
or asymmetric. Furthermore, our forecasts are derived using the power spectrum 
and bispectrum in periodic boxes, instead of a realistic geometry or 
radial selection function of galaxy surveys, which degrades the constraining power. 
Since we use periodic boundary conditions, we also do not account for super-sample
covariance, which in practice results from a non-trivial survey geometry. Lastly, 
our forecast focuses on the halo bispectrum, while the \emph{galaxy} bispectrum 
is what is actually measured from observations. We marginalize over a simplistic bias 
model through $b'$ and $M_{\rm min}$; however, such a model is insufficient to 
flexibly describe how galaxies trace matter. A more realistic bias model 
involve extra parameters that describe the distribution of central and satellite 
galaxies. In the next paper of the series, we include HOD parameters in our
forecasts and quantify the full information content of the redshift-space 
galaxy bispectrum. 

% what are the implications of our results? 
The constraints from our forecasts are derived for a $(1~h^{-1}{\rm Gpc})^3$ 
volume. These constraints roughly scale as $\propto1/\sqrt{V}$ with volume. 
Upcoming spectroscopic galaxy surveys will map out vastly larger cosmic
volumes: PFS $\sim 9~h^{-3}{\rm Gpc}^3$ and DESI $\sim 50~h^{-3}{\rm Gpc}^3$~\citep{takada2014, desicollaboration2016}. 
Euclid and WFIRST, space-based surveys, will expand these volumes to 
higher redshifts. A key science goal in these surveys will be constraining $\smnu$. 
Scaling our 0.0572 eV $\smnu$ constraint from the bispectrum to the survey 
volumes gives 0.0191 eV and 0.0081 eV $1\sigma$ constraints for PFS 
and DESI. With such precision, we would detect $\smnu$ by $>3\sigma$ and 
distinguish between the normal and inverted neutrino mass hierarchy scenarios 
by $>2\sigma$ from the bispectrum alone. Such naive projections, however, 
should be taken with more than a grain of salt. Nevertheless, the substantial 
improvement we find in constraints with the bispectrum over the power spectrum, 
strongly advocate for analyzing future surveys with more than the power spectrum. 
Our results demonstrate the potential of the bispectrum to tightly constrain 
$\smnu$ with unprecedented precision.
