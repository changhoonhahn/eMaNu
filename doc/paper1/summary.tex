\section{Summary} 
% why $\smnu$ constraints with galaxy clustering? 
A precise measurement of $\smnu$ can distinguish between the `normal' 
and `inverted' neutrino mass hierarchy scenarios and reveal physics 
beyond the Standard Model of particle physics. Neutrinos impact the  
expansion history and the growth of cosmic structure in the Universe. 
Measuring these effects from cosmological observables (\emph{e.g.} 
CMB and large-scale structure) has the potential to provide more precise 
$\smnu$ measurements than current and upcoming laboratory based experiments. 
The current tighest cosmological $\smnu$ constraints come from combining 
CMB data with other cosmological probes. The degeneracy between $\smnu$ 
and $\tau$, the optical depth of reionization, however, is a major bottleneck 
for $\smnu$ constraints from CMB data that will {\em not} be addressed 
by upcoming ground-based CMB experiments. Measuring the imprint of neutrinos
on the 3D clustering of galaxies provides, given the unprecedented cosmic 
volumes that will be mapped by upcoming surveys, a promising alternative 
to tightly constrain $\smnu$. 

% why bispectrum?
Previous works examining the impact of $\smnu$ on redshift-space matter
and halo clustering find that the imprint of $\smnu$ and $\sig$ on the power 
spectrum are degenerate and differ only by $< 1\%$.
Although this strong $\smnu$ -- $\sig$ degeneracy poses a serious limitation 
for constraining $\smnu$ with the power spectrum, information in the nonlinear 
regime cascade from the power spectrum to higher-order statistics such as the 
bispectrum. While previous works have demonstrated that the bispectrum
has comparable signal-to-noise as the power spectrum on nonlinear scales~\citep{sefusatti2005, chan2017}
and that it improves constraints on $\Lambda$CDM cosmological parameters~\citep{sefusatti2006, yankelevich2019}, 
no work to date has quantified the full information content and constraining 
power of the bispectrum down to nonlinear scales --- especially for $\smnu$. 

% what's our methodology?
%In this work, we examine the effect of massive neutrinos on the redshift-space 
%halo bispectrum using more than 42000 $N$-body simulations with massive neutrinos. 
%We first demonstrate that the bispectrum helps break the $\smnu$--$\sig$ degeneracy 
%found in the power spectrum. Then we present the full information content of the 
%bispectrum for all triangle configurations down to $k_{\rm max} = 0.5~\mpc$ using 
%a Fisher forecast where we estimate the covariance matrix and derivatives with 
%the large set of simulations from the Quijote suite (Villaescusa-Navarro et al. in preparation). 
%This paper is the first of a series of papers that aim to demonstrate the potential 
%of the galaxy bispectrum analysis in constraining $\smnu$. In the subsequent paper, 
%we will include HOD parameters in our forecasts to quantify the full information 
%content of the galaxy bispectrum. In the series, we will also present methods to 
%tackle challenges that come with analyzing the full galaxy bispectrum, such as data 
%compression for reducing the dimensionality of the bispectrum. 

\begin{itemize}
\item whats our methodlogy 
% what are our results? 
\begin{itemize}
    \item B helps break the $\smnu$-$\sig$ degeneracy 
    \item we quantify for the first time, the full information of the redshift-space bispectrum down to nonlinear scales 
    \item we demonstrate that the bispectrum improves all constraints 
\end{itemize}
% what are the caveats our results? 
\item paragraph on the caveats of our results. we don't marginalize over a full bias model yet and will do that in the next paper. sentence saying regardless our results are awesome.
% what are the implications of our results? 
\item implications for upcoming spectroscopic redshift surveys. We only use 1gpc. DESI PFS, Euclid and WFIRST are going to map out way more. In fact some of the main scientific goals of these surveys will be neutrinos. Based on our results, the bispectrum has the potentially to dramatically improve $\smnu$ constraints. Something about X sigma detection?  
\item 
\end{itemize}


%In this work, we examine the effect of massive neutrinos on the redshift-space 
%halo bispectrum using more than 23000 $N$-body simulations with massive neutrinos. 
%We first demonstrate that the bispectrum helps break the $\smnu$--$\sig$ degeneracy 
%found in the power spectrum. Then we present the full information content of the 
%bispectrum for all triangle configurations down to $k_{\rm max} = 0.5~\mpc$ using 
%a Fisher forecast where we estimate the covariance matrix and derivatives with 
%the large set of simulations from the Quijote suite (Villaescusa-Navarro et al. in preparation). 
%This paper is the first of a series of papers that aim to demonstrate the potential 
%of the galaxy bispectrum analysis in constraining $\smnu$. In the subsequent paper, 
%we will include HOD parameters in our forecasts to quantify the full information 
%content of the galaxy bispectrum. In the series, we will also present methods to 
%tackle challenges that come with analyzing the full galaxy bispectrum, such as data 
%compression for reducing the dimensionality of the bispectrum. 

%Massive neutrinos suppress the growth of structure below their free-streaming scale and leave an 
%imprint on large-scale structure. Measuring this imprint allows us to constrain the sum of neutrino 
%masses, $\smnu$, a key parameter in particle physics beyond the Standard Model. However, degeneracies 
%among cosmological parameters, especially between $\smnu$ and $\sig$, limit the constraining power 
%of standard two-point clustering statistics. In this work, we investigate whether we can break these 
%degerancies and constrain $\smnu$ with the next higher-order correlation function --- the bispectrum. 
%We first examine the redshift-space halo bispectrum of $800$ $N$-body simulations from the HADES suite 
%and demonstrate that the bispectrum helps break the $\smnu$--$\sig$ degeneracy. Then using over 23000
%$N$-body simulations of the Quijote suite, we quantify for the first time the full information content 
%of the redshift-space halo bispectrum using a Fisher matrix forecast of $\{\Om$, $\Ob$, $h$, $n_s$, $\sig$, $\smnu\}$ 
%down to nonlinear scales. For $k_{\rm max}{=}0.5~\mpc$, the bispectrum 
%constrains $\Om$, $\Ob$, $h$, $n_s$, and $\sig$ 3.1, 4.1, 5.1, 5.9, and 3.3 times tighter than the 
%power spectrum. For $\smnu$, the bispectrum improves the 1$\sigma$ constraint from 
%0.1962 to 0.0342 eV --- 6 times tigher than the power spectrum. Even with priors from {\em Planck}, 
%the bispectrum improves $\smnu$ constraints by a factor of 2.7. Although we reserve marginalizing 
%over bias parameters to the next paper of the series, these constraints are derived for a 
%$(1~h^{-1}{\rm Gpc})^3$ box, a substantially smaller volume than upcoming surveys. Thus, our results 
%demonstrate that the bispectrum offers significant improvements over the power spectrum, especially 
%for constraining $\smnu$.  
