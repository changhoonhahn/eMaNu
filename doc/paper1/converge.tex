\section{Fisher Forecasts using $N$-body simulations} \label{sec:numerical}
The two key elements in calculating the Fisher matrices in our bispectrum forecasts
are the bispectrum covariance matrix ($\bfi{C}$; Figure~\ref{fig:bk_cov}) and 
the derivatives of the bispectrum along the cosmological and nuisance parameters 
($\partial B_0/\partial \theta$; Figure~\ref{fig:bk_deriv}). We compute 
both these elements directly using the $N$-body simulations of the Quijote suite 
(Section~\ref{sec:hades}). This exploits the accuracy of $N$-body simulations in 
the nonlinear regime and allows us to accurately quantify the constraining power of 
the bispectrum beyond perturbation theory models. However, we must ensure that 
both $\bfi{C}$ and $\partial B_0/\partial \theta$ have converged and that numerical 
effects do not introduce any biases that impact our results. Below, we tests 
the convergence of $\bfi{C}$ and $\partial B_0/\partial \theta$ and discuss
some of the subtleties and caveats of our $\partial B_0/\partial \theta$ 
calculations. 

We use 15,000 Quijote $N$-body simulations at the fiducial cosmology to 
estimate $\bfi{C}$. This is a \emph{significantly} larger number of simulations 
than any previous bispectrum analyses; however, we also consider 1898 triangle 
configurations out to $k_{\rm max} = 0.5~\mpc$. For reference, \cite{gil-marin2017} 
recently used 2048 simulations to estimate the covariance matrix of the 
bispectrum with $825$ configurations. To check the convergence of covariance 
matrix, we vary the number of simulations used to estimate $\bfi{C}$, 
$N_{\rm cov}$, and determine whether this significantly impacts the elements 
of the Fisher matrix, $F_{ij}$, or the final marginalized Fisher parameter 
constraints, $\sigma_\theta$. In the left panel of Figure~\ref{fig:fij_converge}, 
we present the ratio between $F_{ij}(N_{\rm cov})$, $F_{ij}$ derived from 
$\bfi{C}$ calculated with $N_{\rm cov}$ simulations, and 
$F_{ij}(N_{\rm cov} = 15,000)$ for all 36 elements of the Fisher matrix. We 
shade $\pm5\%$ deviations in the ratios for reference. The $F_{ij}$ elements 
vary by $\lesssim 5\%$ for $N_{\rm cov} > 5000$ and $\lesssim 1\%$ for 
$N_{\rm cov} > 10,000$. Next, we present ratio between $\sigma_\theta(N_{\rm cov})$, 
the marginalized $1\sigma$ constraints for $\{\Om$, $\Ob$, $h$, $n_s$, $\sig$, $\smnu\}$ 
derived from $\bfi{C}$ calculated with $N_{\rm cov}$ simulations, and 
$\sigma_\theta(N_{\rm cov} = 15,000)$ in the left panel of Figure~\ref{fig:converge}. 
The constraints vary by $\lesssim 5\%$ for $N_{\rm cov} > 5000$ and $\lesssim 1\%$ 
for $N_{\rm cov} > 12000$. Hence, $N_{\rm cov} = 15,000$ is sufficient to 
accurately estimate $\bfi{C}$ and its convergence does not impact our 
forecasts.

We estimate $\partial B_0/\partial \theta$ using $N_{\rm deriv} = 1,500$ 
$N$-body simulations at 13 different cosmologies listed in Table~\ref{tab:sims} 
To check the convergence of $\partial B_0/\partial \theta$ and its impact on our 
results, we examine the ratio between $F_{ij}(N_{\rm deriv})$, the Fisher 
matrix element derived from $\partial B_0/\partial \theta$ calculated with 
$N_{\rm deriv}$ simulations, and $F_{ij}(N_{\rm deriv} = 1,500)$ for all 
36 elements of the Fisher matrix in the right panel of Figure~\ref{fig:fij_converge}.
For $N_{\rm deriv} > 1000$, $F_{ij}$ elements vary by $\lesssim 5\%$. Next, we
present the ratio between $\sigma_\theta(N_{\rm deriv})$, the marginalized 
$1\sigma$ constraints for $\{\Om$, $\Ob$, $h$, $n_s$, $\sig$, $\smnu\}$ 
derived from $\partial B_0/\partial \theta$ calculated with $N_{\rm deriv}$ 
simulations, and $\sigma_\theta(N_{\rm deriv} = 1,500)$ in the right panel 
of Figure~\ref{fig:converge}. Unlike $\sigma_\theta(N_{\rm cov})$, 
$\sigma_\theta(N_{\rm deriv})$ depend significantly on $\theta$. For instance, 
$\sigma_{\sig}$ and $\sigma_{\Om}$ vary by $\lesssim 10\%$ for $N_{\rm deriv} > 600$ 
and $\lesssim 1\%$ for $N_{\rm deriv} > 1200$. $\sigma_\theta$ for the other 
parameter vary significantly more. Nonetheless, for $N_{\rm deriv} > 800$ and 
$1200$ they vary by $\lesssim 10$ and $5\%$, respectively.

For $\Om$, $\Ob$, $h$, $n_s$, $\sig$, and also the nuisance parameter $M_{\rm lim}$ 
we estimate $\partial B_0/\partial \theta$ using a centered difference approximation 
(Eq.~\ref{eq:dbkdt}). However, for $\smnu$ we cannot have values below 0.0 eV 
and, thus, cannot estimate the derivative with the same method. If we use the 
forward difference approximation, 
\beq 
\frac{\partial \overline{B}_0}{\partial \smnu} \approx \frac{\overline{B}_0(\theta_{\rm fid}^{\rm ZA}+\delta \smnu) - \overline{B}_0(\smnu^{\rm fid})}{\delta \smnu}, 
\eeq
the error goes as $\mathcal{O}(\delta \smnu)$. Instead, we use a finite difference
approximation with the Quijote simulations at $\smnu^{+}$, $\smnu^{++}$, $\smnu^{+++}$, 
and the fiducial cosmology, a $\mathcal{O}(\delta \smnu^2)$ order approximation 
(Eq.~\ref{eq:dbkdmnu}). We can also use a finite difference approximation with 
simulations at $\smnu^{+}$, $\smnu^{++}$, and the fiducial cosmology. We examine
the stability of $\partial \log B_0/\partial M_\nu$ (right) and 
$\partial \log P_\ell/\partial M_\nu$ (left) by comparing the derivatives computing
using simulations at \{$\theta_{\rm fid}^{\rm ZA}$, $\smnu^{+}$, $\smnu^{++}$, $\smnu^{+++}$\} (blue), 
\{$\theta_{\rm fid}^{\rm ZA}$, $\smnu^{+}$, $\smnu^{++}$\} (orange), and \{$\theta_{\rm fid}^{\rm ZA}$, $\smnu^{+}$\} 
(green) in Figure~\ref{fig:dPBdmnu}. The three $\partial \log B_0/\partial \smnu$
approximations differ from one another by $\sim10\%$ with Eq.~\ref{eq:dbkdmnu} 
producing the largest estimate for both $P_\ell$ and $B_0$. If we use the
\{$\theta_{\rm fid}^{\rm ZA}$, $\smnu^{+}$, $\smnu^{++}$\} derivative and 
\{$\theta_{\rm fid}^{\rm ZA}$, $\smnu^{+}$\} derivative 
instead of Eq.~\ref{eq:dbkdmnu} for our Fisher forecasts, the marginalized constraint on
$\smnu$ for $k_{\rm max} = 0.5~\mpc$ increases to 0.390 and 0.682 eV for $P_\ell$ 
and 0.0754 and 0.1354 eV for $B_0$. Compared to our $\sigma_{\smnu} = 0.0572$ eV 
$B_0$ forecast, these correspond to a $\sim30$ and $130\%$ relative increase. 
While the derivative estimated from \{$\theta_{\rm fid}^{\rm ZA}$, $\smnu^{+}$\} significantly 
impact the forecasts, we emphasize that this is a $\mathcal{O}(\delta \smnu)$ approximation, 
unlike the other $\mathcal{O}(\delta \smnu^2)$ approximations. In fact, the 
\{$\theta_{\rm fid}^{\rm ZA}$, $\smnu^{+}$\} derivatives are better $\mathcal{O}(\delta \smnu^2)$ 
estimates for $\partial P_\ell/\partial \smnu$ and $\partial B_0/\partial \smnu$ at 0.05 eV. 
When we compare the derivative of the linear theory power spectrum, $P^{(\rm LT)}$, 
we find that $\partial P^{(\rm LT)}/\partial \smnu$ at 0.0 is larger than at 0.05 eV. 
Hence, the differences between the \{$\theta_{\rm fid}^{\rm ZA}$, $\smnu^{+}$\} derivatives 
and the other derivatives are not solely due to numerical stability. Moreover, because the 
discrepancies in the derivative propagate similarly to both $P_\ell$ and $B_0$ forecasts, 
the relative improvement of $B_0$ over $P_\ell$ remains roughly the same.
Hence, we conclude that the derivatives with respect to $\smnu$ are sufficiently 
stable and robust for our Fisher forecasts.

\begin{figure}
\begin{center}
    \includegraphics[width=0.75\textwidth]{figs/quijote_bkFij_convergence_reg_dmnu_fin_kmax0_5.pdf}
    \caption{
        {\em Left}: The convergence of all 36 Fisher matrix elements, $F_{ij}$, as 
        as a function of $N_{\rm cov}$, the number of $N$-body simulations used to 
        estimate the covariance matrix, $\bfi{C}$. We present the ratio between 
        $F_{ij}(N_{\rm cov})$, $F_{ij}$ derived from $\bfi{C}$ calculated with 
        $N_{\rm cov}$ simulations, and $F_{ij}(N_{\rm cov} = 15,000)$. We mark 
        $\pm5\%$ deviations in the ratios with the shaded regions for reference. 
        All $F_{ij}$ elements vary by $\lesssim 5\%$ for $N_{\rm cov} > 5000$ and 
        $\lesssim 1\%$ for $N_{\rm cov} > 10,000$.
        {\em Right}: The convergence of $F_{ij}$ as a function of $N_{\rm deriv}$, 
        the number of $N$-body simulations used to estimate the derivatives, 
        $\partial B_0/\partial \theta$. We plot the ratios between $F_{ij}(N_{\rm deriv})$ 
        and $F_{ij}(N_{\rm deriv} = 1,500)$. All $F_{ij}$ elements vary by 
        $\lesssim 5\%$ for $N_{\rm deriv} > 1000$. Hence, {\em $N_{\rm cov} = 15,000$ 
        and $N_{\rm deriv} = 1,500$ are sufficient and the convergence of $\bfi{C}$ 
        and $\partial B_0/\partial \theta$ does not impact $F_{ij}$.}
    }
\label{fig:fij_converge}
\end{center}
\end{figure}

\begin{figure}
\begin{center}
    \includegraphics[width=0.75\textwidth]{figs/quijote_bkFisher_convergence_reg_dmnu_fin_kmax0_5.pdf} 
    \caption{{\em Left}: The convergence of the marginalized 1$\sigma$ constraints
    for \{$\Om$, $\Ob$, $h$, $n_s$, $\sig$, $\smnu$\}, $\sigma_\theta$, as a function of 
    $N_{\rm cov}$, the number of Quijote simulations used to estimate $\bfi{C}$. We 
    plot the ratio between $\sigma_\theta(N_{\rm cov})$, derived from $\bfi{C}$ with 
    $N_{\rm cov}$ simulations, and $\sigma_\theta(N_{\rm cov} = 15,000)$. $\sigma_\theta$ vary by $\lesssim 5$ and $1\%$ for 
    $N_{\rm cov} > 5000$ and $12,000$, respectively. 
    {\em Right}: The convergence of $\sigma_\theta$ as a function of $N_{\rm deriv}$, 
    the number of simulated used to estimate $\partial B_0/\partial \smnu$. We plot 
    the ratio between $\sigma_\theta(N_{\rm deriv})$, derived from $\partial B/\partial \smnu$
    with $N_{\rm deriv}$ simulations, and $\sigma_\theta(N_{\rm deriv} = 1,500)$. Although 
    $\sigma_\theta(N_{\rm deriv})/\sigma_\theta(N_{\rm deriv} = 1,500)$ vary among 
    the parameters, the ratio vary by $\lesssim 10$ and $5\%$ for $N_{\rm deriv} > 800$ 
    and $1200$, respectively. Hence, {\em we have a sufficient number of simulations 
    to estimate $\bfi{C}$ and the derivatives of the bispectrum and our forecasts are 
    robust to their convergence.} 
    }
\label{fig:converge}
\end{center}
\end{figure}

\begin{figure}
\begin{center}
    \includegraphics[width=\textwidth]{figs/quijote_dPBdMnu_dmnu_reg.pdf} 
    \caption{Comparison of $\partial \log B_0(k_1, k_2, k_3)/\partial M_\nu$ (right) 
    and $\partial \log P_\ell(k)/\partial M_\nu$ (left), computed using Quijote simulations 
    at \{$\theta_{\rm fid}^{\rm ZA}$, $\smnu^{+}$, $\smnu^{++}$, $\smnu^{+++}$\} (blue), 
    \{$\theta_{\rm fid}^{\rm ZA}$, $\smnu^{+}$, $\smnu^{++}$\} (orange), and \{$\theta_{\rm fid}^{\rm ZA}$, $\smnu^{+}$\} 
    (green). The derivative approximations differ from one another by $\sim10\%$ with 
    Eq.~\ref{eq:dbkdmnu} producing the largest estimate for both $P_\ell$ and $B_0$.
    Using the \{$\theta_{\rm fid}^{\rm ZA}$, $\smnu^{+}$, $\smnu^{++}$\} derivatives instead of the 
    Eq.~\ref{eq:dbkdmnu} derivatives, increases the marginalized constraint on $\smnu$ 
    by $\sim 30\%$. However, the differences in the derivatives propagate similarly to 
    the $P_\ell$ and $B_0$ forecasts so the relative improvement of $B_0$ over $P_\ell$
    remains the same. Hence, we conclude that {\em the derivatives with respect to $\smnu$ 
    are sufficiently stable and robust for our Fisher forecasts}.
    }
\label{fig:dPBdmnu}
\end{center}
\end{figure}
