\section{Fisher Forecasts using $N$-body simulations} \label{sec:numerical}
The two key elements in calculating the Fisher matrices in our bispectrum forecasts
are the bispectrum covariance matrix ($\bfi{C}$; Figure~\ref{fig:bk_cov}) and 
the derivatives of the bispectrum along the cosmological and nuisance parameters 
($\partial B_0/\partial \theta$; Figure~\ref{fig:bk_deriv}). We compute 
both these elements directly using the $N$-body simulations of the Quijote suite 
(Section~\ref{sec:hades}). This exploits the accuracy of $N$-body simulations in 
the nonlinear regime and allows us to accurately quantify the constraining power of 
the bispectrum beyond perturbation theory models. However, we must ensure that 
both $\bfi{C}$ and $\partial B_0/\partial \theta$ have converged and that numerical 
effects do not introduce any biases that impact our results. Below, we tests 
the convergence of $\bfi{C}$ and $\partial B_0/\partial \theta$ and discuss
some of the subtleties and caveats of our $\partial B_0/\partial \theta$ 
calculations. 

We use 15,000 Quijote $N$-body simulations at the fiducial cosmology to 
estimate $\bfi{C}$. This is a \emph{significantly} larger number of simulations 
than any previous bispectrum analyses; however, we also consider 1898 triangle 
configurations out to $k_{\rm max} = 0.5~\mpc$. For reference, \cite{gil-marin2017} 
recently used 2048 simulations to estimate the covariance matrix of the 
bispectrum with $825$ configurations. To check the convergence of covariance 
matrix, we vary the number of simulations used to estimate $\bfi{C}$, 
$N_{\rm cov}$, and see whether this significantly impacts the elements 
of the Fisher matrix, $F_{ij}$, or the final marginalized Fisher parameter 
constraints, $\sigma_\theta$. In the left panel of Figure~\ref{fig:fij_converge}, 
we present the ratio between $F_{ij}(N_{\rm cov})$, $F_{ij}$ derived from 
$\bfi{C}$ calculated with $N_{\rm cov}$ simulations, and 
$F_{ij}(N_{\rm cov} = 15,000)$ for all 36 elements of the Fisher matrix. We 
shade $\pm5\%$ deviations in the ratios for reference. The $F_{ij}$ elements 
vary by $\lesssim 5\%$ for $N_{\rm cov} > 5000$ and $\lesssim 1\%$ for 
$N_{\rm cov} > 10,000$. Next, we present ratio between $\sigma_\theta(N_{\rm cov})$, 
the marginalized $1\sigma$ constraints for $\{\Om$, $\Ob$, $h$, $n_s$, $\sig$, $\smnu\}$ 
derived from $\bfi{C}$ calculated with $N_{\rm cov}$ simulations, and 
$\sigma_\theta(N_{\rm cov} = 15,000)$ in the left panel of Figure~\ref{fig:converge}. 
The constraints vary by $\lesssim 5\%$ for $N_{\rm cov} > 5000$ and $\lesssim 1\%$ 
for $N_{\rm cov} > 12000$. Hence, $N_{\rm cov} = 15,000$ is sufficient to 
accurately estimate $\bfi{C}$ and its convergence does not impact our 
forecasts.

We estimate $\partial B_0/\partial \theta$ using $N_{\rm deriv} = 1,500$ 
$N$-body simulations at 13 different cosmologies listed in Table~\ref{tab:sims} 
To check the convergence of $\partial B_0/\partial \theta$ and its impact on our 
results, we examine the ratio between $F_{ij}(N_{\rm deriv})$, the Fisher 
matrix element derived from $\partial B_0/\partial \theta$ calculated with 
$N_{\rm deriv}$ simulations, and $F_{ij}(N_{\rm deriv} = 1,500)$ for all 
36 elements of the Fisher matrix in the right panel of Figure~\ref{fig:fij_converge}.
For $N_{\rm cov} > 1000$, $F_{ij}$ elements vary by $\lesssim 5\%$. Next, we
present the ratio between $\sigma_\theta(N_{\rm deriv})$, the marginalized 
$1\sigma$ constraints for $\{\Om$, $\Ob$, $h$, $n_s$, $\sig$, $\smnu\}$ 
derived from $\partial B_0/\partial \theta$ calculated with $N_{\rm deriv}$ 
simulations, and $\sigma_\theta(N_{\rm deriv} = 1,500)$ in the right panel 
of Figure~\ref{fig:converge}. Unlike $\sigma_\theta(N_{\rm cov})$, 
$\sigma_\theta(N_{\rm deriv})$ depend significantly on $\theta$. For instance, 
\ch{$\sigma_\sig$ and $\sigma_\Om$ vary by $< 10\%$ for $N_{\rm deriv} > 300$ 
and $< 2\%$ for $N_{\rm fp} > 450$. $\sigma_\theta$ for the other parameter 
vary significantly more. Nonetheless, for $N_{\rm deriv} > 400$ and $450$ they vary by 
$< 10$ and $5\%$, respectively.}

For $\Om$, $\Ob$, $h$, $n_s$, $\sig$, and also the nuisance parameter $M_{\rm lim}$ 
we estimate $\partial B_0/\partial \theta$ using a centered difference approximation 
(Eq.~\ref{eq:dbkdt}). However, for $\smnu$ we cannot have values below 0.0 eV 
and, thus, cannot estimate the derivative with the same method. If we use the 
forward difference approximation, 
\beq 
\frac{\partial \overline{B}_0}{\partial \smnu} \approx \frac{\overline{B}_0(\smnu^{\rm fid}+\delta \smnu) - \overline{B}_0(\smnu^{\rm fid})}{\delta \smnu}, 
\eeq
the error goes as $\mathcal{O}(\delta \smnu)$. Instead, we use the Quijote 
simulations at $\smnu^{+}$, $\smnu^{++}$, $\smnu^{+++}$, and the fiducial 
cosmology in Eq.~\ref{eq:dbkdmnu}, a $\mathcal{O}(\delta \smnu^2)$ order 
approximation. 
%%%% continue editing 06/19/2019 %%%%
We compare 
$\partial \log B(k_1, k_2, k_3)/\partial M_\nu$ (right) and 
$\partial \log P(k)/\partial M_\nu$ (left), computed using Eq.~\ref{eq:dbkdmnu} (blue) 
and the forward difference approximation (green) in Figure~\ref{fig:dPBdmnu}.
We also include $P$ and $B$ derivatives approximated using only simulations at $\smnu^{+}$,  
$\smnu^{++}$, and the fiducial cosmology in orange. The three approximations for the 
derivatives differ from one another by roughly $10\%$ with Eq.~\ref{eq:dbkdmnu} producing 
the largest estimate for both $P_0$ and $B_0$. If we use the $\smnu^{+}$ and $\smnu^{++}$ 
and forward difference derivatives instead of the Eq.~\ref{eq:dbkdmnu} for our Fisher 
forecasts, we find the following marginalized $\smnu$ constraints for $k_{\rm max} = 0.5~h/{\rm Mpc}$: 
$0.196$ and $0.294$eV for $P_0$ and  $0.0308$ and $0.0483$eV for $B_0$. These correspond 
to a $\sim20$ and $80\%$ relative increase from our forecasts in Section~\ref{sec:forecasts}. 
The forward difference derivatives have a significant impact on our forecasts; however, 
we emphasize that this is a 
$\mathcal{O}(\delta \smnu)$ approximation, unlike the other $\mathcal{O}(\delta \smnu^2)$ 
approximations. Moreover, because the discrepancies in the derivative propogate similarly
to the $P_0$ and $B_0$ constraints, the relative improvement of $B_0$ over $P_0$ remains
roughly the same. Hence we conclude that the derivatives have sufficiently converged and 
robust for our Fisher forecasts.

\begin{figure}
\begin{center}
    \includegraphics[width=0.75\textwidth]{figs/quijote_bkFij_convergence_reg_dmnu_fin_kmax0_5.pdf}
    \caption{{\em Left}: {\em Right}:}
\label{fig:fij_converge}
\end{center}
\end{figure}

\begin{figure}
\begin{center}
    \includegraphics[width=0.8\textwidth]{figs/quijote_bkFisher_convergence_dmnu_fin_kmax05.pdf}
    \caption{{\bf Left}: The ratio of the 1$\sigma$ Fisher constraint for $\theta=\Om$, 
    $\Ob$, $h$, $n_s$, $\sig$, and $\smnu$ calculated using $N_{\rm fid}$ Quijote 
    simulations over the constraint calculated with all 15,000 simulations, 
    $\sigma_\theta(N_{\rm fid})/\sigma_\theta(N_{\rm fid} = 15,000)$, as a 
    function of $N_{\rm fid}$. The $N_{\rm fid}$ simulations are used to estimate 
    $\bfi{C}$ used to calculate the Fisher matrix~(Eq.~\ref{eq:fisher}). The Fisher
    parameter constraints vary by $<10$ and $1\%$ for $N_{\rm fid} > 7000$ and 
    $14,000$, respectively. 
    {\bf Right}: The ratio of the 1$\sigma$ Fisher constraint for $\theta$ 
    calculated using  $N_{\rm fp}$ simulations over the constraint calculated with 
    all 500 fixed paired simulations, $\sigma_\theta(N_{\rm fp})/\sigma_\theta(N_{\rm fp} = 500)$, 
    as a function of $N_{\rm fp}$. The $N_{\rm fp}$ fixed paired simulations are 
    used to numerically estimate $\partial B_0/\partial \theta_i$ in Eq.~\ref{eq:fisher}. 
    Although $\sigma_\theta(N_{\rm fp})/\sigma_\theta(N_{\rm fp} = 500)$ vary among 
    the parameters, for $N_{\rm fp} > 400$ and $450$ they vary by $< 10$ and $5\%$, 
    respectively. Hence, {\em we have a sufficient number of simulations to estimate 
    $\bfi{C}$ and the derivatives of the bipsectrum and our forecasts are robust to 
    their convergence.} 
    }
\label{fig:converge}
\end{center}
\end{figure}

\begin{figure}
\begin{center}
    \includegraphics[width=\textwidth]{figs/quijote_dPBdMnu_dmnu_reg.pdf} 
    \caption{Comparison of $\partial \log B(k_1, k_2, k_3)/\partial M_\nu$ (right) 
    and $\partial \log P(k)/\partial M_\nu$ (left), computed using Eq.\ref{eq:dbkdmnu} (blue), 
    excluding $\smnu^{+++}$ (orange), and the forward difference approximation (green). 
    }
\label{fig:dPBdmnu}
\end{center}
\end{figure}
