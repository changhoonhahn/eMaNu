\section{Fisher Forecasts using $N$-body simulations} \label{sec:numerical}
The two key elements in calculating the Fisher matrices in our bispectrum forecasts
are the bispectrum covariance matrix ($\bfi{C}$; Figure~\ref{fig:bk_cov}) and 
the derivatives of the bispectrum along the cosmological and nuisance parameters 
($\partial B_0/\partial \theta$; Figure~\ref{fig:bk_deriv}). We compute 
both these elements directly using the $N$-body simulations of the Quijote suite 
(Section~\ref{sec:hades}). This exploits the accuracy of $N$-body simulations in 
the nonlinear regime and allows us to accurately quantify the constraining power of 
the bispectrum beyond perturbation theory models. However, we must ensure that 
both $\bfi{C}$ and $\partial B_0/\partial \theta$ have converged and that numerical 
effects do not introduce any biases that impact our results. Below, we tests 
the convergence of $\bfi{C}$ and $\partial B_0/\partial \theta$ and discuss
some of the subtleties and caveats of our $\partial B_0/\partial \theta$ 
calculations. 

We use 15,000 Quijote $N$-body simulations at the fiducial cosmology to 
estimate $\bfi{C}$. This is a \emph{significantly} larger number of simulations 
than any previous bispectrum analyses; however, we also consider 1898 triangle 
configurations out to $k_{\rm max} = 0.5~\mpc$. For reference, \cite{gil-marin2017} 
recently used 2048 simulations to estimate the covariance matrix of the 
bispectrum with $825$ configurations. We check the convergence of covariance 
matrix by varying the number of simulations used to estimate $\bfi{C}$, 
$N_{\rm cov}$,and examining whether this significantly impacts the elements of 
the Fisher matrix, $F_{ij}$ or the ultimate Fisher parameter constraints. 
We present the ratio between $F_{ij}(N_{\rm cov})$, the Fisher matrix element 
derived from $\bfi{C}$ calculated using $N_{\rm cov}$ simulations, and 
$F_{ij}(N_{\rm cov} = 15,000)$ for all 36 elements of the matrix in 
the left panel of Figure~\ref{fig:fij_converge}. We shade $\pm5\%$ deviations 
in the ratios for reference. $F_{ij}$ varies by $\lesssim 5\%$ for $N_{\rm cov} > 5000$
and $\lesssim 1\%$ for $N_{\rm cov} > 10,000$. Next, we present ratio between 
$\sigma_\theta(N_{\rm cov})$, the marginalized $1\sigma$ constraints for 
$\{\Om$, $\Ob$, $h$, $n_s$, $\sig$, $\smnu\}$ derived from $\bfi{C}$ 
calculated using $N_{\rm cov}$ simulations, and $\sigma_\theta(N_{\rm cov} = 15,000)$ 
in the left panel of Figure~\ref{fig:converge}. $\sigma_\theta$ vary by 
$\lesssim 5\%$ for $N_{\rm cov} > 5000$ and $\lesssim 1\%$ for $N_{\rm cov} > 12000$. 
Hence, $N_{\rm cov} = 15,000$ is sufficient to accurately estimate $\bfi{C}$
and the convergence of $\bfi{C}$ does not impact our forecasts.

%%% continue edits from here 06/19/2019 
We estimate $\partial B_0/\partial \theta$ using $N_{\rm deriv} = 1,500$ 
$N$-body simulations at 13 different cosmologies listed in Table~\ref{tab:sims} 
To check the convergence of $\partial B_0/\partial \theta$ and its impact on our 
results, we present the ratio of the 1$\sigma$ Fisher constraint for $\theta$ calculated 
using $N_{\rm fp}$ simulations over the constraint calculated with $N_{\rm fp} = 500$, 
$\sigma_\theta(N_{\rm fp})/\sigma_\theta(N_{\rm fp} = 500)$, as a function of 
$N_{\rm fp}$ (Figure~\ref{fig:converge} right panel). Unlike $\sigma_\theta(N_{\rm fid})$, 
$\sigma_\theta(N_{\rm fp})$ depend significantly on $\theta$. For instance, 
$\sigma_\theta$ for $\sig$ and $\Om$ vary by $< 10\%$ for $N_{\rm fp} > 300$ 
and $< 2\%$ for $N_{\rm fp} > 450$. $\sigma_\theta$ for the other parameter 
vary significantly more. Nonetheless, for $N_{\rm fp} > 400$ and $450$ they vary by 
$< 10$ and $5\%$, respectively. 

% Fij ratio for Nderiv
%[[A[100, 200, 400, 600, 800, 1000, 1200, 1400, 1500]
%--- $\Omega_m$,$\Omega_m$ ---
%[1.04150836 1.02079794 1.00987824 1.0026939  0.99886825 0.99913592 0.99789147 0.99990023 1.        ]
%--- $\Omega_m$,$\Omega_b$ ---
%[1.01348548 1.01308594 1.00136552 0.99852557 0.9937658  0.99689605 0.99705162 0.99710909 1.        ]
%--- $\Omega_m$,$h$ ---
%[1.00063037 1.0002033  1.00621237 0.99876067 0.99968388 0.99881756 0.99846445 1.00143804 1.        ]
%--- $\Omega_m$,$n_s$ ---
%[1.00029743 1.00592034 1.01000104 1.00208885 0.99976727 1.00097346 0.99911279 1.00062347 1.        ]
%--- $\Omega_m$,$\sigma_8$ ---
%[0.99593048 1.00950314 1.00040767 0.99010904 0.98931399 0.99649563 1.0015535  1.00218921 1.        ]
%--- $\Omega_m$,$M_\nu$ ---
%[0.98858663 1.0219267  1.0046164  1.00134123 1.00401471 1.00587072 1.00158579 1.00199991 1.        ]
%--- $\Omega_b$,$\Omega_b$ ---
%[1.6755022  1.31450658 1.10084259 1.04934085 1.02531377 1.00989552 1.00859403 1.00130831 1.        ]
%--- $\Omega_b$,$h$ ---
%[1.01044859 0.9980883  1.00552477 0.9954082  0.99495123 0.99695468 0.99810241 0.99910949 1.        ]
%--- $\Omega_b$,$n_s$ ---
%[0.99762807 1.0101526  1.00461862 0.99785162 0.99438066 0.99819648 0.9985386  0.99746637 1.        ]
%--- $\Omega_b$,$\sigma_8$ ---
%[0.88917392 1.04979285 1.01363184 0.98707969 0.98285492 0.99123858 0.99641063 0.99859811 1.        ]
%--- $\Omega_b$,$M_\nu$ ---
%[0.91143507 1.01030528 1.00556617 0.98751959 0.99935747 1.00978767 1.0035679  0.99834964 1.        ]
%--- $h$,$h$ ---
%[1.16841584 1.0634469  1.03104311 1.01010877 1.00884495 1.00340538 1.00212079 1.00407741 1.        ]
%--- $h$,$n_s$ ---
%[1.01314326 1.0061738  1.01208516 1.00256109 1.00138005 1.00089581 1.00100134 1.00241085 1.        ]
%--- $h$,$\sigma_8$ ---
%[0.915567   0.9971496  1.01038432 0.9813361  0.98721846 0.99421008 1.00069377 1.00529848 1.        ]
%--- $h$,$M_\nu$ ---
%[0.96294557 1.02252845 1.01157962 0.99920413 1.01217822 1.00904731 1.00540234 1.00459001 1.        ]
%--- $n_s$,$n_s$ ---
%[1.10258408 1.05275718 1.02901861 1.01262656 1.00553745 1.00440912 1.00268721 1.00193929 1.        ]
%--- $n_s$,$\sigma_8$ ---
%[0.913902   1.00637357 1.0212924  0.98751793 0.99287654 0.99884804 0.9994306  1.00427712 1.        ]
%--- $n_s$,$M_\nu$ ---
%[0.96637896 1.02963955 1.00728284 1.00214753 1.00902491 1.0117498 1.00510923 1.00368202 1.        ]
%--- $\sigma_8$,$\sigma_8$ ---
%[1.59264666 1.23442712 1.06991172 1.03878806 1.02039282 1.0110555 1.00926118 1.00080884 1.        ]
%--- $\sigma_8$,$M_\nu$ ---
%[1.17573317 1.12574324 1.05511212 1.00759508 0.98095207 0.99292696 1.00026365 1.00006001 1.        ]
%--- $M_\nu$,$M_\nu$ ---
%[3.6368878  2.13824878 1.36119227 1.19264217 1.12645384 1.06109267 1.04652736 1.02067508 1.        ]

For $\Om$, $\Ob$, $h$, $n_s$, $\sig$, and $M_{\rm lim}$ we estimate 
$\partial B_0/\partial \theta_i$ using a centered difference approximation 
(Eq.~\ref{eq:dbkdt}). However, for $\smnu$, where we cannot have values below 
0.0 eV, we cannot estimate the derivative with the same method. If we use the 
forward difference approximation, 
\beq 
\frac{\partial \overline{B}_0}{\partial \smnu} \approx \frac{\overline{B}_0(\smnu^{\rm fid}+\delta \smnu) - \overline{B}_0(\smnu^{\rm fid})}{\delta \smnu}, 
\eeq
the error goes as $\mathcal{O}(\delta \smnu)$. Instead, we use Eq.~\ref{eq:dbkdmnu},
which provides a $\mathcal{O}(\delta \smnu^2)$ order approximation. In our 
$\partial B_0/\partial \smnu$ approximation, we use the Quijote simulations at 
$\smnu^{+}$, $\smnu^{++}$, and $\smnu^{+++}$. We compare 
$\partial \log B(k_1, k_2, k_3)/\partial M_\nu$ (right) and 
$\partial \log P(k)/\partial M_\nu$ (left), computed using Eq.~\ref{eq:dbkdmnu} (blue) 
and the forward difference approximation (green) in Figure~\ref{fig:dPBdmnu}.
We also include $P$ and $B$ derivatives approximated using only the $\smnu^{+}$ and 
$\smnu^{++}$ simulations in orange. \ch{update numbers:} The three approximations for the 
derivatives differ from one another by roughly $10\%$ with Eq.~\ref{eq:dbkdmnu} producing 
the largest estimate for both $P_0$ and $B_0$. If we use the $\smnu^{+}$ and $\smnu^{++}$ 
and forward difference derivatives instead of the Eq.~\ref{eq:dbkdmnu} for our Fisher 
forecasts, we find the following marginalized $\smnu$ constraints for $k_{\rm max} = 0.5~h/{\rm Mpc}$: 
$0.196$ and $0.294$eV for $P_0$ and  $0.0308$ and $0.0483$eV for $B_0$. These correspond 
to a $\sim20$ and $80\%$ relative increase from our forecasts in Section~\ref{sec:forecasts}. 
The forward difference derivatives have a significant impact on our forecasts; however, 
we emphasize that this is a 
$\mathcal{O}(\delta \smnu)$ approximation, unlike the other $\mathcal{O}(\delta \smnu^2)$ 
approximations. Moreover, because the discrepancies in the derivative propogate similarly
to the $P_0$ and $B_0$ constraints, the relative improvement of $B_0$ over $P_0$ remains
roughly the same. Hence we conclude that the derivatives have sufficiently converged and 
robust for our Fisher forecasts.

\begin{figure}
\begin{center}
    \includegraphics[width=0.75\textwidth]{figs/quijote_bkFij_convergence_reg_dmnu_fin_kmax0_5.pdf}
    \caption{{\em Left}: {\em Right}:}
\label{fig:fij_converge}
\end{center}
\end{figure}

\begin{figure}
\begin{center}
    \includegraphics[width=0.8\textwidth]{figs/quijote_bkFisher_convergence_dmnu_fin_kmax05.pdf}
    \caption{{\bf Left}: The ratio of the 1$\sigma$ Fisher constraint for $\theta=\Om$, 
    $\Ob$, $h$, $n_s$, $\sig$, and $\smnu$ calculated using $N_{\rm fid}$ Quijote 
    simulations over the constraint calculated with all 15,000 simulations, 
    $\sigma_\theta(N_{\rm fid})/\sigma_\theta(N_{\rm fid} = 15,000)$, as a 
    function of $N_{\rm fid}$. The $N_{\rm fid}$ simulations are used to estimate 
    $\bfi{C}$ used to calculate the Fisher matrix~(Eq.~\ref{eq:fisher}). The Fisher
    parameter constraints vary by $<10$ and $1\%$ for $N_{\rm fid} > 7000$ and 
    $14,000$, respectively. 
    {\bf Right}: The ratio of the 1$\sigma$ Fisher constraint for $\theta$ 
    calculated using  $N_{\rm fp}$ simulations over the constraint calculated with 
    all 500 fixed paired simulations, $\sigma_\theta(N_{\rm fp})/\sigma_\theta(N_{\rm fp} = 500)$, 
    as a function of $N_{\rm fp}$. The $N_{\rm fp}$ fixed paired simulations are 
    used to numerically estimate $\partial B_0/\partial \theta_i$ in Eq.~\ref{eq:fisher}. 
    Although $\sigma_\theta(N_{\rm fp})/\sigma_\theta(N_{\rm fp} = 500)$ vary among 
    the parameters, for $N_{\rm fp} > 400$ and $450$ they vary by $< 10$ and $5\%$, 
    respectively. Hence, {\em we have a sufficient number of simulations to estimate 
    $\bfi{C}$ and the derivatives of the bipsectrum and our forecasts are robust to 
    their convergence.} 
    }
\label{fig:converge}
\end{center}
\end{figure}

\begin{figure}
\begin{center}
    \includegraphics[width=\textwidth]{figs/quijote_dPBdMnu_dmnu_reg.pdf} 
    \caption{Comparison of $\partial \log B(k_1, k_2, k_3)/\partial M_\nu$ (right) 
    and $\partial \log P(k)/\partial M_\nu$ (left), computed using Eq.\ref{eq:dbkdmnu} (blue), 
    excluding $\smnu^{+++}$ (orange), and the forward difference approximation (green). 
    }
\label{fig:dPBdmnu}
\end{center}
\end{figure}
