\documentclass[12pt, letterpaper, preprint]{aastex62}
%%% This file is generated by the Makefile.
\newcommand{\giturl}{\url{https://github.com/changhoonhahn/eMaNu}}
\newcommand{\githash}{112c8e2}\newcommand{\gitdate}{2019-09-20}\newcommand{\gitauthor}{changhoonhahn}

%\usepackage[breaklinks,colorlinks, urlcolor=blue,citecolor=blue,linkcolor=blue]{hyperref}
\usepackage{color}
\usepackage{amsmath}
\usepackage{natbib}
\usepackage{ctable}
\usepackage{bm}
\usepackage[normalem]{ulem} % Added by MS for \sout -> not required for final version
\usepackage{xspace}

% typesetting shih
\linespread{1.08} % close to 10/13 spacing
\setlength{\parindent}{1.08\baselineskip} % Bringhurst
\setlength{\parskip}{0ex}
\let\oldbibliography\thebibliography % killin' me.
\renewcommand{\thebibliography}[1]{%
  \oldbibliography{#1}%
  \setlength{\itemsep}{0pt}%
  \setlength{\parsep}{0pt}%
  \setlength{\parskip}{0pt}%
  \setlength{\bibsep}{0ex}
  \raggedright
}
\setlength{\footnotesep}{0ex} % seriously?

% citation alias

% math shih
\newcommand{\setof}[1]{\left\{{#1}\right\}}
\newcommand{\given}{\,|\,}
\newcommand{\lss}{{\small{LSS}}\xspace}

\newcommand{\Om}{\Omega_{\rm m}} 
\newcommand{\Ob}{\Omega_{\rm b}} 
\newcommand{\OL}{\Omega_\Lambda}
\newcommand{\smnu}{\sum m_\nu} 
\newcommand{\sig}{\sigma_8} 
\newcommand{\hmpc}{\,h/\mathrm{Mpc}}

\newcommand{\specialcell}[2][c]{%
  \begin{tabular}[#1]{@{}c@{}}#2\end{tabular}}
% text shih
\newcommand{\foreign}[1]{\textsl{#1}}
\newcommand{\bfi}[1]{\textbf{\textit{#1}}}
\newcommand{\etal}{\foreign{et~al.}}
\newcommand{\opcit}{\foreign{Op.~cit.}}
\newcommand{\documentname}{\textsl{Article}}
\newcommand{\equationname}{equation}
\newcommand{\bitem}{\begin{itemize}}
\newcommand{\eitem}{\end{itemize}}
\newcommand{\beq}{\begin{equation}}
\newcommand{\eeq}{\end{equation}}

%% MS: To make collaborating easier
\definecolor{orange}{rgb}{1,0.5,0}

%%% Here are the things that make collaborative effort easier
\newcommand{\ch}[1]{{\color{orange}{\bf CH:} #1}}

\begin{document}\sloppy\sloppypar\frenchspacing 

\title{Breaking $\smnu$ Parameter Degeneracies with the Bispectrum} 
\date{\texttt{DRAFT~---~\githash~---~\gitdate~---~NOT READY FOR DISTRIBUTION}}

\newcounter{affilcounter}
\author{ChangHoon Hahn}
\altaffiliation{hahn.changhoon@gmail.com}
\affil{Lawrence Berkeley National Laboratory, 1 Cyclotron Rd, Berkeley CA 94720, USA}
\affil{Berkeley Center for Cosmological Physics, University of California, Berkeley, CA 94720, USA}

\author{Francisco Villaescusa-Navarro} 
\affil{Center for Computational Astrophysics, Flatiron Institute, 162 5th Avenue, New York, NY 10010, USA} 

\author{Emanuele Castorina} 
\affil{Berkeley Center for Cosmological Physics, University of California, Berkeley, CA 94720, USA}
\affil{Lawrence Berkeley National Laboratory, 1 Cyclotron Rd, Berkeley CA 94720, USA}

\author{Roman Scoccimarro} 
\affil{Center for Cosmology and Particle Physics, Department of Physics, New York University, NY 10003, New York, USA}

\begin{abstract}
    abstract
\end{abstract}

\keywords{
cosmology: 
---
}

\section{Introduction}
very brief into on neutrinos

Brief intro on the impact of massive active neutrinos on the matter powerspectrum 
and how that's detectable with CMB and LSS


Quick summary of current constraints and where they come from. Talk about the 
CMB-LSS lever arm. The degeneracy between As and tau and how that's a bottleneck
short thing about how $\tau$ is hard to constrain.

Fortunately the imprint of neutrinos on the matter distribution leaves imprints
on clustering. So with clustering measurements alone we can derive constraints 
on $\smnu$ and at the very least tighten constraints.  

Brief summary of previous works that look at the powerspectrum. Then Discuss the 
shortcomings of the powerspectrum only analysis-- Not good enough. 

However, we don't have to settle for just two point statistics, three-point 
statistics such as the bispectrum and 3PCF... 

In Section blah 


%%%%%%%%%%%%%%%%%%%%%%%%%%%%%%%%%%%%%%%%%%%%%%%%%%%%%%%%%%%%%%%%%%%%%%%%
% Section   
%%%%%%%%%%%%%%%%%%%%%%%%%%%%%%%%%%%%%%%%%%%%%%%%%%%%%%%%%%%%%%%%%%%%%%%%
\section{HADES and Quijote Simulation Suites} \label{sec:hades} 
We use a subset of the HADES\footnote{https://franciscovillaescusa.github.io/hades.html} 
and Quijote simulation suites. Below, we briefly describe these simulations; 
a brief summary of the simulations can be found in Table~\ref{tab:sims}. 
The HADES simulations start from Zel'dovich approximated initial conditions 
generated at $z=99$ using the~\cite{zennaro2017a} rescaling method and follow 
the gravitational evolution of $N_{\rm cdm}=512^3$ CDM, plus $N_{\nu}=512^3$ 
neutrino particles (for massive neutrino models), to $z=0$. They are run using 
the {\sc GADGET-III} TreePM+SPH code~\citep{springel2005} in a periodic 
$(1 h^{-1}{\rm Gpc})^3$ box. All of the HADES simulations share the following 
cosmological parameter values, which are in good agreement with Planck 
constraints~\cite{ade2016a}: $\Om{=}0.3175, \Ob{=}0.049, \OL{=}0.6825, n_s{=}0.9624, h{=}0.6711$, 
and $k_{\rm pivot} = 0.05~h{\rm Mpc}^{-1}$. 

The HADES suite includes models with degenerate massive neutrinos of different 
masses: $\smnu = $ 0.06, 0.10, and 0.15 eV. These massive neutrino models are run 
using the ``particle method'', where neutrinos are described as a collisionless 
and pressureless fluid and therefore modeled as particles, same as 
CDM~\citep{brandbyge2008,viel2010}. HADES also includes models with massless 
neutrino and different values of $\sigma_8$ to examine the $\smnu-\sigma_8$ 
degeneracy. The $\sigma_8$ values were chosen to match either $\sigma_8^m$ or 
$\sigma_8^{c}$ --- $\sigma_8$ computed with respect to total matter 
(CDM + baryons + $\nu$) or CDM + baryons --- of the massive neutrino models: 
$\sigma_8 = 0.822, 0.818, 0.807$, and $0.798$. Each model has $100$ independent 
realizations and we focus on the snapshots saved at $z = 0$. Halos closely 
trace the CDM+baryon field rather than the total matter field and neutrinos 
have negligible contribution to halo masses~\citep[\emph{e.g.}][]{ichiki2012, castorina2014, loverde2014, villaescusa-navarro2014}.
Hence, dark matter halos are identified in each realization using the Friends-of-Friends 
algorithm~\cite[FoF;][]{davis1985} with linking length $b=0.2$ on the CDM + baryon
distribution; only halos with masses $> 3.2\times 10^{13} h^{-1}M_\odot$ 
are included. For further details on the HADES simulations, we refer readers 
to~\cite{villaescusa-navarro2018}. 

\ch{describe quijote simulations} 
\bitem 
\item fixed-pair  
\eitem

%%%%%%%%%%%%%%%%%%%%%%%%%%%%%%%%%%%%%%%%%%
% simulation table
%%%%%%%%%%%%%%%%%%%%%%%%%%%%%%%%%%%%%%%%%%
\begin{table}
\caption{Specifications of the HADES and Quijote simulation suites.} 
\begin{center}
\begin{tabular}{ccccccccccc} \toprule
    Name  &$\smnu$ & $\Omega_m$ & $\Omega_b$ & $h$ & $n_s$ & $\sigma^m_8$ & $\sigma^c_8$ & $m_{\rm cdm}$ & $m_\nu$ & realizations \\
      &(eV) & & & & & & & ($10^{10}h^{-1}M_\odot$) & ($10^{10}h^{-1}M_\odot$) & \\[3pt] \hline\hline
    \multicolumn{11}{c}{HADES suite} \\ \hline
    Fiducial    & 0.0   & 0.3175 & 0.049 & 0.6711 & 0.9624 & 0.833 & 0.833 & 65.66 & 0      &100 \\ 
                & 0.06  & 0.3175 & 0.049 & 0.6711 & 0.9624 & 0.819 & 0.822 & 65.36 & 29.57  &100 \\ 
                & 0.10  & 0.3175 & 0.049 & 0.6711 & 0.9624 & 0.809 & 0.815 & 65.16 & 49.28  &100 \\ 
                & 0.15  & 0.3175 & 0.049 & 0.6711 & 0.9624 & 0.798 & 0.806 & 64.92 & 73.95  &100 \\ 
                & 0.0   & 0.3175 & 0.049 & 0.6711 & 0.9624 & 0.822 & 0.822 & 65.66 & 0      &100 \\ 
                & 0.0   & 0.3175 & 0.049 & 0.6711 & 0.9624 & 0.818 & 0.818 & 65.66 & 0      &100 \\ 
                & 0.0   & 0.3175 & 0.049 & 0.6711 & 0.9624 & 0.807 & 0.807 & 65.66 & 0      &100 \\ 
                & 0.0   & 0.3175 & 0.049 & 0.6711 & 0.9624 & 0.798 & 0.798 & 65.66 & 0      &100 \\[3pt]
    \hline \hline
    \multicolumn{11}{c}{Qujiote suite} \\ \hline
    Fiducial 	    & 0.0         & 0.3175 & 0.049 & 0.6711 & 0.9624 & 0.834 & 0.834 & & & 15,000 \\ 
    $\smnu^+$       & \underline{0.1}   & 0.3175 & 0.049 & 0.6711 & 0.9624 & 0.834 & 0.834 & & & 500 \\ 
    $\smnu^{++}$    & \underline{0.2}   & 0.3175 & 0.049 & 0.6711 & 0.9624 & 0.834 & 0.834 & & & 500 \\ 
    $\smnu^{+++}$   & \underline{0.4}   & 0.3175 & 0.049 & 0.6711 & 0.9624 & 0.834 & 0.834 & & & 500 \\ 
    $\Omega_m^+$    & 0.0   & \underline{ 0.3275} & 0.049 & 0.6711 & 0.9624 & 0.834 & 0.834 & & & 500 \\ 
    $\Omega_m^-$    & 0.0   & \underline{ 0.3075} & 0.049 & 0.6711 & 0.9624 & 0.834 & 0.834 & & & 500 \\ 
    $\Omega_b^+$    & 0.0   & 0.3175 & \underline{0.050} & 0.6711 & 0.9624 & 0.834 & 0.834 & & & 500 \\ 
    $\Omega_b^-$    & 0.0   & 0.3175 & \underline{0.048} & 0.6711 & 0.9624 & 0.834 & 0.834 & & & 500 \\ 
    $h^+$           & 0.0   & 0.3175 & 0.049 & \underline{0.6911} & 0.9624 & 0.834 & 0.834 & & & 500 \\ 
    $h^-$           & 0.0   & 0.3175 & 0.049 & \underline{0.6511} & 0.9624 & 0.834 & 0.834 & & & 500 \\ 
    $n_s^+$         & 0.0   & 0.3175 & 0.049 & 0.6711 & \underline{0.9824} & 0.834 & 0.834 & & & 500 \\ 
    $n_s^-$         & 0.0   & 0.3175 & 0.049 & 0.6711 & \underline{0.9424} & 0.834 & 0.834 & & & 500 \\ 
    $\sigma_8^+$    & 0.0   & 0.3175 & 0.049 & 0.6711 & 0.9624 & \underline{0.849} & \underline{0.849} & & & 500 \\ 
    $\sigma_8^-$    & 0.0   & 0.3175 & 0.049 & 0.6711 & 0.9624 & \underline{0.819} & \underline{0.819} & & & 500 \\[3pt]
    \hline
\end{tabular} \label{tab:sims}
\end{center}
    {\bf Notes:} \ch{description of the table} 
\end{table}
%%%%%%%%%%%%%%%%%%%%%%%%%%%%%%%%%%%%%%%%%%

\begin{figure}
\begin{center}
\includegraphics[width=0.9\textwidth]{figs/haloPlk_rsd_ratio.pdf}
    \caption{Impact of $\smnu$ and $\sig$ on the redshift-space halo power 
    spectrum monopole and quadrupole measured using the HADES simulation suite. 
    $\smnu$ and $\sig$ produce almost identical effects on halo clustering on 
    small scales ($k > 0.1\hmpc$). This degeneracy can be partially broken 
    through the quadrupole; however, {\em $\smnu$ and $\sig$ produce almost 
    the same effect on two-point clustering --- within a few percent}.
    }
\label{fig:plk}
\end{center}
\end{figure}

%%%%%%%%%%%%%%%%%%%%%%%%%%%%%%%%%%%%%%%%%%%%%%%%%%%%%%%%%%%%%%%%%%%%%%%%
% Section   
%%%%%%%%%%%%%%%%%%%%%%%%%%%%%%%%%%%%%%%%%%%%%%%%%%%%%%%%%%%%%%%%%%%%%%%%
\section{Bispectrum} \label{sec:bk} 
We're interested in breaking parameter degeneracies that limit the constraining 
power on $\smnu$ of two-point clustering analyses using three-point clustering 
statistics --- \emph{i.e.} the bispectrum. In this section, we describe the 
bispectrum estimator used throughout the paper. We focus on the bispectrum monopole 
($\ell = 0$) and use an estimator that exploits Fast Fourier Transpforms (FFTs). 
Our estimator is similar to the estimators described in \cite{scoccimarro2015,sefusatti2016}; 
we also follow their formalism in our description below. Although \cite{sefusatti2016} 
and \cite{scoccimarro2015} respectively describe estimators in redshift- and real-space,  
since we focus on the bispectrum monopole, we note that there is no difference. 

To measure the bispectrum of our halo catalogs, we begin by interpolating the halo
positions to a grid, $\delta({\bfi x})$ and Fourier transforming the grid to get $\delta({\bfi k})$.
We use a fourth-order interpolation to interlaced grids, which has advantageous anti-aliasing 
properties~\citep{hockney1981,sefusatti2016} that allow unbias measurements up to the Nyquist 
frequency. Then using $\delta({\bfi k})$, we measure the bispectrum monopole as 
\beq \label{eq:bk} 
\widehat{B}_{\ell = 0}(k_1,k_2, k_3) = \frac{1}{V_B} \int\limits_{k_1}{\rm d}^3q_1 \int\limits_{k_2}{\rm d}^3q_2 \int\limits_{k_3}{\rm d}^3q_3~\delta_{\rm D}({\bfi q_{123}})~\delta({\bfi q_1})~\delta({\bfi q_2})~\delta({\bfi q_3}) - B^{\rm SN}_{\ell = 0}
\eeq
$\delta_{\rm D}$ above is a Dirac delta function and hence $\delta_{\rm D}({\bfi q_{123}}) = \delta_{\rm D}({\bfi q_1} + {\bfi q_2} + {\bfi q_3})$ 
ensures that the ${\bfi q}_i$ triplet actually form a closed triangle. Each of the integrals 
above represent an integral over a spherical shell in $k$-space with radius $\delta k$ 
centered at ${\bfi k}_i$ --- \emph{i.e.} 
\beq
\int_{k_i}{\rm d}^3q \equiv \int\limits_{k_i-\delta k/2}^{k_i+\delta k/2}{\rm d}q~q^2\int {\rm d}\Omega.
\eeq
$V_B$ is a normalization factor proportional to the number of triplets ${\bfi q_1}$, 
${\bfi q_2}$, and ${\bfi q_3}$ that can be found in the triangle bin defined by 
$k_1$, $k_2$, and $k_3$ with width $\delta k$: 
\beq
V_B = \int\limits_{k_1}{\rm d}^3q_1 \int\limits_{k_2}{\rm d}^3q_2 \int\limits_{k_3}{\rm d}^3q_3~\delta_{\rm D}({\bfi q_{123}})
\eeq
Lastly, $B^{\rm SN}_{\ell=0}$ is the correction for the Poisson shot noise, which 
contributes due to the self-correlation of individual objects: 
\beq \label{eq:bk_sn} 
B^{\rm SN}_{\ell=0}(k_1, k_2, k_3) = \frac{1}{\bar{n}} \big(P_0(k_1) + P_0(k_2) + P_0(k_3) \big) + \frac{1}{\bar{n}^2}. 
\eeq
$\bar{n}$ is the number density of objects (halos) and $P_0$ is the powerspectrum monopole. 

In order to evaluate the integrals in Eq.~\ref{eq:bk}, we take advantage of the plane-wave 
representation of the Dirac delta function and rewrite the equation as
\begin{align} \label{eq:bk2} 
    \widehat{B}_{\ell = 0}(k_1,k_2, k_3) &= \frac{1}{V_B} \int\frac{{\rm d}^3x}{(2\pi)^3} \int\limits_{k_1}{\rm d}^3q_1 \int\limits_{k_2}{\rm d}^3q_2 \int\limits_{k_3}{\rm d}^3q_3~\delta({\bfi q_1})~\delta({\bfi q_2})~\delta({\bfi q_3})~e^{i{\bfi q_{123}}\cdot{\bfi x}} - B^{\rm SN}_{\ell = 0}\\ 
    &= \frac{1}{V_B} \int\frac{{\rm d}^3x}{(2\pi)^3}\prod\limits_{i=1}^{3} I_{k_i}({\bfi x}) - B^{\rm SN}_{\ell = 0} 
\end{align}
where 
\beq
I_{k_i}({\bfi x}) = \int\limits_k {\rm d}^3q~\delta({\bfi q})~e^{i {\bfi q}\cdot{\bfi x}}. 
\eeq
At this point, we measure $\widehat{B}_{\ell = 0}(k_1,k_2, k_3)$ by calcuating the 
$I_{k_i}$s with inverse FFTs and summing over in real space. 
\ch{note the specifications of the bispectrum calculation; fft grid size etc} 

We present the redshift-space halo bispectrum of the HADE simulations measured using 
the estimator above in Figures~\ref{fig:bk_shape} and~\ref{fig:bk_amp}. In Figure~\ref{fig:bk_shape}, 
we plot $B(k_1, k_2, k_3)$ as a function of $k_2/k_1$ and $k_3/k_1$, which describe the
triangle configuration shape, for all configurations with $k_1, k_2, k_3 < k_{\rm max} = 0.5~h/{\rm Mpc}$. 
In the upper panels, we present $B(k_1, k_2, k_3)$ of HADES models with 
$\smnu = 0.0, 0.06, 0.10$, and $0.15\,\mathrm{eV}$; in the lower panels, we present
$B(k_1, k_2, k_3)$ of HADES models with $\smnu 0.0$ eV and $\sig = 0.822, 0.818$, and 
$0.807$. The top and bottom panels of the three right-most columns have matching $\sig$
values (Section~\ref{sec:hades}). The $B(k_1, k_2, k_3)$ amplitude (colormap) in each 
($k_2/k_1$, $k_3/k_1$) bin is average amplitude of all triangle configurations in the bin. 
The upper left bin contains squeezed triangles ($k_1 = k_2 \gg k_3$); the upper right bin 
contains equilateral triangles ($k_1 = k_2 = k_3$); and the bottom center bin contains 
folded triangles ($k_1 = 2 k_2 = 2 k_3$). 

Next, in Figure~\ref{fig:bk_amp}, we plot $B(k_1, k_2, k_3)$ for all possible triangle 
configurations with $k_1, k_2, k_3 < k_{\rm max} = 0.5~h/{\rm Mpc}$ where we loop through 
the configurations with $k_3$ in the inner most loop and $k_1$ in the outer most loop 
satisfying $k_1 \leq k_2 \leq k_3$. In the top panel, we present $B(k_1, k_2, k_3)$ of 
HADES models with $\smnu = 0.0, 0.06, 0.10$, and $0.15\,\mathrm{eV}$; in the lower panel, 
we present $B(k_1, k_2, k_3)$ of HADES models with $\smnu 0.0$ eV and $\sig = 0.822, 0.818$, 
and $0.807$. We zoom into triangle configurations with $k_1 = 0.113$, $0.226 \leq k_2 \leq 0.283$, 
and $0.283 \leq k_3 \leq 0.377~h/{\rm Mpc}$ in the insets of the panels. 

\begin{figure}
\begin{center}
    \includegraphics[width=\textwidth]{figs/haloBk_shape_kmax05_rsd.pdf} 
    \caption{The redshift-space halo bispectrum, $B(k_1, k_2, k_3)$, as a 
    function of triangle configuration shape for $\smnu = 0.0, 0.06, 0.10$, 
    and $0.15\,\mathrm{eV}$ (upper panels) and $\sig = 0.822, 0.818$, and 
    $0.807$ (lower panels). The HADES simulations of the top and bottom 
    panels in the three right-most columns, have matching $\sig$ values 
    (Section~\ref{sec:hades}). We describe the triangle configuration shape 
    by the ratio of the triangle sides: $k_3/k_1$ and $k_2/k_1$. The upper 
    left bin contains squeezed triangles ($k_1 = k_2 \gg k_3$); the upper 
    right bin contains equilateral triangles ($k_1 = k_2 = k_3$); and the 
    bottom center bin contains folded triangles ($k_1 = 2 k_2 = 2 k_3$). 
    We include all triangle configurations with $k_1, k_2, k_3 \leq k_{\rm max} = 0.5~h/{\rm Mpc}$. 
    and use the $B(k_1, k_2, k_3)$ estimator described in Section~\ref{sec:bk}.}
\label{fig:bk_shape}
\end{center}
\end{figure}

\begin{figure}
\begin{center}
\includegraphics[width=0.9\textwidth]{figs/haloBk_amp_kmax05_rsd.pdf}
    \caption{The redshift-space halo bispectrum, $B(k_1, k_2, k_3)$, as a
    function of triangle configurations for $\smnu = 0.0, 0.06, 0.10$, 
    and $0.15\,\mathrm{eV}$ (top panel) and $\smnu = 0.0$ eV, $\sig = 0.822, 0.818, 0.807$, 
    and $0.798$ (lower panel). We include all possible triangle configurations 
    with $k_1, k_2, k_3 \leq k_{\rm max} = 0.5~h/{\rm Mpc}$ where we loop 
    through the configurations with $k_3$ in the inner most loop and 
    $k_1$ in the outer most loop satisfying $k_1 \leq k_2 \leq k_3$. In the 
    insets of the panels we zoom into triangle configurations with 
    $k_1 = 0.113$, $0.226 \leq k_2 \leq 0.283$, and $0.283 \leq k_3 \leq 0.377~h/{\rm Mpc}$.}
\label{fig:bk_amp}
\end{center}
\end{figure}

%%%%%%%%%%%%%%%%%%%%%%%%%%%%%%%%%%%%%%%%%%%%%%%%%%%%%%%%%%%%%%%%%%%%%%%%
% Section   
%%%%%%%%%%%%%%%%%%%%%%%%%%%%%%%%%%%%%%%%%%%%%%%%%%%%%%%%%%%%%%%%%%%%%%%%
\section{Results} \label{sec:results} 
\subsection{Breaking the $\smnu$-- $\sig$ degeneracy} \label{sec:mnusig}

\begin{figure}
\begin{center}
\includegraphics[width=0.8\textwidth]{figs/haloBk_dshape_kmax05_rsd.pdf} 
    \caption{The shape dependence of the $\smnu$ and $\sig$ imprint on 
    the redshift-space halo bispectrum, $\Delta B/B^\mathrm{(fid)}$. 
    We align the $\smnu{=}~0.06, 0.10$, and $0.15\,\mathrm{eV}$ HADES models 
    (upper panels) with $\smnu{=}~0.0$ eV, $\sig{=}~0.822$, $0.818$, and $0.807$ 
    models (bottom panels) such that the top and bottom panels in each column 
    have matching $\sig^{c}$, which produce mostly degenerate imprints on the 
    redshift-space power spectrum. The difference between the top and bottom 
    panels highlight that $\smnu$ leaves a distinct imprint on elongated and 
    isosceles triangles (bins along the bottom left and bottom right edges, 
    respectively) from $\sig$. {\em The imprint of $\smnu$ has an overall distinct 
    shape dependence on the bispectrum that cannot be replicated by varying $\sig$}. 
    }
\label{fig:dbk_shape}
\end{center}
\end{figure}

\begin{figure}
\begin{center}
\includegraphics[width=0.9\textwidth]{figs/haloBk_residual_kmax05_rsd.pdf}
    \caption{The impact of $\smnu$ and $\sig$ on the redshift-space 
    halo bispectrum, $\Delta B/B^\mathrm{(fid)}$, for all triangle configurations 
    with $k_1, k_2, k_3 \leq 0.5 h/{\rm Mpc}$. We compare $\Delta B/B^\mathrm{(fid)}$ 
    of the $\smnu = 0.06$ (top), $0.10$ (middle), and $0.15$ eV (bottom) HADES models
    to $\Delta B/B^\mathrm{(fid)}$ of $\smnu{=}~0.0$ eV, $\sig{=}~0.822$, $0.818$, and 
    $0.807$ models. The impact of $\smnu$ on the bispectrum has a significantly different 
    amplitude than the impact of $\sig$. For instance, $\smnu{=}~0.15\,\mathrm{eV}$ (red) 
    has a $\sim 5\%$ stronger impact on the bispectrum than $\smnu{=}~0.0$ eV, 
    $\sig{=}~0.798$ (black). Meanwhile, these two simulations have powerspectrums that 
    differ by $< 1\%$ (Figure~\ref{fig:plk}). Combined with the shape-dependence of 
    Figure~\ref{fig:dbk_shape}, {\em the distinct impact of $\smnu$ and $\sig$ on the 
    redshift-space halo bispectrum illustrate that the bispectrum can break the degeneracy 
    between $\smnu$ and $\sig$ that degrade constraints from two-point analyses}.
    }
\label{fig:dbk_amp}
\end{center}
\end{figure}


%\begin{figure}
%\begin{center}
%\includegraphics[width=0.95\textwidth]{figs/haloBk_triangles_rsd.pdf}
%    \caption{The impact of $\smnu$ and $\sig$ on the redshift-space halo 
%    bispectrum, $\Delta B/B^\mathrm{(fid)}$, for triangles that have  
%    $[k_1, k_2] = [0.19, 0.11], [0.11, 0.11]$ and $[0.08, 0.06]~h/{\rm Mpc}$ 
%    (left to right) with different angles in between, $\theta_{12}$. Again, 
%    the HADES simulations with $\smnu{=}~0.06$, $0.10$, and $0.15$ eV 
%    (orange, green, red) have matching $\sig^{c}$ with the $\smnu{=}~0.0$ eV, 
%    $\sig{=}~0.822$, $0.818$, and $0.807$ (purple, pink, yellow dashed). Comparison
%    between  these two sets of simulations further illustrate the distinct 
%    imprint of the $\smnu$ on the bispectrum. We also include $\smnu{=}~0.0$ eV, 
%    $\sig{=}~0.798$ (black dashed) to highlight that change in $\sig$ cannot 
%    reproduce the imprint of $\smnu$. 
%    }
%\label{fig:dbk_amp}
%\end{center}
%\end{figure}

\subsection{Forecasts} \label{sec:forecasts}

\begin{figure}
\begin{center}
    \includegraphics[width=0.6\textwidth]{figs/quijote_bkCov_kmax05.png} 
    \caption{Covariance matrix of the redshift-space halo bispectrum estimated 
    using the 15,000 realizations of the Qujiote simulation suite with the 
    fiducial cosmology: $\Om{=}0.3175$, $\Ob{=}0.049$, $h{=}0.6711$, $n_s{=}0.9624$, $\sig{=}0.834$, 
    and $\smnu{=}0.0$ eV. The triangle configurations (the bins) have the same 
    ordering as in Figures~\ref{fig:bk_amp} and~\ref{fig:dbk_amp}.
    }
\label{fig:bk_cov}
\end{center}
\end{figure}

\begin{figure}
\begin{center}
    \includegraphics[width=\textwidth]{figs/quijote_pbkFisher_001_05.pdf}
    \caption{Fisher forecast constraints on cosmological parameters from the 
    redshift-space halo power spectrum monopole (blue) and bispectrum (orange) 
    derived using the Quijote simulation suite. For both the power spectrum 
    and bispectrum constraints, we set $k_{\rm max} = 0.5~h/{\rm Mpc}$. The 
    contours mark the $68\%$ and $95\%$ confidence interals. The bispectrum 
    {\em substantially} improves constraints on all of the cosmological parameters 
    over the power spectrum. {\em For $\smnu$, the bispectrum improves the constraint
    from $\sigma_{\smnu}{=}~0.279$ to $0.0258$ --- over an order of magnitude 
    improvement over the power spectrum}.}
\label{fig:bk_fish}
\end{center}
\end{figure}


\begin{figure}
\begin{center}
    \includegraphics[width=\textwidth]{figs/quijote_pkFisher_kmax.pdf} 
    \caption{Fisher forecast constraints on cosmological parameters from the 
    redshift-space halo power spectrum monopole for $k_{\rm max} = 0.2$ (green) 0.3 (orange), 
    and $0.5~h/{\rm Mpc}$ (blue). The contours mark the $68\%$ (black dashed) and $95\%$ 
    confidence intervals. The contours illustrate the degeneracy between $\smnu$ and $\sig$ 
    we find  in the power spectrum comparisons of the HADES simulations (Figure~\ref{fig:plk}).}
\label{fig:pk_fish_kmax}
\end{center}
\end{figure}


\begin{figure}
\begin{center}
    \includegraphics[width=\textwidth]{figs/quijote_bkFisher_kmax.pdf} 
    \caption{Constraints on cosmological parameters from the redshift-space
    halo bispectrum for $k_1, k_2, k_3 \leq k_{\rm max} = 0.2$ (green) 0.3 (orange), 
    and $0.5~h/{\rm Mpc}$ (blue). The contours mark the $68\%$ (black dashed) and $95\%$ 
    confidence intervals. As we find with the bispectrum comparisons of the HADES 
    simulations (Section~\ref{sec:mnusig}, the bispectrum breaks the degeneracy 
    with $\sig$. \ch{probably want to say more things}}
\label{fig:bk_fish_kmax}
\end{center}
\end{figure}

%%%%%%%%%%%%%%%%%%%%%%%%%%%%%%%%%%%%%%%%%%%%%%%%%%%%%%%%%%%%%%%%%%%%%%%%
% Section   
%%%%%%%%%%%%%%%%%%%%%%%%%%%%%%%%%%%%%%%%%%%%%%%%%%%%%%%%%%%%%%%%%%%%%%%%
\section{Summary} 


\section*{Acknowledgements}
It's a pleasure to thank 
    Daniel Eisenstein, 
    Simone Ferraro, 
    Shirley Ho, 
    David N.~Spergel, 
    Benjamin D.~Wandelt


\appendix
\section{Redshift-space Bispectrum} 
\begin{figure}
\begin{center}
    \includegraphics[width=0.95\textwidth]{figs/haloBk_amp_kmax05_rsd_comparison.pdf}
    \caption{Comparison of the fiducial HADES simuluations real and redshift-space halo
    bispectrum for triangle configurations with $k_1, k_2, k_3 \leq k_{\rm max} = 0.5 h/{\rm Mpc}$ 
    (blue and orange respectively). We mark equilateral triangle configurations (empty 
    triangle marker) along with their side lengths $k$.
    }
\label{fig:cov_converge}
\end{center}
\end{figure}

\section{Testing Convergence} 
\begin{figure}
\begin{center}
    \includegraphics[width=0.5\textwidth]{figs/quijote_Fisher_nmocks_001_05.pdf}
    \caption{}
\label{fig:cov_converge}
\end{center}
\end{figure}

\begin{figure}
\begin{center}
    \includegraphics[width=0.5\textwidth]{figs/quijote_Fisher_dBk_nmocks_001_05.pdf} 
    \caption{}
\label{fig:dbk_converge}
\end{center}
\end{figure}

\bibliographystyle{yahapj}
\bibliography{emanu} 
\end{document}
