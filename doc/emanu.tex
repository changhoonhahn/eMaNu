\documentclass[12pt, letterpaper, preprint]{aastex62}
%%% This file is generated by the Makefile.
\newcommand{\giturl}{\url{https://github.com/changhoonhahn/eMaNu}}
\newcommand{\githash}{f673b16}\newcommand{\gitdate}{2019-07-23}\newcommand{\gitauthor}{changhoonhahn}

%\usepackage[breaklinks,colorlinks, urlcolor=blue,citecolor=blue,linkcolor=blue]{hyperref}
\usepackage{color}
\usepackage{amsmath}
\usepackage{natbib}
\usepackage{ctable}
\usepackage{bm}
\usepackage[normalem]{ulem} % Added by MS for \sout -> not required for final version
\usepackage{xspace}

% typesetting shih
\linespread{1.08} % close to 10/13 spacing
\setlength{\parindent}{1.08\baselineskip} % Bringhurst
\setlength{\parskip}{0ex}
\let\oldbibliography\thebibliography % killin' me.
\renewcommand{\thebibliography}[1]{%
  \oldbibliography{#1}%
  \setlength{\itemsep}{0pt}%
  \setlength{\parsep}{0pt}%
  \setlength{\parskip}{0pt}%
  \setlength{\bibsep}{0ex}
  \raggedright
}
\setlength{\footnotesep}{0ex} % seriously?

% citation alias

% math shih
\newcommand{\setof}[1]{\left\{{#1}\right\}}
\newcommand{\given}{\,|\,}
\newcommand{\pseudo}{{\mathrm{pseudo}}}
\newcommand{\Var}{\mathrm{Var}}
\newcommand{\Nd}{{100}\xspace}

\newcommand{\smnu}{\sum m_\nu} 
\newcommand{\sig}{\sigma_8} 
\newcommand{\hmpc}{\,h/\mathrm{Mpc}}

\newcommand{\specialcell}[2][c]{%
  \begin{tabular}[#1]{@{}c@{}}#2\end{tabular}}
% text shih
\newcommand{\foreign}[1]{\textsl{#1}}
\newcommand{\etal}{\foreign{et~al.}}
\newcommand{\opcit}{\foreign{Op.~cit.}}
\newcommand{\documentname}{\textsl{Article}}
\newcommand{\equationname}{equation}
\newcommand{\bitem}{\begin{itemize}}
\newcommand{\eitem}{\end{itemize}}
\newcommand{\beq}{\begin{equation}}
\newcommand{\eeq}{\end{equation}}

%% MS: To make collaborating easier
\definecolor{orange}{rgb}{1,0.5,0}
%\setlength{\marginparwidth}{1.2in}
%\let\oldmarginpar\marginpar
%\renewcommand\marginpar[1]{\-\oldmarginpar[\raggedleft\footnotesize #1]%
%  {\raggedright\footnotesize #1}}

%%% Here are the things that make collaborative effort easier
\newcommand{\ch}[1]{{\color{orange}{\bf CH:} #1}}
\newcommand{\lss}{{\small{LSS}}\xspace}
\newcommand{\gmm}{{\small{GMM}}\xspace}
\newcommand{\gmms}{{\small{GMM}s}\xspace}
\newcommand{\EM}{{\small{EM}}\xspace}
\newcommand{\bic}{{\small{BIC}}\xspace}
\newcommand{\pca}{{\small{PCA}}\xspace}
\newcommand{\ica}{{\small{ICA}}\xspace}
\newcommand{\patchy}{{\fontshape\scdefault\selectfont patchy}}

\begin{document}\sloppy\sloppypar\frenchspacing 

\title{eMa$\nu$ 1.: Breaking $\sum m_\nu$ Parameter Degeneracies with Three-point Statistics} 
\date{\texttt{DRAFT~---~\githash~---~\gitdate~---~NOT READY FOR DISTRIBUTION}}

\newcounter{affilcounter}
% \altaffiltext{1}{}

%\setcounter{affilcounter}{1}
%
%\edef \lbl {\arabic{affilcounter}}\stepcounter{affilcounter}
%\altaffiltext{\lbl}{Lawrence Berkeley National Laboratory, 1 Cyclotron Rd, Berkeley CA 94720, USA}
%
%\edef \bccp {\arabic{affilcounter}}\stepcounter{affilcounter}
%\altaffiltext{\bccp}{Berkeley Center for Cosmological Physics, University of California, Berkeley, CA 94720, USA}

\author{ChangHoon Hahn}
\altaffiliation{hahn.changhoon@gmail.com}
\affil{Lawrence Berkeley National Laboratory, 1 Cyclotron Rd, Berkeley CA 94720, USA}
\affil{Berkeley Center for Cosmological Physics, University of California, Berkeley, CA 94720, USA}

\author{Francisco Villaescusa-Navarro} 
%\affil{Center for Computational Astrophysics, Flatiron Institute, 162 5th Avenue, New York, NY 10010, USA} 

\author{Zachary Slepian} 
%\affil{Center for Computational Astrophysics, Flatiron Institute, 162 5th Avenue, New York, NY 10010, USA} 

\begin{abstract}
blah 
\end{abstract}

\keywords{
cosmology: 
---
}

\section{Introduction}
talk about the impact of massive active neutrinos on the matter powerspectrum and how that's detectable with CMB and LSS. 

one big roadblock is the degerenacy among $\tau$ $\sigma_8$ and $\sum m_\nu$.
short thing about how $\tau$ is hard to constrain

%%%%%%%%%%%%%%%%%%%%%%%%%%%%%%%%%%%%%%%%%%%%%%%%%%%%%%%%%%%%%%%%%%%%%%%%
% Section   
%%%%%%%%%%%%%%%%%%%%%%%%%%%%%%%%%%%%%%%%%%%%%%%%%%%%%%%%%%%%%%%%%%%%%%%%
\section{HADES simulations} \label{sec:hades} 
brief description of the hades simulation and the halo catalogs
%We use a subset of the HADES12 simulations. We briefly describe the characteristics of this subset here and refer the reader to F. Villaescusa-Navarro et al. (2017, in preparation) for further details. The simulations have been run using the TreePM+SPH code GADGET-III (Springel 2005). All simula- tions are run in a periodic box size of 1 h−1 Gpc, and all models share the value of the following cosmological parameters:
%Ωm=0.3175, Ωb=0.049, ΩΛ=0.6825, ns = 0.9624, h = 0.6711, and kpivot=0.05 h Mpc−1, which are in very good agreement with those obtained by Planck (Planck Collaboration et al. 2016). All our massive neutrino runs assume degenerate masses.13 A summary of the different simulations can be found in Table 1.
%We present convergence tests for the N-body simulations in Appendix A. We notice that while our lower-resolution simulations cannot be used to provide absolute theory predictions that are accurate at 1% down to the smallest scales we consider, k=0.5 h Mpc−1, their relative difference will be much more accurate (see Appendix A for details). Thus, in this work we either show relative differences or show absolute values for all models. We emphasize that for the latter, even if the absolute value from our simulations is not converged on a given scale, the relative differences between models should almost be.

\begin{figure}
\begin{center}
\includegraphics[width=0.95\textwidth]{figs/haloPlk_rsd_ratio.pdf}
    \caption{Impact of $\smnu$ and $\sig$ on the halo power spectrum monopole 
    and quadrupole. $\smnu$ and $\sig$ produce almost identical effects on halo 
    clustering on small scales ($k > 0.1\hmpc$). This degeneracy can be partially 
    broken through the quadrupole; however, {\em $\smnu$ and $\sig$ produce, within 
    a few percent, almost the same effect on two-point clustering}.
    \ch{update to coarser binning}}
\label{fig:plk}
\end{center}
\end{figure}

%%%%%%%%%%%%%%%%%%%%%%%%%%%%%%%%%%%%%%%%%%%%%%%%%%%%%%%%%%%%%%%%%%%%%%%%
% Section   
%%%%%%%%%%%%%%%%%%%%%%%%%%%%%%%%%%%%%%%%%%%%%%%%%%%%%%%%%%%%%%%%%%%%%%%%
\section{Bispectrum} \label{sec:bk} 
Brief description of the Scoccimarro et al. bispectrum estimator here 

\begin{figure}
\begin{center}
\includegraphics[width=\textwidth]{figs/haloB123_shape_001_05_rsd.pdf}
    \caption{The redshift-space halo bispectrum, $B(k_1, k_2, k_3)$ as a 
    function of triangle configuration shape for $\smnu = 0.0, 0.06, 0.10$, 
    and $0.15\,\mathrm{eV}$ (top panels) and $\sig = 0.822, 0.818, 0.807$, 
    and $0.798$ (lower panels). 
    \ch{details on the triangle configurations and the colormap}.
    We describe the estimator used to calculate $B(k_1, k_2, k_3)$ in 
    Section~\ref{sec:bk}.}
\label{fig:bk_shape}
\end{center}
\end{figure}

\begin{figure}
\begin{center}
\includegraphics[width=0.95\textwidth]{figs/haloB123_amp_001_05_rsd.pdf}
    \caption{The redshift-space halo bispectrum, $B(k_1, k_2, k_3)$, as a
    function of all triangle configurations for $\smnu = 0.0, 0.06, 0.10$, 
    and $0.15\,\mathrm{eV}$ (top panel) and $\sig = 0.822, 0.818, 0.807$, 
    and $0.798$ (lower panel). 
    \ch{details on the ordering of the triangle configurations; also mention
    how it's roughly scale dependence}.
    We describe the estimator used to calculate $B(k_1, k_2, k_3)$ in 
    Section~\ref{sec:bk}.}
\label{fig:bk_amp}
\end{center}
\end{figure}


%%%%%%%%%%%%%%%%%%%%%%%%%%%%%%%%%%%%%%%%%%%%%%%%%%%%%%%%%%%%%%%%%%%%%%%%
% Section   
%%%%%%%%%%%%%%%%%%%%%%%%%%%%%%%%%%%%%%%%%%%%%%%%%%%%%%%%%%%%%%%%%%%%%%%%
\section{Results} \label{sec:results} 

\begin{figure}
\begin{center}
\includegraphics[width=\textwidth]{figs/halodB123_relative_shape_001_05_rsd.pdf} 
    \caption{The shape depdence of the $\smnu$ and $\sig$ impact on the 
    redshift-space halo bispectrum, $\Delta B/B^\mathrm{(fid)}$. 
    $\smnu = 0.06, 0.10$, and $0.15\,\mathrm{eV}$ (top panels; left to right) are aligned
    with $\sig = 0.822, 0.818$, and $0.807\,\mathrm{eV}$ (bottom planes; left to right),
    which produce mostly degenerate imprints on the redshift-space power 
    spectrum. \ch{Details on the shape dependence}. 
    The difference between the top and bottom panels illustrate 
    that {\em $\smnu$ induces a sigificantly different impact on the 
    shape-dependence of the halo bispectrum than $\sig$}. 
    }
\label{fig:dbk_shape}
\end{center}
\end{figure}


\begin{figure}
\begin{center}
\includegraphics[width=\textwidth]{figs/halodB123_relative_001_05_rsd.pdf}
    \caption{The impact of $\smnu$ and $\sig$ on the redshift-space 
    halo bispectrum for all triangle configurations: $\Delta B/B^\mathrm{(fid)}$. 
    The impact of $\smnu$ differs significantly from the impact of $\sig$ 
    both in amplitude and scale dependence. For instance, $\smnu = 0.15\,\mathrm{eV}$ 
    (red) has a $\sim 5\%$ stronger impact on the bispectrum than $\sig = 0.798$ 
    (cyan dotted), which has little difference in the power spectrum (Figure~\ref{fig:plk}). 
    Combined with the shape-dependence of Figure~\ref{fig:dbk_shape}, {\em the 
    contrasting impact of $\smnu$ and $\sig$ on the redshift-space halo bispectrum 
    illustrate that the bispectrum break the degeneracy between $\smnu$
    and $\sig$ that degrade constraints from two-point analyses}. 
    }
\label{fig:dbk_amp}
\end{center}
\end{figure}


%%%%%%%%%%%%%%%%%%%%%%%%%%%%%%%%%%%%%%%%%%%%%%%%%%%%%%%%%%%%%%%%%%%%%%%%
% Section   
%%%%%%%%%%%%%%%%%%%%%%%%%%%%%%%%%%%%%%%%%%%%%%%%%%%%%%%%%%%%%%%%%%%%%%%%
\section{Summary} 

%\bibliographystyle{yahapj}
\end{document}
