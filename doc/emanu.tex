\documentclass[12pt, letterpaper, preprint]{aastex62}
%%% This file is generated by the Makefile.
\newcommand{\giturl}{\url{https://github.com/changhoonhahn/eMaNu}}
\newcommand{\githash}{f673b16}\newcommand{\gitdate}{2019-07-23}\newcommand{\gitauthor}{changhoonhahn}

%\usepackage[breaklinks,colorlinks, urlcolor=blue,citecolor=blue,linkcolor=blue]{hyperref}
\usepackage{color}
\usepackage{amsmath}
\usepackage{natbib}
\usepackage{ctable}
\usepackage{bm}
\usepackage[normalem]{ulem} % Added by MS for \sout -> not required for final version
\usepackage{xspace}

% typesetting shih
\linespread{1.08} % close to 10/13 spacing
\setlength{\parindent}{1.08\baselineskip} % Bringhurst
\setlength{\parskip}{0ex}
\let\oldbibliography\thebibliography % killin' me.
\renewcommand{\thebibliography}[1]{%
  \oldbibliography{#1}%
  \setlength{\itemsep}{0pt}%
  \setlength{\parsep}{0pt}%
  \setlength{\parskip}{0pt}%
  \setlength{\bibsep}{0ex}
  \raggedright
}
\setlength{\footnotesep}{0ex} % seriously?

% citation alias

% math shih
\newcommand{\setof}[1]{\left\{{#1}\right\}}
\newcommand{\given}{\,|\,}
\newcommand{\lss}{{\small{LSS}}\xspace}

\newcommand{\Om}{\Omega_{\rm m}} 
\newcommand{\Ob}{\Omega_{\rm b}} 
\newcommand{\OL}{\Omega_\Lambda}
\newcommand{\smnu}{\sum m_\nu} 
\newcommand{\sig}{\sigma_8} 
\newcommand{\hmpc}{\,h/\mathrm{Mpc}}

\newcommand{\specialcell}[2][c]{%
  \begin{tabular}[#1]{@{}c@{}}#2\end{tabular}}
% text shih
\newcommand{\foreign}[1]{\textsl{#1}}
\newcommand{\etal}{\foreign{et~al.}}
\newcommand{\opcit}{\foreign{Op.~cit.}}
\newcommand{\documentname}{\textsl{Article}}
\newcommand{\equationname}{equation}
\newcommand{\bitem}{\begin{itemize}}
\newcommand{\eitem}{\end{itemize}}
\newcommand{\beq}{\begin{equation}}
\newcommand{\eeq}{\end{equation}}

%% MS: To make collaborating easier
\definecolor{orange}{rgb}{1,0.5,0}

%%% Here are the things that make collaborative effort easier
\newcommand{\ch}[1]{{\color{orange}{\bf CH:} #1}}

\begin{document}\sloppy\sloppypar\frenchspacing 

\title{Breaking $\smnu$ Parameter Degeneracies with the Bispectrum} 
\date{\texttt{DRAFT~---~\githash~---~\gitdate~---~NOT READY FOR DISTRIBUTION}}

\newcounter{affilcounter}
\author{ChangHoon Hahn}
\altaffiliation{hahn.changhoon@gmail.com}
\affil{Lawrence Berkeley National Laboratory, 1 Cyclotron Rd, Berkeley CA 94720, USA}
\affil{Berkeley Center for Cosmological Physics, University of California, Berkeley, CA 94720, USA}

\author{Francisco Villaescusa-Navarro} 
\affil{Center for Computational Astrophysics, Flatiron Institute, 162 5th Avenue, New York, NY 10010, USA} 

\author{Emanuele Castorina} 
\affil{Berkeley Center for Cosmological Physics, University of California, Berkeley, CA 94720, USA}
\affil{Lawrence Berkeley National Laboratory, 1 Cyclotron Rd, Berkeley CA 94720, USA}

\begin{abstract}
    abstract
\end{abstract}

\keywords{
cosmology: 
---
}

\section{Introduction}
very brief into on neutrinos

Brief intro on the impact of massive active neutrinos on the matter powerspectrum 
and how that's detectable with CMB and LSS


Quick summary of current constraints and where they come from. Talk about the 
CMB-LSS lever arm. The degeneracy between As and tau and how that's a bottleneck
short thing about how $\tau$ is hard to constrain.

Fortunately the imprint of neutrinos on the matter distribution leaves imprints
on clustering. So with clustering measurements alone we can derive constraints 
on $\smnu$ and at the very least tighten constraints.  

Brief summary of previous works that look at the powerspectrum. Then Discuss the 
shortcomings of the powerspectrum only analysis-- Not good enough. 

However, we don't have to settle for just two point statistics, three-point 
statistics such as the bispectrum and 3PCF... 

In Section blah 


%%%%%%%%%%%%%%%%%%%%%%%%%%%%%%%%%%%%%%%%%%%%%%%%%%%%%%%%%%%%%%%%%%%%%%%%
% Section   
%%%%%%%%%%%%%%%%%%%%%%%%%%%%%%%%%%%%%%%%%%%%%%%%%%%%%%%%%%%%%%%%%%%%%%%%
\section{HADES and Quijote simulations} \label{sec:hades} 
We use a subset of the HADES\footnote{https://franciscovillaescusa.github.io/hades.html} 
and Quijote simulation suites. The HADES simulations have been run
using the {\sc GADGET-III} TreePM+SPH code~\citep{springel2005} in a 
periodic $(1 h^{-1}{\rm Gpc})^3$ box. All of the HADES simulations 
share the values of the following cosmological parameters:
$\Om{=}0.3175, \Ob{=}0.049, \OL{=}0.6825, n_s{=}0.9624, h{=}0.6711$, 
and $k_{\rm pivot} = 0.05~h{\rm Mpc}^{-1}$. These parameters are in
good agreement with Planck constraints~\cite{planck2016}. 

%All our massive neutrino runs assume degenerate masses.13 A summary of the different simulations can be found in Table 1.
%We present convergence tests for the N-body simulations in Appendix A. We notice that while our lower-resolution simulations cannot be used to provide absolute theory predictions that are accurate at 1% down to the smallest scales we consider, k=0.5 h Mpc−1, their relative difference will be much more accurate (see Appendix A for details). Thus, in this work we either show relative differences or show absolute values for all models. We emphasize that for the latter, even if the absolute value from our simulations is not converged on a given scale, the relative differences between models should almost be.

\ch{describe quijote simulations} 

\begin{figure}
\begin{center}
\includegraphics[width=0.9\textwidth]{figs/haloPlk_rsd_ratio.pdf}
    \caption{Impact of $\smnu$ and $\sig$ on the redshift-space halo power 
    spectrum monopole and quadrupole. $\smnu$ and $\sig$ produce almost identical 
    effects on halo clustering on small scales ($k > 0.1\hmpc$). This degeneracy 
    can be partially broken through the quadrupole; however, {\em $\smnu$ and $\sig$ 
    produce, within a few percent, almost the same effect on two-point clustering}.
    }
\label{fig:plk}
\end{center}
\end{figure}

%%%%%%%%%%%%%%%%%%%%%%%%%%%%%%%%%%%%%%%%%%%%%%%%%%%%%%%%%%%%%%%%%%%%%%%%
% Section   
%%%%%%%%%%%%%%%%%%%%%%%%%%%%%%%%%%%%%%%%%%%%%%%%%%%%%%%%%%%%%%%%%%%%%%%%
\section{Bispectrum} \label{sec:bk} 
Brief description of the Scoccimarro et al. bispectrum estimator here 

\begin{figure}
\begin{center}
    \includegraphics[width=\textwidth]{figs/haloBk_shape_001_05_rsd.pdf} 
    \caption{The redshift-space halo bispectrum, $B(k_1, k_2, k_3)$ as a 
    function of triangle configuration shape for $\smnu = 0.0, 0.06, 0.10$, 
    and $0.15\,\mathrm{eV}$ (upper panels) and $\sig = 0.822, 0.818$, and 
    $0.807$ (lower panels). We describe the triangle configuration shape 
    by the ratio of the triangle sides: $k_3/k_1$ and $k_2/k_1$. In each 
    of the panels, the upper left bin contains squeezed triangles 
    ($k_1 = k_2 \gg k_3$); the upper right bin contains equilateral triangles 
    ($k_1 = k_2 = k_3$); and the bottom center bin contains folded triangles 
    ($k_1 = 2 k_2 = 2 k_3$). We include all triangle configurations with 
    $k_1, k_2, k_3 \leq k_{\rm max} = 0.5~h/{\rm Mpc}$. In the three 
    right-most columns, the HADES simulations of the top and bottom panels 
    have matching $\sig$ values. We describe the estimator used to calculate 
    $B(k_1, k_2, k_3)$ in Section~\ref{sec:bk}.}
\label{fig:bk_shape}
\end{center}
\end{figure}

\begin{figure}
\begin{center}
\includegraphics[width=0.9\textwidth]{figs/haloBk_amp_001_05_rsd.pdf}
    \caption{The redshift-space halo bispectrum, $B(k_1, k_2, k_3)$, as a
    function of triangle configurations for $\smnu = 0.0, 0.06, 0.10$, 
    and $0.15\,\mathrm{eV}$ (top panel) and $\smnu = 0.0$ eV, $\sig = 0.822, 0.818, 0.807$, 
    and $0.798$ (lower panel). We include all possible triangle configurations 
    with $k_1, k_2, k_3 \leq k_{\rm max} = 0.5~h/{\rm Mpc}$ where we loop 
    through the configurations with $k_3$ in the inner most loop and 
    $k_1$ in the outer most loop satisfying $k_1 \leq k_2 \leq k_3$. In the 
    insets of the panels we zoom into triangle configurations with 
    $k_1 = 0.113$, $0.226 \leq k_2 \leq 0.283$, and $0.283 \leq k_3 \leq 0.377~h/{\rm Mpc}$. 
    The blue shaded regions represent the uncertainties estimated using 
    the $15,000$ fiducial Quijote simulations and illustrate how triangle 
    configurations on small scales are dominated by shot noise.}
\label{fig:bk_amp}
\end{center}
\end{figure}

%%%%%%%%%%%%%%%%%%%%%%%%%%%%%%%%%%%%%%%%%%%%%%%%%%%%%%%%%%%%%%%%%%%%%%%%
% Section   
%%%%%%%%%%%%%%%%%%%%%%%%%%%%%%%%%%%%%%%%%%%%%%%%%%%%%%%%%%%%%%%%%%%%%%%%
\section{Results} \label{sec:results} 
\subsection{Breaking the $\smnu$-- $\sig$ degeneracy} \label{sec:mnusig}

\begin{figure}
\begin{center}
\includegraphics[width=\textwidth]{figs/haloBk_dshape_001_05_rsd.pdf} 
    \caption{The shape dependence of the $\smnu$ and $\sig$ imprint on 
    the redshift-space halo bispectrum, $\Delta B/B^\mathrm{(fid)}$. 
    We align the $\smnu{=}~0.06, 0.10$, and $0.15\,\mathrm{eV}$ (upper panels) 
    with $\smnu{=}~0.0$ eV, $\sig{=}~0.822$, $0.818$, and $0.807$ (bottom panels) 
    such that simulations of top and bottom panels in each of the three columns 
    have matching $\sig^{c}$, which produce mostly degenerate imprints on 
    the redshift-space power spectrum. The difference between the top and 
    bottom panels highlight, for instance, that $\smnu$ leaves a distinct
    imprint on elongated and isosceles triangles (bins along the bottom left
    and bottom right edges, respectively) from $\sig$. {\em The imprint of 
    $\smnu$ has an overall distinct shape dependence on the bispectrum that 
    cannot be replicated by a change of $\sig$}. 
    }
\label{fig:dbk_shape}
\end{center}
\end{figure}

\begin{figure}
\begin{center}
\includegraphics[width=0.9\textwidth]{figs/haloBk_residual_001_05_rsd.pdf}
    \caption{The impact of $\smnu$ and $\sig$ on the redshift-space 
    halo bispectrum, $\Delta B/B^\mathrm{(fid)}$, for triangle configurations with 
    $k_1, k_2, k_3 \leq 0.5 h/{\rm Mpc}$. We compare $\Delta B/B^\mathrm{(fid)}$ 
    of the HADES simulations with $\smnu = 0.06$ (top), $0.10$ (middle), and 
    $0.15$ eV (bottom) to $\Delta B/B^\mathrm{(fid)}$ of $\smnu=0.0$ eV HADES 
    simulations with matching $\sig$. The impact of $\smnu$ on the bispectrum
    has a significantly different amplitude than the impact of $\sig$. For 
    instance, $\smnu{=}~0.15\,\mathrm{eV}$ (red) has a $\sim 5\%$ stronger impact 
    on the bispectrum than $\smnu{=}~0.0$ eV, $\sig{=}~0.798$ (black). Meanwhile, 
    these two simulations have power spectrums that differ by $< 1\%$ (Figure~\ref{fig:plk}). 
    Combined with the shape-dependence of Figure~\ref{fig:dbk_shape}, {\em the 
    distinct impact of $\smnu$ and $\sig$ on the redshift-space halo bispectrum 
    illustrate that the bispectrum can break the degeneracy between $\smnu$
    and $\sig$ that degrade constraints from two-point analyses}. 
    }
\label{fig:dbk_amp}
\end{center}
\end{figure}


\begin{figure}
\begin{center}
\includegraphics[width=0.95\textwidth]{figs/haloBk_triangles_rsd.pdf}
    \caption{The impact of $\smnu$ and $\sig$ on the redshift-space halo 
    bispectrum, $\Delta B/B^\mathrm{(fid)}$, for triangles that have  
    $[k_1, k_2] = [0.19, 0.11], [0.11, 0.11]$ and $[0.08, 0.06]~h/{\rm Mpc}$ 
    (left to right) with different angles in between, $\theta_{12}$. Again, 
    the HADES simulations with $\smnu{=}~0.06$, $0.10$, and $0.15$ eV 
    (orange, green, red) have matching $\sig^{c}$ with the $\smnu{=}~0.0$ eV, 
    $\sig{=}~0.822$, $0.818$, and $0.807$ (purple, pink, yellow dashed). Comparison
    between  these two sets of simulations further illustrate the distinct 
    imprint of the $\smnu$ on the bispectrum. We also include $\smnu{=}~0.0$ eV, 
    $\sig{=}~0.798$ (black dashed) to highlight that change in $\sig$ cannot 
    reproduce the imprint of $\smnu$. 
    }
\label{fig:dbk_amp}
\end{center}
\end{figure}

\subsection{Forecasts} \label{sec:forecasts}

\begin{figure}
\begin{center}
    \includegraphics[width=0.6\textwidth]{figs/quijote_bkCov_001_05.png} 
    \caption{Covariance matrix of the redshift-space halo bispectrum estimated 
    using the 15,000 realizations of the Qujiote simulation suite with the 
    fiducial cosmology: $\Om{=}0.3175$, $\Ob{=}0.049$, $h{=}0.6711$, $n_s{=}0.9624$, $\sig{=}0.834$, 
    and $\smnu{=}0.0$ eV. The triangle configurations (the bins) have the same 
    ordering as in Figures~\ref{fig:bk_amp} and~\ref{fig:dbk_amp}.
    }
\label{fig:bk_cov}
\end{center}
\end{figure}

\begin{figure}
\begin{center}
    \includegraphics[width=\textwidth]{figs/quijote_pbkFisher_001_05.pdf}
    \caption{Fisher forecast constraints on cosmological parameters from the 
    redshift-space halo power spectrum monopole (blue) and bispectrum (orange) 
    derived using the Quijote simulation suite. For both the power spectrum 
    and bispectrum constraints, we set $k_{\rm max} = 0.5~h/{\rm Mpc}$. The 
    contours mark the $68\%$ and $95\%$ confidence interals. The bispectrum 
    {\em substantially} improves constraints on all of the cosmological parameters 
    over the power spectrum. {\em For $\smnu$, the bispectrum improves the constraint
    from $\sigma_{\smnu}{=}~0.279$ to $0.0258$ --- over an order of magnitude 
    improvement over the power spectrum}.}
\label{fig:bk_fish}
\end{center}
\end{figure}


\begin{figure}
\begin{center}
    \includegraphics[width=\textwidth]{figs/quijote_pkFisher_kmax.pdf} 
    \caption{Fisher forecast constraints on cosmological parameters from the 
    redshift-space halo power spectrum monopole for $k_{\rm max} = 0.2$ (green) 0.3 (orange), 
    and $0.5~h/{\rm Mpc}$ (blue). The contours mark the $68\%$ (black dashed) and $95\%$ 
    confidence intervals. The contours illustrate the degeneracy between $\smnu$ and $\sig$ 
    we find  in the power spectrum comparisons of the HADES simulations (Figure~\ref{fig:plk}).}
\label{fig:pk_fish_kmax}
\end{center}
\end{figure}


\begin{figure}
\begin{center}
    \includegraphics[width=\textwidth]{figs/quijote_bkFisher_kmax.pdf} 
    \caption{Constraints on cosmological parameters from the redshift-space
    halo bispectrum for $k_1, k_2, k_3 \leq k_{\rm max} = 0.2$ (green) 0.3 (orange), 
    and $0.5~h/{\rm Mpc}$ (blue). The contours mark the $68\%$ (black dashed) and $95\%$ 
    confidence intervals. As we find with the bispectrum comparisons of the HADES 
    simulations (Section~\ref{sec:mnusig}, the bispectrum breaks the degeneracy 
    with $\sig$. \ch{probably want to say more things}}
\label{fig:bk_fish_kmax}
\end{center}
\end{figure}

%%%%%%%%%%%%%%%%%%%%%%%%%%%%%%%%%%%%%%%%%%%%%%%%%%%%%%%%%%%%%%%%%%%%%%%%
% Section   
%%%%%%%%%%%%%%%%%%%%%%%%%%%%%%%%%%%%%%%%%%%%%%%%%%%%%%%%%%%%%%%%%%%%%%%%
\section{Summary} 


\section*{Acknowledgements}
It's a pleasure to thank 
    Simone Ferraro, 
    Shirley Ho, 

\appendix
\section{Redshift-space Bispectrum} 
\begin{figure}
\begin{center}
    \includegraphics[width=0.95\textwidth]{figs/haloBk_amp_001_05_rsd_comparison.pdf}
    \caption{Comparison of the fiducial HADES simuluation real and redshift-space halo
    bispectrum (blue and orange).}
\label{fig:cov_converge}
\end{center}
\end{figure}

\section{Testing Convergence} 

\begin{figure}
\begin{center}
    \includegraphics[width=0.5\textwidth]{figs/quijote_Fisher_nmocks_001_05.pdf}
    \caption{}
\label{fig:cov_converge}
\end{center}
\end{figure}

\begin{figure}
\begin{center}
    \includegraphics[width=0.5\textwidth]{figs/quijote_Fisher_dBk_nmocks_001_05.pdf} 
    \caption{}
\label{fig:dbk_converge}
\end{center}
\end{figure}

%\bibliographystyle{yahapj}
\end{document}
