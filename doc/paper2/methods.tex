\section{Bispectrum and Cosmological Parameter Forecasts} \label{sec:methods}
We measure the galaxy bispectrum and calculate the parameter constraints using the 
same methods as \cite{hahn2020}. For further details, we  refer readers to \cite{hahn2020}. 

To measure $\Bg$, we use a Fast Fourier Transform (FFT) based estimator similar to
the ones in \cite{sefusatti2005}, \cite{scoccimarro2015}, and
\cite{sefusatti2016}. Galaxy positions are first interpolated onto a grid,
$\delta(\bfi{x})$, using a fourth-order interpolation scheme, which has advantageous
anti-aliasing properties that allow unbiased measurements up to the Nyquist
frequency~\citep{hockney1981, sefusatti2016}. After Fourier transforming 
$\delta(\bfi{x})$ to get $\delta(\bfi{k})$, we measure the bispectrum monopole
\beq \label{eq:bk} 
\Bg(k_1, k_2, k_3) = \frac{1}{V_B} \int\limits_{k_1}{\rm d}^3q_1
\int\limits_{k_2}{\rm d}^3q_2 \int\limits_{k_3}{\rm d}^3q_3~\delta_{\rm
D}({\bfi q_{123}})~\delta({\bfi q_1})~\delta({\bfi q_2})~\delta({\bfi q_3}) -
B^{\rm SN}_0.
\eeq
$\delta_D$ is the Dirac delta function, $V_B$ is the normalization factor
proportional to the number of triplets that can be found in the $k_1, k_2, k_3$
triangle bin, and $B^{\rm SN}_0$ is the correction term for the Poisson shot
noise. Throughout the paper, we use $\delta(\bfi{x})$ grids with $N_{\rm grid}
= 360$ and triangle configurations defined by $k_1, k_2, k_3$ bins of width
$\Delta k = 3 k_f = 0.01885\hmpc$, where $k_f = 2\pi/(1000~h^{-1}{\rm Mpc})$. 

In Figure~\ref{fig:bgh}, we present the redshift-space galaxy power spectrum multipoles 
($\Pgl$; left) and bispectrum ($\Bg$; right) of the fiducial HOD galaxy catalog (blue). The $\Pgl$ and 
$\Bg$ are averaged over one set of HOD realizations run on 15,000 $N$-body
\quij~simulations at the fiducial cosmology. In the left panel, we
plot both the power spectrum monopole ($\ell = 0$; solid) and quadrupole 
($\ell = 2$; dashed). In the right panel, we plot $\Bg$ for all 1898 triangle
configurations with $k_1, k_2, k_3 \le k_{\rm max} = 0.5\hmpc$. The configurations 
are ordered by looping through $k_3$ in the inner most loop and $k_1$ in the outer
most loop satisfying $k_1 \le k_2 \le k_3$. For comparison, we include the
redshift-space halo power spectrum and bispectrum at the fiducial cosmology 
from \cite{hahn2020} (black dotted). 

\begin{figure}
\begin{center}
    \includegraphics[width=\textwidth]{figs/PBg_reg.pdf} 
    \caption{The redshift-space galaxy power spectrum multipoles ($\Pgl$; left)
    and bispectrum monopole ($\Bg$; right) of the fiducial HOD galaxy catalog (blue).
    The $\Pgl$ and $\Bg$ are averaged over one set of HOD realizations run on
    15,000 $N$-body \quij~simulations measured using the same FFT-based estimator as \cite{hahn2020}. In the 
    left panel, we plot both the power spectrum monopole ($\ell = 0$; solid) and quadrupole 
    ($\ell = 2$; dashed). In the right panel, we plot $\Bg$ for all 1898 triangle
    configurations with $k_1, k_2, k_3 \le k_{\rm max} = 0.5\hmpc$. The
    configurations are ordered by looping through $k_3$ in the inner most loop
    and $k_1$ in the outer most loop satisfying $k_1 \le k_2 \le k3$.
    We include for comparison the \cite{hahn2020} halo $\Phl$ and $\Bh$ at the 
    fiducial cosmology (black). %We note that the fiducial galaxy catalog has $\bar{n}_g \sim 1.63\times 10^{-4}~h^3{\rm Gpc}^{-3}$ and linear bias of $b_g \sim 2.55$.
    }
\label{fig:bgh}
\end{center}
\end{figure}

To estimate the constraining power of $\Pgl$ and $\Bg$, we use Fisher information
matrices, which have been ubiquitously used in
cosmology~\citep[\emph{e.g.}][]{jungman1996,tegmark1997,dodelson2003,heavens2009,verde2010}: 
\beq 
F_{ij} = - \bigg \langle \frac{\partial^2 \mathrm{ln} \mathcal{L}}{\partial \theta_i \partial \theta_j} \bigg \rangle,
\eeq
As in \cite{hahn2020}, we assume that the $\Bg$ likelihood is Gaussian and
neglect the covariance derivative term~\citep{carron2013} and estimate the
Fisher matrix as 
\beq \label{eq:fisher}
F_{ij} = \frac{1}{2}~\mathrm{Tr} \Bigg[\bfi{C}^{-1}
\left(\parti{\Bg}\partj{\Bg}^T + \parti{\Bg}^T \partj{\Bg} \right)\Bigg].
\eeq
We derive the covariance matrix, $\bfi{C}$, using $15,000$ fiducial galaxy
catalogs. The derivatives along the cosmological and HOD
parameters, $\partial \Bg/\partial \theta_i$, are estimated using finite
difference. For all parameters other than $\smnu$, we estimate 
\beq 
\frac{\partial \Bg}{\partial \theta_i} \approx \frac{\Bg(\theta_i^{+})-\Bg(\theta_i^{-})}{\theta_i^+ - \theta_i^-}, 
\eeq
where $\Bg(\theta_i^{+})$ and $\Bg(\theta_i^{-})$ are the average bispectrum of the 
$(500~{\rm simulations})\times(3~{\rm RSD~axes})\times(5~{\rm
HOD~realizations}) = 7,500$
realizations at $\theta_i^{+}$ and $\theta_i^{-}$, the HOD or 
cosmological parameter values above and below the fiducial parameters.  
For $\smnu$, where the fiducial value is 0.0 eV, we use the galaxy catalogs 
at $\smnu^+$, $\smnu^{++}$, $\smnu^{+++}=0.1, 0.2, 0.4$ eV (Table~\ref{tab:sims}) 
to estimate 
\beq \label{eq:dbkdmnu} 
\frac{\partial \Bg}{\partial \smnu} \approx \frac{-21 \Bg(\theta_{\rm fid}^{\rm ZA}) + 
32 \Bg(\smnu^{+}) - 12 \Bg(\smnu^{++}) + \Bg(\smnu^{+++})}{1.2}, 
\eeq
which provides a $\mathcal{O}(\delta \smnu^2)$ order approximation. 
Since the simulations at $\smnu^+$, $\smnu^{++}$, and $\smnu^{+++}$ are generated 
from Zel'dovich initial conditions, we use simulations at the fiducial cosmology 
also generated from Zel'dovich initial conditions ($\theta_{\rm fid}^{\rm ZA}$). 
Our simulation-based approach with galaxy catalogs constructed from
$N$-body simulations is essential for accurately quantifying the constraining power
of the bispectrum beyond the limitations of analytic methods down to nonlinear
regimes.
