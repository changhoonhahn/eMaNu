\begin{figure}
\begin{center}
    \includegraphics[width=0.45\textwidth]{figs/hod_fid.pdf} 
    \caption{Our fiducial halo occupation (black) parameterized using the
    standard \cite{zheng2007} HOD model. The parameter values of our fiducial
    HOD model (Eq.~\ref{eq:hod_fid}) are based on by the best-fit HOD
    parameters of the SDSS $M_r < -21.5$ and $< -22.$ samples from
    \cite{zheng2007} modified to accommodate the $M_{\rm lim} = 3.2\times
    10^{13} h^{-1}M_\odot$ halo mass limit of the \quij simulations (black
    dotted). We include the best-fit halo occupations of the SDSS  $M_r <
    -21.5$ (blue dashed) and $< -22.$ samples (orange dashed) from
    \cite{zheng2007} for reference. Since our HOD parameters are based on the
    high luminosity SDSS samples, we do not include assembly bias.  Our
    fiducial HOD sample has a galaxy number density of $\overline{n}_g \sim
    1.63\times10^{-4}~h^3/{\rm Mpc}^3$ and linear bias of $b_g \sim 2.55$.
    }\label{fig:hod}
\end{center}
\end{figure}

\section{Halo Occupation Distribution} \label{sec:hod}  
We are interested in quantifying the information content of the galaxy bispectrum. 
For a perturbation theory approach, this involves incorporating a bias model 
for galaxies~\citep[\emph{e.g.}][]{sefusatti2006, yankelevich2019, chudaykin2019}.
Perturbation theory approaches, however, break down on small scales and limit
the constraining power from nonlinear regime. Instead, in our simulation based 
approach we use the halo occupation distribution (HOD) 
framework~\citep[\emph{e.g.}][]{benson2000, peacock2000, seljack2000, berlind2002,
cooray2002, zheng2005, leauthaud2012, tinker2013, zentner2016, vakili2019}.
HOD models statistically populate galaxies in dark matter halos by specifying
the probability of a given halo hosting a certain number of galaxies. This 
statistical prescription for connecting galaxies to halos has been remarkably 
successful in reproducing the observed galaxy clustering and, as a result, is the standard approach for constructing 
simulated galaxy mock catalogs in galaxy clustering analyses to estimate covariance 
matrices and test systematic effects~\citep[\emph{e.g.}][]{rodriguez-torres2016, rodriguez-torres2017, beutler2017}. 
More importantly, HOD is the primary framework used in simulation-based galaxy
clustering analyses: for instance, in emulation~\citep[\eg][]{mcclintock2018,
zhai2019} or evidence modeling~\citep[\eg][]{lange2019}. The forecasts we
present in this paper are aimed at quantifying the constraining power of the
galaxy bispectrum for such simulation-based analyses. Therefore, the HOD model
is particularly well-suited for our purpose.

In HOD models, the probability of a given halo hosting $N$ galaxies of a
certain class is dictated by its halo mass --- $P(N|M_h)$. We use the standard
HOD model from \cite{zheng2007}, which specifies the mean number of galaxies in
a halo as
\beq
\langle N_{\rm gal} \rangle = \langle N_{\rm cen} \rangle + \langle N_{\rm sat} \rangle
\eeq
with mean central galaxy occupation
\beq \label{eq:Ncen}
\langle N_{\rm cen} \rangle  = \frac{1}{2}\Bigg[1 + {\rm erf}\bigg(\frac{\log M_h - \log M_{\rm min}}{\sigma_{\log M}}\bigg) \Bigg]
\eeq
and mean satellite galaxy occupation
\beq \label{eq:Nsat}
\langle N_{\rm sat} \rangle = \langle N_{\rm cen} \rangle \bigg(\frac{M_h - M_0}{M_1}\bigg)^\alpha.
\eeq
The mean number of centrals in a halo transitions smoothly from 0 to 1 for halos 
with mass $M_h > M_{\rm min}$. The width of the transition is dictated by 
$\sigma_{\log M}$, which reflects the scatter between stellar mass/luminosity and 
halo mass~\citep{citecite}. For $M_h > M_{\rm min}$, 
$\langle N_{\rm sat} \rangle$ follows a power law with slope $\alpha$. $M_0$ 
is the halo mass cut-off for satellite occupation and $M_h = M_0 + M_1$ is 
the typical mass scale for halos to host one satellite galaxy. The numbers 
of centrals and satellites for each halo are drawn from Bernoulli and Poisson 
distribution, respectively. Central galaxies are placed at the center of the
halo while position and velocity of the satellite galaxies are sampled from a 
\cite{navarro1997} (NFW) profile. 

For the fiducial parameters of our HOD model we use values based on the 
best-fit HOD parameters for the SDSS $M_r < -21.5$  and $-22$ samples from \cite{zheng2007}: 
\beq \label{eq:hod_fid}
\{M_{\rm min}, \sigma_{\log M}, \log M_0, \alpha, \log M_1 \} = \{13.65,~0.2,~14.,~1.1,~14.\}.
\eeq
In Figure~\ref{fig:hod}, we present the halo occupation of our fiducial 
HOD parameters (black). We include the best-fit halo occupations of 
the SDSS $M_r < -21.5$ (blue)  and $-22$ (orange) samples from \cite{zheng2007}
for comparison. We also mark the halo mass limit, $M_{\rm lim}$, of the \quij 
simulations (black dotted). At $M_h \sim 10^{13} M_\odot$, the best-fit halo 
occupations of the SDSS samples extend below $M_{\rm lim}$ so halos below $M_{\rm lim}$ host galaxies. This prevents us from 
directly using the values from the literature and instead, we reduce 
$\sigma_{\log M}$ to 0.2 dex.
As we mention above, $\sigma_{\log M}$ reflects the scatter between stellar 
mass/luminosity and halo mass. The high $\sigma_{\log M}$ in the $M_r < -21.5$ 
and $-22$ SDSS samples is caused by the turnover in the stellar-to-halo mass relation at high stellar 
masses~\citep{mandelbaum2006a, conroy2007, more2011, leauthaud2012, tinker2013,
zu2015, hahn2019a}. Our fiducial halo occupation, with its lower $\sigma_{\log M}$, 
produces a galaxy sample with a tighter scatter than the samples selected based 
on $M_r$ or $M_*$ cuts, which were used in SDSS and BOSS. To select such a sample
would require selecting based on observable galaxy properties that correlate 
more strong with $M_h$ than luminosity or $M_*$. While there is evdience
that such observables are availabe~\citep[\eg~$L_{\rm sat}$; ][]{alpaslan2019},
they have not been adopted for selecting galaxy samples. Regardless, in this work
we focus on quantifying the information content of the galaxy bispectrum 
and not on analyzing a specific observed galaxy sample. We therefore opt for 
a more conservative set of HOD parameters with respect to $M_{\rm lim}$, even
if the resulting galaxy sample is less reflective of observations. For our
fiducial halo occupation at the fiducial cosmology, the galaxy catalog has 
$\bar{n}_g \sim 1.63\times 10^{-4}~h^3~{\rm Gpc}^{-3}$ and linear bias of 
$b_g \sim 2.55$.
 %We confirm using \quij simulations with higher mass resolution ($1024^3$ CDM particles) that $M_{\rm lim}$ does not impact the observables or their derivatives in our analysis for our fiducial HOD parameters. 

The halo occupation in the \cite{zheng2007} model depends soley on $M_h$. 
Simulations, however, find evidence that secondary halo properties such as
concentration or formation history correlate with spatial distribution of
halos --- a phenomenon referred to as ``halo assembly bias''~\citep[\eg][]{sheth2004,
gao2005, harker2006, wechsler2006, dalal2008, wang2009a, lacerna2014,
contreras2020, hadzhiyska2020}.
A model that only depends on $M_h$, does not account for this halo assembly 
bias and may not sufficient describe the connection between galaxies and 
halos. Moreover, if unaccounted for in the HOD model, and thus not marginalized 
over, halo assembly bias can impact the cosmological parameter constraints. 
However, for the high luminosity samples in SDSS ($M_r < -21.5$  and $<-21$), 
\cite{zentner2016} and \cite{vakili2019} find little evidence for assembly bias 
in the galaxy clustering. Similarly, \cite{beltz-mohrmann2020} also find that
the \cite{zheng2007} HOD model is sufficent to reproduce galaxy clustering of
luminous galaxies in hydrodynamic simulations. Since we base our HOD parameters
on the high luminosity SDSS samples, we do not include assembly bias and use
the \cite{zheng2007} model. 

We construct galaxy catalogs using $22,000$ $N$-body simulations of the \quij
suite: $15,000$ at the fiducial cosmology and $500$ at the 14 other cosmologies
listed in Table 1 of \cite{hahn2020}. First, to construct the mock catalogs
used to estimate the covariance matrices and the derivatives with respect to
cosmological parameters, we these simulations with the fiducial HOD parameters.
Next, to construct the mocks for estimating the derivatives with respect to the
5 HOD parameters, we use 500 \quij simulations at the fiducial cosmology with
10 additional sets of HOD parameters --- a pair per parameter. Similar to the
cosmologies in the \quij suite, for each pair we vary one HOD paramter above
and below the fiducial value by step sizes:
\beq
\{\Delta M_{\rm min}, \Delta \sigma_{\log M}, \Delta \log M_0, \Delta \alpha,
\Delta \log M_1 \} = \{0.05, 0.2, 0.2, 0.2, 0.2\}.
\eeq
For the covariance matrix mocks (fiducial cosmology and fiducial HOD), we
generate one set of HOD realizations and apply RSD along the z-axis. For the
rest, used to estimate derivatives, we generate 5 sets of HOD realizations and 
apply RSD along all 3 directions. {\em In total, we construct and use 195,000
galaxy catalogs in our analysis.} All of the galaxy catalogs are publicly available
at \todo{where to access the galaxy catalogs}.
