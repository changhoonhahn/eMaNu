\section{Summary} \label{sec:summary} 

Afterwards, we will apply it to future surveys. 
\ch{rough numbers of DESI, PFS, and Euclid} 

% what are the implications of our results? 
%The constraints from our forecasts are derived for a $(1~h^{-1}{\rm Gpc})^3$ 
%volume. These constraints roughly scale as $\propto1/\sqrt{V}$ with volume. 
%Upcoming spectroscopic galaxy surveys will map out vastly larger cosmic
%volumes: PFS $\sim 9~h^{-3}{\rm Gpc}^3$ and DESI $\sim 50~h^{-3}{\rm Gpc}^3$~\citep{takada2014, desicollaboration2016}. 
%Euclid and WFIRST, space-based surveys, will expand these volumes to 
%higher redshifts. A key science goal in these surveys will be constraining $\smnu$. 
%Scaling our 0.0572 eV $\smnu$ constraint from the bispectrum to the survey 
%volumes gives 0.0191 eV and 0.0081 eV $1\sigma$ constraints for PFS and DESI. 
%Galaxy samples in these surveys will also have significantly higher 
%number densities than the halos used in our forecast: \emph{e.g.} 
%the DESI BGS at $z{\sim}0.3$, DESI LRG, and PFS at $z \sim 1.3$ 
%will have ${\sim}20, 3$, and $5\times$ higher number densities, 
%respectively. With such precision, we would detect $\smnu$ by $>3\sigma$ and 
%distinguish between the normal and inverted neutrino mass hierarchy scenarios 
%by $>2\sigma$ from the bispectrum alone. Such naive projections, however, 
%should be taken with more than a grain of salt. Nevertheless, the substantial 
%improvement we find in constraints with the bispectrum over the power spectrum, 
%strongly advocate for analyzing future surveys with more than the power spectrum. 
%Our results demonstrate the potential of the bispectrum to tightly constrain 
%$\smnu$ with unprecedented precision.
