\section{Summary} \label{sec:summary} 
Tight constraints on the total mass of neutrinos, $\smnu$, inform particle
physics beyond the Standard Model and can potentially distinguish
between the `normal' and `inverted' neutrino mass hierarchies. The current
tightest constraints 
come from measuring the impact of $\smnu$ on the expansion history and the 
growth of cosmic structure in the Universe using cosmological observables --- 
combinations of CMB with other cosmological probes. However, constraints from 
upcoming ground-based CMB experiments will be severely limited by the degeneracy
between $\smnu$ and $\tau$, the optical depth of reionization. Meanwhile, 
measuring the $\smnu$ imprint on the 3D clustering of galaxies provides a 
complementary and opportune avenue for improving $\smnu$ constraints. Progress
in modeling nonlinear structure formation of simulations and in new
simulation-based approaches now enables us to tractably exploit the accuracy of
$N$-body simulations to analyze galaxy clustering. Furthermore, in the next few
years, upcoming surveys such as DESI, PFS, Euclid, and the Roman Space Telescope
will probe unprecedented cosmic volumes with galaxy redshifts. Together, these 
development present the opportunity to go beyond traditional perturbation theory methods, unlock the
information content in nonlinear clustering where the impact of $\smnu$ is
strongest, and tightly constrain $\smnu$ and other cosmological parameters. 

In \cite{hahn2020}, the previous paper of the series, we demonstrated that the 
bispectrum breaks parameter degeneracies (\eg~$\smnu$--$\sig$ degeneracy) that 
serious limit $\smnu$ constraints with traditional two-point clustering statistics. 
We also illustrated the substantial constraining power of the bispectrum in nonlinear regimes.
\cite{hahn2020}, however, focused on the redshift-space halo bispectrum while 
constraints on $\smnu$ will come from galaxy distributions. %Therefore, in this paper, we extend the \cite{hahn2020} forecasts to include a realistic and complete galaxy bias model. 
In this work, we extend the \cite{hahn2020} bispectrum forecasts to
include a realistic galaxy bias model. With our eyes set on
simulation-based analyses, we use the halo occupation distribution (HOD) galaxy
bias framework and construct the \molino~suite---75,000 galaxy mock catalogs from the \quij~$N$-body 
simulations. Using these mocks, we present for the first time the total information
content and constraining power of the {\em redshift-space galaxy bispectrum} 
down to nonlinear regimes. More specifically, we find
\begin{itemize}
    \item $\Bg$ substantial improves in cosmological parameter constraints ---
        especially $\smnu$ --- even after marginalizing over galaxy bias through
        the HOD parameters. Combining $\Pgl$ and $\Bg$ further improves
        constraints by breaking several key parameter degeneracies. For 
        $\kmax{=}0.5\hmpc$, $\Bg$ improves constraints on 
        $\Om$, $\Ob$, $h$, $n_s$, and $\sig$ by 2.8, 3.1, 3.8, 4.2, 4.2, and 
        $4.7{\times}$ over power spectrum. For $\smnu$, we achieve $5.6\times$ 
        tighter constraints with $\Bg$.

    \item Even with priors from \planck, $\Bg$ significantly improves
        cosmological constraints. For $\kmax{=}0.5\hmpc$, including 
        $\Bg$~achieves 2.0, 2.1, 1.9, 1.2, 2.2, and $2.3\times$ tighter
        constraints on $\Om$, $\Ob$, $h$, $n_s$, $\sig$, and $\smnu$ than with $\Pgl$
        and \planck. $\Bg$ also substantially improves constraints at mildly non-linear regimes:
        for $\kmax\sim0.2\hmpc$, $\Bg$ achieves $1.4$ and $2.8\times$ tighter
        $\smnu$ constraints than $\Pgl$ with and without \planck~priors. 

    \item $\Bg$ has substantial constraining power on non-linear regime beyond
        $\kmax > 0.2\hmpc$. This makes $\Bg$ particularly valuable when we include
        \planck~priors: the constraining power of $\Pgl$ completely saturates 
        at $\kmax \gtrsim 0.12\hmpc$ while with $\Bg$, constraints improve out to 
        $\kmax=0.5\hmpc$. For $\Pgl$ and $\Bg$ out to $\kmax{=}0.5\hmpc$, with
        \planck~priors, we achieve a $1\sigma$ $\smnu$ constraint of 0.048 eV.
\end{itemize}

Overall, our results clearly demonstrate the significant advantages of the
galaxy bispectrum for more precisely constraining cosmological parameters ---
especially $\smnu$. There are, however, a few caveats in our forecast. 
Fisher matrix forecasts assume that the posterior is approximately Gaussian and
can overestimate the constraints for highly non-elliptical or asymmetric
posteriors. We also do not consider realistic survey geometry, selection
effects, or super-sample covariance. Lastly, we include galaxy bias through the
standard \cite{zheng2007} HOD model. Although, this model is sufficiently
accurate for a high luminosity galaxy sample that we consider, for galaxy 
samples from upcoming surveys additional effects such as assembly bias and
velocity biases will need to be included. While these effects will impact the
constraining power of $\Bg$, they also impact the constraining power of
$\Pg$. Hence, we nonetheless expect significant improvements from including 
the galaxy bispectrum.

There is, in fact, room for more optimism. All the constraints we present in
this paper are for a $1~h^{-3}{\rm Gpc}^3$ volume and for a galaxy sample with number
density $\bar{n}_g \sim 1.63\times 10^{-4}~h^3{\rm Gpc}^{-3}$. Upcoming surveys
will probe {\em vastly} larger cosmic volumes and with higher number densities.
For instance, PFS will probe $\sim 9~h^{-3}{\rm Gpc}^3$ with ${\sim}5\times$
higher $n_g$ at $z{\sim}1.3$~\citep{takada2014}; DESI will probe ${\sim}50~h^{-3}{\rm Gpc}^3$
and its Bright Galaxy Survey and LRG sample will have ${\sim}20$ and $3\times$ 
higher $n_g$, respectively~\citep{desicollaboration2016,ruiz-macias2020}. 
Euclid and the Roman Space Telescope, space-based surveys, will expand these 
volumes to higher redshifts. Constraints {\em conservatively} scale as $\propto 1/\sqrt{V}$
with volume and higher $\bar{n}_g$ samples will achieve higher signal-to-noise. 
Combined with our results, this suggests that analyzing the galaxy bispectrum in 
upcoming surveys has the potential to tightly constrain $\smnu$ with unprecedented 
precision. 

Now that we have demonstrated the total information content and constraining
power of $\Bg$, in the following paper of this series we will address a major 
practical challenge for a $\Bg$ analysis --- its 
large dimensionality. We will present how data compression can be used to reduce 
the dimensionality and tractably estimate the covariance matrix in a $\Pgl$ and
$\Bg$ analysis using a simulation-based approach. Afterward, we will conduct a 
fully simulation-based $\Pgl$ and $\Bg$ reanalysis of SDSS-III BOSS. The series
will ultimately culminate in extending this simulation-based $\Pgl$ and $\Bg$
analysis to constrain $\smnu$ using the DESI survey. 
