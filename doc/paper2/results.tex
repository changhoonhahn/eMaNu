%%%%%%%%%%%%%%%%%%%%%%%%%%%%%%%%%%%%%%%%%%
% forecast table
%%%%%%%%%%%%%%%%%%%%%%%%%%%%%%%%%%%%%%%%%%
\begin{table}
    \caption{Marginalized Fisher parameter constraints from the redshift-space 
    $P_\ell$, $B_0$, and $P_\ell$ + $B_0$. We list constraints for cosmological 
    parameters $\smnu$, $\Omega_m$, $\Omega_b$, $h$, $n_s$, and $\sig$ as well 
    as HOD and nuisance parameters.} 
\begin{center} 
    \begin{tabular}{ccccc} \toprule
            & $P_\ell$ & $B_0$ & $P_\ell$ + $B_0$ \\[3pt]
\hline 
$\smnu$     & & & \\
$\Om$       & & & \\
$\Ob$       & & & \\
$h$         & & & \\
$n_s$       & & & \\
$\sig$      & & & \\\hline
$M_{\rm min}$       & & & \\
$\sigma_{\log M}$   & & & \\ 
$\log M_0$          & & & \\
$\alpha$            & & & \\ 
$\log M_1$          & & & \\ [3pt]
\hline            
\end{tabular} \label{tab:forecast}
\end{center}
\end{table}
%%%%%%%%%%%%%%%%%%%%%%%%%%%%%%%%%%%%%%%%%%

%\begin{figure}
%    \begin{center}
%        \includegraphics[width=\textwidth]{figs/quijote_dP02Bdtheta_reg_fin.pdf}
%        \caption{}
%        \label{fig:db}
%    \end{center}
%\end{figure}
In Figure~\ref{fig:kmax_forecast} We only include the $k_{\rm max}$ ranges where we have more data bins than
number of parameters. 

\begin{figure}
    \begin{center}
        \includegraphics[width=\textwidth]{figs/Fisher_p02bk_dmnu_fin_kmax0_50_reg.pdf}
        \caption{Fisher matrix constraints for $\smnu$ and other cosmological
        parameters for the redshift-space galaxy $\Pgl$ (blue), $\Bg$
        (green), and combined $\Pgl$ and $\Bg$ (orange) for $k_{\rm max} =
        0.5\hmpc$. Our forecasts marginalizes over the full set of \cite{zheng2007}
        HOD parameters: $\{M_{\rm min}, \sigma_{\log M}, \log M_0, \alpha \log
        M_1 \}$ (bottom panels). The bispectrum substantially improves
        constraints on all of the cosmological parameters over $\Pgl$.
        Constraints on $\Om$, $\Ob$, $h$, $n_s$, and $\sig$ improve by factors
        of \ch{1.9, 2.6, 3.1, 3.6, and 2.6}, respectively. For $\smnu$, the
        bispectrum improves $\sigma_{\smnu}$ from \ch{0.2968 to 0.0572 eV ---
        over a factor of $\sim5$ improvement over the power spectrum}.
        }
        \label{fig:forecast}
    \end{center}
\end{figure}

\begin{figure}
    \begin{center}
        \includegraphics[width=\textwidth]{figs/Fisher_kmax_p02bk_dmnu_fin_reg_planck.pdf}
        \caption{Marginalized $1\sigma$ constraints, $\sigma_\theta$, of the
        cosmological parameters $\Om$, $\Ob$, $h$, $n_s$, $\sig$, and $\smnu$
        as a function of $k_{\rm max}$ for the redshift-space $\Pgl$ (blue),
        $\Bg$ (green), and combined $\Pgl$ and $\Bg$ (orange). Even after
        marginalizing over the \cite{zheng2007} HOD parameters
        (Eq.~\ref{eq:hod_fid}), the bispectrum {\em significantly} improves
        cosmological parameter constraints above $k_{\rm max} > 0.1\hmpc$.
        While constraints improve for higher $k_{\rm max}$, throughout
        $0.2 < k_{\rm max} < 0.5$, the $\Pgl + \Bg$ constraint improves the $\smnu$ constraint by \ch{X, Y \%}
        over the $\Pgl$ constraints

        When we include {\em Planck} priors (dotted), the improvement from
        $\Bg$ is even more evident. The constraining power of $\Pgl$ complete
        saturates for $k_{\rm max} \gtrsim 0.12\hmpc$. Adding $\Bg$ not only improves the
        cosmological constraints, but the constraints continue to improve for
        higher $k_{\rm max}$. At $k_{\rm max} = 0.2$ and $0.5\hmpc$, the $\Pgl
        + \Bg$ constraint improves the $\smnu$ constraint by \ch{X, Y \%}
        over the $\Pgl$ constraints. We emphasize that the constraints above
        are for $1~({\rm Gpc}/h)^3$ box and thus are underestatimes for
        upcoming surveys. 
        }
        \label{fig:kmax_forecast}
    \end{center}
\end{figure}

\section{Results} \label{sec:results} 
