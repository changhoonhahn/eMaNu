%%%%%%%%%%%%%%%%%%%%%%%%%%%%%%%%%%%%%%%%%%
% forecast table
%%%%%%%%%%%%%%%%%%%%%%%%%%%%%%%%%%%%%%%%%%
\begin{table}
    \caption{Marginalized Fisher parameter constraints from the redshift-space 
    $P_\ell$, $B_0$, and $P_\ell$ + $B_0$. We list constraints for cosmological 
    parameters $\smnu$, $\Omega_m$, $\Omega_b$, $h$, $n_s$, and $\sig$ as well 
    as HOD and nuisance parameters.} 
\begin{center} 
    \begin{tabular}{c|ccc|ccc} \toprule
        & & $\kmax=0.2$ & & & $\kmax=0.5$ & \\
        & $P_\ell$ & $B_0$ & $P_\ell + B_0$ & $P_\ell$ & $B_0$ & $P_\ell + B_0$ \\[3pt]
\hline 
$\smnu$     & & & \\
$\Om$       & & & \\
$\Ob$       & & & \\
$h$         & & & \\
$n_s$       & & & \\
$\sig$      & & & \\[3pt] \hline
$M_{\rm min}$       & & & \\
$\sigma_{\log M}$   & & & \\ 
$\log M_0$          & & & \\
$\alpha$            & & & \\ 
$\log M_1$          & & & \\ [3pt]
\hline            
\end{tabular} \label{tab:forecast}
\end{center}
\end{table}
%%%%%%%%%%%%%%%%%%%%%%%%%%%%%%%%%%%%%%%%%%

\begin{figure}
    \begin{center}
        \includegraphics[width=\textwidth]{figs/Fisher_p02bk_dmnu_fin_kmax0_50.pdf}
        \caption{Fisher matrix constraints for $\smnu$ and other cosmological
        parameters for the redshift-space galaxy $\Pgl$ (blue), $\Bg$
        (green), and combined $\Pgl$ and $\Bg$ (orange) for $k_{\rm max} =
        0.5\hmpc$. Our forecasts marginalizes over the \cite{zheng2007}
        HOD parameters: $M_{\rm min}, \sigma_{\log M}, \log M_0, \alpha \log
        M_1$ (bottom panels). The contours mark the $68\%$ and $95\%$
        confidence intervals. The bispectrum substantially improves
        constraints on all of the cosmological parameters over the power
        spectrum. $\Om$, $\Ob$, $h$, $n_s$, and $\sig$ constraints improve by factors
        of \ch{1.9, 2.6, 3.1, 3.6, and 2.6}, respectively. For $\smnu$, the
        bispectrum improves $\sigma_{\smnu}$ from \ch{0.2968 to 0.0572 eV ---
        over a factor of $\sim5$ improvement over the power spectrum}.
        }
        \label{fig:forecast}
    \end{center}
\end{figure}

\begin{figure}
    \begin{center}
        \includegraphics[width=\textwidth]{figs/Fisher_kmax_p02bk_dmnu_fin.pdf}
        \caption{Marginalized $1\sigma$ constraints, $\sigma_\theta$, of the
        cosmological parameters $\Om$, $\Ob$, $h$, $n_s$, $\sig$, and $\smnu$
        as a function of $k_{\rm max}$ for the redshift-space $\Pgl$ (blue)
        and combined $\Pgl + \Bg$ (orange). Even after marginalizing over
        HOD parameters (Eq.~\ref{eq:hod_fid}), the galaxy bispectrum {\em
        significantly} improves cosmological parameter constraints above
        $k_{\rm max} > 0.1\hmpc$. Constraints from $\Pgl$ and
        $\Pgl + \Bg$ improve with higher $k_{\rm max}$. {\em Throughout 
        $0.2 < k_{\rm max} < 0.5$, including the bispectrum improves 
        $\{\Om, \Ob, h, n_s, \sig, \smnu\}$ by \ch{X, Y \%}.} When we include 
        \planck priors (dotted), the improvement from $\Bg$ is even more
        evident. The constraining power of $\Pgl$ complete
        saturates for $\kmax \gtrsim 0.12\hmpc$. Adding $\Bg$ not only 
        improves constraints, but the constraints continue to improve for
        higher $k_{\rm max}$. At $k_{\rm max} = 0.2$ and $0.5\hmpc$, the $\Pgl
        + \Bg$ improves the $\smnu$ constraint by \ch{X, Y \%} over $\Pgl$. 
        We emphasize that the constraints above are for $1~({\rm Gpc}/h)^3$ box
        and thus underestimate the constraining power of upcoming galaxy
        clustering surveys.
        }
        \label{fig:kmax_forecast}
    \end{center}
\end{figure}

\section{Results} \label{sec:results} 
We present the Fisher matrix constraints for $\smnu$ and other cosmological
parameters from the redshift-space galaxy $\Pgl$ (blue), $\Bg$ (green), and 
combined $\Pgl + \Bg$ (orange) in Figure~\ref{fig:forecast}. These
constraints marginalize over the \cite{zheng2007} HOD parameters %$\{M_{\rm min}, \sigma_{\log M}, \log M_0, \alpha \log M_1 \}$ 
(bottom panels) and extends to $k_{\rm max} = 0.5\hmpc$. The contours mark the $68\%$ and
$95\%$ confidence intervals. With the redshift-space $\Pgl$
alone, we derive the following $1\sigma$ constraints for $\{\Om, \Ob, h, n_s,
\sig, \smnu\}$: 
\ch{0.03443, 0.01219, 0.14069, 0.16552, 0.07701, 0.58119}
With the redshift-space $\Bg$ alone, we get: 
\ch{0.01422, 0.00371, 0.03409, 0.03543, 0.02494, 0.10599}.
{\em The galaxy bispectrum produces tighter constraints on all cosmological
parameters over the power spectrum}.

Furthermore, we find that by combining $\Pgl$ and $\Bg$, we derive even better
constraints by breaking a number of parameter degeneracies. Among the cosmological 
parameters, in addition to breaking the $\sig - \smnu$ degeneracy, which limits
power spectrum analyses, the $\Om-\sig$ degeneracy is also broken and leads to
significant improvements in both $\Om$ and 
$\sig$ constraints. Meanwhile, for the HOD parameters, degeneracies with 
$\log M_0$, $\alpha$, and $\log M_1$ are substantially reduced. 
Combining $\Pgl$ and $\Bg$, we get the following $1\sigma$ constraints for 
 $\Om$, $\Ob$, $h$, $n_s$, $\sig$, and $\smnu$: 
 \ch{0.01033, 0.00336, 0.03150, 0.03361, 0.01641, and 0.10299}
{\em With $\Pgl$ and $\Bg$ combined, we improve $\Om$, $\Ob$, $h$,
$n_s$, and $\sig$ constraints by factors of \ch{3.3, 3.6, 4.5, 4.9, and 4.7}
and $\smnu$ constraint by a factor of 5.6 over the $\Pg$ constraints}

In Figure~\ref{fig:kmax_forecast}, we present the marginalized $1\sigma$
constraints, $\sigma_\theta(\kmax)$, of the cosmological parameters $\Om$,
$\Ob$, $h$, $n_s$, $\sig$, and $\smnu$ as a function of $\kmax$ for $\Pgl$
(blue) and the combined $\Pgl + \Bg$ (orange). Again, these constraints are
marginalized over the \cite{zheng2007} HOD parameters. For both $\Pgl$ and
$\Pgl + \Bg$, parameter constraints expectedly improve as
we include higher $\kmax$. More importantly, {\em the galaxy bispectrum significantly 
improves cosmological parameter constraints throughout $\kmax > 0.1\hmpc$
and not only at high $\kmax$}. Even for $\kmax\sim 0.2\hmpc$, including 
$\Bg$ improves $\Om$, $\Ob$, $h$, $n_s$, $\sig$ and $\smnu$ constraints by 
factors of \ch{2.6, 2.4, 2.7, 3.2, 4.1, and 3.6}.

In Figure~\ref{fig:kmax_forecast}, we also present $\sigma_\theta(\kmax)$ for
$\Pgl$ (blue dashed) and $\Pgl + \Bg$ 
(orange dashed) {\em with priors from Planck}. Once we include \planck priors,
$\Pgl$ constraints do not significantly improve for $\kmax \gtrsim 0.12\hmpc$.
On the other hand, the constraints from $\Pgl + \Bg$ continues to improve for $\kmax > 0.15\hmpc$. 
At $\kmax=0.2\hmpc$, $\Bg$ improves the $\Pgl$ + \planck priors constraints on  
$\Om$, $\Ob$, $h$, $n_s$, $\sig$ and $\smnu$ constraint by factors of
\ch{1.5, 1.4, 1.4, 1.1, 1.3, and $1.4\times$}.
At $\kmax = 0.5\hmpc$, $\Bg$ improves the $\Pgl$ + \planck priors constraints on  
$\Om$, $\Ob$, $h$, $n_s$, $\sig$ and $\smnu$ constraint by factors of 
\ch{2.1, 2.1, 2.0, 1.2, 2.2, and $2.2\times$} 
Therefore, even with \planck priors, the galaxy bispectrum significantly improves cosmological 
constraints. In fact, we emphasize that the constraints in Figure~\ref{fig:kmax_forecast} 
are for a $1~({\rm Gpc}/h)^3$ box, a much smaller cosmic volume than upcoming
galaxy redshift surveys (\eg~DESI, Euclid). Our forecasts with \planck priors
{\em underestimate} the constraining power contribution from galaxy clustering 
that we expect from upcoming surveys.  With more constraining power coming 
from galaxy clustering, improvements from including $\Bg$ to $\Pgl$ and \planck
will be larger.

%The improvements of $\Bg$ come from breaking parameter degeneracies. 
%Most noteably, $\Om$ and $\sig$ degeneracies are broken by
%$\Pgl + \Bg$, which improves $\Om - \sig$, $\Om - \smnu$, $\sig-\smnu$ that we care about. 

% comparison to literature 
% ------------------------
% comparison to halo forecast
% SN kmax=0.20: Pg:92.984343, Bg:24.182641, Ph:121.561879, Bh:34.068850
% SN kmax=0.51: Pg:140.263416, Bg:30.561088, Ph:143.901659, Bh:75.042306
In the previous paper of the series~\citep{hahn2020}, we presented the full
information content of the redshift-space halo bispectrum, $\Bh$. For $\Bh$ to
$\kmax=0.5\hmpc$, \cite{hahn2020} derived $1\sigma$ constraints of 0.012, 0.004,
0.04, 0.036, 0.014, and 0.057 for $\Om$, $\Ob$, $h$, $n_s$, $\sig$ and $\smnu$. 
In comparison, we find constraints of \ch{XXXXXXX} for $\Bg$ to
$\kmax=0.5\hmpc$. $\Bg$ produces comparable constraints for $\Ob$, $h$, and
$n_s$ but has less constraining power for $\Om$, $\sig$, and $\smnu$. This is
the same for $\kmax=0.2\hmpc$. A comparison of the signal-to-noise ratios (SNR)
of $\Bg$ and $\Bh$, estimated from the covariance
matrix~\citep[\eg][]{sefusatti2005,sefusatti2006,chan2017}, also confirm the lower
constraining power of $\Bg$. Furthermore, while both $\Bh$ and $\Bg$ SNRs increase 
at higher $\kmax$, however the increase is lower for $\Bg$ than $\Bh$.
Marginalizing over HOD parameters reduces some of the constraining power of 
the bispectrum. Fingers-of-god (FoG), the elongation of satellite galaxies
in redshift-space along the line-of-sight due to their virial velocities inside 
halos, also contributes to this reduction. 
Nevertheless, $\Bg$ significantly improves parameters constraints over $\Pgl$.
In fact, marginalizing over HOD parameters and FoG reduces the constraining
power of the power spectrum more than the bispectrum. Therefore, {\em we find 
larger improvements in the parameter constraints from $\Bg$ over $\Pgl$ than
from $\Bh$ over $\Phl$}.

%comparison to the literature.
Other previous works have also quantified the information content of the
bispectrum:~\citep[\eg][]{scoccimarro2004, sefusatti2006, sefusatti2007,
song2015, tellarini2016, yamauchi2017a, karagiannis2018, yankelevich2019,
chudaykin2019, coulton2019, reischke2019}. 
We focus our comparison to \cite{sefusatti2006}, \cite{yankelevich2019}, 
and \cite{chudaykin2019}, which provide bispectrum forecasts for full sets of
cosmological parameters.
\cite{sefusatti2006} present \lcdm~forecasts for a joint likelihood analysis of
$\Bg$ with $\Pg$ and WMAP. For $\kmax = 0.2\hmpc$, they find that including
$\Bg$ improves constraints on $\Om$, $\Ob$, $h$, $n_s$, and $\sig$ by 1.6, 1.2,
1.5, 1.4, and 1.5 times from the $\Pg$ and WMAP constraints. In comparison, for 
$\kmax = 0.2\hmpc$ and with \planck priors, we find $\Bg$ improves constraints
by  1.5, 1.4, 1.4, 1.1, and $1.3\times$, which is in good agreement. There are,
however, some significant differences between our analyses. First, \cite{sefusatti2006} 
uses the WMAP likelihood while we use priors from {\em Planck}. Furthermore, 
in our simulation-based approach, we marginalizes over the HOD parameters
whereas \cite{sefusatti2006} marginalize over the linear and quadratic bias
terms ($b_1, b_2$) in their perturbation theory approach. Nevertheless, our
results are consistent with the improvement \cite{sefusatti2006} find in
parameter constraints with $\Bg$. 

%comparison to the Yankelevic 
Next, \cite{yankelevich2019} present \lcdm, $w$CDM and $w_0w_a$CDM Fisher
forecasts for a Euclid-like survey\cite{laureijs2011} over $0.65 < z < 2.05$.
Focusing only on their \lcdm~forecasts, they find that for $\kmax = 0.15\hmpc$, 
$\Pg+\Bg$ produces constraints on $\Omega_{\rm cdm}$, $\Ob$, $A_s$, $h$, $n_s$ 
that are ${\sim}1.3\times$ tighter than $\Pg$ alone. In contrast, we find even
at $\kmax=0.15\hmpc$ significantly larger improvement in the parameter constraints 
from including $\Bg$. We note that \cite{yankelevich2019} present forecasts for
a significantly different galaxy sample. For instance, their $z = 0.7$ redshift
bin has $\bar{n}_g = 2.76 \times 10^{-3}~h^3{\rm Gpc}^{-3}$ and linear bias of
$b_g = 1.18$. Meanwhile our galaxy sample is at $z=0$ with $\bar{n}_g \sim 1.63\times
10^{-4}~h^3{\rm Gpc}^{-3}$ and linear bias of $b_g \sim 2.55$
(Section~\ref{sec:hod}). Furthermore, while we use the HOD framework, they use
a bias expansion with linear, non-linear, and tidal bias ($b_1$, $b_2$, and
$b_{s^2}$). They also marginalize over 56 nuisance parameters since they
jointly analyze $14$ $z$ bins, each with $4$ nuisance parameters.  Lastly,
\cite{yankelevich2019} use perturbation theory models and, therefore, limit
their forecast to $\kmax = 0.15\hmpc$ due
to theoretical uncertainties. Despite the differences, when they estimate the constraining
power beyond $\kmax > 0.15\hmpc$ using Figure of Merit they find that the
constraining power of $\Bg$ relative to $\Pg$ increases for higher $\kmax$
consistent with our results. 

%comparison to the Chudaykin 
Finally, \cite{chudaykin2019} present $\smnu$ + \lcdm~forecasts for the power
spectrum and bispectrum of a Euclid-like survey over $0.5 < z < 2.1$. For
$\omega_{\rm cdm}$, $\omega_b$, $h$, $n_s$, $A_s$, and $\smnu$ they find
${\sim}1.2, 1.5, 1.4, 1.3$, and $1.1\times$ tighter constraints from $\Pgl$ and
$\Bg$ than from $\Pgl$ alone. For $\smnu$, they find a factor of 1.4 improvement, 
from 0.038 eV to 0.028 eV. With \planck, they get ${\sim}2, 1.1, 2.3, 1.5$,
$1.1$, and $1.3\times$ tighter constraints for $\omega_{\rm cdm}$, $\omega_b$,
$h$, $n_s$, $A_s$, and $\smnu$ from including $\Bg$. Overall, \cite{chudaykin2019} 
find significant improvements from including $\Bg$ --- consistent with our
results. However, they find more modest improvements. 
Again, there are significant differences between our anlayses. First, like
\cite{yankelevich2019}, \cite{chudaykin2019} present forecasts for a
Euclid-like survey, which is significantly different than our galaxy sample.
Their $z = 0.6$ redshift bin, for instance, has $\bar{n}_g = 3.83 \times 10^{-3}~h^3{\rm
Gpc}^{-3}$ and linear bias of $b_g = 1.14$. Next, they include the
Alcock-Paczynski (AP) effect for $\Pgl$ but not for $\Bg$. They find that
including the AP effect significantly improves $\Pgl$ constraints (e.g.
tightens $\smnu$ constraints by ${\sim}30\%$); this reduces the improvement
they report from including $\Bg$. 

Another difference between our analyses is that although \cite{chudaykin2019} use 
a more accurate Markov-Chain Monte-Carlo (MCMC) approach to derive parameter
constraints, they neglect the non-Gaussian contributions to both $\Pgl$ and
$\Bg$ covariance matrices and also do not include the covariance between $\Pgl$
and $\Bg$ for the joint constraints. We find that neglecting the covariance
between $\Pgl$ and $\Bg$ overestimates constraints by \ch{XXXX} for our
$\kmax=0.2\hmpc$ constraints. Lastly, \cite{chudaykin2019} use a one-loop 
and tree-level perturbation theory to model $\Pgl$ and $\Bg$, respectively.
Rather than imposing a $\kmax$ cutoff to restrict their forecasts to scales
where their perturbation theory models can be trusted, they use a theoretical
error covariance model approach from \cite{baldauf2016}. With a tree-level
$\Bg$ model, theoretical errors quickly dominate at $\kmax \gtrsim 0.1\hmpc$,
where one- and two-loop contribute significantly~\citep[\eg][]{lazanu2018}. 
So effectively, their forecasts do not include the constraining power on
those scales. If we restrict our forecast to $\kmax = 0.25\hmpc$ 
for $\Pgl$ and $\kmax = 0.1\hmpc$ for $\Bg$, our $\Om$, $\Ob$, $h$, $n_s$, 
$\sig$, and $\smnu$ constraints improve by 1.3, 1.2, 1.2, 1.4, 2.0, and 
$1.7\times$ from including $\Bg$, roughly consistent with \cite{chudaykin2019}. 

%Various differences between our forecast and previous work prevent more thorough comparisons. However, crucial aspects of our simulation based approach distinguish our forecasts from other works. We present the first bispectrum forecasts for a full set of cosmological parameters using bispectrum measured entirely from N-body simulations. By using the simulations, we go beyond perturbation the- ory models and accurately model the redshift-space bispectrum to the nonlinear regime. Furthermore, by exploiting the immense number of simulations, we accurately estimate the full high-dimensional covariance matrix of the bispectrum. With these advantages, we present the first forecast of cosmo- logical parameters from the bispectrum down to nonlinear scales and demonstrate the constraining power of the bispectrum for Mν. Below, we underline a few caveats of our forecasts.

% caveats paragraph 
Among the various differences between our forecast and previous works, we
emphasize that we use a simulationed-based approach. This allows us to go beyond previous
perturbation theory approaches and accurately quantify the constraining power
in the nonlinear regime. A simulation-based approach, however, has a few caveats. 
First, our forecasts rely on the stability and convergence of the covariance 
matrix and numerical derivatives. 
For our constraints we use $195000$ galaxy catalogs (Section~\ref{sec:hod}): 
$15,000$ for the covariance matrices and $180,000$ for the derivatives with 
respect to 11 parameters. To ensure the robustness of our results, we conduct the
same set of convergence tests as \cite{hahn2020}. 
First, we test whether our results have sufficiently converged by deriving our 
constraints using different numbers of galaxy catalogs to estimate the covariance 
matrix and derivatives: $N_{\rm cov}$ and $N_{\rm deriv}$. For $N_{\rm cov}$,
we find $< XXXX\%$ variation $\sigma_\theta$ for $N_{\rm cov} > 12000$.
For $N_{\rm deriv}$, 
we find $< XXXX\%$ variation $\sigma_\theta$ for $N_{\rm cov} > 12000$.
vary by $< 10\%$, we conclude that the conergence of the covariance matrix or
deriviatves do not significantly impact our forecast. 
\ch{fill this in once we have the convergence test}. 

% derivative w.r.t. Mnu
Besides the convergence of the numerical derivatives, the $\smnu$ derivatives
can be evaluated using different sets of cosmologies. In our anlaysis, we
evaluate ${\partial \Pgl}/{\partial \smnu}$ and ${\partial \Bg}/{\partial
\smnu}$ using simulations at the $\{\theta_{\rm ZA}, \smnu^+, \smnu^{++},
\smnu^{+++}\}$ cosmologies. They can, however, also be estimated using 
two other sets of cosmologies: (i) $\{\theta_{\rm ZA}, \smnu^+\}$ and (ii)
$\{\theta_{\rm ZA}, \smnu^+, \smnu^{++}\}$. If we used (i) estimates for 
${\partial \Pgl}/{\partial \smnu}$ and ${\partial \Bg}/{\partial \smnu}$, 
compared to our forecasts, we get \ch{XXXXX}. For (ii), we get \ch{XXXX}. 
\ch{check this once the multiple HOD seeds finish.} 

% derivative w.r.t. sigma_log M 
For our fiducial HOD, we chose parameter values based on \cite{zheng2007} 
fits to the SDSS $M_r < -21.5$  and $-22$ samples, except for the tighter
scatter $\sigma_{\log M} = 0.2$ dex, due to the halo mass limit of \quij
(Section~\ref{sec:hod}). As a result, our HOD galaxy 
catalogs have a different selection function than observed samples, typically
selected based on $M_r$ or $M_*$ cuts (\eg~SDSS or BOSS). To test the impact
of the fiducial $\sigma_{\log M}$ choice, \ch{in Appendix~\ref{sec:sig}, we compare 
${\partial \Pgl}/{\partial \sigma_{\log M}}$ and 
${\partial \Bg}/{\partial \sigma_{\log M}}$ at $\sigma_{\log M} = 0.2$ dex to
the derivates evaluated at $\sigma_{\log M} = $, estimated using higher
resolution \quij simulations. Fill in after we do the comparison.}
\ch{what can we say about the sigma8-sigmalogM degeneracy?}

% usual Fisher Forecast caveat  
Besides convergence and stability, our forecasts are derived from Fisher
matrices. We, therefore, assume that the posterior is approximately Gaussian. 
When posteriors are highly non-elliptical or asymmetric, Fisher forecasts 
significantly underestimate the constraints~\citep{wolz2012}. We note that in
this paper we do not derive actual parameter constraints. Instead, we focus on
quantifying the information content and constraining power of $\Bg$ relative to
$\Pgl$. Hence, we do not explore beyond the Fisher forecast. When we analyze
the SDSS-III BOSS data using a simulation-based approach later in the series,
we will use a robust method to sample the posterior. 

% limiations of current work and future works
In addition to the caveats above, a number of extra steps and complications remain
between this work and a full galaxy bispectrum analysis. For instance, we use
the standard \cite{zheng2007} HOD model,
which does not include assembly bias. \cite{zentner2016} and \cite{vakili2019}
find little evidence for assembly bias in the galaxy clustering
of the SDSS $M_r < -21.5$  and $-21$ samples. \cite{beltz-mohrmann2020} 
also found that the basic HOD is sufficent to reproduce several galaxy
clustering statistics (\eg~projected and 3D 2-point correlation functions, group multiplicity function)
of high luminosity galaxies in the Illustris and EAGLE hydrodynamic
simulations. While the standard HOD is sufficient for our forecast, 
many works have demonstrated that assembly bias impacts galaxy
clustering for lower luminsoity/mass samples both using
observations~\citep{pujol2014, hearin2016, pujol2017, zentner2019, vakili2019, obuljen2020}
and hydrodynamic simulations~\citep{chaves-montero2016, beltz-mohrmann2020}. 
 
Central and satellite velocity biases, which are not included in the
\cite{zheng07} HOD can also impact galaxy clustering~\citep{guo2015a,guo2015}. 
Central galaxies, both in observations and simulations, are not found to be 
at rest in the centers of the host 
halos~\citep[\eg][]{berlind2003, yoshikawa2003, vandenbosch2005, skibba2011}. 
Similarly, satellite galaxies in simulations do not have the same velocities as
the underlying dark matter~\citep[\eg][]{diemand2004, gao2004, lau2010,
munari2013, wu2013}. The central velocity bias reduces the Kaiser effect and
the satellite velocity bias reduces the FoG effect; hence both can impact
galaxy clustering. However, for the high luminosity SDSS samples,
\cite{guo2015} find little satellite velocity bias.
%While they find some central velocity bias, their constraints are based on galaxy clustering on very small scales (${\sim}0.1-25h^{-1}{\rm Mpc}$). 
In simulations, \cite{beltz-mohrmann2020} find that removing central and
satellite velocity biases in the Illustris and EAGLE simulations has
little impact on various clustering measurements of high luminosity
samples. Although assembly bias and velocity bias likely do not impact our 
forecasts, they must be included for lower luminosity/mass galaxy samples 
and for higher precision measurements of observations. Therefore, when 
we reanalyze BOSS with a simulation-based approach later in the series,
we will use a decorated HOD framework~\citep[\eg][]{hearin2016, vakili2019,
zhai2019} that includes both assembly bias and velocity biases. 
Given the improvements we see in HOD parameter constraints from $\Bg$ in
Figure~\ref{fig:forecast}, $\Bg$ also has the potential to better
constrain the assembly bias parameters and improve our understanding of the 
galaxy-halo connection. 

% Baryonic effects? 
Our analysis also does not include baryonic effects. Although they have been 
typically neglected in galaxy clustering analyses, baryonic effects, such as
feedback from active galactic nuclei (AGN), can impact the matter distribution
at cosmological distances~\citep[\eg][]{white2004, zhan2004, jing2006,
rudd2008, harnois-deraps2015}. %(White 2004; Zhan & Knox 2004; Jing et al. 2006; Rudd et al. 2008; van Daalen et al. 2011)
For AGN feedback in particular, various works find an impact on the matter 
power spectrum~\citep[\eg][]{vandaalen2011, vogelsberger2014, hellwing2016, peters2018,
springel2018, chisari2018, vandaalen2020}. % also Vogelsberger et al. 2014a, Hellwing et al. 2016, Peters
% et al. 2018, Springel et al. 2018, Chisari et al. 2018, vandaalen2020)
Although there is no consensus on the magnitude of the effect, ultimately, a 
more effective AGN feedback increases the impact on the matter 
clustering~\citep{barreira2019}. In state-of-the-art hydrodynamical simulations,
\cite{foreman2019} find $\lesssim 1\%$ impact on 
the matter power spectrum at $k \lesssim 0.5\hmpc$. For the matter bispectrum, 
\cite{foreman2019} find that the effect of baryons is peaked at $k=3\hmpc$ and, 
similarly, a $\lesssim1\%$ effect at $k \lesssim 0.5\hmpc$. Although there is growing
evidence of baryon impacting the matter clustering, the effect is mainly
on scales smaller than what is probed by galaxy clustering analyses with
spectroscopic redshift surveys. We, therefore, do not include baryonic effects
in our forecasts and do not consider it further in the series. 

% other things we want to take into account
%alcock pacinzsky effect
In our forecasts, we use $\Bg$ with triange defined in $k_1,k_2,k_3$ bins of
width $\Delta k = 3 k_f$ (Section~\ref{sec:methods}).
\cite{gagrani2017} find that for the growth rate parameter bispectrum
multipoles beyond the monopole have significant constraining power.  
\cite{yankelevich2019}, with figure-of-merit (FoM) estimates, also find
significant information content beyond the monopole. Furthermore, 
\cite{yankelevich2019} also find that coarser binning of the triangle
configurations reduces the information content of the bispectrum: binning by
$\Delta k = 3 k_f$ has ${\sim}10\%$ less constraining power than binning by
$\Delta k = k_f$. While including higher order multipoles and increase the binning 
are straightforward to implement, they both increase the dimensionality of the 
data vector. $\Bg$ alone binned by $\Delta k = 3 k_f$ already has 1898
dimensions. Including the bispectrum multipoles and increasing the
binning would not be feasible for a full bispectrum analysis without the use of data
compression~\citep[\eg][]{byun2017, gualdi2018, gualdi2019a, gualdi2019}. 
Thus, in the next paper in the series, we present how data compression can be
incorporated in a galaxy bispectrum analysis.

Lastly, our forecasts are derived using periodic boxes and do not consider a
realistic geometry or radial selection function of galaxy surveys. A realistic 
selection function will smooth the triangle configuration dependence and degrade 
the constraining power of the bispectrum~\citep{sefusatti2005}. We also do not
account for super-sample covariance, which may also impact our
constraints~\citep{hamilton2006, sefusatti2006, takada2013, li2018, wadekar2019}. 
Since these effects also affect the power spectrum, we still expect to find
substantial improvements in cosmological parameter constraints from including
the bispectrum, especially for $\smnu$. 

In this paper, we present the total information content and constraining power
of the galaxy bispectrum down to the nonlinear regime. Even after marginalizing
over galaxy bias, through the HOD parameters, including $\Bg$ substantial
improves in cosmological parameter constraints --- especially $\smnu$.
Combining $\Pgl$ and $\Bg$ breaks several key parameter degeneracies
and further improves cosmological parameter constraints. We find significant
improvements from $\Bg$ even at $\kmax \sim 0.2\hmpc$ and with \planck priors. 
Furthermore, the constraints we present is for a $1h^{-1}{\rm Gpc}$ 
box and $\bar{n}_g \sim 1.63\times 10^{-4}~h^3{\rm Gpc}^{-3}$ and upcoming surveys
will probe substantially larger cosmic volumes with higher number densities.
We discuss a number of factors that will impact the constraining power of $\Bg$
for actual galaxy clustering analyses, such as assembly bias, survey geometry,
super-sample covariance, and etc. However, our forecasts suggest that even if
the constraining power is reduced the galaxy bispectrum will significantly improve
cosmology parameter constraints.

Now that we have demonstrated the constraining power of $\Bg$, in the following 
paper of this series we will address a major practical challenge for a $\Bg$
analysis --- its large dimensionality. We will present how data compression can
be used to reduce the dimensionality and tractably estimate the covariance
matrix in a $\Pgl$ and $\Bg$ analysis using a simulation-based approach. Afterwards, 
the series will culminate in fully simulation-based $\Pgl$ and $\Bg$
reanalysis of SDSS-III BOSS. 

%A key part of the improvement from $\Bg$ comes from the degeneracies that are
%broken among HOD parameters. 
%Furthermore, including $\Bg$ allows us to break a number of degeneracies among
%HOD parameters. This is interesting because there are many questions regarding
%HOD parameters that are still up in the air. For instance, the impact and
%importance of assembly bias, which can impact cosmological analyses. Moreover,
%tighter constraints on the halo occupation in general allows us to better 
%understand the galaxy-halo connection, which will improve our understanding of
%galaxy formation and evolution. While we consider a relatively simplistic halo
%occupation model, our results clearly demonstrate the advantages of
%higher-order statistic bispectrum over the standard two-point statistics. 
