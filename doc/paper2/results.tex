%%%%%%%%%%%%%%%%%%%%%%%%%%%%%%%%%%%%%%%%%%
% forecast table
%%%%%%%%%%%%%%%%%%%%%%%%%%%%%%%%%%%%%%%%%%
\begin{table}
    \caption{Marginalized Fisher parameter constraints from the redshift-space 
    $P_\ell$, $B_0$, and $P_\ell$ + $B_0$. We list constraints for cosmological 
    parameters $\smnu$, $\Omega_m$, $\Omega_b$, $h$, $n_s$, and $\sig$ as well 
    as HOD and nuisance parameters.} 
\begin{center} 
    \begin{tabular}{c|ccc|ccc} \toprule
        & & $\kmax=0.2$ & & & $\kmax=0.5$ & \\
        & $P_\ell$ & $B_0$ & $P_\ell + B_0$ & $P_\ell$ & $B_0$ & $P_\ell + B_0$ \\[3pt]
\hline 
$\smnu$     & & & \\
$\Om$       & & & \\
$\Ob$       & & & \\
$h$         & & & \\
$n_s$       & & & \\
$\sig$      & & & \\[3pt] \hline
$M_{\rm min}$       & & & \\
$\sigma_{\log M}$   & & & \\ 
$\log M_0$          & & & \\
$\alpha$            & & & \\ 
$\log M_1$          & & & \\ [3pt]
\hline            
\end{tabular} \label{tab:forecast}
\end{center}
\end{table}
%%%%%%%%%%%%%%%%%%%%%%%%%%%%%%%%%%%%%%%%%%

\begin{figure}
    \begin{center}
        \includegraphics[width=\textwidth]{figs/Fisher_p02bk_dmnu_fin_kmax0_50.pdf}
        \caption{Fisher matrix constraints for $\smnu$ and other cosmological
        parameters for the redshift-space galaxy $\Pgl$ (blue), $\Bg$
        (green), and combined $\Pgl$ and $\Bg$ (orange) for $k_{\rm max} =
        0.5\hmpc$. Our forecasts marginalizes over the \cite{zheng2007}
        HOD parameters: $\{M_{\rm min}, \sigma_{\log M}, \log M_0, \alpha \log
        M_1 \}$ (bottom panels). The contours mark the $68\%$ and $95\%$
        confidence intervals. The bispectrum substantially improves
        constraints on all of the cosmological parameters over the power
        spectrum. $\Om$, $\Ob$, $h$, $n_s$, and $\sig$ constraints improve by factors
        of \ch{1.9, 2.6, 3.1, 3.6, and 2.6}, respectively. For $\smnu$, the
        bispectrum improves $\sigma_{\smnu}$ from \ch{0.2968 to 0.0572 eV ---
        over a factor of $\sim5$ improvement over the power spectrum}.
        }
        \label{fig:forecast}
    \end{center}
\end{figure}

\begin{figure}
    \begin{center}
        \includegraphics[width=\textwidth]{figs/Fisher_kmax_p02bk_dmnu_fin.pdf}
        \caption{Marginalized $1\sigma$ constraints, $\sigma_\theta$, of the
        cosmological parameters $\Om$, $\Ob$, $h$, $n_s$, $\sig$, and $\smnu$
        as a function of $k_{\rm max}$ for the redshift-space $\Pgl$ (blue)
        and combined $\Pgl + \Bg$ (orange). Even after marginalizing over
        HOD parameters (Eq.~\ref{eq:hod_fid}), the galaxy bispectrum {\em
        significantly} improves cosmological parameter constraints above
        $k_{\rm max} > 0.1\hmpc$. Constraints from $\Pgl$ and
        $\Pgl + \Bg$ improve with higher $k_{\rm max}$. {\em Throughout 
        $0.2 < k_{\rm max} < 0.5$, including the bispectrum improves 
        $\{\Om, \Ob, h, n_s, \sig, \smnu\}$ by \ch{X, Y \%}.} When we include 
        \planck priors (dotted), the improvement from $\Bg$ is even more
        evident. The constraining power of $\Pgl$ complete
        saturates for $\kmax \gtrsim 0.12\hmpc$. Adding $\Bg$ not only 
        improves constraints, but the constraints continue to improve for
        higher $k_{\rm max}$. At $k_{\rm max} = 0.2$ and $0.5\hmpc$, the $\Pgl
        + \Bg$ improves the $\smnu$ constraint by \ch{X, Y \%} over $\Pgl$. 
        We emphasize that the constraints above are for $1~({\rm Gpc}/h)^3$ box
        and thus underestimate the constraining power of upcoming galaxy
        clustering surveys.
        }
        \label{fig:kmax_forecast}
    \end{center}
\end{figure}

\section{Results} \label{sec:results} 
We present the Fisher matrix constraints for $\smnu$ and other cosmological
parameters from the redshift-space galaxy $\Pgl$ (blue), $\Bg$ (green), and 
combined $\Pgl + \Bg$ (orange) in Figure~\ref{fig:forecast}. These
constraints marginalize over the \cite{zheng2007} HOD parameters %$\{M_{\rm min}, \sigma_{\log M}, \log M_0, \alpha \log M_1 \}$ 
(bottom panels) and extends to $k_{\rm max} = 0.5\hmpc$. The contours mark the $68\%$ and
$95\%$ confidence intervals. With the redshift-space $\Pgl$
alone, we derive the following $1\sigma$ constraints for $\{\Om, \Ob, h, n_s,
\sig, \smnu\}$: 
\ch{0.03443, 0.01219, 0.14069, 0.16552, 0.07701, 0.58119}
With the redshift-space $\Bg$ alone, we get: 
\ch{0.01422, 0.00371, 0.03409, 0.03543, 0.02494, 0.10599}.
{\em The galaxy bispectrum substantially improves the constraints
on all cosmological parameters over the power spectrum.} 

With $\Pgl$ and $\Bg$, we derive even better constraints by breaking a number
of parameter degeneracies. Among the cosmological parameters, the $\Om-\sig$ 
degeneracy is broken and leads to significant improvements in both $\Om$ and 
$\sig$ constraints. Meanwhile, for the HOD parameters, degeneracies with 
$\log M_0$, $\alpha$, and $\log M_1$ are substantially reduced. 
Combining $\Pgl$ and $\Bg$, we get the following $1\sigma$ constraints for 
 $\Om$, $\Ob$, $h$, $n_s$, $\sig$, and $\smnu$: 
 \ch{0.01033, 0.00336, 0.03150, 0.03361, 0.01641, and 0.10299}
{\em With $\Pgl$ and $\Bg$ combined, we improve $\Om$, $\Ob$, $h$,
$n_s$, and $\sig$ constraints by factors of \ch{3.3, 3.6, 4.5, 4.9, and 4.7}
and $\smnu$ constraint by a factor of 5.6.}

In Figure~\ref{fig:kmax_forecast}, we present the marginalized $1\sigma$
constraints, $\sigma_\theta(\kmax)$, of the cosmological parameters $\Om$,
$\Ob$, $h$, $n_s$, $\sig$, and $\smnu$ as a function of $\kmax$ for $\Pgl$
(blue) and the combined $\Pgl + \Bg$ (orange). Again, these constraints are
marginalized over the \cite{zheng2007} HOD parameters (Eq.~\ref{eq:hod_fid}). 
For both $\Pgl$ and $\Pgl + \Bg$, parameter constraints improve at higher
$\kmax$. More importantly, {\em the galaxy bispectrum significantly 
improves constraints on all cosmological parameters throughout the $\kmax$
range and not only at high $\kmax$}. Even for $\kmax\sim 0.2\hmpc$, including 
$\Bg$ improves $\Om$, $\Ob$, $h$, $n_s$, $\sig$ and $\smnu$ constraints by 
factors of 
\ch{2.6, 2.4, 2.7, 3.2, 4.1, and 3.6}.

We also present $\sigma_\theta(\kmax)$ for $\Pgl$ (blue dashed) and $\Pgl + \Bg$ 
(orange dashed) {\em with priors from \planck}. Once we include \planck priors,
$\Pgl$ constraints do not improve for $\kmax \gtrsim 0.12\hmpc$. However, the 
constraining power of $\Pgl + \Bg$ continues to increase for $\kmax > 0.15\hmpc$. 
At $\kmax=0.2\hmpc$, $\Bg$ improves the $\Pgl$ + \planck priors constraints on  
$\Om$, $\Ob$, $h$, $n_s$, $\sig$ and $\smnu$ constraint by factors of
\ch{1.5, 1.4, 1.4, 1.1, 1.3, and $1.4\times$}.
At $\kmax = 0.5\hmpc$, $\Bg$ improves the $\Pgl$ + \planck priors constraints on  
$\Om$, $\Ob$, $h$, $n_s$, $\sig$ and $\smnu$ constraint by factors of 
\ch{2.1, 2.1, 2.0, 1.2, 2.2, and $2.2\times$} 
Even with \planck priors, the galaxy bispectrum significantly improves cosmological 
constraints. In fact, we emphasize that the constraints in Figure~\ref{fig:kmax_forecast} 
are for a $1~({\rm Gpc}/h)^3$ box. Hence, they {\em underestimate} the 
constraining power contribution from galaxy clustering that we expect from upcoming
galaxy redshift surveys, which will probe a much larger volume (\emph{e.g.} DESI, Euclid). 
With more constraining power coming from galaxy clustering, improvements from
including $\Bg$ to $\Pgl$ and \planck will be larger. 

% comparison to literature 
% ------------------------
% comparison to halo forecast
% SN kmax=0.20: Pg:92.984343, Bg:24.182641, Ph:121.561879, Bh:34.068850
% SN kmax=0.51: Pg:140.263416, Bg:30.561088, Ph:143.901659, Bh:75.042306
In the previous paper of the series, \cite{hahn2020} presents the full
information content of the redshift-space halo bispectrum, $\Bh$. For $\Bh$ to
$\kmax=0.5\hmpc$, \cite{hahn2020} find $1\sigma$ constraints of 0.012, 0.004,
0.04, 0.036, 0.014, and 0.057 for $\Om$, $\Ob$, $h$, $n_s$, $\sig$ and $\smnu$. 
In comparison, we find that $\Bg$ produces comparable constraints for $\Ob$, 
$h$, and $n_s$. On the other hand, $\Bg$ has less constraining power than 
$\Bh$ for $\Om$, $\sig$, and $\smnu$. This is the same for $\kmax=0.2\hmpc$. 
When we compare the signal-to-noise ratios (SNR) of $\Bg$ and $\Bh$, estimated 
from the covariance matrix~\citep[\eg][]{sefusatti2005,sefusatti2006,chan2017}, 
we find lower SNR for $\Bg$, consistent with the constraints. We also find that 
the increase in SNR with $\kmax$ is lessened for $\Bg$. This demonstrates that 
marginalizing over HOD parameters reduces some of the constraining power of 
$\Bg$.  Fingers-of-god (FoG), also contributes to this reduction. 
\ch{elaborate on how FoG}.
Nevertheless, $\Bg$ significantly improves parameters constraints over $\Pgl$.
In fact, marginalizing over HOD parameters and FoG reduces the constraining
power of the power spectrum more so than the bispectrum. Therefore, we find 
larger improvements in the parameter constraints from $\Bg$ over $\Pgl$ than
from $\Bh$ over $\Phl$.

%comparison to the literature.
A number of previous works have quantified the information content of the
bispectrum:~\citep[\eg][]{scoccimarro2004, sefusatti2006, sefusatti2007,
song2015, tellarini2016, yamauchi2017a, karagiannis2018, yankelevich2019,
chudaykin2019, coulton2019, reischke2019}. 
We, however, focus our comparison to \cite{sefusatti2006}, \cite{yankelevich2019}, 
and \cite{chudaykin2019}, which provide bispectrum forecasts for full sets of
cosmological parameters.
\cite{sefusatti2006} present \lcdm forecasts for a joint likelihood analysis of
$\Bg$ with $\Pg$ and WMAP. For $\kmax = 0.2\hmpc$, they find that including
$\Bg$ improves constraints on $\Om$, $\Ob$, $h$, $n_s$, and $\sig$ by 1.6, 1.2,
1.5, 1.4, and 1.5 times from the $\Pg$ and WMAP constraints. In comparison, for 
$\kmax = 0.2\hmpc$ and with \planck priors, we find $\Bg$ improves constraints
by  1.5, 1.4, 1.4, 1.1, and $1.3\times$, which is in good agreement with
\cite{sefusatti2006}. We note, however, that there are
some significant differences in our analyses. First, \cite{sefusatti2006} uses 
the WMAP likelihood while we use priors from \planck. Furthermore, 
in our simulation-based approach, we marginalizes over the HOD parameters. On
the other hand, \cite{sefusatti2006} use a perturbation theory approach and 
marginalize over the linear and quadratic bias terms ($b_1, b_2$). 
Nevertheless, the improvement \cite{sefusatti2006} find in parameter constraints 
including $\Bg$ is in good agreement with our results. 

%comparison to the Yankelevic 
Next, \cite{yankelevich2019} present \lcdm, $w$CDM and $w_0w_a$CDM Fisher
forecasts for a Euclid-like survey\cite{laureijs2011} over $0.65 < z < 2.05$.
Focusing only on their \lcdm forecasts, they find that for $\kmax = 0.15\hmpc$, 
$\Pg+\Bg$ produces constraints on $\Omega_{\rm cdm}$, $\Ob$, $A_s$, $h$, $n_s$ 
that are ${\sim}1.3\times$ tighter than $\Pg$ alone. With \planck priors, they 
find parameter constraints improve by a factor of ${\sim}1.1$. In contrast, we 
find even at $\kmax=0.15\hmpc$ significantly larger improvement in the parameter 
constraints from including $\Bg$. However, we similarly find that the improvement 
descreases once we include \planck priors (Figure~\ref{fig:kmax_forecast}). 

Although \cite{yankelevich2019} find significantly less improvement in
parameter constraints from $\Bg$, we emphasize that \cite{yankelevich2019}
present forecasts for a significantly different galaxy sample --- \ie the
Euclid survey over $0.65 < z < 2.05$. For instance, their $z = 0.7$ redshift
bin has $\bar{n}_g = 2.76 \times 10^{-3}~h^3{\rm Gpc}^{-3}$ and linear bias of
$b_g = 1.18$. Meanwhile our galaxy sample is at $z=0$ with $\bar{n}_g \sim 1.63\times
10^{-4}~h^3{\rm Gpc}^{-3}$ and linear bias of $b_g \sim 2.55$
(Section~\ref{sec:hod}). Furthermore, while we use the HOD framework, they use
a bias expansion with linear, non-linear, and tidal bias ($b_1$, $b_2$, and
$b_{s^2}$). They also marginalize over 56 nuisance parameters since they
jointly analyze $14$ $z$ bins, each with $4$ nuisance parameters.  Lastly,
unlike our simulation-based approach, \cite{yankelevich2019} use perturbation
theory models and, therefore, limit their forecast to $\kmax = 0.15\hmpc$ due
to theoretical uncertainties. Nevertheless, when they estimate the constraining
power beyond $\kmax > 0.15\hmpc$ using Figure of Merit they find that the
constraining power of $\Bg$ relative to $\Pg$ increases for higher $\kmax$
consistent with our results. 

%comparison to the Chudaykin 
Finally, \cite{chudaykin2019} present $\smnu$ + \lcdm forecasts for the power
spectrum and bispectrum of a Euclid-like survey over $0.5 < z < 2.1$. For
$\omega_{\rm cdm}$, $\omega_b$, $h$, $n_s$, $A_s$, and $\smnu$ they find
${\sim}1.2, 1.5, 1.4, 1.3$, and $1.1\times$ tighter constraints from $\Pgl$ and
$\Bg$ than from $\Pgl$ alone. For $\smnu$, they find a factor of 1.4 improvement, 
from 0.038 eV to 0.028 eV. With \planck priors, 

The improvement from including $\Bg$ Overall,
\cite{chudaykin2019} find more modest improvements from including $\Bg$ than
we find in Figures~\ref{fig:forecast} and~\ref{fig:kmax_forecast}. 

However, we again note significant differences between our anlayses. Like 
\cite{yankelevich2019}, \cite{chudaykin2019} present a forecast for a Euclid survey and therefore is a forecast for a significantly different
galaxy sample than ours: their $z = 0.6$ redshift bin, for instance, has 
$\bar{n}_g = 3.83 \times 10^{-3}~h^3{\rm Gpc}^{-3}$ and linear bias of
$b_g = 1.14$. 


% differences
%- PT model with theoretical errors so it's not obvious what kmax we're getting to with P vs B. 
%Because they use the tree-level perturbation theory model for B0, theoretical errors for B0 quickly dominate at k 􏱘 0.1 h/Mpc, where the one- and two-loop contribute significantly (e.g. Lazanu & Liguori 2018). Hence, their forecasts do not include the constraining power on nonlinear scales.
%- they include Alcock-Paczynski (AP) effect for $P_\ell$, which significantly improve the Pl parameter constraints (e.g. tightens Mν constraints by \sim30\%) but not for $\Bg$. 

%Finally, Chudaykin & Ivanov (2019) present power spectrum and bispectrum forecasts for ωcdm, ωb, h, ns, As, ns, and Mν of a Euclid-like survey in 8 non-overlapping redshift bins over 0.5 < z < 2.1. They use a one-loop perturbation theory model for the redshift-space power spectrum multipoles (l = 0, 2, 4) and a tree-level bispectrum monopole model. Also, rather than imposing a kmax cutoff to restrict their forecasts to scales where their perturbation theory model can be trusted, they use a theoretical error covariance model approach from Baldauf et al. (2016).They find ∼1.4, 1.5, 1.2, 1.5, and 1.3 times tighter constraints from Pl and B0 than from Pl alone. For Mν, they find a factor of 1.4 improvement, from 0.038 eV to 0.028 eV. Overall, Chudaykin & Ivanov (2019) find more modest improvements from including B0 than the improvements we find in our kmax = 0.5 h/Mpc B0 constraints over the Pl constraints.

%Among the differences from our forecast, most notably, Chudaykin & Ivanov (2019) marginalize over 64 parameters of their bias model (4 bias and 4 counterterm normalization parameters at each redshift bin). They also include the Alcock-Paczynski (AP) effect for Pl, which significantly improve the Pl parameter constraints (e.g. tightens Mν constraints by ∼ 30%). They however, do not include AP effects in B0. Furthermore, Chudaykin & Ivanov (2019) use theoretical error covariance to quantify the uncertainty of their perturbation theory models. Because they use the tree-level perturbation theory model for B0, theoretical errors for B0 quickly dominate at k 􏱘 0.1 h/Mpc, where the one- and two-loop contribute significantly (e.g. Lazanu & Liguori 2018). Hence, their forecasts do not include the constraining power on nonlinear scales. Also different from our analysis, Chudaykin & Ivanov (2019) use an Markov-Chain Monte-Carlo (MCMC) approach, which derives more accurate parameter constraints than our Fisher approach. They, however, neglect non-Gaussian contributions to both the Pl and B0 covariance matrices and do not include the covariance between Pl and B0 for their joint constraints. While their theoretical error covariance, which couples different k-modes, may partly take this into account, we find that neglecting the covariance between Pl and B0 overestimates constraints, tighter by ∼20% for kmax = 0.2 h/Mpc. Nonetheless, Chudaykin & Ivanov (2019) and our results both find significant improves in cosmological parameter constraints from including B0. In fact, taking the theoretical errors into account, the improvements from B0 they find are loosely consistent with our results at kmax ∼ 0.2 h/Mpc.


%Various differences between our forecast and previous work prevent more thorough comparisons. However, crucial aspects of our simulation based approach distinguish our forecasts from other works. We present the first bispectrum forecasts for a full set of cosmological parameters using bispectrum measured entirely from N-body simulations. By using the simulations, we go beyond perturbation the- ory models and accurately model the redshift-space bispectrum to the nonlinear regime. Furthermore, by exploiting the immense number of simulations, we accurately estimate the full high-dimensional covariance matrix of the bispectrum. With these advantages, we present the first forecast of cosmo- logical parameters from the bispectrum down to nonlinear scales and demonstrate the constraining power of the bispectrum for Mν. Below, we underline a few caveats of our forecasts.
\begin{itemize}
    \item a discussion on the impact of off diagonal terms of the covariance matrix. 
\end{itemize} 

% caveats paragraph 
We derive, for the first time, the total information cotent of the redshift-space 
galaxy power spectrum and bispectrum. Our simulation based approach, allows us
to go beyond previous perturbation theory approaches and accurately quantify the 
constraining power in the nonlinear regime. They, however, rely on the
stability and convergence of our covariance matrix and numerical derivatives.  
For our constraints we use $195000$ galaxy catalogs (Section~\ref{sec:hod}): 
$15,000$ for the covariance matrices and $180,000$ for the derivatives with 
respect to 11 parameters. To ensure that our results are robust, we conduct the
same set of convergence tests as \cite{hahn2020}. 

First, we test whether our results have sufficiently converged by deriving our 
constraints using different numbers of galaxy catalogs to estimate the covariance 
matrix and derivatives: $N_{\rm cov}$ and $N_{\rm deriv}$. For $N_{\rm cov}$,
we find $< XXXX\%$ variation $\sigma_\theta$ for $N_{\rm cov} > 12000$.
For $N_{\rm deriv}$, 
we find $< XXXX\%$ variation $\sigma_\theta$ for $N_{\rm cov} > 12000$.
vary by $< 10\%$, we conclude that the conergence of the covariance matrix or
deriviatves do not significantly impact our forecast. 
\ch{fill this in once we have the convergence test}. 

We evaluate the $\Pgl$ and $\Bg$ derivatives with respect to $\smnu$ using
simulations at the $\{\theta_{\rm ZA}, \smnu^+, \smnu^{++}, \smnu^{+++}\}$
cosmologies. To test the stability of the $\smnu$ derivatives, we compare
parameter constraints derived using two different sets of cosmologies: 
$\{\theta_{\rm ZA}, \smnu^+\}$ and $\{\theta_{\rm ZA}, \smnu^+\}$. 
\ch{check this once the multiple HOD seeds finish} 

% usual Fisher Forecast caveat  
Besides their convergence and stability, our forecasts are derived from Fisher
matrices. We, therefore, assume that the posterior is approximately Gaussian. 
When posteriors are highly non-elliptical or asymmetric, Fisher forecasts 
significantly underestimate the constraints~\citep{wolz2012}. We note that in
this paper we do not derive actual parameter constraints. Instead, we focus on
quantifying the information content and constraining power of $\Bg$ relative to
$\Pgl$. Hence, we do not explore beyond the Fisher forecast. In a later paper of 
the series, when we analyze the SDSS-III BOSS data using a simulation-based 
approach, we will use a robust method to sample the posterior.  

\ch{anything else that could be wrong that we've checked?} 


\ch{what's missing in the analysis: all the graints of salt/ future works}
\begin{itemize}
    \item how do the unrealistic hod parameters we use map onto real galaxy
        samples. We tried our best to get the SDSS-like sample. The main
        ``unrealistic'' parameter choice is $\sigma_{\log M}$ --- use the
        high resolution simulations to compare the derivative w.r.t sigma logM
        and make some statements regarding this. \ch{appendix?}
    \item we use a simplistic HOD model which doesn't take into account
        assembly bias. Overview the impact of assembly bias in previous galaxy
        clustering results and discuss how it may analogously apply to our
        constraints.
    \item along this line, we do not consider baryonic effects. However recent
        papers that have looked into this find the impact only matters at very
        high k.  
    \item including some extra nuisance parameters doesn't do anything for
        cosmological constraints. We tried b1, Asn, Bsn. \ch{try b2, gamma2} 
    \item alcock pacinzsky effect
    \item no geometry obviously. repeat of how realistic geometry lowers the
        signal to noise. 
    \item we take broad triangle bins, \cite{yankelevich2019} finds higher
        constraining power for narrower bins. $\Delta k = 3 k_f$ has
        ${\sim}10\%$ less constrainig power according to their FoM estimate.  
    \item beyond the monopole \cite{yankelevich2019} 50\% more information
        beyond monopole 
\end{itemize}

\ch{restatement and discusssion of key findings} 
The improvements of $\Bg$ come from breaking parameter degeneracies. 
Most noteably, $\Om$ and $\sig$ degeneracies are broken by
$\Pgl + \Bg$, which improves $\Om - \sig$, $\Om - \smnu$, $\sig-\smnu$ that we care about. 

A key part of the improvement from $\Bg$ comes from the degeneracies that are
broken among HOD parameters. 
Furthermore, including $\Bg$ allows us to break a number of degeneracies among
HOD parameters. This is interesting because there are many questions regarding
HOD parameters that are still up in the air. For instance, the impact and
importance of assembly bias, which can impact cosmological analyses. Moreover,
tighter constraints on the halo occupation in general allows us to better 
understand the galaxy-halo connection, which will improve our understanding of
galaxy formation and evolution. While we consider a relatively simplistic halo
occupation model, our results clearly demonstrate the advantages of
higher-order statistic bispectrum over the standard two-point statistics. 

what's next in the paper series 
\begin{itemize}
    \item data compression 
    \item outline the overall forward modeling approach that will culminate in
        an analysis of BOSS. 
    \item paragraph outline the obvious extensions to DESI and PFS with rough
        numbers 
\end{itemize}
