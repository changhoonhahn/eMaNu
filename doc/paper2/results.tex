%%%%%%%%%%%%%%%%%%%%%%%%%%%%%%%%%%%%%%%%%%
% forecast table
%%%%%%%%%%%%%%%%%%%%%%%%%%%%%%%%%%%%%%%%%%
\begin{table}
    \caption{Marginalized Fisher parameter constraints from the redshift-space 
    $P_\ell$, $B_0$, and $P_\ell$ + $B_0$. We list constraints for cosmological 
    parameters $\smnu$, $\Omega_m$, $\Omega_b$, $h$, $n_s$, and $\sig$ as well 
    as HOD and nuisance parameters.} 
\begin{center} 
    \begin{tabular}{ccccc} \toprule
            & $P_\ell$ & $B_0$ & $P_\ell$ + $B_0$ \\[3pt]
\hline 
$\smnu$     & & & \\
$\Om$       & & & \\
$\Ob$       & & & \\
$h$         & & & \\
$n_s$       & & & \\
$\sig$      & & & \\\hline
$M_{\rm min}$       & & & \\
$\sigma_{\log M}$   & & & \\ 
$\log M_0$          & & & \\
$\alpha$            & & & \\ 
$\log M_1$          & & & \\ [3pt]
\hline            
\end{tabular} \label{tab:forecast}
\end{center}
\end{table}
%%%%%%%%%%%%%%%%%%%%%%%%%%%%%%%%%%%%%%%%%%

%\begin{figure}
%    \begin{center}
%        \includegraphics[width=\textwidth]{figs/quijote_dP02Bdtheta_reg_fin.pdf}
%        \caption{}
%        \label{fig:db}
%    \end{center}
%\end{figure}
\begin{figure}
    \begin{center}
        \includegraphics[width=\textwidth]{figs/Fisher_p02bk_dmnu_fin_kmax0_50_reg.pdf}
        \caption{Fisher matrix constraints for $\smnu$ and other cosmological
        parameters for the redshift-space galaxy $\Pgl$ (blue), $\Bg$
        (green), and combined $\Pgl$ and $\Bg$ (orange) for $k_{\rm max} =
        0.5\hmpc$. Our forecasts marginalizes over the \cite{zheng2007}
        HOD parameters: $\{M_{\rm min}, \sigma_{\log M}, \log M_0, \alpha \log
        M_1 \}$ (bottom panels). The bispectrum substantially improves
        constraints on all of the cosmological parameters over the power
        spectrum. $\Om$, $\Ob$, $h$, $n_s$, and $\sig$ constraints improve by factors
        of \ch{1.9, 2.6, 3.1, 3.6, and 2.6}, respectively. For $\smnu$, the
        bispectrum improves $\sigma_{\smnu}$ from \ch{0.2968 to 0.0572 eV ---
        over a factor of $\sim5$ improvement over the power spectrum}.
        }
        \label{fig:forecast}
    \end{center}
\end{figure}

\begin{figure}
    \begin{center}
        \includegraphics[width=\textwidth]{figs/Fisher_kmax_p02bk_dmnu_fin_reg.pdf}
        \caption{Marginalized $1\sigma$ constraints, $\sigma_\theta$, of the
        cosmological parameters $\Om$, $\Ob$, $h$, $n_s$, $\sig$, and $\smnu$
        as a function of $k_{\rm max}$ for the redshift-space $\Pgl$ (blue)
        and combined $\Pgl + \Bg$ (orange). Even after marginalizing over
        HOD parameters (Eq.~\ref{eq:hod_fid}), the galaxy bispectrum {\em
        significantly} improves cosmological parameter constraints above
        $k_{\rm max} > 0.1\hmpc$. Constraints from $\Pgl$ and
        $\Pgl + \Bg$ improve with higher $k_{\rm max}$. {\em Throughout 
        $0.2 < k_{\rm max} < 0.5$, including the bispectrum improves 
        $\{\Om, \Ob, h, n_s, \sig, \smnu\}$ by \ch{X, Y \%}.} When we include 
        {\em Planck} priors (dotted), the improvement from $\Bg$ is even more
        evident. The constraining power of $\Pgl$ complete
        saturates for $k_{\rm max} \gtrsim 0.12\hmpc$. Adding $\Bg$ not only 
        improves constraints, but the constraints continue to improve for
        higher $k_{\rm max}$. At $k_{\rm max} = 0.2$ and $0.5\hmpc$, the $\Pgl
        + \Bg$ improves the $\smnu$ constraint by \ch{X, Y \%} over $\Pgl$. 
        We emphasize that the constraints above are for $1~({\rm Gpc}/h)^3$ box
        and thus underestimate the constraining power of upcoming galaxy
        clustering surveys.
        }
        \label{fig:kmax_forecast}
    \end{center}
\end{figure}

\section{Results} \label{sec:results} 
presenting the results 

Figure~\ref{fig:forecast}
\begin{itemize}
    \item obviously the galaxy bispectrum blows the galaxy power spectrum out
        of the water.  
    \item parameter degeneracies in the power spectrum and bispectrum alone
        that go away once you combine them
\end{itemize}


Figure~\ref{fig:kmax_forecast}
\begin{itemize}
    \item We only include the $k_{\rm max}$ ranges where we have more data bins than number of parameters. 
    \item emphasize that the improvement is for consistent for $k_{\rm max} >
        0.2$, not just at the 0.5. 
    \item emphasize that combining with planck is actually underestimating the
        constarining power of galaxy clustering
\end{itemize}


discussions

first things that could be wrong with the results: 
\begin{itemize}
    \item convergence: we conducted the same set of convergence tests as the halo paper and
        found \ch{some quantifying numbers} \ch{appendix or is it too
        repetive?}
    \item derivative w.r.t. $\smnu$ is sensitive to the choice of step size.
        However, we use enough realizations so the results don't change
        significantly. \ch{actually check this with the multiple HOD seeds} 
        \ch{appendix?}
\end{itemize}

interesting findings 
\begin{itemize}
    \item signal-to-noise is lower for the galaxy power and bispectrums than
        the halo power and bispectra --- fingers of god. \ch{figure?}
    \item re-emphasize the impact of bispectrum in breaking HOD parameter
        degeneracies. 
    \item \ch{do we want H0 forecast and discussion on H0 tension} 
\end{itemize}

what's missing in the analysis: all the graints of salt
\begin{itemize}
    \item how do the unrealistic hod parameters we use map onto real galaxy
        samples. We tried our best to get the SDSS-like sample. The main
        ``unrealistic'' parameter choice is $\sigma_{\log M}$ --- use the
        high resolution simulations to compare the derivative w.r.t sigma logM
        and make some statements regarding this. \ch{appendix?}
    \item we use a simplistic HOD model which doesn't take into account
        assembly bias. Overview the impact of assembly bias in previous galaxy
        clustering results and discuss how it may analogously apply to our
        constraints.
    \item including some extra nuisance parameters doesn't do anything for
        cosmological constraints. We tried b1, Asn, Bsn. \ch{try b2, gamma2} 
    \item baryonic effects are not considered and some papers have looked at
        this recently. 
    \item no geometry obviously. repeat of how realistic geometry lowers the
        signal to noise. 
\end{itemize}

what's next in the paper series 
\begin{itemize}
    \item data compression 
    \item outline the overall forward modeling approach that will culminate in
        an analysis of BOSS. 
    \item paragraph outline the obvious extensions to DESI and PFS. 
\end{itemize}
