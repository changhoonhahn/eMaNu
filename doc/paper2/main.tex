\documentclass[12pt, letterpaper, preprint]{aastex62}
%%% This file is generated by the Makefile.
\newcommand{\giturl}{\url{https://github.com/changhoonhahn/eMaNu}}
\newcommand{\githash}{112c8e2}\newcommand{\gitdate}{2019-09-20}\newcommand{\gitauthor}{changhoonhahn}

%\usepackage[breaklinks,colorlinks, urlcolor=blue,citecolor=blue,linkcolor=blue]{hyperref}
\usepackage{color}
\usepackage{amsmath}
\usepackage{natbib}
\usepackage{ctable}
\usepackage{bm}
\usepackage[normalem]{ulem} % Added by MS for \sout -> not required for final version
\usepackage{xspace}
\usepackage{csvsimple} 

% typesetting shih
\linespread{1.08} % close to 10/13 spacing
\setlength{\parindent}{1.08\baselineskip} % Bringhurst
\setlength{\parskip}{0ex}
\let\oldbibliography\thebibliography % killin' me.
\renewcommand{\thebibliography}[1]{%
  \oldbibliography{#1}%
  \setlength{\itemsep}{0pt}%
  \setlength{\parsep}{0pt}%
  \setlength{\parskip}{0pt}%
  \setlength{\bibsep}{0ex}
  \raggedright
}
\setlength{\footnotesep}{0ex} % seriously?

% math shih
\newcommand{\setof}[1]{\left\{{#1}\right\}}
\newcommand{\given}{\,|\,}
\newcommand{\lss}{{\small{LSS}}\xspace}

\newcommand{\lcdm}{$\Lambda$CDM} 
\newcommand{\Om}{\Omega_{\rm m}} 
\newcommand{\Ob}{\Omega_{\rm b}} 
\newcommand{\OL}{\Omega_\Lambda}
\newcommand{\smnu}{M_\nu}
\newcommand{\sig}{\sigma_8} 
\newcommand{\mmin}{M_{\rm min}}
\newcommand{\BOk}{\widehat{B}_0} 
\newcommand{\hmpc}{\,h/\mathrm{Mpc}}
\newcommand{\bfi}[1]{\textbf{\textit{#1}}}
\newcommand{\parti}[1]{\frac{\partial #1}{\partial \theta_i}}
\newcommand{\partj}[1]{\frac{\partial #1}{\partial \theta_j}}
\newcommand{\mpc}{h/{\rm Mpc}}

% specific to this project
\newcommand{\quij}{{\sc Quijote}}
\newcommand{\planck}{{\em Planck}}
\newcommand{\kmax}{k_{\rm max}}
\newcommand{\Pg}{P^g} 
\newcommand{\Pgl}{P^g_{\ell}} 
\newcommand{\Bg}{B^g_0} 
\newcommand{\Phl}{P^h_{\ell}} 
\newcommand{\Bh}{B^h_0} 

\newcommand{\specialcell}[2][c]{%
  \begin{tabular}[#1]{@{}c@{}}#2\end{tabular}}
% text shih
\newcommand{\foreign}[1]{\textsl{#1}}
\newcommand{\etal}{\foreign{et~al.}}
\newcommand{\opcit}{\foreign{Op.~cit.}}
\newcommand{\documentname}{\textsl{Article}}
\newcommand{\equationname}{equation}
\newcommand{\bitem}{\begin{itemize}}
\newcommand{\eitem}{\end{itemize}}
\newcommand{\beq}{\begin{equation}}
\newcommand{\eeq}{\end{equation}}
\newcommand{\eg}{\emph{e.g.}}
\newcommand{\ie}{\emph{i.e.}}

%% collaborating
\newcommand{\todo}[1]{\marginpar{\color{red}TODO}{\color{red}#1}}
\definecolor{orange}{rgb}{1,0.5,0}
\newcommand{\ch}[1]{{\color{orange}{\bf CH:} #1}}

\begin{document}\sloppy\sloppypar\frenchspacing 

\title{Constraining $\smnu$ with the Bispectrum II: the Total Information Content of the Galaxy Bispectrum} 
\date{\texttt{DRAFT~---~\githash~---~\gitdate~---~NOT READY FOR DISTRIBUTION}}

\newcounter{affilcounter}
\author{ChangHoon Hahn}
\altaffiliation{hahn.changhoon@gmail.com}
\affil{Lawrence Berkeley National Laboratory, 1 Cyclotron Rd, Berkeley CA 94720, USA}
\affil{Berkeley Center for Cosmological Physics, University of California, Berkeley, CA 94720, USA}
\affil{Department of Astrophysical Sciences, Princeton University, Peyton Hall, Princeton NJ 08544, USA} 

\author{Francisco Villaescusa-Navarro} 
\affil{Center for Computational Astrophysics, Flatiron Institute, 162 5th Avenue, New York, NY 10010, USA} 
\affil{Department of Astrophysical Sciences, Princeton University, Peyton Hall, Princeton NJ 08544, USA} 

\author{...}

\begin{abstract}
    Massive neutrinos suppress the growth of structure on small scales %below their free-streaming scale 
    and leave an imprint on large-scale structure that can be measured to
    constrain their total mass, $\smnu$. With standard analyses of two-point
    clustering stastics, $\smnu$ constraints are severely limited by parameter
    degeneracies. \cite{hahn2020} demonstrated that the bispectrum, the
    next higher-order statistic, can break these degeneracies and dramatically
    improve constraints on $\smnu$ and other cosmological parameters. In this
    paper, we present the constraining power of the {\em redshift-space galaxy 
    bispectrum}, $\Bg$. We construct $195,000$ mock galaxy catalogs from the
    \quij~$N$-body simulation suite using the halo occupation distribution (HOD) model,
    which provides an effective galaxy bias framework well-suited for simulation-based
    approaches. Using these mocks, we present the Fisher matrix forecasts for 
    $\{\Om$, $\Ob$, $h$, $n_s$, $\sig$, $\smnu\}$ and
    quantify, for the first time, the total information content of the $\Bg$
    down to nonlinear scales. For $\kmax{=}0.5\hmpc$, $\Bg$ improves constraints 
    on $\Om$, $\Ob$, $h$, $n_s$, and $\sig$ by 2.8, 3.1, 3.8, 4.2, 4.2, and $4.7{\times}$ 
    over power spectrum, after marginalizing over HOD parameters. For $\smnu$, 
    we achieve $5.6\times$ tighter constraints with $\Bg$. Even with priors from
    \planck, $\Bg$ improves cosmological constraints by $\gtrsim 2\times$. 
    While effects such as survey geometry and assembly bias will have an
    impact, these %  the constraining power for galaxy surveys
    constraints are derived for $(1~h^{-1}{\rm Gpc})^3$, a substantially
    smaller volume than upcoming surveys. Therefore, we conclude that including
    the galaxy bispectrum will significantly improve cosmological constraints
    for upcoming galaxy surveys --- especially for $\smnu$.
\end{abstract}

\keywords{
cosmology: cosmological parameters 
--- 
cosmology: large-scale structure of Universe.
---
cosmology: theory
}

% --- intro ---
\section{Introduction} \label{sec:intro} 

\cite{pontzen2016}
\cite{angulo2016}
\cite{villaescusa-navarro2018a}

\beq
\delta(\bfi{x}) = \frac{\rho(\bfi{x}) - \bar{\rho}}{\bar{\rho}}
\eeq

\beq
\delta(\bfi{k}) = \frac{1}{(2\pi)^3} \int {\rm d}^3\bfi{x} e^{-i\bfi{k} \cdot \bfi{x}} \delta(\bfi{x}) = A e^{i\theta}
\eeq

For Gaussian random field, $\theta$ is uniformly sampled from $0$ to $2\pi$ 
and $A$ is sampled from Rayleigh distribution:
\beq
p(A) {\rm d}A = \frac{A}{\sigma^2} e^{-A^2/2\sigma^2} {\rm d}A
\eeq
where $\sigma^2 = V~P(k)/(16\pi^3)$. The mean of this distribution is 
\beq
\langle A\rangle = \int_0^\infty \frac{A^2}{\sigma^2} e^{-A^2/2\sigma^2} {\rm d}A = \sqrt{\frac{V\,P(k)}{32\pi^2}}.
\eeq
Also, 
\beq
\langle \delta(\bfi{k})\delta^*(\bfi{k}) \rangle = \langle A^2\rangle = \int_0^\infty \frac{A^3}{\sigma^2} e^{-A^2/2\sigma^2} {\rm d}A = \frac{V\,P(k)}{(2\pi)^3}.
\eeq

A paired Gaussian field is where you have two fields $\delta_1$ and $\delta_2$ 
where $\delta_2(k) = A e^{i(\theta + \pi)} = - \delta_1(k)$. 

A fixed field is when the amplitude is fixed, 
\beq
A = \sqrt{\frac{V~P(k)}{(2\pi)^3}},
\eeq
such that the power spectrum is the same. 

Paired fixed is when you do both. 


% --- quijote ---
% brief summary of the Quijote suite. 
\section{The Quijote Simulation Suite} \label{sec:sims}
We use a subset of simulations from the \quij suite, a set of 43,000 $N$-body 
simulations that spans over 7000 cosmological models and contains, at a single 
redshift, over 8.5 trillion particles. The \quij suite was designed to quantify
the information content of cosmological observables and also to train machine 
learning algorithms. Hence, the suite includes enough realizations to accurately 
estimate the covariance matrices of high-dimensional observables such as the 
bispectrum as well as the derivatives of these observables with respect to 
cosmological parameters. For the derivatives, the suite includes sets of 
simulations run at different cosmologies where only one parameter is varied 
from the fiducial cosmology  
($\Om{=}0.3175$, $\Ob{=}0.049$, $h{=}0.6711$, $n_s{=}0.9624$, $\sig{=}0.834$, 
and $\smnu{=}0.0$ eV). Along $\Om$, $\Ob$, $h$, $n_s$, and $\sig$, the fiducial 
cosmology is adjusted by either a small step above or below the fiducial value:
$\{\Om^{+}, \Om^{-}, \Ob^{+}, \Ob^{-}, h^+, h^-, n_s^{+}, n_s^{-}, \sig^{+}, \sig^-\}$ . 
Along $\smnu$, because $\smnu \ge 0.0$ eV and the derivative of certain observable 
with respect to $\smnu$ is noisy, \quij includes sets of simulations for 
$\{\smnu^{+}, \smnu^{++}, \smnu^{+++} \} = \{0.1, 0.2, 0.4~{\rm eV}\}$.
At each of these 14 cosmologies, \quij includes sets of standard $N$-body and 
suppressed variance simulations. 

The initial conditions for all the simulations were generated at $z=127$ using 
second-order perturbation theory for simulations with massless neutrinos 
($\smnu = 0.0$ eV) and the Zel’dovich approximation for massive neutrinos 
($\smnu > 0.0$ eV). The initial conditions with massive neutrinos take 
their scale-dependent growth factors/rates into account using the 
\cite{zennaro2017a} method, while for the massless neutrino case we use 
the traditional scale-independent rescaling. 
%The \textsc{Quijote} simulations contain standard, fixed and paired fixed simulations. The differences between those is the way the initial conditions are generated. Consider a Fourier-space mode, $\delta(\vec{k})$. Since it is in general a complex number, we can write it as $\delta(\vec{k})=Ae^{i\theta}$, where both the amplitude $A$, and the phase $\theta$, depends on the considered wavenumber $\vec{k}$. In standard simulations, $A$ follows a Rayleigh distribution and $\theta$ is drawn from an uniform distribution between 0 and $\pi$. We note that this is the traditional way to generate ICs for cosmological simulations. In fixed simulations, while $\theta$ is still drawn from an uniform distribution between 0 and $2\pi$, the value of $A$ is fixed to the square root of the variance of the previous Rayleigh distribution. Finally, paired fixed simulations are two fixed simulations where the phases of the two pairs differ by $\pi$. We refer the reader to \citep{Pontzen_2016,RP_16,Paco_2018b} for further details.


From the initial conditions, 
the simulations follow the gravitational evolution of $512^3$ dark matter
particles, and $512^3$ neutrino particles for massive neutrino models, to 
$z=0$ using {\sc Gadget-III} TreePM+SPH code~\citep{springel2005}. 
Simulations with massive neutrinos are run using the ``particle method'', 
where neutrinos are described as a collisionless and pressureless fluid 
and therefore modeled as particles, same as CDM~\citep{brandbyge2008,viel2010}.
Halos are then identified using the Friends-of-Friends algorithm~\citep[FoF;][]{davis1985} 
with linking length $b=0.2$ on the CDM + baryon distribution. We limit 
the halo catalogs to halos with masses above 
$M_{\rm lim} = 3.2\times 10^{13} h^{-1}M_\odot$.
For the fiducial cosmology, the halo catalogs have ${\sim}156,000$ halos 
($\bar{n} \sim 1.56 \times 10^{-4}~h^3{\rm Gpc}^{-3}$) with $\bar{n} P_0(k=0.1)\sim 3.23$. 
We refer readers to Villaescusa-Navarro et al. (in preparation) and Hahn et al. (2019) 
for further details on the \quij simulations.



% --- hod ---
\begin{figure}
\begin{center}
    \includegraphics[width=0.45\textwidth]{figs/hod_fid.pdf} 
    \caption{
        The halo occupation our fiducial halo occupation (black) parameterized with 
        the standard \cite{zheng2007} HOD model. For the parameter values we modify
        the best-fit HOD parameters of the SDSS $M_r < -21.5$ or $< -22.$ sample
        best-fit parameters from \cite{zheng2007}. The halo mass limit of the \quij 
        simulations,  $M_{\rm lim} = 3.2\times 10^{13} h^{-1}M_\odot$ (black dashed), 
        prevent us from directly adopting the best-fit HOD parameters from \cite{zheng2007}. 
        We include the halo occupation of the \cite{zheng2007} best-fits of the 
        $M_r < -21.5$ (blue dashed) or $< -22.$ sample (orange dashed) for reference.
    }\label{fig:hod}
\end{center}
\end{figure}

\section{Halo Occupation Distribution} \label{sec:hod}  
We are interested in quantifying the information content of the galaxy bispectrum. 
For a perturbation theory approach, this involves incorporating a bias model 
for galaxies~\citep[\emph{e.g.}][]{sefusatti2006, yankelevich2019, chudaykin2019}.
Perturbation theory approaches, however, break down on small scales and limit
the constraining power from nonlinear regime. Instead, in our simulation based 
approach we use the halo occupation distribution (HOD) 
framework~\citep[\emph{e.g.}][]{zheng2005, leauthaud2012, tinker2013, zentner2016, vakili2019}.% Berlind & Weinberg 2002; Peacock & Smith 2000; Seljak 2000; Benson et al. 2000; White et al. 2001; Cooray & Sheth 2002 
HOD models statistically populate galaxies in dark matter halos by specifying
the probability of a given halo hosting a certain number of galaxies. This 
statistical prescription for connecting galaxies to halos has been remarkably 
successful in reproducing the observational statistics of galaxies (\emph{e.g.} 
galaxy clustering) and, as a result, is the standard approach for constructing 
simulated galaxy mock catalogs in galaxy clustering analyses to estimate covariance 
matrices and test systematic effects~\citep[\emph{e.g.}][]{rodriguez-torres2016, rodriguez-torres2017, beutler2017}. 
More importantly, HOD models in simulations for build galaxy clustering 
emulators~\citep[see the Aemulus project][]{mcclintock2018, zhai2018}. Emulation, 
as we mention above, is one of the most promising approaches for modeling small 
scale galaxy clustering and is what we're trying to forecast in this work. 

In the simplest HOD models, the probability of a given halo hosting $N$ galaxies 
of a certain class is dictated by its halo mass --- $P(N|M_h)$.%~\citep[\emph{e.g.}][]{zheng2007}. 
We use the standard $P(N|M_h)$ model from \cite{zheng2007}, which has been 
ubiquitously used in galaxy clustering analyses~\citep[\emph{e.g.}][\todo{many more}]{sinha2018}. 
The model specifies the mean number of galaxies in a halo as
\beq
\langle N_{\rm gal} \rangle = \langle N_{\rm cen} \rangle + \langle N_{\rm sat} \rangle
\eeq
with mean central galaxy occupation
\beq \label{eq:Ncen}
\langle N_{\rm cen} \rangle  = \frac{1}{2}\Bigg[1 + {\rm erf}\bigg(\frac{\log M_h - \log M_{\rm min}}{\sigma_{\log M}}\bigg) \Bigg]
\eeq
and mean satellite galaxy occupation
\beq \label{eq:Nsat}
\langle N_{\rm sat} \rangle = \langle N_{\rm cen} \rangle \bigg(\frac{M_h - M_0}{M_1}\bigg)^\alpha.
\eeq
The mean number of centrals in a halo transitions smoothly from 0 to 1 for halos 
with mass $M_h > M_{\rm min}$. The width of the transition is dictated by 
$\sigma_{\log M}$, which reflects the scatter between stellar mass/luminosity and 
halo mass~\citep{citecite}. For $M_h > M_{\rm min}$, 
$\langle N_{\rm sat} \rangle$ follows a power law with slope $\alpha$. $M_0$ 
is the halo mass cut-off for satellite occupation and $M_h = M_0 + M_1$ is 
the typical mass scale for halos to host one satellite galaxy. The numbers 
of centrals and satellites for each halo are drawn from Bernoulli and Poisson 
distribution, respectively. Central galaxies are placed at the center of the
halo while position and velocity of the satellite galaxies are sampled from a 
\cite{navarro1997} (NFW) profile. 

The halo occupation in the \cite{zheng2007} model depends soley on $M_h$. 
Simulations, however, find evidence that secondary halo properties such as
concentration or formation history correlate with spatial distribution of
halos --- a phenomenon referred to as ``halo assembly bias''~\citep{sheth2004, gao2005, harker2006, wechsler2006}.
A model that only depends on $M_h$, does not account for this halo assembly 
bias and may not sufficient describe the connection between galaxies and 
halos. Moreover, if unaccounted for in the HOD model, and thus not marginalized 
over, halo assembly bias may impact the cosmological parameter constraints. 
\cite{zentner2016} and \cite{vakili2019} recently examined evidence for 
assembly bias in the observed clustering measurements of the Sloan Digital 
Sky Survey (SDSS) DR 7 main galaxy sample. 
\todo{short description of the papers and how they compare with a model with assembly bias}. 
However, they find little evidence for assembly bias in the galaxy clustering
of the SDSS $M_r < -21.5$  and $-21$ samples. Therefore, in this work we use 
the standard \cite{zheng2007} HOD model and assume it is sufficient for modeling 
the galaxy--halo connection. 

For the fiducial parameter values of the HOD model we used values motivated by 
best-fit HOD parameters from the literature, namely the \cite{zheng2007} 
fits to the SDSS $M_r < -21.5$  and $-22$ samples: 
\beq \label{eq:hod_fid}
\{M_{\rm min}, \sigma_{\log M}, \log M_0, \alpha \log M_1 \} = \{13.65, 0.2, 14., 1.1, 14.\}.
\eeq
In Figure~\ref{fig:hod} we present the halo occupation of our fiducial 
HOD parameters (black). We include the best-fit halo occupations of 
the SDSS $M_r < -21.5$ (blue)  and $-22$ (orange) samples from \cite{zheng2007}
for comparison. We also mark the halo mass limit, $M_{\rm lim}$, of the \quij 
simulations (black dotted). At $M_h \sim 10^{13} M_\odot$, the best-fit halo 
occupations of the SDSS samples extend below $M_{\rm lim}$ --- \emph{i.e.} 
they have halos below $M_{\rm lim}$ that host galaxies. This prevents us from 
directly using the values from the literature and instead, we reduce 
$\sigma_{\log M}$ to 0.2 dex. We confirm using \quij simulations with higher 
mass resolution ($1024^3$ CDM particles) that $M_{\rm lim}$ does not impact
the observables or their derivatives in our analysis for our fiducial HOD 
parameters. 

As we mention above, $\sigma_{\log M}$ reflects the scatter between stellar 
mass/luminosity and halo mass. The high $\sigma_{\log M}$ in the $M_r < -21.5$ 
and $-22$ SDSS samples is caused by the turnover in this relation at high stellar 
mass/luminosity. Our fiducial halo occupation, with its lower $\sigma_{\log M}$, 
results in a galaxy sample with a tighter scatter than the samples selected based 
on $M_r$ or $M_*$ cuts, \emph{e.g.} used in SDSS and BOSS. Hence, such a sample
would require selecting based on observable galaxy properties that correlate 
more strong with $M_h$ than luminosity or $M_*$. \todo{Alpaslan et al. in prep. find
that $L_{\rm sat}$ is more correlated to $M_h$ than luminosity or $M_*$; however, 
it has yet to be used for galaxy sample selection.} Regardless, in this work
we are interested in quantifying the information content of the galaxy bispectrum 
and not analyzing an observed galaxy sample. We therefore opt for a more 
conservative set of HOD parameters with respect to $M_{\rm lim}$. 
 

% --- methods ---
\section{Bispectrum and Cosmological Parameter Forecasts} \label{sec:methods}
We measure the galaxy bispectrum and calculate the parameter constraints using the 
same methods as \cite{hahn2020}. For further details, we  refer readers to \cite{hahn2020}. 

To measure $\Bg$, we use a Fast Fourier Transform (FFT) based estimator similar to
the ones in \cite{sefusatti2005}, \cite{scoccimarro2015}, and
\cite{sefusatti2016}. Galaxy positions are first interpolated onto a grid,
$\delta(\bfi{x})$, using a fourth-order interpolation scheme, which has advantageous
anti-aliasing properties that allow unbiased measurements up to the Nyquist
frequency~\citep{hockney1981, sefusatti2016}. After Fourier transforming 
$\delta(\bfi{x})$ to get $\delta(\bfi{k})$, we measure the bispectrum monopole
\beq \label{eq:bk} 
\Bg(k_1, k_2, k_3) = \frac{1}{V_B} \int\limits_{k_1}{\rm d}^3q_1
\int\limits_{k_2}{\rm d}^3q_2 \int\limits_{k_3}{\rm d}^3q_3~\delta_{\rm
D}({\bfi q_{123}})~\delta({\bfi q_1})~\delta({\bfi q_2})~\delta({\bfi q_3}) -
B^{\rm SN}_0.
\eeq
$\delta_D$ is the Dirac delta function, $V_B$ is the normalization factor
proportional to the number of triplets that can be found in the $k_1, k_2, k_3$
triangle bin, and $B^{\rm SN}_0$ is the correction term for the Poisson shot
noise. Throughout the paper, we use $\delta(\bfi{x})$ grids with $N_{\rm grid}
= 360$ and triangle configurations defined by $k_1, k_2, k_3$ bins of width
$\Delta k = 3 k_f = 0.01885\hmpc$, where $k_f = 2\pi/(1000~h^{-1}{\rm Mpc})$. 

In Figure~\ref{fig:bgh}, we present the redshift-space galaxy power spectrum multipoles 
($\Pgl$; left) and bispectrum ($\Bg$; right) of the fiducial HOD galaxy catalog (blue). The $\Pgl$ and 
$\Bg$ are averaged over one set of HOD realizations run on 15,000 $N$-body
\quij~simulations at the fiducial cosmology. In the left panel, we
plot both the power spectrum monopole ($\ell = 0$; solid) and quadrupole 
($\ell = 2$; dashed). In the right panel, we plot $\Bg$ for all 1898 triangle
configurations with $k_1, k_2, k_3 \le k_{\rm max} = 0.5\hmpc$. The configurations 
are ordered by looping through $k_3$ in the inner-most loop and $k_1$ in the 
outer-most loop satisfying $k_1 \le k_2 \le k_3$. For comparison, we include the
redshift-space halo power spectrum and bispectrum at the fiducial cosmology 
from \cite{hahn2020} (black dotted). 

\begin{figure}
\begin{center}
    \includegraphics[width=\textwidth]{figs/PBg.pdf} 
    \caption{The redshift-space galaxy power spectrum multipoles ($\Pgl$; left)
    and bispectrum monopole ($\Bg$; right) of the fiducial HOD galaxy catalog (blue).
    The $\Pgl$ and $\Bg$ are averaged over one set of HOD realizations run on
    15,000 $N$-body \quij~simulations measured using the same FFT-based estimator as \cite{hahn2020}. In the 
    left panel, we plot both the power spectrum monopole ($\ell = 0$; solid) and quadrupole 
    ($\ell = 2$; dashed). In the right panel, we plot $\Bg$ for all 1898 triangle
    configurations with $k_1, k_2, k_3 \le k_{\rm max} = 0.5\hmpc$. The
    configurations are ordered by looping through $k_3$ in the inner most loop
    and $k_1$ in the outer most loop satisfying $k_1 \le k_2 \le k3$.
    We include for comparison the \cite{hahn2020} halo $\Phl$ and $\Bh$ at the 
    fiducial cosmology (black). %We note that the fiducial galaxy catalog has $\bar{n}_g \sim 1.63\times 10^{-4}~h^3{\rm Gpc}^{-3}$ and linear bias of $b_g \sim 2.55$.
    }
\label{fig:bgh}
\end{center}
\end{figure}

To estimate the constraining power of $\Pgl$ and $\Bg$, we use Fisher information
matrices, which have been ubiquitously used in
cosmology~\citep[\emph{e.g.}][]{jungman1996,tegmark1997,dodelson2003,heavens2009,verde2010}: 
\beq 
F_{ij} = - \bigg \langle \frac{\partial^2 \mathrm{ln} \mathcal{L}}{\partial \theta_i \partial \theta_j} \bigg \rangle,
\eeq
As in \cite{hahn2020}, we assume that the $\Bg$ likelihood is Gaussian and
neglect the covariance derivative term~\citep{carron2013} and estimate the
Fisher matrix as 
\beq \label{eq:fisher}
F_{ij} = \frac{1}{2}~\mathrm{Tr} \Bigg[\bfi{C}^{-1}
\left(\parti{\Bg}\partj{\Bg}^T + \parti{\Bg}^T \partj{\Bg} \right)\Bigg].
\eeq
We derive the covariance matrix, $\bfi{C}$, using $15,000$ fiducial galaxy
catalogs. The derivatives along the cosmological and HOD
parameters, $\partial \Bg/\partial \theta_i$, are estimated using finite
difference. For all parameters other than $\smnu$, we estimate 
\beq 
\frac{\partial \Bg}{\partial \theta_i} \approx \frac{\Bg(\theta_i^{+})-\Bg(\theta_i^{-})}{\theta_i^+ - \theta_i^-}, 
\eeq
where $\Bg(\theta_i^{+})$ and $\Bg(\theta_i^{-})$ are the average bispectrum of the 
$(500~{\rm simulations})\times(3~{\rm RSD~axes})\times(5~{\rm
HOD~realizations}) = 7,500$
realizations at $\theta_i^{+}$ and $\theta_i^{-}$, the HOD or 
cosmological parameter values above and below the fiducial parameters.  
For $\smnu$, where the fiducial value is 0.0 eV, we use the galaxy catalogs 
at $\smnu^+$, $\smnu^{++}$, $\smnu^{+++}=0.1, 0.2, 0.4$ eV (Table~\ref{tab:sims}) 
to estimate 
\beq \label{eq:dbkdmnu} 
\frac{\partial \Bg}{\partial \smnu} \approx \frac{-21 \Bg(\theta_{\rm fid}^{\rm ZA}) + 
32 \Bg(\smnu^{+}) - 12 \Bg(\smnu^{++}) + \Bg(\smnu^{+++})}{1.2}, 
\eeq
which provides a $\mathcal{O}(\delta \smnu^2)$ order approximation. 
Since the simulations at $\smnu^+$, $\smnu^{++}$, and $\smnu^{+++}$ are generated 
from Zel'dovich initial conditions, we use simulations at the fiducial cosmology 
also generated from Zel'dovich initial conditions ($\theta_{\rm fid}^{\rm ZA}$). 
Our simulation-based approach with galaxy catalogs constructed from
$N$-body simulations is essential for accurately quantifying the constraining power
of the bispectrum beyond the limitations of analytic methods down to the nonlinear
regime.


% --- results ---
\section{Results} \label{sec:results} 

\begin{figure}
\begin{center}
    \includegraphics[width=0.8\textwidth]{figs/compressedFisher_PCA_kmax0_3_tdist_corr_1sigma.pdf}
    \caption{}
\label{fig:pca}
\end{center}
\end{figure}

\begin{figure}
\begin{center}
    \includegraphics[width=0.8\textwidth]{figs/compressedFisher_KL_kmax0_3_tdist_corr_1sigma.pdf} 
    \caption{}
\label{fig:kl}
\end{center}
\end{figure}

\begin{figure}
\begin{center}
    \includegraphics[width=0.8\textwidth]{figs/compressedFisher_PCA_KL_kmax0_3_tdist_corr_contour.pdf} 
    \caption{}
\label{fig:contour}
\end{center}
\end{figure}


% --- summary ---
\section{Summary} \label{sec:summary} 

Afterwards, we will apply it to future surveys. 
\ch{rough numbers of DESI, PFS, and Euclid} 

% what are the implications of our results? 
%The constraints from our forecasts are derived for a $(1~h^{-1}{\rm Gpc})^3$ 
%volume. These constraints roughly scale as $\propto1/\sqrt{V}$ with volume. 
%Upcoming spectroscopic galaxy surveys will map out vastly larger cosmic
%volumes: PFS $\sim 9~h^{-3}{\rm Gpc}^3$ and DESI $\sim 50~h^{-3}{\rm Gpc}^3$~\citep{takada2014, desicollaboration2016}. 
%Euclid and WFIRST, space-based surveys, will expand these volumes to 
%higher redshifts. A key science goal in these surveys will be constraining $\smnu$. 
%Scaling our 0.0572 eV $\smnu$ constraint from the bispectrum to the survey 
%volumes gives 0.0191 eV and 0.0081 eV $1\sigma$ constraints for PFS and DESI. 
%Galaxy samples in these surveys will also have significantly higher 
%number densities than the halos used in our forecast: \emph{e.g.} 
%the DESI BGS at $z{\sim}0.3$, DESI LRG, and PFS at $z \sim 1.3$ 
%will have ${\sim}20, 3$, and $5\times$ higher number densities, 
%respectively. With such precision, we would detect $\smnu$ by $>3\sigma$ and 
%distinguish between the normal and inverted neutrino mass hierarchy scenarios 
%by $>2\sigma$ from the bispectrum alone. Such naive projections, however, 
%should be taken with more than a grain of salt. Nevertheless, the substantial 
%improvement we find in constraints with the bispectrum over the power spectrum, 
%strongly advocate for analyzing future surveys with more than the power spectrum. 
%Our results demonstrate the potential of the bispectrum to tightly constrain 
%$\smnu$ with unprecedented precision.
 

\section*{Acknowledgements}
It's a pleasure to thank 
    Mehmet Alpaslan, 
    Arka Banerjee, 
    Joseph DeRose, 
    Daniel Eisenstein, 
    Mikhail Ivanov, 
    Andrew Hearin,
    Elena Massara,
    Jeremy L. Tinker,
    Roman Scoccimarro, 
    Uro{\u s}~Seljak,
    Zachary Slepian, 
    Digvijay Wadekar,
    Risa Wechsler
    ...
for valuable discussions and comments. 

\appendix

\bibliographystyle{yahapj}
\bibliography{emanu_hod} 
\end{document}
