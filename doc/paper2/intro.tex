\section{Introduction} \label{sec:intro}
Over two decades ago, neutrino oscillation experiments discovered the lower
bound on the sum of neutrino masses ($\smnu \gtrsim 0.06$ eV) and confirmed
physics beyond the Standard Model~\citep{fukuda1998, forero2014, gonzalez-garcia2016}. 
Since then, experiments have sought to more precisely measure $\smnu$ in order
to distinguish between the `normal' and `inverted' neutrino mass hierarchy
scenarios and further reveal the physics of neutrinos. Upcoming laboratory 
experiments (\eg~double beta decay and tritium beta decay), however, will not
be sufficient to distinguish between the mass hierarchies~\citep{bonn2011, drexlin2013}.
Fortunately, complementary and more precise constraints on $\smnu$ can be
placed by measuring the effect of neutrinos on the expansion history and growth
of cosmic structure. 

In the early Universe, neutrinos are relativistic and contribute to the 
energy density of radiation. Later, as they become non-relativistic, 
they contribute to the energy density of matter. This transition affects 
the expansion history of the Universe and leaves imprints on the cosmic
microwave background (CMB) anisotropy spectrum~\citep{lesgourgues2012,
lesgourgues2014}. Massive neutrinos also impact the growth of structure. 
While neutrino perturbations are indistinguishable from cold dark matter (CDM)
perturbations on large scales, on scales smaller than their free-streaming
scale, neutrinos do not contribute to the clustering and reduce the 
amplitude of the total matter power spectrum. They also reduce the growth 
rate of CDM perturbations at late times. This combined suppression of 
the small-scale matter power spectrum leaves measurable imprints 
on the CMB as well as large-scale structure~\citep[for further details see][]{lesgourgues2012, lesgourgues2014, gerbino2018}. 

%\todo{transition to why LSS neutrino mass measurement is important} (condensed version of paper 1) 
The tightest cosmological constraints on $\smnu$ currently come from 
combining CMB temperature and large angle polarization data from the 
\planck satellite with Baryon Acoustic Oscillation and CMB lensing: 
$\smnu < 0.13$ eV~\citep{planckcollaboration2018}. Future improvements
will likely continue to come from combining CMB data on large scales 
with clustering/lensing data on small scales and low redshifts, where 
the suppression of power by neutrinos is strongest~\citep{brinckmann2019}. 
But they will heavily rely on a better determination of $\tau$, the optical
depth of reionization since CMB experiments measure the combined quantity $A_s
e^{-2\tau}$~\citep{allison2015, liu2016, archidiacono2017}.
Most upcoming CMB experiments, however, are ground-based (\eg~CMB-S4) and 
will not directly constrain $\tau$~\citep{abazajian2016}. Meanwhile, proposed
future space-based experiments such as
LiteBIRD\footnote{http://litebird.jp/eng/} and 
LiteCOrE\footnote{http://www.core-mission.org/}, which have the greatest 
potential to precisely measure $\tau$, have yet to be confirmed. 

Despite the limited progress we can expect on constraining $\tau$ in the
near future, measuring the imprint of $\smnu$ on the 3D clustering of galaxies
provides a promising avenue for improving $\smnu$ constraints. With the
unprecedented cosmic volumes they will probe, upcoming galaxy surveys 
such as DESI\footnote{https://www.desi.lbl.gov/}, PFS\footnote{https://pfs.ipmu.jp/}, 
EUCLID\footnote{http://sci.esa.int/euclid/}, and WFIRST\footnote{https://wfirst.gsfc.nasa.gov/} 
have the potential to tightly constrain $\smnu$~\citep{audren2013, font-ribera2014, petracca2016, sartoris2016, boyle2018}.
Constraining $\smnu$ from 3D galaxy clustering, however, faces two limiting
challenges: (1) accurate theoretical modeling beyond linear scales, for bias tracers, and
in redshift-space and (2) parameter degeneracies that limit the constraining
power of standard two-point clustering analyses. 

Regarding the former, simulations have made huge strides in accurately modeling 
nonlinear structure formation with massive neutrinos~\citep[\emph{e.g.}][]{brandbyge2008, 
villaescusa-navarro2013, castorina2015, adamek2017, emberson2017, banerjee2018, 
villaescusa-navarro2018, villaescusa-navarro2019}. Moreover, new simulation-based
approaches, such as `emulation', enable us to tractably exploit the accuracy of 
$N$-body simulations in analyzing galaxy clustering. A nubmer of recent works have
applied these simulation-based approaches to analyze small-scale galaxy
clustering with remarkable success~\citep[\eg][]{heitmann2009, kwan2015,
euclidcollaboration2018, lange2019, zhai2019, wibking2019}. These developments 
present the opportunity to unlock the information content in nonlinear
clustering to constrain $\smnu$. 
\ch{uros would remark that there isn't much constraining power on small scales
once you marginalize over galaxy bias. edit this to handle that} 

Among the various works that have examined the impact of $\smnu$ on nonlinear 
clustering~\citep[\eg][]{brandbyge2008, saito2008, wong2008, saito2009,
viel2010, agarwal2011, marulli2011, bird2012, castorina2015, banerjee2016,
upadhye2016}, \cite{villaescusa-navarro2018} recently used a suite of more than 1000
$N$-body simulations to examine the redshift-space matter and halo power spectrum. 
They found that the imprint of $\smnu$ and $\sig$ on the redshift-space halo power
spectrum are degenerate and differ by $< 1\%$ and, thus, the $\smnu$ -- $\sig$ 
degeneracy poses a serious limitation on constraining $\smnu$ with the power spectrum. 
Information in the nonlinear regime, however, cascades from the power spectrum to 
higher-order statistics --- \eg~the bispectrum. In fact, studies have long
demonstrated the potential of the bispectrum for improving cosmological parameter 
constraints~\citep{sefusatti2005, sefusattti2006, chan2017, yankelevich2019}. 
\cite{chudaykin2019}, in particular, included $\smnu$ in their forecast and
found that the bispectrum significantly improves constraints on $\smnu$.
However, their perturbation theory based forecast does not include the
constraining power on nonlinear scales. 

In the previous paper of this series~\citep{hahn2020}, we used 22,000 $N$-body
simulations of the \quij suite to quantify the total information content and
constraining power of the redshift-space halo bispectrum down to nonlinear scales. 
For $k_{\rm max}{=}0.5~\mpc$, we found that the bispectrum produces $\Om$,
$\Ob$, $h$, $n_s$, and $\sig$ constraints 1.9, 2.6, 3.1, 3.6, and 2.6 times
tighter than the power spectrum. For $\smnu$, the bispectrum improves
constraints by 5 times over the power spectrum. In the \cite{hahn2020}
forecasts, we include 
% we marginalized over some nuisance parameters and found that results stuck
% but we didn't include a complete galaxy bias model. Therefore, in this work we include 


paragraph on others including galaxy bias for $\smnu$ constraints, but they're all in the 
perturbation theory framework so none extend to nonlinear scales. In this paper we include galaxy 
bias in a simulation-based approach with emulation and LFI in mind. We use HODs, which are 
(a sentence on hods) 

In Section~\ref{sec:hades}, we describe the two $N$-body simulation suites, HADES and Quijote, 
and the halo catalogs constructed from them. We then describe in Section~\ref{sec:bk}, 
how we measure the bispectrum of these simulations. 
Afterwards, we use the redshift-space halo bispectra to demonstrate the distinct imprint of 
$\smnu$ on the bispectrum, which allows it to break the degeneracy between 
$\smnu$ and $\sig$, in Section~\ref{sec:mnusig}. Finally, in Section~\ref{sec:forecasts} 
we present the full information content of the halo bispectrum with a Fisher forecast 
of cosmological parameters and demonstrate how the bispectrum significantly improves 
the constraints on the cosmological parameters: $\Om$, $\Ob$, $h$, $n_s$, $\sig$, and {\em especially} $\smnu$. 
