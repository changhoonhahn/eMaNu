\section{Introduction} \label{sec:intro}

\todo{Neutrino mass background} (condensed version of paper 1)

The discovery of the lower bound on the sum of neutrino masses ($\smnu \gtrsim 0.06$ eV), 
provides conclusive evidence of 
physics beyond the Standard Model of particle physics~\citep{forero2014, gonzalez-garcia2016}. 
A more precise measurement of $\smnu$ has the potential to distinguish 
between the `normal' and `inverted' neutrino mass hierarchy scenarios 
and further reveal the physics of neutrinos. 
Neutrino oscillation experiments, which are insensitive to $\smnu$, and 
other upcoming laboratory experiments (\emph{e.g.} double beta 
decay and tritium beta decay) are not
sufficient to distinguish between the mass hierarches

Neutrino oscillation 
experiments, however, are insensitive to the absolute neutrino mass scales. 
Other laboratory experiments sensitive to $\smnu$  have the potential to place upper 
bounds of $\smnu < 0.2$ eV in upcoming experiments~\citep{bonn2011, drexlin2013}. 
However, these upper bound alone are not sufficient to distinguish between the mass
hierarchies. Neutrinos, through the cosmic neutrino background, affect the 
expansion history and the growth of cosmic structure. Measuring these effects 
with cosmological observables provides complementary and potentially more 
precise measurements of $\smnu$. 

Neutrinos, in the early Universe, are relativistic and contribute to the 
energy density of radiation. Later as they become non-relativistic, 
they contribute to the energy density of matter. This transition affects 
the expansion history of the Universe and leaves imprints observable in 
the cosmic microwave background (CMB) anisotropy spectrum~\citep{lesgourgues2012, lesgourgues2014}. 
Massive neutrinos also impact the growth of structure. On large scales, 
neutrino perturbations are indistinguishable from perturbations of cold 
dark matter (CDM). However, on scales smaller than their free-streaming 
scale, neutrinos do not contribute to the clustering and thereby reduce 
the amplitude of the total matter power spectrum. In addition, they also reduce the growth 
rate of CDM perturbations at late times. This combined suppression of 
the small-scale matter power spectrum leaves measurable imprints 
on the CMB as well as large-scale structure. For more on details the effect
of neutrinos in cosmological observables, we refer readers to 
\cite{lesgourgues2012,lesgourgues2014} and~\cite{gerbino2018}. 


\todo{transition to why LSS neutrino mass measurement is important} (condensed version of paper 1) 
The tightest cosmological constraints on $\smnu$ currently come from 
combining CMB data with other cosmological probes. Temperature and large 
angle polarization data from the {\em Planck} satellite places an upper 
bound of $\smnu < 0.54$ eV with 95\% confidence level~\citep{planckcollaboration2018}. 
Adding the Baryon Acoustic Oscillation (BAO) to the {\em Planck} 
likelihood breaks geometrical degeneracies (among $\smnu$, $h$, $\Om$) 
and significantly tightens the upper bound to $\smnu < 0.16$ eV. CMB 
lensing further tightens the bound to $\smnu < 0.13$ eV, though 
not as significantly. Future improvements will likely continue to come from combining CMB data 
on large scales with clustering/lensing data on small scales and low 
redshifts, where the suppression of power by neutrinos is strongest~\citep{brinckmann2019}. 
CMB experiments, however, measure the combined quantity $A_s e^{-2\tau}$, 
where $\tau$ is the optical depth of reionization. Hence, improvements in 
neutrino mass constraints obtained from comparing the power spectrum 
on small and large scales will heavily rely on a better determination of
$\tau$~\citep{allison2015, liu2016, archidiacono2017}. The best constraints on $\tau$ 
currently come from {\em Planck} --- $\tau = 0.054\pm0.007$. However, most
upcoming ground-based CMB experiments ({\em e.g.} CMB-S4) will not observe 
scales larger than $\ell < 30$, and therefore will not directly constrain 
$\tau$~\citep{abazajian2016}. While the upcoming CLASS experiment aims to 
improve $\tau$ constraints~\citep{watts2018}, proposed future space-based 
experiments such as LiteBIRD\footnote{http://litebird.jp/eng/} and 
LiteCOrE\footnote{http://www.core-mission.org/}, which have the greatest 
potential to precisely measure $\tau$, have yet to be confirmed. 
CMB data, however, is not the only way to improve $\smnu$ constraints. The 
imprint of neutrinos on 3D clustering of galaxies can be measured to constrain 
$\smnu$ and with the sheer cosmic volumes mapped, upcoming surveys such 
as DESI\footnote{https://www.desi.lbl.gov/}, PFS\footnote{https://pfs.ipmu.jp/}, 
EUCLID\footnote{http://sci.esa.int/euclid/}, and WFIRST\footnote{https://wfirst.gsfc.nasa.gov/} 
will be able tightly constrain 
$\smnu$~\citep{audren2013, font-ribera2014, petracca2016, sartoris2016, boyle2018}.

\todo{why the bispectrum: condensed version of paper1}
A major limitation of using 3D clustering is obtaining accurate theoretical 
predictions beyond linear scales, for bias tracers, and in redshift space. 
Simulations have made huge strides in accurately and efficiently modeling 
nonlinear structure formation with massive neutrinos~\citep[\emph{e.g.}][]{brandbyge2008, 
villaescusa-navarro2013, castorina2015, adamek2017, emberson2017, villaescusa-navarro2018}. 
In conjunction, new simulation based `emulation' models that exploit the 
accuracy of $N$-body simulations while minimizing the computing budget 
have been applied to analyze small-scale galaxy clustering with remarkable 
success~\citep[\emph{e.g.}][]{heitmann2009, kwan2015, euclidcollaboration2018, mcclintock2018, zhai2018, wibking2019}. 
Developments on these fronts have the potential to unlock the information 
content in nonlinear clustering to constrain $\smnu$. 

Various works have examined the impact of neutrino masses on nonlinear clustering 
of matter in 
real-space~\citep[\emph{e.g.}][]{brandbyge2008, saito2008, wong2008, saito2009, viel2010, agarwal2011, bird2012, castorina2015, banerjee2016} 
and in redshift-space~\citep{marulli2011, castorina2015, upadhye2016}. Most recently, 
using a suite of more than 1000 simulations, \cite{villaescusa-navarro2018} 
examined the impact of $\smnu$ on the redshift-space matter and halo power 
spectrum to find that the imprint of $\smnu$ and $\sig$ on the power spectrum are 
degenerate and differ by $< 1\%$ (see also Figure~\ref{fig:plk}). The strong $\smnu$ -- $\sig$ degeneracy
poses a serious limitation on constraining $\smnu$ with the power spectrum. 
However, information in the nonlinear regime cascades from the power spectrum 
to higher-order statistics --- \emph{e.g.} the bispectrum. 
In fact, the 
bispectrum has a comparable signal-to-noise ratio to the power spectrum
on nonlinear scales~\citep{sefusatti2005, chan2017}. \todo{summary of paper 1 results} 


paragraph on others including galaxy bias for $\smnu$ constraints, but they're all in the 
perturbation theory framework so none extend to nonlinear scales. In this paper we include galaxy 
bias in a simulation-based approach with emulation and LFI in mind. We use HODs, which are 
(a sentence on hods) 

Furthermore, although $\smnu$ 
is not included in their analyses, \cite{sefusatti2006} and \cite{yankelevich2019} 
have shown that including the bispectrum significantly improves constraints on 
cosmological parameters.
Including $\smnu$, \cite{chudaykin2019} find that the bispectrum significantly 
improves constraints for $\smnu$. Their forecasts, however, do not include the 
constraining power on nonlinear scales (Section~\ref{sec:forecasts}). 
No work to date has quantified the total information content and constraining 
power of the full redshift-space bispectrum down to nonlinear scales --- especially for $\smnu$. 

