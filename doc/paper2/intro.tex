\section{Introduction} \label{sec:intro}
More than two decades ago, neutrino oscillation experiments discovered the lower
bound on the sum of neutrino masses ($\smnu \gtrsim 0.06$ eV) and confirmed
physics beyond the Standard Model~\citep{fukuda1998, forero2014, gonzalez-garcia2016}. 
Since then, experiments have sought to measure $\smnu$ more precisely in order
to distinguish between the `normal' and `inverted' neutrino mass hierarchy
scenarios and further reveal the physics of neutrinos. Upcoming laboratory 
experiments (\eg~double beta decay and tritium beta decay), however, will {\em
not} place the most stringent constraints on $\smnu$~\citep{bonn2011,
drexlin2013}.
%be sufficient to distinguish between the mass hierarchies~\citep[ (\eg~double beta decay and tritium beta decay)][]{bonn2011, drexlin2013}.
Complementary and more precise constraints on $\smnu$ can be
placed by measuring the effect of neutrinos on the expansion history and growth
of cosmic structure. 

In the early Universe, neutrinos are relativistic and contribute to the 
energy density of radiation. Later, as they become non-relativistic, 
they contribute to the energy density of matter. This transition affects 
the expansion history of the Universe and leaves imprints on the cosmic
microwave background~\citep[CMB;][]{lesgourgues2012, lesgourgues2014}. 
Massive neutrinos also impact the growth of structure. 
While neutrino perturbations are indistinguishable from cold dark matter (CDM)
perturbations on large scales, below their free-streaming scale, neutrinos 
do not contribute to the clustering and reduce the 
amplitude of the total matter power spectrum. They also reduce the growth 
rate of CDM perturbations on small scales. This combined suppression of 
the small-scale matter power spectrum leaves measurable imprints 
on the CMB as well as large-scale structure~\citep[for further details see][]{lesgourgues2012, lesgourgues2014, gerbino2018}. 

%\todo{transition to why LSS neutrino mass measurement is important} (condensed version of paper 1) 
The tightest cosmological constraints on $\smnu$ currently come from 
combining CMB temperature and large-angle polarization data from the 
\planck~satellite with Baryon Acoustic Oscillation and CMB lensing: 
$\smnu < 0.13$ eV~\citep{planckcollaboration2018}. Future improvements
will likely continue to come from combining CMB data on large scales 
with clustering/lensing data on small scales and low redshifts, where 
the suppression of power by neutrinos is strongest~\citep{brinckmann2019}. 
But they will heavily rely on a better determination of $\tau$, the optical
depth of reionization since CMB experiments measure the combined quantity $A_s
e^{-2\tau}$~\citep{allison2015, liu2016, archidiacono2017}.
Major upcoming CMB experiments, however, are ground-based (\eg~CMB-S4) and 
will not directly constrain $\tau$~\citep{abazajian2016}. The CLASS
experiment~\citep{xu2020} aims to improve $\tau$ constraints from the ground;
however, future space-based experiments such as LiteBIRD\footnote{http://litebird.jp/eng/} 
and LiteCOrE\footnote{http://www.core-mission.org/}, which have the greatest 
potential to precisely measure $\tau$, have yet to be confirmed. 

Despite the $\tau$ bottleneck in the near future, measuring the $\smnu$ imprint 
on the 3D clustering of galaxies provides a promising avenue for improving $\smnu$ constraints. 
Upcoming galaxy surveys such as DESI\footnote{https://www.desi.lbl.gov/}, 
PFS\footnote{https://pfs.ipmu.jp/}, EUCLID\footnote{http://sci.esa.int/euclid/}, 
and the Roman Space Telescope\footnote{https://roman.gsfc.nasa.gov/}, 
with the unprecedented cosmic volumes they will probe, 
have the potential to tightly constrain 
$\smnu$~\citep{audren2013, font-ribera2014, petracca2016, sartoris2016, boyle2018}.
Constraining $\smnu$ from 3D galaxy clustering, however, faces two major 
challenges: (1) accurate theoretical modeling beyond linear scales, for biased
tracers in redshift-space and (2) parameter degeneracies that limit the
constraining power of standard two-point clustering analyses. 

For the former, simulations have made huge strides in accurately modeling 
nonlinear structure formation with massive neutrinos~\citep[\eg][]{brandbyge2008, 
villaescusa-navarro2013, castorina2015, adamek2017, emberson2017, banerjee2018, 
villaescusa-navarro2018a, yoshikawa2020, villaescusa-navarro2020a}. Moreover, new simulation-based
approaches to modeling such as `emulation' enable us to tractably exploit the accuracy of 
$N$-body simulations and analyze galaxy clustering on nonlinear scales beyond
traditional perturbation theory methods. Recent works have applied
these simulation-based approaches to analyze small-scale galaxy clustering with
remarkable success~\citep[\eg][]{heitmann2009a, kwan2015, euclidcollaboration2018, lange2019, zhai2019, wibking2019}. 
These developments present the opportunity to significantly improve $\smnu$
constraints by unlocking the information content in nonlinear clustering, where
the impact of massive neutrinos is strongest~\citep[\eg][]{brandbyge2008,
saito2008, wong2008, saito2009, viel2010a, agarwal2011, marulli2011, bird2012,
castorina2015, banerjee2016, upadhye2016, banerjee2020, allys2020, massara2020,
uhlemann2020}.

For the latter, parameter degeneracies such as the $\smnu$--$\sig$
degeneracy pose serious limitations on constraining 
%\cite{villaescusa-navarro2018} recently used more than 1000 $N$-body
%simulations from the {\sc Hades} suite to examine the redshift-space matter and
%halo power spectrum.  They found that the imprint of $\smnu$ and $\sig$ on the
%redshift-space halo power spectrum are degenerate and differ by $< 1\%$. This
%$\smnu$ -- $\sig$ degeneracy poses a serious limitation on constraining 
$\smnu$ with the power spectrum~\citep{villaescusa-navarro2018a}. However, 
information in the nonlinear regime cascades
from the power spectrum to higher-order statistics such as the bispectrum 
and help break these degeneracies~\citep{hahn2020}. Previous studies have 
already demonstrated the potential of the bispectrum for improving cosmological 
parameter constraints~\citep{sefusatti2005, sefusatti2006, chan2017, yankelevich2019,
agarwal2020, kamalinejad2020}.
\cite{chudaykin2019}, in particular, included $\smnu$ in their forecast and
found that the bispectrum significantly improves constraints on $\smnu$.
However, none of these perturbation theory based forecast includes the
constraining power on nonlinear scales. 

In \cite{hahn2020}, the previous paper of this series, we used 22,000 $N$-body
simulations from the \quij~suite to quantify the total information content and
constraining power of the redshift-space halo bispectrum down to nonlinear scales. 
For $k_{\rm max}{=}0.5~\mpc$, we found that the bispectrum achieves $\Om$,
$\Ob$, $h$, $n_s$, and $\sig$ constraints 1.9, 2.6, 3.1, 3.6, and 2.6 times
tighter than the power spectrum. For $\smnu$, the bispectrum improved 
constraints by 5 times over the power spectrum. In this forecast, we marginalized 
over linear bias, $b_1$, and halo mass limit, $M_{\rm lim}$, parameters. We also found that the
improvements from the bispectrum are not impacted when we include quadratic 
and nonlocal bias parameters in the forecast. Nevertheless, \cite{hahn2020}
focused on the halo bispectrum. Actual constraints on $\smnu$, however, will be 
derived from the distribution of galaxies and therefore require a more 
realistic and complete galaxy bias model, which we provide in this paper.

In this work, we present the total information content and constraining power
of the {\em redshift-space galaxy bispectrum} down to $k_{\rm max}= 0.5~\hmpc$. For our galaxy
bias model, we use the halo occupation distribution (HOD) framework, which provides a
statistical prescription for populating dark matter halos with central and satellite
galaxies. The HOD model has been successful in reproducing the observed galaxy
clustering~\citep[\emph{e.g.}][]{zheng2005, leauthaud2012, tinker2013, zentner2016, vakili2019}. 
It is also the primary framework used in simulation-based clustering
analyses~\citep[\eg][]{mcclintock2018, zhai2019, lange2019, wibking2019}. 
We first construct 75,000 galaxy mock catalogs from the \quij~$N$-body
simulations then use them to calculate Fisher matrix forecasts. Afterward, we
present the constraining power of the galaxy bispectrum on $\smnu$ and other 
cosmological parameters after marginalizing over the HOD parameters. This work
is the second paper in a series that aims to demonstrate the potential for
simulation-based galaxy bispectrum analyses in constraining $\smnu$. Later in
the series, we will also present methods to tackle challenges that come with
analyzing the full galaxy bispectrum, such as data compression to reduce its
dimensionality. The series will culminate in a fully simulation-based galaxy
power spectrum and bispectrum reanalysis of SDSS-III BOSS. 

In Sections~\ref{sec:sims} and~\ref{sec:hod}, we describe the \quij~$N$-body simulation 
suite and the HOD framework we use to construct the \molino~suite of galaxy mock 
catalogs from them.  We then describe in Section~\ref{sec:methods}, how we measure the bispectrum and
calculate the Fisher forecasts of the cosmological parameters from the galaxy
mocks. Finally, in Section~\ref{sec:results}, we present the full information
content of the galaxy bispectrum and demonstrate how it significantly improves
the constraints on the cosmological parameters: $\Om$, $\Ob$, $h$, $n_s$,
$\sig$, and {\em especially} $\smnu$. 
