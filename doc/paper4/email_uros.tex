\documentclass[12pt, letterpaper, preprint]{aastex62}
%\usepackage[breaklinks,colorlinks, urlcolor=blue,citecolor=blue,linkcolor=blue]{hyperref}
\usepackage{color}
\usepackage{amsmath}
\usepackage{natbib}
\usepackage{ctable}
\usepackage{bm}
\usepackage[normalem]{ulem} % Added by MS for \sout -> not required for final version
\usepackage{xspace}

% typesetting shih
\linespread{1.08} % close to 10/13 spacing
\setlength{\parindent}{1.08\baselineskip} % Bringhurst
\setlength{\parskip}{0ex}
\let\oldbibliography\thebibliography % killin' me.
\renewcommand{\thebibliography}[1]{%
  \oldbibliography{#1}%
  \setlength{\itemsep}{0pt}%
  \setlength{\parsep}{0pt}%
  \setlength{\parskip}{0pt}%
  \setlength{\bibsep}{0ex}
  \raggedright
}
\setlength{\footnotesep}{0ex} % seriously?

% citation alias

% math shih
\newcommand{\setof}[1]{\left\{{#1}\right\}}
\newcommand{\given}{\,|\,}
\newcommand{\pseudo}{{\mathrm{pseudo}}}
\newcommand{\Var}{\mathrm{Var}}
\newcommand{\Nd}{{100}\xspace}

\newcommand{\Om}{\Omega_{\rm m}} 
\newcommand{\Ob}{\Omega_{\rm b}} 
\newcommand{\OL}{\Omega_\Lambda}
\newcommand{\smnu}{\sum m_\nu} 
\newcommand{\sig}{\sigma_8} 
\newcommand{\hmpc}{\,h/\mathrm{Mpc}}
\newcommand{\lss}{{\small{LSS}}\xspace}
\newcommand{\mpc}{h/{\rm Mpc}}
\newcommand{\quij}{{\sc Quijote}~}

\newcommand{\specialcell}[2][c]{%
  begin{tabular}[#1]{@{}c@{}}#2\end{tabular}}
% text shih
\newcommand{\foreign}[1]{\textsl{#1}}
\newcommand{\etal}{\foreign{et~al.}}
\newcommand{\opcit}{\foreign{Op.~cit.}}
\newcommand{\documentname}{\textsl{Article}}
\newcommand{\equationname}{equation}
\newcommand{\bitem}{\begin{itemize}}
\newcommand{\eitem}{\end{itemize}}
\newcommand{\beq}{\begin{equation}}
\newcommand{\eeq}{\end{equation}}

\begin{document}\sloppy\sloppypar\frenchspacing 
We're using Paco's \quij suite (Table~\ref{tab:sims}) to determine
whether using paired-fixed simulations instead of standard $N$-body 
simulations can bias clustering analyses with the redshift-space
power spectrum and bispectrum. More specifically, the context we have
in mind is to see whether an emulator constructed using paired-fixed
would bias parameter constraints. We quantify the bias for some observable 
$X$ as
\beq
\beta_X = \frac{\overline{X}_{\rm std} - \overline{X}_{\rm pf}}{\sqrt{\sigma_{\rm std}^2 + \sigma_{\rm pf}^2}}
\eeq
where $\overline{X}_{\rm std} = \frac{1}{N_{\rm std}}\sum\limits_{i=1}^{N_{\rm std}} X_{{\rm std}, i}$ 
and $\overline{X}_{\rm pf} = \frac{1}{N_{\rm pf}}\sum\limits_{i=1}^{N_{\rm pf}} X_{{\rm pf}, i}$. 
$X_{{\rm pf},i}$ is the mean observable for a pair of paired-fixed simulations: 
$X_{{\rm pf},i} = \frac{1}{2} \big[ X_{{\rm pf},i}^{(1)} + X_{{\rm pf},i}^{(2)}\big]$.
For the most part $N_{\rm std} = 500$ and $N_{\rm pf} = 250$. 

We first look at the amplitudes of the power spectrum and bispectrum in 
Figures~\ref{fig:pk} and~\ref{fig:bk_rsd}. There's some noticable discrepancies
for the bispectrum at the $h^+$ and $M_\nu^{+++}$ cosmologies, but nothing 
statistically significant. Next we look at the derivatives of the power 
spectrum and bispectrum with respect to $\Om$, $\Ob$, $h$, $n_s$, $\sig$, 
and $\smnu$. Here we find more significant biases in the derivatives 
with respect to $h$ and $\sig$.



%%%%%%%%%%%%%%%%%%%%%%%%%%%%%%%%%%%%%%%%%%
% simulation table
%%%%%%%%%%%%%%%%%%%%%%%%%%%%%%%%%%%%%%%%%%
\begin{table}
    \caption{
    The \quij suite includes 15,000 standard $N$-body simulations at the fiducial cosmology to 
    accurately estimate the covariance matrices. It also includes sets of 500 simulations at 13 
    different cosmologies, where only one parameter is varied from the fiducial value (underlined), 
    to estimate derivatives of observables along the cosmological parameters. At every cosmology,
    the \quij suite also includes 250 pairs of paired-fixed simulations.  
    } 
    \begin{center}
    \begin{tabular}{cccccccccc} \toprule
    Name  &$\smnu$ & $\Omega_m$ & $\Omega_b$ & $h$ & $n_s$ & $\sigma_8$ & ICs & standard & paired-fixed \\
    & & & & & & & & realizations & pairs \\[3pt] \hline\hline
    Fiducial 	& 0.0         & 0.3175 & 0.049 & 0.6711 & 0.9624 & 0.834 & 2LPT & 15,000 & 250\\ 
    Fiducial ZA     & 0.0         & 0.3175 & 0.049 & 0.6711 & 0.9624 & 0.834 & Zel'dovich& 500 & 250\\ 
    $\smnu^+$       & \underline{0.1} eV & 0.3175 & 0.049 & 0.6711 & 0.9624 & 0.834 & Zel'dovich & 500 & 250\\ 
    $\smnu^{++}$    & \underline{0.2} eV & 0.3175 & 0.049 & 0.6711 & 0.9624 & 0.834 & Zel'dovich & 500 & 250\\ 
    $\smnu^{+++}$   & \underline{0.4} eV & 0.3175 & 0.049 & 0.6711 & 0.9624 & 0.834 & Zel'dovich & 500 & 250\\ 
    $\Omega_m^+$    & 0.0   & \underline{ 0.3275} & 0.049 & 0.6711 & 0.9624 & 0.834 & 2LPT & 500 & 250\\ 
    $\Omega_m^-$    & 0.0   & \underline{ 0.3075} & 0.049 & 0.6711 & 0.9624 & 0.834 & 2LPT & 500 & 250\\ 
    $\Omega_b^+$    & 0.0   & 0.3175 & \underline{0.051} & 0.6711 & 0.9624 & 0.834 & 2LPT & 500 & 250\\ 
    $\Omega_b^-$    & 0.0   & 0.3175 & \underline{0.047} & 0.6711 & 0.9624 & 0.834 & 2LPT & 500 & 250\\ 
    $h^+$           & 0.0   & 0.3175 & 0.049 & \underline{0.6911} & 0.9624 & 0.834 & 2LPT & 500 & 250\\ 
    $h^-$           & 0.0   & 0.3175 & 0.049 & \underline{0.6511} & 0.9624 & 0.834 & 2LPT & 500 & 250\\ 
    $n_s^+$         & 0.0   & 0.3175 & 0.049 & 0.6711 & \underline{0.9824} & 0.834 & 2LPT & 500 & 250\\ 
    $n_s^-$         & 0.0   & 0.3175 & 0.049 & 0.6711 & \underline{0.9424} & 0.834 & 2LPT & 500 & 250\\ 
    $\sigma_8^+$    & 0.0   & 0.3175 & 0.049 & 0.6711 & 0.9624 & \underline{0.849} & 2LPT & 500 & 250\\ 
    $\sigma_8^-$    & 0.0   & 0.3175 & 0.049 & 0.6711 & 0.9624 & \underline{0.819} & 2LPT & 500 & 250\\[3pt]
    \hline
    \end{tabular} \label{tab:sims}
    \end{center}
\end{table}
%%%%%%%%%%%%%%%%%%%%%%%%%%%%%%%%%%%%%%%%%%


\begin{figure}
\begin{center}
    \includegraphics[width=0.8\textwidth]{figs/pf_Pk.pdf} 
    \caption{Comparison of the power spectra of standard $N$-body simulations 
    and paired-fixed simulations. We compare the real-space power spectrum, 
    redshift-space power spectrum monopole, and quadrupole in the left, center, 
    and right columns, respectively. In the top panels, we compare the average 
    power spectra of the standard simulations (black line) to the paired-fixed
    simulations (orange scatter) at the fiducial cosmology. In the bottom panels, 
    we compare the bias, $\beta$ of the paired-fixed simulations for the power 
    spectra for all 14 cosmologies (Table~\ref{tab:sims}). We mark bias within 
    $\pm2\sigma$ within the shaded region. At the fiducial 
    cosmology we use 15,000 standard $N$-body and $250$ pairs of paired-fixed 
    simulations. For the other cosmologies, we use 500 standard $N$-body and $250$ pairs 
    of paired-fixed simulations. Consistent with previous results, {\em we find no
    significant bias in the real and redshift-space power spectra of paired-fixed
    simulations.}
    }
\label{fig:pk}
\end{center}
\end{figure}


\begin{figure}
\begin{center}
    \includegraphics[width=0.8\textwidth]{figs/pf_Bk_rsd.pdf} 
    \caption{Comparison of the redshift-space bispectrum of standard $N$-body 
    and paired-fixed simulations. In the top panel, we compare the average 
    bispectrum of the standard simulations (black line) to the paired-fixed
    simulations (orange scatter) at the fiducial cosmology. We include all 
    1898 triangles out to $k_{\rm max} = 0.5$. In the bottom panels, we 
    compare the bias, $\beta$ of the paired-fixed simulations for the bispectrum 
    for all 14 cosmologies, as a function of triangle configuration (left) 
    and their distribution (right). The $\beta$ distributions of the different cosmologies are mostly in good 
    agreement with $\mathcal{N}(0,1)$ (lower right panel). A few cosmologies, 
    $h^+$, $M_\nu^{+++}$, have noticable discrepancies; however, these 
    discrepancies are within $1\sigma$. {\em Hence, we also find no significant 
    bias in the redshift-space bispectrum of paired-fixed simulations.}
    }
\label{fig:bk_rsd}
\end{center}
\end{figure}

\begin{figure}
\begin{center}
    \includegraphics[width=\textwidth]{figs/pf_dPdtheta_rsd.pdf}
    \caption{Comparison of the redshift-space power spectrum monopole (top panels) 
    and quadrupole (center panels) derivatives with respect to $\Om$, $\Ob$, $h$, 
    $n_s$, $\sigma_8$, and $M_\nu$ derived from standard $N$-body versus 
    paired-fixed simulations. In the top panels, we compare the derivatives from 
    the standard simulations (black) to the paired-fixed simulations (orange). 
    In the bottom panels, we compare the bias, $\beta$ of each $P_\ell$ derivative. 
    We find significant biases for the derivatives with respect to $h$, $\sig$ and 
    $M_\nu$. The bias for both ${\rm d} P_0/{\rm d}h$ and ${\rm d} P_2/{\rm d}h$ exceed 
    $2\sigma$ in the range $k > 0.04~\mpc$.} 
\label{fig:dpk_rsd}
\end{center}
\end{figure}

\begin{figure}
\begin{center}
    \includegraphics[width=\textwidth]{figs/pf_dBdtheta_rsd_kmax0_5.pdf}
    \caption{Comparison of the redshift-space bispectrum derivatives with respect
    to $\Om$, $\Ob$, $h$, $n_s$, $\sigma_8$, and $M_\nu$ derived from standard 
    $N$-body simulations versus paired-fixed simulations. In the left panels, we 
    compare the derivatives from the standard simulations (black) to the paired-fixed
    simulations (orange). In the right panels, we compare the bias, $\beta$ of
    the derivatives for each parameter. The bias for ${\rm d} B_0/{\rm d}h$ exceeds 
    $2\sigma$ for triangle configurations with $0.1 \lesssim k_1, k_2, k_3 \lesssim 0.3~\mpc$.}
\label{fig:dbk_rsd}
\end{center}
\end{figure}

\begin{figure}
\begin{center}
    \includegraphics[width=\textwidth]{figs/pf_P_posterior_rsd_kmax0_5_dmnu_fin_dh_pm.pdf}
    \caption{{to be updated}}
\label{fig:ppost_rsd}
\end{center}
\end{figure}

\begin{figure}
\begin{center}
    \includegraphics[width=\textwidth]{figs/pf_B_posterior_rsd_kmax0_5_dmnu_fin_dh_pm.pdf} 
    \caption{{to be updated}}
\label{fig:bpost_rsd}
\end{center}
\end{figure}

\end{document}
