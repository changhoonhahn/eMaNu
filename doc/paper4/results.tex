\begin{figure}
\begin{center}
    \includegraphics[width=0.8\textwidth]{figs/pf_Pk.pdf} 
    \caption{Comparison of the power spectra of standard $N$-body simulations 
    and paired-fixed simulations. We compare the real-space power spectrum, 
    redshift-space power spectrum monopole, and quadrupole in the left, center, 
    and right columns, respectively. In the top panels, we compare the average 
    power spectra of the standard simulations (black line) to the paired-fixed
    simulations (orange scatter) at the fiducial cosmology. In the bottom panels, 
    we compare the bias, $\beta$ of the paired-fixed simulations for the power 
    spectra for all 14 cosmologies (Table~\ref{tab:sims}). We mark bias within 
    $\pm2\sigma$ within the shaded region. At the fiducial 
    cosmology we use 15,000 standard $N$-body and $250$ pairs of paired-fixed 
    simulations. For the other cosmologies, we use 500 standard $N$-body and $250$ pairs 
    of paired-fixed simulations. Consistent with previous results, {\em we find no
    significant bias in the real and redshift-space power spectra of paired-fixed
    simulations.}
    }
\label{fig:pk}
\end{center}
\end{figure}


\begin{figure}
\begin{center}
    \includegraphics[width=0.8\textwidth]{figs/pf_Bk_real.pdf} 
    \caption{Comparison of the real-space bispectrum of standard $N$-body simulations 
    and paired-fixed simulations. In the top panel, we compare the average 
    bispectrum of the standard simulations (black line) to the paired-fixed
    simulations (orange scatter) at the fiducial cosmology. We include all 
    1898 triangles out to $k_{\rm max} = 0.5$. In the bottom panels, we 
    compare the bias, $\beta$ of the paired-fixed simulations for the bispectrum 
    for all 14 cosmologies, as a function of triangle configuration (left) 
    and their distribution (right). The $\beta$ distributions are in good 
    agreement with a Gaussian distribution, $\mathcal{N}(0,1)$ (blackdashed). 
    {\em Hence, we find no significant bias in the real-space bispectrum of 
    paired-fixed simulations.}
}
\label{fig:bk_real}
\end{center}
\end{figure}

\begin{figure}
\begin{center}
    \includegraphics[width=0.8\textwidth]{figs/pf_Bk_rsd.pdf} 
    \caption{Same as Figure~\ref{fig:bk_real} but for the redshift-space bispectrum.
    The $\beta$ distributions of the different cosmologies are mostly in good 
    agreement with $\mathcal{N}(0,1)$ (lower right panel). A few cosmologies, 
    $h^+$, $M_\nu^{+++}$, have noticable discrepancies; however, these 
    discrepancies are within $1\sigma$. {\em Hence, we also find no significant 
    bias in the redshift-space bispectrum of paired-fixed simulations.}
    }
\label{fig:bk_rsd}
\end{center}
\end{figure}

\begin{figure}
\begin{center}
    \includegraphics[width=\textwidth]{figs/pf_dPdtheta_real.pdf}
    \caption{Comparison of the real-space power spectrum derivatives with respect
    to $\Om$, $\Ob$, $h$, $n_s$, $\sigma_8$, and $M_\nu$ derived from standard 
    $N$-body simulations versus paired-fixed simulations. In the top panels, we 
    compare the derivatives from the standard simulations (black) to the paired-fixed
    simulations (orange). In the bottom panels, we compare the bias, $\beta$ of 
    each power spectrum derivative. The derivatives with respect 
    to $\Om$ and $\sig$ have noticably non-zero biases; however these deviations are
    within $1\sigma$. Therefore, {\em we find no significant bias in the real-space 
    power spectrum derivatives calculated from paired-fixed simulations.}
}
\label{fig:dpk_real}
\end{center}
\end{figure}

\begin{figure}
\begin{center}
    \includegraphics[width=\textwidth]{figs/pf_dPdtheta_rsd.pdf}
    \caption{Same as Figure~\ref{fig:dpk_real} but for the derivatives of the power
    spectrum monopole (top panels) and quadrupole (center panels) with respect
    to $\Om$, $\Ob$, $h$, $n_s$, $\sigma_8$, and $M_\nu$. The bottom panel reveals
    significantly larger biases for ${\rm d} P_0/{\rm d}\theta$ and ${\rm d} P_2/{\rm d}\theta$
    than the real-space ${\rm d} P/{\rm d}\theta$, especially for $h$, $\sig$ and $M_\nu$. 
    In fact, the bias for both ${\rm d} P_0/{\rm d}h$ and ${\rm d} P_2/{\rm d}h$ exceed 
    $2\sigma$ in the range $k > 0.04~\mpc$. \ch{some nuanced discussion about how this still
    is not a cause for concern because this is the bias for a 500 Gpc volume so it's well 
    within the tolerance.} 
}
\label{fig:dpk_rsd}
\end{center}
\end{figure}

\begin{figure}
\begin{center}
    \includegraphics[width=\textwidth]{figs/pf_dBdtheta_real_kmax0_5.pdf}
    \caption{Comparison of the real-space bispectrum derivatives with respect
    to $\Om$, $\Ob$, $h$, $n_s$, $\sigma_8$, and $M_\nu$ derived from standard 
    $N$-body simulations versus paired-fixed simulations. In the left panels, we 
    compare the derivatives from the standard simulations (black) to the paired-fixed
    simulations (orange). In the right panels, we compare the bias, $\beta$ of
    the derivatives for each parameter. {\em We find no significant bias in the
    real-space bispectrum derivatives calculated from paired-fixed simulations.}
}
\label{fig:dbk_real}
\end{center}
\end{figure}


\begin{figure}
\begin{center}
    \includegraphics[width=\textwidth]{figs/pf_dBdtheta_rsd_kmax0_5.pdf}
    \caption{Same as Figure~\ref{fig:dbk_real} but for the derivatives of the
    redshift-space bispectrum monopole with respect to the cosmological parameters. 
    Similar to the power spectrum derivatives, we find significantly larger biases
    for redshift-space derivatives than in real-space. The bias for 
    ${\rm d} B_0/{\rm d}h$ exceeds $2\sigma$ for triangle configurations with 
    $0.1 \lesssim k_1, k_2, k_3 \lesssim 0.3~\mpc$. \ch{again some nuanced 
    discussion about how this still is not a cause for concern because this is 
    the bias for a 500 Gpc volume so it's well within the tolerance.} 
    }
\label{fig:dbk_rsd}
\end{center}
\end{figure}

\begin{figure}
\begin{center}
    \includegraphics[width=\textwidth]{figs/pf_P_posterior_rsd_kmax0_5.pdf}
    \caption{\ch{to be updated}}
\label{fig:ppost_rsd}
\end{center}
\end{figure}

\begin{figure}
\begin{center}
    \includegraphics[width=\textwidth]{figs/pf_B_posterior_rsd_kmax0_5.pdf}
    \caption{\ch{to be updated}}
\label{fig:bpost_rsd}
\end{center}
\end{figure}

\section{Results} \label{sec:results} 
We're interesting in determining whether paired-fixed simulations can be used, instead of
standard $N$-body simulations, to reduce the statistical uncertainty from cosmic variance 
in applications where we want to accurately estimates of the mean of observables, such as
in emulator-based approach. More specifically, we examine whether using paired-fixed 
simulations introduce bias for the full real and redshift-space bispectrum with the 
unprecedented statistical precision from over $22,000$ \quij simulatons. Furthermore,
we exmaine whether such biases can propagate to constraints on inferred cosmological 
parameters. We also we reexamine whether paired-fixed simulations introduce bias for
the power spectrum. 

We quantify the bias of paired-fixed simulations for some observable $X$ as
\beq
\beta_X = \frac{\overline{X}_{\rm std} - \overline{X}_{\rm pf}}{\sqrt{\sigma_{\rm std}^2 + \sigma_{\rm pf}^2}}
\eeq
where the mean observable of the standard $N$-body and paired-fixed 
simulations: $\overline{X}_{\rm std} = \frac{1}{N}\sum\limits_{i=1}^{N} X_{{\rm std}, i}$ 
and $\overline{X}_{\rm pf} = \frac{1}{N}\sum\limits_{i=1}^{N} X_{{\rm pf}, i}$. 
$X_{{\rm pf},i}$ is the mean observable of a pair of paired-fixed simulations: 
\beq
X_{{\rm pf},i} = \frac{1}{2} \big[ X_{{\rm pf},i}^{(1)} + X_{{\rm pf},i}^{(2)}\big].
\eeq



