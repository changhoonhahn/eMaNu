%%%%%%%%%%%%%%%%%%%%%%%%%%%%%%%%%%%%%%%%%%
% simulation table
%%%%%%%%%%%%%%%%%%%%%%%%%%%%%%%%%%%%%%%%%%
\begin{table}
\caption{
The \quij suite includes 15,000 standard $N$-body simulations at the fiducial cosmology to 
accurately estimate the covariance matrices. It also includes sets of 500 simulations at 13 
different cosmologies, where only one parameter is varied from the fiducial value (underlined), 
to estimate derivatives of observables along the cosmological parameters. At every cosmology,
the \quij suite also includes 250 pairs of paired-fixed simulations.  
} 
\begin{center}
\begin{tabular}{cccccccccc} \toprule
Name  &$\smnu$ & $\Omega_m$ & $\Omega_b$ & $h$ & $n_s$ & $\sigma_8$ & ICs & standard & paired-fixed \\
& & & & & & & & realizations & pairs \\[3pt] \hline\hline
Fiducial 	& 0.0         & 0.3175 & 0.049 & 0.6711 & 0.9624 & 0.834 & 2LPT & 15,000 & 250\\ 
Fiducial ZA     & 0.0         & 0.3175 & 0.049 & 0.6711 & 0.9624 & 0.834 & Zel'dovich& 500 & 250\\ 
$\smnu^+$       & \underline{0.1} eV & 0.3175 & 0.049 & 0.6711 & 0.9624 & 0.834 & Zel'dovich & 500 & 250\\ 
$\smnu^{++}$    & \underline{0.2} eV & 0.3175 & 0.049 & 0.6711 & 0.9624 & 0.834 & Zel'dovich & 500 & 250\\ 
$\smnu^{+++}$   & \underline{0.4} eV & 0.3175 & 0.049 & 0.6711 & 0.9624 & 0.834 & Zel'dovich & 500 & 250\\ 
$\Omega_m^+$    & 0.0   & \underline{ 0.3275} & 0.049 & 0.6711 & 0.9624 & 0.834 & 2LPT & 500 & 250\\ 
$\Omega_m^-$    & 0.0   & \underline{ 0.3075} & 0.049 & 0.6711 & 0.9624 & 0.834 & 2LPT & 500 & 250\\ 
$\Omega_b^+$    & 0.0   & 0.3175 & \underline{0.051} & 0.6711 & 0.9624 & 0.834 & 2LPT & 500 & 250\\ 
$\Omega_b^-$    & 0.0   & 0.3175 & \underline{0.047} & 0.6711 & 0.9624 & 0.834 & 2LPT & 500 & 250\\ 
$h^+$           & 0.0   & 0.3175 & 0.049 & \underline{0.6911} & 0.9624 & 0.834 & 2LPT & 500 & 250\\ 
$h^-$           & 0.0   & 0.3175 & 0.049 & \underline{0.6511} & 0.9624 & 0.834 & 2LPT & 500 & 250\\ 
$n_s^+$         & 0.0   & 0.3175 & 0.049 & 0.6711 & \underline{0.9824} & 0.834 & 2LPT & 500 & 250\\ 
$n_s^-$         & 0.0   & 0.3175 & 0.049 & 0.6711 & \underline{0.9424} & 0.834 & 2LPT & 500 & 250\\ 
$\sigma_8^+$    & 0.0   & 0.3175 & 0.049 & 0.6711 & 0.9624 & \underline{0.849} & 2LPT & 500 & 250\\ 
$\sigma_8^-$    & 0.0   & 0.3175 & 0.049 & 0.6711 & 0.9624 & \underline{0.819} & 2LPT & 500 & 250\\[3pt]
\hline
\end{tabular} \label{tab:sims}
\end{center}
\end{table}
%%%%%%%%%%%%%%%%%%%%%%%%%%%%%%%%%%%%%%%%%%
% brief summary of the Quijote suite. 
\section{The Quijote Simulation Suite} \label{sec:sims}
We use a subset of simulations from the \quij suite, a set of 43,000 $N$-body 
simulations that spans over 7000 cosmological models and contains, at a single 
redshift, over 8.5 trillion particles. The \quij suite was designed to quantify
the information content of cosmological observables and also to train machine 
learning algorithms. Hence, the suite includes enough realizations to accurately 
estimate the covariance matrices of high-dimensional observables such as the 
bispectrum as well as the derivatives of these observables with respect to 
cosmological parameters. For the derivatives, the suite includes sets of 
simulations run at different cosmologies where only one parameter is varied 
from the fiducial cosmology  
($\Om{=}0.3175$, $\Ob{=}0.049$, $h{=}0.6711$, $n_s{=}0.9624$, $\sig{=}0.834$, 
and $\smnu{=}0.0$ eV). Along $\Om$, $\Ob$, $h$, $n_s$, and $\sig$, the fiducial 
cosmology is adjusted by either a small step above or below the fiducial value:
$\{\Om^{+}, \Om^{-}, \Ob^{+}, \Ob^{-}, h^+, h^-, n_s^{+}, n_s^{-}, \sig^{+}, \sig^-\}$ . 
Along $\smnu$, because $\smnu \ge 0.0$ eV and the derivative of certain observable 
with respect to $\smnu$ is noisy, \quij includes sets of simulations for 
$\{\smnu^{+}, \smnu^{++}, \smnu^{+++} \} = \{0.1, 0.2, 0.4~{\rm eV}\}$.
At each of these 14 cosmologies, \quij includes sets of standard $N$-body and 
suppressed variance simulations. 

The initial conditions for all the simulations were generated at $z=127$ using 
second-order perturbation theory for simulations with massless neutrinos 
($\smnu = 0.0$ eV) and the Zel’dovich approximation for massive neutrinos 
($\smnu > 0.0$ eV). The initial conditions with massive neutrinos take 
their scale-dependent growth factors/rates into account using the 
\cite{zennaro2017a} method, while for the massless neutrino case we use 
the traditional scale-independent rescaling. 
%The \textsc{Quijote} simulations contain standard, fixed and paired fixed simulations. The differences between those is the way the initial conditions are generated. Consider a Fourier-space mode, $\delta(\vec{k})$. Since it is in general a complex number, we can write it as $\delta(\vec{k})=Ae^{i\theta}$, where both the amplitude $A$, and the phase $\theta$, depends on the considered wavenumber $\vec{k}$. In standard simulations, $A$ follows a Rayleigh distribution and $\theta$ is drawn from an uniform distribution between 0 and $\pi$. We note that this is the traditional way to generate ICs for cosmological simulations. In fixed simulations, while $\theta$ is still drawn from an uniform distribution between 0 and $2\pi$, the value of $A$ is fixed to the square root of the variance of the previous Rayleigh distribution. Finally, paired fixed simulations are two fixed simulations where the phases of the two pairs differ by $\pi$. We refer the reader to \citep{Pontzen_2016,RP_16,Paco_2018b} for further details.


From the initial conditions, 
the simulations follow the gravitational evolution of $512^3$ dark matter
particles, and $512^3$ neutrino particles for massive neutrino models, to 
$z=0$ using {\sc Gadget-III} TreePM+SPH code~\citep{springel2005}. 
Simulations with massive neutrinos are run using the ``particle method'', 
where neutrinos are described as a collisionless and pressureless fluid 
and therefore modeled as particles, same as CDM~\citep{brandbyge2008,viel2010}.
Halos are then identified using the Friends-of-Friends algorithm~\citep[FoF;][]{davis1985} 
with linking length $b=0.2$ on the CDM + baryon distribution. We limit 
the halo catalogs to halos with masses above 
$M_{\rm lim} = 3.2\times 10^{13} h^{-1}M_\odot$.
For the fiducial cosmology, the halo catalogs have ${\sim}156,000$ halos 
($\bar{n} \sim 1.56 \times 10^{-4}~h^3{\rm Gpc}^{-3}$) with $\bar{n} P_0(k=0.1)\sim 3.23$. 
We refer readers to Villaescusa-Navarro et al. (in preparation) and Hahn et al. (2019) 
for further details on the \quij simulations.

