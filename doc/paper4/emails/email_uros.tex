\documentclass[12pt, letterpaper, preprint]{aastex62}
%\usepackage[breaklinks,colorlinks, urlcolor=blue,citecolor=blue,linkcolor=blue]{hyperref}
\usepackage{color}
\usepackage{amsmath}
\usepackage{natbib}
\usepackage{ctable}
\usepackage{bm}
\usepackage[normalem]{ulem} % Added by MS for \sout -> not required for final version
\usepackage{xspace}

% typesetting shih
\linespread{1.08} % close to 10/13 spacing
\setlength{\parindent}{1.08\baselineskip} % Bringhurst
\setlength{\parskip}{0ex}
\let\oldbibliography\thebibliography % killin' me.
\renewcommand{\thebibliography}[1]{%
  \oldbibliography{#1}%
  \setlength{\itemsep}{0pt}%
  \setlength{\parsep}{0pt}%
  \setlength{\parskip}{0pt}%
  \setlength{\bibsep}{0ex}
  \raggedright
}
\setlength{\footnotesep}{0ex} % seriously?

% citation alias

% math shih
\newcommand{\setof}[1]{\left\{{#1}\right\}}
\newcommand{\given}{\,|\,}
\newcommand{\pseudo}{{\mathrm{pseudo}}}
\newcommand{\Var}{\mathrm{Var}}
\newcommand{\Nd}{{100}\xspace}

\newcommand{\Om}{\Omega_{\rm m}} 
\newcommand{\Ob}{\Omega_{\rm b}} 
\newcommand{\OL}{\Omega_\Lambda}
\newcommand{\smnu}{\sum m_\nu} 
\newcommand{\sig}{\sigma_8} 
\newcommand{\hmpc}{\,h/\mathrm{Mpc}}
\newcommand{\lss}{{\small{LSS}}\xspace}
\newcommand{\mpc}{h/{\rm Mpc}}
\newcommand{\quij}{{\sc Quijote}~}

\newcommand{\specialcell}[2][c]{%
  begin{tabular}[#1]{@{}c@{}}#2\end{tabular}}
% text shih
\newcommand{\foreign}[1]{\textsl{#1}}
\newcommand{\etal}{\foreign{et~al.}}
\newcommand{\opcit}{\foreign{Op.~cit.}}
\newcommand{\documentname}{\textsl{Article}}
\newcommand{\equationname}{equation}
\newcommand{\bitem}{\begin{itemize}}
\newcommand{\eitem}{\end{itemize}}
\newcommand{\beq}{\begin{equation}}
\newcommand{\eeq}{\end{equation}}

\begin{document}\sloppy\sloppypar\frenchspacing 
We're using Paco's \quij suite (Table~\ref{tab:sims}) to determine
whether using paired-fixed simulations instead of standard $N$-body 
simulations can bias clustering analyses with the redshift-space
power spectrum and bispectrum. More specifically, the context we have
in mind is: would an emulator constructed using paired-fixed sims
bias parameter constraints. For some observable, $X$, we quantify the 
bias as
\beq
\beta_X = \frac{\overline{X}_{\rm std} - \overline{X}_{\rm pf}}{\sqrt{\sigma_{\rm std}^2 + \sigma_{\rm pf}^2}}
\eeq
where $\overline{X}_{\rm std} = \frac{1}{N_{\rm std}}\sum\limits_{i=1}^{N_{\rm std}} X_{{\rm std}, i}$ 
and $\overline{X}_{\rm pf} = \frac{1}{N_{\rm pf}}\sum\limits_{i=1}^{N_{\rm pf}} X_{{\rm pf}, i}$. 
$X_{{\rm pf},i}$ is the mean observable for a pair of paired-fixed simulations: 
$X_{{\rm pf},i} = \frac{1}{2} \big[ X_{{\rm pf},i}^{(1)} + X_{{\rm pf},i}^{(2)}\big]$.
For the fiducial cosmology we use $N_{\rm std} = 15000$ standard sims and 
$N_{\rm pf} = 250$ pairs of paired fixed sims. For the other cosmologies, 
we use $N_{\rm std} = 500$ and $N_{\rm pf} = 250$ pairs but for observables
in redshift-space, we average the observables for redshift-space distortions
imposed along the three axes. 

We first look at the amplitudes of the power spectrum and bispectrum in 
Figures~\ref{fig:pk} and~\ref{fig:bk_rsd}. Among the 14 different 
cosmologies we have in \quij, we find significant bias for $P_2$ at
the $h^+$ cosmology. We find similar discrepancies for the bispectrum  
at the $h^+$ and $M_\nu^{+++}$ cosmologies, but nothing statistically 
significant. Next we look at the derivatives of the power 
spectrum and bispectrum with respect to $\Om$, $\Ob$, $h$, $n_s$, $\sig$, 
and $\smnu$ in Figures~\ref{fig:dpk_rsd} and~\ref{fig:dbk_rsd}. Here we 
find more significant biases, especially for the derivatives with respect 
to $h$ and $\sig$. Lastly in Figures~\ref{fig:ppost_rsd} and~\ref{fig:ppost_rsd}, 
we examine how the discrepancies in the derivatives propagate to parameter
constraints estimated using Fisher. We find $>10\%$ discrepancies in the 
marginalized parameter constraints. 


%%%%%%%%%%%%%%%%%%%%%%%%%%%%%%%%%%%%%%%%%%
% simulation table
%%%%%%%%%%%%%%%%%%%%%%%%%%%%%%%%%%%%%%%%%%
\begin{table}
    \caption{
    The \quij suite includes 15,000 standard $N$-body simulations at the fiducial cosmology to 
    accurately estimate the covariance matrices. It also includes sets of 500 simulations at 13 
    different cosmologies, where only one parameter is varied from the fiducial value (underlined), 
    to estimate derivatives of observables along the cosmological parameters. At every cosmology,
    the \quij suite also includes 250 pairs of paired-fixed simulations.  
    } 
    \begin{center}
    \begin{tabular}{cccccccccc} \toprule
    Name  &$\smnu$ & $\Omega_m$ & $\Omega_b$ & $h$ & $n_s$ & $\sigma_8$ & ICs & standard & paired-fixed \\
    & & & & & & & & realizations & pairs \\[3pt] \hline\hline
    Fiducial 	& 0.0         & 0.3175 & 0.049 & 0.6711 & 0.9624 & 0.834 & 2LPT & 15,000 & 250\\ 
    Fiducial ZA     & 0.0         & 0.3175 & 0.049 & 0.6711 & 0.9624 & 0.834 & Zel'dovich& 500 & 250\\ 
    $\smnu^+$       & \underline{0.1} eV & 0.3175 & 0.049 & 0.6711 & 0.9624 & 0.834 & Zel'dovich & 500 & 250\\ 
    $\smnu^{++}$    & \underline{0.2} eV & 0.3175 & 0.049 & 0.6711 & 0.9624 & 0.834 & Zel'dovich & 500 & 250\\ 
    $\smnu^{+++}$   & \underline{0.4} eV & 0.3175 & 0.049 & 0.6711 & 0.9624 & 0.834 & Zel'dovich & 500 & 250\\ 
    $\Omega_m^+$    & 0.0   & \underline{ 0.3275} & 0.049 & 0.6711 & 0.9624 & 0.834 & 2LPT & 500 & 250\\ 
    $\Omega_m^-$    & 0.0   & \underline{ 0.3075} & 0.049 & 0.6711 & 0.9624 & 0.834 & 2LPT & 500 & 250\\ 
    $\Omega_b^+$    & 0.0   & 0.3175 & \underline{0.051} & 0.6711 & 0.9624 & 0.834 & 2LPT & 500 & 250\\ 
    $\Omega_b^-$    & 0.0   & 0.3175 & \underline{0.047} & 0.6711 & 0.9624 & 0.834 & 2LPT & 500 & 250\\ 
    $h^+$           & 0.0   & 0.3175 & 0.049 & \underline{0.6911} & 0.9624 & 0.834 & 2LPT & 500 & 250\\ 
    $h^-$           & 0.0   & 0.3175 & 0.049 & \underline{0.6511} & 0.9624 & 0.834 & 2LPT & 500 & 250\\ 
    $n_s^+$         & 0.0   & 0.3175 & 0.049 & 0.6711 & \underline{0.9824} & 0.834 & 2LPT & 500 & 250\\ 
    $n_s^-$         & 0.0   & 0.3175 & 0.049 & 0.6711 & \underline{0.9424} & 0.834 & 2LPT & 500 & 250\\ 
    $\sigma_8^+$    & 0.0   & 0.3175 & 0.049 & 0.6711 & 0.9624 & \underline{0.849} & 2LPT & 500 & 250\\ 
    $\sigma_8^-$    & 0.0   & 0.3175 & 0.049 & 0.6711 & 0.9624 & \underline{0.819} & 2LPT & 500 & 250\\[3pt]
    \hline
    \end{tabular} \label{tab:sims}
    \end{center}
\end{table}
%%%%%%%%%%%%%%%%%%%%%%%%%%%%%%%%%%%%%%%%%%


\begin{figure}
\begin{center}
    \includegraphics[width=0.8\textwidth]{figs/pf_Pk.pdf} 
    \caption{Comparison of the power spectra of standard $N$-body and paired-fixed
    simulations. We compare the real-space power spectrum, redshift-space power 
    spectrum monopole, and quadrupole in the left, center, and right columns, 
    respectively. In the top panels, we compare the average power spectra of t
    he standard (black line) to the paired-fixed simulations (orange scatter) 
    at the fiducial cosmology. In the bottom panels, we compare the bias, $\beta$ 
    in the power spectrum of the paired-fixed simulations for all 14 cosmologies 
    (Table~\ref{tab:sims}). We mark bias within $\pm2\sigma$ within the shaded 
    region. At the fiducial cosmology we use 15,000 standard $N$-body and $250$ 
    pairs of paired-fixed simulations. For the other cosmologies, we use 500 
    standard $N$-body and $250$ pairs of paired-fixed simulations; for the 
    redshift-space $P_\ell$, we average the $P_\ell$s for redshift-space distortions
    imposed along the three axes for each realization. We find differences between
    $P_2$ of  standard and paired fixed simulations for the $h^+$ cosmology but
    find no significant bias in other cosmologies.  
    }
\label{fig:pk}
\end{center}
\end{figure}


\begin{figure}
\begin{center}
    \includegraphics[width=0.8\textwidth]{figs/pf_Bk_rsd.pdf} 
    \caption{Comparison of the redshift-space bispectrum of standard $N$-body 
    and paired-fixed simulations. In the top panel, we compare the average 
    bispectrum of the standard (black line) to the paired-fixed simulations 
    (orange scatter) at the fiducial cosmology. We include all 1898 triangles 
    out to $k_{\rm max} = 0.5$. In the bottom panels, we compare the bias, 
    $\beta$, in the bispectrum of the paired-fixed simulations for all 14 
    cosmologies, as a function of triangle configuration (left) and their 
    distribution (right). The $\beta$ distributions of the different cosmologies 
    are mostly in good agreement with $\mathcal{N}(0,1)$ (lower right panel). 
    A few cosmologies, $h^+$, $M_\nu^{+++}$, have noticable discrepancies; however, 
    these discrepancies are within $1\sigma$. {\em Hence, we also find no significant 
    bias in the redshift-space bispectrum of paired-fixed simulations.}
    }
\label{fig:bk_rsd}
\end{center}
\end{figure}

\begin{figure}
\begin{center}
    \includegraphics[width=\textwidth]{figs/pf_dPdtheta_rsd_dmn_fin_dh_pm.pdf} 
    \caption{Comparison of the redshift-space power spectrum monopole (top panels) 
    and quadrupole (center panels) derivatives with respect to $\Om$, $\Ob$, $h$, 
    $n_s$, $\sigma_8$, and $M_\nu$ derived from standard $N$-body versus 
    paired-fixed simulations. In the top panels, we compare the derivatives from 
    the standard simulations (black) to the paired-fixed simulations (orange). 
    In the bottom panels, we compare the bias, $\beta$ of ${\rm d} P_\ell/{\rm d}\theta$.
    We find significant biases for the derivatives with respect to $h$ and $\sig$. 
    For ${\rm d} P_\ell/{\rm d}h$, we find that bias exceeds $2\sigma$.} 
\label{fig:dpk_rsd}
\end{center}
\end{figure}

\begin{figure}
\begin{center}
    \includegraphics[width=\textwidth]{figs/pf_dBdtheta_rsd_kmax0_5_dmnu_fin_dh_pm.pdf} 
    \caption{Comparison of the redshift-space bispectrum derivatives with respect
    to $\Om$, $\Ob$, $h$, $n_s$, $\sigma_8$, and $M_\nu$ derived from standard 
    $N$-body simulations versus paired-fixed simulations. In the left panels, we 
    compare the derivatives from the standard simulations (black) to the paired-fixed
    simulations (orange). In the right panels, we compare the bias, $\beta$ of
    the derivatives for each parameter. We find significant bias for ${\rm d} B_0/{\rm d}h$
    and ${\rm d} B_0/{\rm d}\sig$. For ${\rm d} B_0/{\rm d}h$, the bias exceeds 
    $2\sigma$ for triangle configurations with $0.1 \lesssim k_1, k_2, k_3 \lesssim 0.3~\mpc$.}
\label{fig:dbk_rsd}
\end{center}
\end{figure}

\begin{figure}
\begin{center}
    \includegraphics[width=\textwidth]{figs/pf_P_posterior_rsd_kmax0_5_dmnu_fin_dh_pm.pdf}
    \caption{$P_\ell$ Fisher forecast parameter constraints derived from standard $N-body$ 
    (blue) versus paired-fixed simulations (orange). We find $\ge 20\%$ discrepancies in the
    marginalized parameter constraints for $h$ and $n_s$ from using paired-fixed simulations.
    These discrepancies are significantly larger than $\sim 5-10\%$ level precision expected 
    from Fisher forecasts. 
    }
\label{fig:ppost_rsd}
\end{center}
\end{figure}

\begin{figure}
\begin{center}
    \includegraphics[width=\textwidth]{figs/pf_B_posterior_rsd_kmax0_5_dmnu_fin_dh_pm.pdf} 
    \caption{Bispectrum Fisher forecast parameter constraints derived from standard $N-body$ 
    (blue) versus paired-fixed simulations (orange). We find $> 10\%$ discrepancies in the
    marginalized parameter constraints for $n_s$ from using paired-fixed simulations.
    This is larger than $\sim 5-10\%$ level precision expected from Fisher forecasts. 
    } 
\label{fig:bpost_rsd}
\end{center}
\end{figure}

\end{document}
