\documentclass[12pt, letterpaper, preprint]{aastex62}
%\usepackage[breaklinks,colorlinks, urlcolor=blue,citecolor=blue,linkcolor=blue]{hyperref}
\usepackage{color}
\usepackage{amsmath}
\usepackage{natbib}
\usepackage{ctable}
\usepackage{bm}
\usepackage[normalem]{ulem} % Added by MS for \sout -> not required for final version
\usepackage{xspace}

% typesetting shih
\linespread{1.08} % close to 10/13 spacing
\setlength{\parindent}{1.08\baselineskip} % Bringhurst
\setlength{\parskip}{0ex}
\let\oldbibliography\thebibliography % killin' me.
\renewcommand{\thebibliography}[1]{%
  \oldbibliography{#1}%
  \setlength{\itemsep}{0pt}%
  \setlength{\parsep}{0pt}%
  \setlength{\parskip}{0pt}%
  \setlength{\bibsep}{0ex}
  \raggedright
}
\setlength{\footnotesep}{0ex} % seriously?

% citation alias

% math shih
\newcommand{\setof}[1]{\left\{{#1}\right\}}
\newcommand{\given}{\,|\,}
\newcommand{\pseudo}{{\mathrm{pseudo}}}
\newcommand{\Var}{\mathrm{Var}}
\newcommand{\Nd}{{100}\xspace}

\newcommand{\Om}{\Omega_{\rm m}} 
\newcommand{\Ob}{\Omega_{\rm b}} 
\newcommand{\OL}{\Omega_\Lambda}
\newcommand{\smnu}{\sum m_\nu} 
\newcommand{\sig}{\sigma_8} 
\newcommand{\hmpc}{\,h/\mathrm{Mpc}}
\newcommand{\lss}{{\small{LSS}}\xspace}

\newcommand{\specialcell}[2][c]{%
  begin{tabular}[#1]{@{}c@{}}#2\end{tabular}}
% text shih
\newcommand{\foreign}[1]{\textsl{#1}}
\newcommand{\etal}{\foreign{et~al.}}
\newcommand{\opcit}{\foreign{Op.~cit.}}
\newcommand{\documentname}{\textsl{Article}}
\newcommand{\equationname}{equation}
\newcommand{\bitem}{\begin{itemize}}
\newcommand{\eitem}{\end{itemize}}
\newcommand{\beq}{\begin{equation}}
\newcommand{\eeq}{\end{equation}}

\begin{document}\sloppy\sloppypar\frenchspacing 
To provide a bit of context, we’re using Paco’s HADES and Quijote simulations 
for the analysis. HADES are the simulations from Paco’s (and your) 2018 paper: 
100 realizations at different ($\smnu$, $\sig$) pairs. Quijote is Paco’s new 
suite of sims that has 15,000 realizations at a fiducial cosmology and 
6,500 more realizations run with all the parameters ($\Om, \Ob, h, n_s, \sig, \smnu$) 
fixed but changing one to either higher or lower than the fiducial value. This is 
to calculate the derivatives for Fisher forecasts. All of the realizations are 
$1~h^{-1}{\rm Gpc}$ periodic boxes at $z=0$. The analysis is for halo bispectrum 
where we impose a $M_h > 3.2\times10^13~M_\odot/h$ limit. The number densities of 
the realizations are $\sim 1.5\times10^{-4}$. 

First, here are the redshift-space halo bispectrum $B(k_1, k_2, k_3)$ 
plots for the HADES simulations: the shape dependence in 
Figure~\ref{fig:bk_shape} and the amplitude in Figure~\ref{fig:bk_amp}. 
More importantly, in Figure~\ref{fig:dbk_shape} we plot the 
shape-depenence of the $\smnu$ and $\sig$ impact on the bispectrum.
The top and bottom panels algined in the three leftmost columns have 
almost the same CDM+baryon $\sig$. Similarly, in Figure~\ref{fig:dbk_amp}, 
we plot the amplitude of the $\smnu$ and $\sig$ impact on the bispectrum.
Figures~\ref{fig:dbk_shape} and~\ref{fig:dbk_amp} illustrate that 
the imprint of $\smnu$ is more distinguishable from $\sig$ in the 
bispectrum than in the powerspectrum. 

Next we use the Quijote simulation suite to calculate the Fisher 
matrix and forecast $\smnu$ constraints. In Figure~\ref{fig:bk_fish}, 
we plot the constraints on the cosmological parameters from the Fisher 
forecast for the powerspectrum monopole (blue) and bispectrum (orange) 
over the ranges $k < 0.5$ and $0.01 \leq k_1, k_2, k_3 \leq 0.5$, 
respectively. The bispectrum substantially improves all the constraints. 
For $\smnu$, the bispectrum improves constraints from $\sigma_{\smnu} = 0.279$ 
to $0.0258$, which is over an order of magnitude improvement. In 
Figures~\ref{fig:pk_fish_kmax} and~\ref{fig:bk_fish_kmax}, we plot the 
Fisher forecast constraints for $k_{\rm max} = 0.2, 0.3, 0.5$. 
Regardless of $k_{\rm max}$, the bispectrum improves the $\smnu$ constraint 
by roughly an order of magntiude. 

Here's a few tests we've run to check our results:
\begin{itemize} 
    \item We've tested the convergence of the covariance matrix by calculating the 
        Fisher constraints where we estimate the covariance matrix with different 
        number of Quijote simulations. We find that the constraints converge after 
        $\sim 10,000$ mocks. 
    \item We've also tested the convergence of the derivatives $\frac{\partial B}{\partial \theta_i}$ 
        by varying the number of Quijote simulations. Although we have 15,000 realizations
        at the fiducial parameter, for the derivatives we only have 500. Still, we find 
        only small changes $ < 10\%$ in the constraints from using 400 to 500. 
    \item We've calculated $\Delta\chi^2$ values for the HADES simulations and 
        they're in agreement with the contours that we find.  
    \item We've also run the Fisher forecasts using select triangle configurations,
        such as equilateral and squeezed, for different $k_{\rm max}$. We find that 
        the improvements on $\smnu$ saturate at $k_{\rm max} \sim 0.4$, which 
        correctly corresponds to the scale where we find that shot noise dominates. 
\end{itemize}

\begin{figure}
\begin{center}
    \includegraphics[width=\textwidth]{figs/haloBk_shape_001_05_rsd.pdf} 
    \caption{The redshift-space halo bispectrum, $B(k_1, k_2, k_3)$ as a 
    function of triangle configuration shape for $\smnu = 0.0, 0.06, 0.10$, 
    and $0.15\,\mathrm{eV}$ (top panels) and $\sig = 0.822, 0.818, 0.807$, 
    and $0.798$ (lower panels). We include all triangle configurations within
    the $k$ range: $0.01 \leq k_1, k_2, k_3 \leq 0.5$.}
\label{fig:bk_shape}
\end{center}
\end{figure}

\begin{figure}
\begin{center}
\includegraphics[width=0.9\textwidth]{figs/haloBk_amp_001_05_rsd.pdf}
    \caption{The redshift-space halo bispectrum, $B(k_1, k_2, k_3)$, as a
    function of all triangle configurations within $0.01 \leq k_1, k_2, k_3 \leq 0.5$.
    for $\smnu = 0.0, 0.06, 0.10$, and $0.15\,\mathrm{eV}$ (top panel) and 
    $\sig = 0.822, 0.818, 0.807$, and $0.798$ (lower panel). The triangle
    configurations are ordered such that $k_1 \leq k_2 \leq k_3$, the same 
    ordering as Hector's plots. The shaded region in plots the uncertainties 
    from the covariance matrix estimated from the Quijote simulations 
    (Figure~\ref{fig:bk_cov}).}
\label{fig:bk_amp}
\end{center}
\end{figure}

\begin{figure}
\begin{center}
\includegraphics[width=\textwidth]{figs/haloBk_dshape_001_05_rsd.pdf} 
    \caption{The shape dependence of the impact $\smnu$ and $\sig$ have on
    the redshift-space halo bispectrum, $\Delta B/B^\mathrm{(fid)}$. 
    $\smnu = 0.06, 0.10$, and $0.15\,\mathrm{eV}$ (top panels; left to right) 
    are aligned with $\sig = 0.822, 0.818$, and $0.807\,\mathrm{eV}$ 
    (bottom planes; left to right). In each of the three leftmost columns,
    the top and bottom panels have matching CDM + baryon $\sig$, which produce 
    mostly degenerate imprints on the redshift-space power spectrum. The 
    differences between the top and bottom panels in every column illustrate 
    that {\em $\smnu$ induces a sigificantly different impact on the shape-dependence 
    of the halo bispectrum than $\sig$}. 
    }
\label{fig:dbk_shape}
\end{center}
\end{figure}


\begin{figure}
\begin{center}
\includegraphics[width=0.9\textwidth]{figs/haloBk_residual_001_05_rsd.pdf}
    \caption{The impact of $\smnu$ and $\sig$ on the redshift-space 
    halo bispectrum for all triangle configurations within $0.01 \leq k_1, k_2, k_3 \leq 0.5$: 
    $\Delta B/B^\mathrm{(fid)}$. $\Delta B/B^\mathrm{(fid)}$s plotted in each panel 
    have matching $\sig$ values. The impact of $\smnu$ differs significantly from the 
    impact of $\sig$ both in amplitude and scale dependence. For instance, in the 
    bottom panel, $\smnu = 0.15\,\mathrm{eV}$ (red) has a $\sim 5\%$ larger impact 
    on the bispectrum than $\sig = 0.807$ (black). Combined with the shape-dependence 
    of Figure~\ref{fig:dbk_shape}, {\em the contrasting impact of $\smnu$ and $\sig$ 
    on the redshift-space halo bispectrum illustrate that the bispectrum break the 
    degeneracy between $\smnu$ and $\sig$ that degrade constraints from two-point 
    analyses}. 
    }
\label{fig:dbk_amp}
\end{center}
\end{figure}

\begin{figure}
\begin{center}
    \includegraphics[width=0.6\textwidth]{figs/quijote_bkCov_001_05.png} 
    \caption{Covariance matrix of the redshift-space halo bispectrum estimated 
    using the 15,000 realizations of the Qujiote simulation suite with the 
    fiducial cosmology: $\Om{=}0.3175, \Ob{=}0.049, h{=}0.6711, n_s{=}0.9624, \sig{=}0.834$, 
    and $\smnu{=}0.0$ eV. The triangle configurations (the bins) are ordered in 
    the same fashion as Figures~\ref{fig:bk_amp} and~\ref{fig:dbk_amp}.
    }
\label{fig:bk_cov}
\end{center}
\end{figure}

\begin{figure}
\begin{center}
    \includegraphics[width=\textwidth]{figs/quijote_pbkFisher_001_05.pdf}
    \caption{Constraints on cosmological parameters from the redshift-space halo 
    powerspectrum monopole (blue) and bispectrum (orange) derived using Fisher 
    matrices computed from the Quijote simulation suite. For the powerspectrum 
    we include modes with $k \leq 0.5$; for the bispectrum we include triangle 
    configurations with $0.01 \leq k_1, k_2, k_3 \leq 0.5$. The bispectrum 
    {\em substantially} improves constraints on the cosmological parameters. 
    The improvement over the powerspectrum is particularly evident for $\smnu$, 
    where the $\smnu$ constraint improves from $\sigma_{\smnu} = 0.279$ to 
    $0.0258$ with the bispectrum --- {\em over an order of magnitude improvement}.}
\label{fig:bk_fish}
\end{center}
\end{figure}


\begin{figure}
\begin{center}
    \includegraphics[width=\textwidth]{figs/quijote_pkFisher_kmax.pdf} 
    \caption{Constraints on cosmological parameters from the redshift-space
    halo powerspectrum monopole for $k_{\rm max} = 0.2$ (green) 0.3 (orange), 
    and 0.5 (blue).}
\label{fig:pk_fish_kmax}
\end{center}
\end{figure}


\begin{figure}
\begin{center}
    \includegraphics[width=\textwidth]{figs/quijote_bkFisher_kmax.pdf} 
    \caption{Constraints on cosmological parameters from the redshift-space
    halo bispectrum for $0.01 \leq k_1, k_2, k_3 \leq k_{\rm max} = 0.2$ (green) 0.3 (orange), 
    and 0.5 (blue).}
\label{fig:bk_fish_kmax}
\end{center}
\end{figure}
\end{document}
